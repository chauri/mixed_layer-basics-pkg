%% LyX 2.1.0 created this file.  For more info, see http://www.lyx.org/.
%% Do not edit unless you really know what you are doing.
\documentclass[english]{article}
\usepackage[T1]{fontenc}
\usepackage[latin9]{inputenc}
\usepackage{babel}
\usepackage{refstyle}
\usepackage{amsmath}
\usepackage{graphicx}
\usepackage{esint}
\usepackage[authoryear]{natbib}
\usepackage[unicode=true,pdfusetitle,
 bookmarks=true,bookmarksnumbered=false,bookmarksopen=false,
 breaklinks=false,pdfborder={0 0 0},backref=false,colorlinks=false]
 {hyperref}

\makeatletter

%%%%%%%%%%%%%%%%%%%%%%%%%%%%%% LyX specific LaTeX commands.

\AtBeginDocument{\providecommand\subref[1]{\ref{sub:#1}}}
\AtBeginDocument{\providecommand\eqref[1]{\ref{eq:#1}}}
\AtBeginDocument{\providecommand\figref[1]{\ref{fig:#1}}}
\AtBeginDocument{\providecommand\tabref[1]{\ref{tab:#1}}}
\RS@ifundefined{subref}
  {\def\RSsubtxt{section~}\newref{sub}{name = \RSsubtxt}}
  {}
\RS@ifundefined{thmref}
  {\def\RSthmtxt{theorem~}\newref{thm}{name = \RSthmtxt}}
  {}
\RS@ifundefined{lemref}
  {\def\RSlemtxt{lemma~}\newref{lem}{name = \RSlemtxt}}
  {}


%%%%%%%%%%%%%%%%%%%%%%%%%%%%%% User specified LaTeX commands.
%-----------------------------
% Slightly less formal
% appearance
%-----------------------------

% Adjust margins

\usepackage{fullpage}

% Change font to Times

%\usepackage{times}

% New maketitle: Date right-aligned at top of page, followed by title and author left-aligned

\makeatletter
\renewcommand{\maketitle}{
\begin{flushleft}
\textbf{\@title} \\
\@author
\end{flushleft}
}
\makeatother

% Section titles are same font size as text but bold

\usepackage[rm,bf,small,raggedright]{titlesec}

% Other

\usepackage{paralist}

%\setlength{\parindent}{0pt} % if you don't want indented paragraphs
\setlength{\parskip}{2 ex plus 0.5ex minus 0.2ex}

%-----------------------------
% Additional
%-----------------------------

% Some packages

\usepackage{booktabs}  % Prettier tables
\usepackage{lineno}
\usepackage{url}
\usepackage{array}  % Extend options for table column format
\usepackage{sparklines} % Table sparklines
\usepackage{longtable}  % Multipage tables
\usepackage[singlelinecheck=off]{caption} % Left-align captions
\usepackage[section]{placeins} % Float barrier at section change

% Watermark for drafts

%\usepackage[firstpage]{draftwatermark}
%\SetWatermarkText{\today}

% Always capitalize Fig and Table references

\renewcommand{\tabref}{\Tabref}
\renewcommand{\figref}{\Figref}

\newref{sec}{refcmd={Section \ref{#1}}}
\newref{sub}{refcmd={Section \ref{#1}}}

% Put figures and tables at the end

% \usepackage{endfloat,longtable}
% \DeclareDelayedFloatFlavor*{longtable}{table}

% Check size of tables (for drafts)

%\usepackage{showframe} 

% Authors and affiliations

\usepackage{authblk} % For author affiliations
\author[1]{Kelly A. Kearney}
\author[2]{Charles Stock}
\affil[1]{Cooperative Institute for Marine and Atmospheric Studies, University of Miami Rosenstiel School of Marine and Atmospheric Science, 4301 Rickenbacker Causeway Miami, FL 33149 (corresponding author: kkearney@rsmas.miami.edu)}
\affil[2]{NOAA Geophysical Fluid Dynamics Laboratory, 201 Forrestal Road, Princeton, NJ 08540-6649}

% For tables

\usepackage{sparklines}
\usepackage{array}
\usepackage{ragged2e}

% Bibliography style

\bibpunct[; ]{(}{)}{;}{a}{,}{;} 

% For float barrier

\usepackage{placeins}

% Longtable caption width

\setlength{\LTcapwidth}{\textwidth}

\makeatother

\begin{document}

\title{The Kearney ecosystem model}

\maketitle
\tableofcontents{}

\listoftables



\section{Introduction}

The model described in this document is a fully-coupled physical,
biogeochemical, and ecosystem model. The generic model framework consists
of a simple physical water column model, within which I run a biological
module that adds in all the biologically-relevant state variables
to this physical model. I never gave this model (or set of models)
a proper name; I have a personal aversion to the trend of unintelligible
acronyms but nothing better ever came to mind. I generally refer to
it by the names of the Matlab functions it uses: the physical model
is \texttt{mixed\_layer.m}, the ecosystem model is \texttt{wce.m}
(for water column ecosystem... damn, and unpronounceable and uninformative
acronym!), and the NPZ-only biological module that I use for some
setup and comparison is \texttt{nemurokak.m} (my own variant of the
popular NEMURO model, with my initials appended). Taken as a whole,
I've just named the thing after myself, since that's how it shows
up for citation purposes.

Earlier versions of this documentation have appeared as an appendix
to my Ph.D. thesis \citep{Kearney2012a} and as supplementary material
to an earlier journal article \citep{Kearney2013}. Since then, the
model has continued to evolve, and likely will continue to do so.
I've spent a lot of time pulling my own hair out in attempts to understand
the inner workings of other people's models, so I'm trying to keep
things as transparent as possible with my own little pet model. This
document will likely continue to accompany any publications that use
this model as supplementary material, since all of this is far too
detailed to go into journal articles themselves.

The basic model framework is designed to be generic to any pelagic
ocean ecosystem. But in practice, it requires a huge number of input
variables and datasets, so its development has been very closely tied
to the specifics of my test ecosystem, the eastern subarctic gyre
of the Pacific Ocean. The parameters and input datasets used to run
the model for that ecosystem are listed here in full.


\section{The physical model}


\subsection{Model framework and equations}

The mixed-layer model simulates the evolution of water column properties
under specified forcing by wind, heat, and salinity forcing. Allowance
is also made for currents via a depth-independent pressure acceleration.
There are six physical state variables in the physical model formulation:
$U$ and $V$ are the east to west and south to north current velocities,
and $T$ and $S$ are the temperature and salinity. The turbulence
closure scheme introduces the remaining two state variables; : $q^{2}$
is a turbulent quantity equal to twice the turbulent kinetic energy,
and $\ell$ is a turbulent length scale. The evolution of these quantities
are governed according to:

\begin{equation}
\frac{{\partial U}}{dt}-fV=-\frac{{1}}{\rho_{_{0}}}\frac{{\partial p}}{\partial x}+\frac{\partial}{\partial z}K_{M}\frac{{\partial U}}{\partial z}-\epsilon U\label{eq:dU-1}
\end{equation}


\begin{equation}
\frac{{\partial V}}{dt}+fU=-\frac{{1}}{\rho_{_{0}}}\frac{{\partial p}}{\partial y}+\frac{\partial}{\partial z}K_{M}\frac{{\partial V}}{\partial z}-\epsilon V\label{eq:dV-1}
\end{equation}


\begin{equation}
\frac{\partial q^{2}}{\partial t}=\frac{\partial}{\partial z}K_{q}\frac{\partial q^{2}}{\partial z}+2K_{M}\left[\left(\frac{\partial U}{\partial z}\right)^{2}+\left(\frac{\partial V}{\partial z}\right)^{2}\right]+\frac{2g}{\rho_{0}}K_{H}\frac{\partial\rho}{\partial z}-\frac{2q^{3}}{B_{1}\ell}\label{eq:dq2}
\end{equation}


\begin{equation}
\frac{\partial q^{2}\ell}{\partial t}=\frac{\partial}{\partial z}K_{q}\frac{\partial q^{2}\ell}{\partial z}+E_{1}\ell\left(K_{M}\left[\left(\frac{\partial U}{\partial z}\right)^{2}+\left(\frac{\partial V}{\partial z}\right)^{2}\right]+E_{3}\frac{g}{\rho_{0}}K_{H}\frac{\partial\rho}{\partial z}\right)-\frac{q^{3}}{B_{1}}W\label{eq:dq2l}
\end{equation}


Boundary forcing for $U$ and $V$ are set by the wind and bottom
stress (see \subref{Wind-forcing} and \subref{Bottom-stress}). The
depth-independent pressure acceleration terms is specified as part
of the model forcing. The bottom stress generated by currents arising
from this term help prevent the bottom layers of the water column
from stagnating. A dissipation term is included in \eqref{dU} and
\eqref{dV} as a surrogate for horizontal momentum divergence. This
term removes energy from past storm events over a specified time-scale
as though energy was being transferred to more quiescent surrounding
waters. Energy tends to accumulate unrealistically in 1D water columns
without this effect \citep{mellor2001one}. The value of $\epsilon$
can be tuned such that the energy in the modeled currents is consistent
with that observed. Values comparable to the time scales of storm
events (0.2-1 day$^{-1}$) seem to yield reasonable results.

The turbulent quantities $q^{2}$ and $q^{2}\ell$ in \eqref{dq2}
and \eqref{dq2l} are used to derive the mixing coefficients for momentum
($K_{M}$), turbulence ($K_{q}$), and tracers ($K_{H}$) using the
Mellor-Yamada 2.5 turbulence closure scheme \citep{mellor1982development}.
Terms on the right hand side of these equations represent the transport
of turbulence, generation of turbulence via shear, and suppression
of turbulence via stratification and dissipation. $B_{1}$, $E_{1},$
$E_{2}$, and $E_{3}$ are constants equal to 16.6, 1.8, 1.33 and
1.0, respectively. $g$ is the acceleration due to gravity, and $W$
is a \textquotedbl{}wall proximity function\textquotedbl{} which limits
the eddy length scale near boundaries. Further details of the turbulence
closure and boundary conditions for the turbulent quantities are described
below.

The temperature and salinity equations are given by: 

\begin{equation}
\frac{\partial T}{\partial t}=\frac{\partial}{\partial z}K_{V}\frac{{\partial T}}{\partial z}+ss\label{eq:dT-1}
\end{equation}


\begin{equation}
\frac{\partial S}{\partial t}=\frac{\partial}{\partial z}K_{V}\frac{{\partial S}}{\partial z}+ss\label{eq:dS-1}
\end{equation}


In the \eqref{dT-1} and \eqref{dS-1}, $ss$ represents a variety
of source and sink terms for temperature and salinity. Temperature
changes are driven by incident radiation and sensible, latent, and
longwave heat fluxes. A net heat transport via advection can also
be specified. Salinity changes are presently driven via relaxation
to specified values. These sources and sinks are described in detail
in the sections that follow. $T$ and $S$ are translated to a density
using a routine from Phil Morgan's CSIRO seawater toolbox (UNESCO
1983 polynomial). Water is modeled as incompressible for this calculation.


\subsubsection{Wind forcing\label{sub:Wind-forcing}}

Wind input is provded as u- and v-components of speed in m/s, and
translated to wind stresses using \citet{large1981open}. The stress
(N m$^{-2}$) is calculated as

\begin{equation}
\tau_{x}=\rho_{air}C_{d}|W_{10}|U_{10}
\end{equation}


\begin{equation}
\tau_{y}=\rho_{air}C_{d}|W_{10}|V_{10}
\end{equation}
where $W_{10}$ is the wind speed at 10m, $C_{d}$ is a dimensionless
drag coefficient, and $\rho_{air}$ is the air density (which is assumed
to be 1.25 kg m$^{-3}$). $W_{10}$ is calculated from the measured
wind speed assuming a logarithmic velocity profile. $C_{d}$ is an
empirical function of the wind speed:

\begin{equation}
C_{d}=\begin{cases}
1.2\times10^{-3} & W_{10}<11\text{ ms}^{-1}\\
1.0\times10^{-3}(0.49+0.065W_{10}) & W_{10}\ge11\text{ ms}^{-1}
\end{cases}
\end{equation}


The \texttt{wstress.m} routine from Rich Signell's Matlab toolbox
is used for this calculation.


\subsubsection{Bottom stresses\label{sub:Bottom-stress}}

The Princeton Ocean Model Formulation from of \citet{mellor2004user}
is used to calculate the bottom stress:

\begin{equation}
\tau_{x,b}=\rho C_{bot}\left(U^{2}+V^{2}\right)^{1/2}U
\end{equation}


\begin{equation}
\tau_{y,b}=\rho C_{bot}\left(U^{2}+V^{2}\right)^{1/2}V
\end{equation}
where $C_{bot}$ is a bottom drag coefficient calculated as:

\begin{equation}
C_{bot}=\max\left(\frac{\kappa^{2}}{\ln\left({\frac{0.5dz}{z_{o}b}}\right)},0.0025\right)
\end{equation}
where $\kappa$ is Von-Karman's constant (0.4, dimensionless), $dz$
is the width of the bottom grid cell (m), and $z_{ob}$ is a bottom
roughness coefficient. The basic form of this relationship comes from
a logarithmic boundary layer solution. $z_{ob}$ is given a value
of 0.01 by default, which is indicative of a rather smooth boundary.
The lower $C_{bot}$ limit comes into play only when $dz$ resolution
is very large relative to the roughness length scale. It was inserted
to be consistent with the POM formulation, but can be avoided as long
as $dz/z_{ob}$ \textasciitilde{} 5000.


\subsubsection{The turbulence closure sub-model}

Vertical mixing is determined using the Mellor-Yamada 2.5 turbulence
closure scheme \citep{mellor1982development} with adjustments and
simplifications described in \citet{Galperin1988} and \citet{mellor2004user}.
Mixing coefficients are derived from the turbulent quantities $q^{2}$
and $\ell$ which are tracked prognostically via \eqref{dq2} and
\eqref{dq2l}. The boundary condition for $q^{2}$ accounts for waves
and is taken from equation (10) of \citet{mellor2004wave}:

\begin{equation}
q^{2}(0)=(15.8\alpha_{CB})^{2/3}u_{\tau}^{2}(z=0)
\end{equation}
where $u_{\tau}$ is the surface \textquotedbl{}friction velocity\textquotedbl{}
driven by the wind ($\sqrt{{\tau_{wind}/\rho_{w}}}$, where $\rho_{w}$
is the surface water density). \citet{mellor2004wave} suggest using
$\alpha_{CB}=100$. A minimum value of 0.0001 is applied, which corresponds
to a wind \textasciitilde{}1 m/s. The bottom boundary condition for
$q^{2}$ is calculated from equation 16a of \citet{mellor2004user}:

\begin{equation}
q^{2}(z=z_{bot})=B_{1}^{2/3}u_{\tau}^{2}(z=z_{bot})
\end{equation}
where $u_{\tau}(z=z_{bot})$ is the bottom friction velocity. The
length scale $\ell$ at the surface is from equations 5a and 6a of
\citet{mellor2004wave}:

\begin{equation}
\ell(z=0)=\kappa z_{w}=\kappa\beta\frac{{u_{tau}^{2}(z=0)}}{g}
\end{equation}
where $\beta$ is 2.0$\times10^{5}$ and $\kappa$ is von Karman's
constant (0.41). The non-zero length scale at the ocean surface reflects
the surface wave roughness. A minimum $z_{w}$ value of 0.02 was imposed.
This corresponds to waves generated by a wind of 1 m s$^{-1}$ and
is meant to mimic the persistence of some wave energy through calm
periods. 

$\ell$ at the bottom boundary was set to 0.

The wall proximity function $W$ limits $\ell$ near the boundaries
by enhancing the dissipation. The formulation used is:

\begin{equation}
1+E_{2}\left(\frac{{\ell}}{\kappa L}\right)^{2}
\end{equation}
where $E_{2}$ is a constant (1.33) and $L$ is the distance from
the surface or the bottom of the water column:

\begin{equation}
L=\frac{1}{|z|}+\frac{1}{|z_{bot}-z|}
\end{equation}
Various wall proximity functions can be found in the literature and
model codes. This one is very similar to those found in various versions
of the Princeton Ocean Model and was taken from \citet{Williams2006}.

$\ell$ is limited in stably stratified flows in accordance with eqs.
(22), (26) and (30) of \citet{Galperin1988}. This is done using the
quantity:

\begin{equation}
G_{H}=\frac{\ell^{2}g}{q^{2}\rho_{0}}\frac{\partial\rho}{\partial z}
\end{equation}
which is the ratio of potential ($\frac{{g}}{\rho_{0}}\frac{{\partial\rho}}{\partial z}$)
to kinetic energy ($\frac{{\ell^{2}}}{q^{2}}$). Since $\frac{{\partial\rho}}{\partial z}$
is negative in stable water columns (i.e. z=0 at the water surface
and becomes more negative with depth), $G_{H}$ is negative in stable
water columns. \citet{Galperin1988} found it necessary to limit $\ell$
in highly stable flows such that stayed below 0.28. Minimum values
of 1e-9 are imposed on both $q^{2}$ and $q^{2}\ell$ in accordance
with the POM2K code.

The diffusivities are calculated from $q^{2}$ and following \citet{mellor1982development}
and \citet{Galperin1988}:

\begin{equation}
K_{M}=q\ell S_{M}
\end{equation}


\begin{equation}
K_{H}=q\ell S_{H}
\end{equation}


\begin{equation}
K_{q}=0.41q\ell S_{H}
\end{equation}


The factors $S_{M}$ and $S_{H}$ all depend on $G_{H}$ according
to:

\begin{equation}
S_{H}=\frac{A_{2}\left(1-6\frac{A_{1}}{B_{1}}\right)}{1-(3A_{2}B_{2}-18A_{1}A_{2})G_{H}}\sim\frac{0.49}{1-34.68G_{H}}
\end{equation}
$A_{1}$, $A_{2}$, $B_{1}$, and $B_{2}$ are empirical constants
that are 0.92, 0.74, 16.6, and 10.1, respectively. Because $S_{H}\rightarrow\infty$
for $G_{H}$ values approaching 0.028 (unstable, high energy conditions),
values of $G_{H}$ are capped at 0.028. $S_{H}\rightarrow0$ as $G_{H}\rightarrow-\infty$
(highly stable or low energy conditions). For $S_{M}$:

\begin{equation}
S_{M}=\frac{A_{1}\left(1-3C_{1}-6\frac{A_{1}}{B_{1}}\right)+S_{H}\left[\left(18A_{1}^{2}+9A_{1}A_{2}\right)G_{H}\right]}{1-9A_{1}A_{2}G_{H}}\sim\frac{0.39+21.36S_{H}G_{H}}{1-6.1G_{H}}
\end{equation}
where $C_{1}$ is 0.08. This factor also approaches 0 as $G_{H}\rightarrow-\infty$.


\subsubsection{Heat forcing}

The heat forcing input calls for the incident flux, air temperature,
and the dew point temperature. Sensible, latent, and longwave fluxes
are calculated using various bulk formulae described below. This has
the advantages of relying on commonly measured quantities and allowing
feedbacks between observed atmosphere conditions and the calculated
ocean temperature. Such feedbacks reduce the probability of developing
large, persistent biases over long model runs.

The incident irradiance ($Q_{i}$) is entered in watts m$^{-2}$ and
the model assumes that data represents energy over the wavelength
range measured by most pyranometers (300-5000 nanometers). This is
adjusted downward according to the albedo specified in the input data.
45\% of the $Q_{i}$ is assumed to be from the photosynthetically
available range (350-700 nm, \citealp{Baker1987}). By default, this
energy is assumed to have an attenuation coefficient of 0.15 m$^{-1}$.
The remaining fraction is attributed to infrared wavelengths and is
absorbed rapidly (1.67 m$^{-1}$). Self-shading by phytoplankton is
applied within the primary production calculations in the ecosystem
model but does not feed back to the physical state variables.

The \citet{friehe1976parameterization} formulae are used to estimate
the sensible heat flux ($Q_{s}$, watts m$^{-2}$):

\begin{equation}
Q_{s}=\rho_{air}C_{p}\bar{{\omega\theta}}
\end{equation}
where $\rho_{air}$ is the air density (assumed to be \textasciitilde{}1.2
kg m$^{-2}$), $C_{p}$ is the specific heat capacity of air (1000
J kg$^{-1}$ K$^{-1}$), and $\bar{{\omega\theta}}$ is the mean vertical
velocity-temperature covariance (m K s$^{-1}$). This latter term
is estimated as a function of wind speed at 10m ($W_{10}$) and rhe
sea-air temperature difference ($\Delta T=T_{ocean}-T_{air})$:

\begin{equation}
\bar{\omega\theta}=\begin{cases}
0.002+(0.97\times10^{-3})W_{10}\Delta T & 0<W_{10}\Delta T<25\text{ (unstable)}\\
(1.46\times10^{-3})W_{10}\Delta T & W_{10}\Delta T\ge25\text{ (highly unstable)}\\
0.0026+(0.86\times10^{-3})W_{10}\Delta T & W_{10}\Delta T\le0\text{ (stable)}
\end{cases}
\end{equation}
The sensible heat goes exclusively into the top grid cell.

The formulae of \citet{friehe1976parameterization} are also used
to estimate the latent heat flux ($Q_{L}$, watts m$^{-2}$):

\begin{equation}
Q_{L}=L_{e}\bar{{wq}}
\end{equation}
where $\bar{{wq}}$ is the mean vertical velocity-water vapor density
covariance (m s$^{-1}$ g m$^{-3}$) and $L_{e}$ is the latent heat
of evaporation (2250 J g$^{-1}$). This is estimated with the formula:

\begin{equation}
Q_{L}=L_{e}(1.32\times10^{-3})W_{10}(q_{s}-q_{a})
\end{equation}


where $q_{s}$ is the water vapor density (g m$^{-3}$) at the sea
surface, and $q_{a}$ is the water vapor density (g m$^{-3}$) at
a reference height above the ocean surface (\textasciitilde{}10m).
$q_{s}$ is calculated as the saturation humidity at the sea surface
temperature (following the approach used in the Princeton Ocean Model).
$q_{a}$ is determined from the measure dew point temperature in the
model forcing. The approximation $e_{s}=Ae^{\beta T}$ is used to
calculate the saturated vapor pressure at temperature $T$ for these
calculations, where $A$ = 611 Pa and $\beta$ = 0.067 $^{\circ}$C$^{-1}$.
This was taken from \citet[p. 6]{Marshall2008a} and gives of the
saturated vapor pressure at typical atmospheric conditions.

The longwave flux ($Q_{LW}$) is calculated using the Efimova formula
as reported by \citet{simpson1979mid}. This calculates the clear-sky
longwave radiation flux as:

\begin{equation}
Q_{LW,clear}=\varepsilon\sigma T_{srf}^{4}\left(0.254-0.00495e_{a}\right)
\end{equation}
where $\varepsilon$ is the emissivity of the surface (0.97), $\sigma$
is the Stefan-Boltzman constant (5.67e-8 W m$^{2}$K$^{-4}$), $T_{srf}$
is the surface ocean temperature (K), and $e_{a}$ is the atmospheric
vapor pressure (in mb = 100 Pascals). The clear sky longwave radiation
loss is adjusted downward by a cloud correction factor of the form:

\begin{equation}
Q_{LW}=Q_{LW,clear}(1-0.8C)
\end{equation}
where $C$ is the ratio of the observed incident radiation ($Q_{i}$)
over 24 hours and the clear-sky value estimated using the Smithsonian
formulas as reported in \citet{Reed1977}. These calculations are
done during the model initialization. The clear-sky irradiance is
calculated as a function of latitude and time of year. As of the writing
of this description, only formulae good for latitudes from 20$^{\circ}$S-60$^{\circ}$N
are included. The details can be found in the routine \texttt{clearsky.m}.

A net heat flux due to advection ($Q_{adv}$) can also be considered.
This heat is added as a source/sink term over a depth scale ($\delta$)
following the approach of \citet{Umoh1994}:

\begin{equation}
Q_{adv}(z)=Q_{adv,o}\frac{{\exp\left({\frac{{z}}{\delta}}\right)}}{\delta\left(1-\exp\left({-\frac{{h}}{\delta}}\right)\right)}
\end{equation}
This assumed that the majority of the depth-integrated heat flux ($Q_{adv,o}$)
is distributed over the depth scale $\delta$, which is effectively
an e-folding scale for the heat transport. $h$ in the above expression
it the total depth of the water column.




\subsection{Input variables for the Eastern Subarctic Gyre}

For the most recent set of simulations (the volcano study), we required
realistic interannual variability, so we used a combination of reanalysis
products for our input forcing. For atmospheric forcing (wind speed
and direction, solar radiation, air temperature, and dew point temperature),
we used the ECMWF ERA-Interim dataset. Water column properties, including
temperature and salinity profiles, derived from ocean state estimates
from ECCO Version 4 Release 1.

\begin{figure}
\includegraphics[width=6.5in]{/Users/kakearney/Documents/Research/Working/fishBgcCoupling/ensembles/setup/erainterimForcing}

\protect\caption{Input datasets used to force the physical model. Atmospheric forcing
data is shown with a 30-day moving average applied for clarity.}
\end{figure}



\section{The biological module}


\subsection{Model framework and equations}

The water column ecosystem model consists of 8 non-living state variables
(particulate and dissolved nutrients) and 3 classes of living state
variables (phytoplankton, zooplankton, and nekton), coupled together
by three sets of ordinary differential equations, one for each of
the three nutrients included in the model.

The designation of living functional groups as either planktonic or
nektonic reflects both their relationship to the physical model and
their interactions with other functional groups. The planktonic label
refers to any group whose movement is strongly influenced by the movement
of the water in which they reside; these groups are resolved with
depth, can feed only on functional groups occupying the same depth
layer as themselves, and are subject to mixing via advection and diffusion
in the same manner as all physical tracers. The nektonic label refers
to all other living organisms, including those that do not live in
the water but feed on marine organisms (such as birds); these groups
are not subject to any mixing, and they feed on the integrated sum
over depth of their prey groups.

Physical and biological ODEs are solved consecutively for the biological
variables. At each model time step, the biological state variables
are first mixed, where applicable, following the same formulation
as \eqref{dT}, with the source minus sink term representing any additional
vertical movement. In most of my simulations, this term is used to
apply sinking velocities to the particulate state variables ($PON$,
$Opal$, and $POFe$), and to prescribe vertical migration behavior
to the copepod group. Following this, the set of ODE equations describing
the exchange of biomass between the biological state variables is
solved using a fourth-order fixed-step solver with a 3-hour time step.

The biological ODEs are based on the addition of several processes:
gross primary production, extracellular excretion, respiration, grazing,
predation, excretion, egestion, non-predatory mortality, decomposition
and remineralization, iron uptake, and iron scavenging (\figref{circleflux}).
The following sections describe each of these processes in detail.

\begin{figure}
\includegraphics[width=1\textwidth]{/Users/kakearney/Documents/Research/Working/fishBgcCoupling/ensembles/analysis/eechoPaperFigs/eecho_circleplot_colorflux}

\protect\caption{Ecosystem processes linking the various biological state variables,
using the Eastern Subarctic food web model.\label{fig:circleflux}}


\end{figure}


A note on subscripts, which are plentiful in this documentation: subscripts
indicate the index of a functional group to which a given process
variable or parameter pertains. I typically use $i$ as this index,
and expand to $j$ and $k$ when I need to represent more than one
functional group in a single equation. Where variables are related
to fluxes between groups, they are denoted by two subscripts (e.g.
$x_{ij}$), with the source group index followed by the sink group
index. For clarity I often separate multiple subscripts from each
other with commas (e.g. $x_{NH_{4},NO_{3}}$ represents a parameter
relating to a flux from the $NH_{4}$ state variable to the $NO_{3}$
variable); commas also separate parameter-name subscripts from functional
group subscripts (e.g. $V_{max,PS}$ is the $V_{max}$ growth rate
parameter associated with the $PS$ state variable).

The values of the various biological state variables are referred
to as $B_{i}$ in all the equations below; depending on the particular
state variable, these can be thought of as either biomass values or
nutrient concentrations. All phytoplankton groups are represented
by three state variables: phytoplankton nitrogen ($B_{i}$), phytoplankton
silica ($B_{Si,i}$), and phytoplankton iron ($B_{Fe,i}$), while
zooplankton and nekton groups consist of only nitrogen state variables. 


\subsubsection{Gross Primary Production (Gpp)}

Gross primary production fluxes flow from the $NO_{3}$ and $NH_{4}$
variables to each phytoplankton group, following \citet{Kishi2007a},
with the addition of iron limitation. The uptake of nitrogen is described
by: 

\begin{equation}
Gpp_{i}=V_{max,i}\exp({K_{gpp,i}T})\cdot L_{nut,i}\cdot L_{light,i}\cdot B_{i}
\end{equation}
where $B_{i}$ is the nitrogen-based biomass of group $i$, resolved
with depth. See \tabref{nemparamgpp} and \tabref{wcenemDerivedParams}
for parameter values.


\subsubsection{Respiration (Res)}

Respiration applies to all phytoplankton groups, and flows from the
phytoplankton to the $NO_{3}$ and $NH_{4}$ groups following the
same f-ratio as uptake via primary production: 

\begin{equation}
Res_{i}=Res_{0i}\exp({K_{res,i}T})B_{i}
\end{equation}


See \tabref{nemparamres} for parameter values.


\subsubsection{Extracellular excretion (Ex)}

Extracellular excretion applies to all phytoplankton groups, and flows
from the phytoplankton to the $DON$ group. Following \citet{Kishi2007a},
extracellular excretion is proportional to the flux due to gross primary
production:

\begin{equation}
Ex_{i}=\gamma_{i}\cdot Gpp{}_{i}
\end{equation}


See \tabref{nemparamexx} for parameter values.


\subsubsection{Consumption (Con)}

Predator/prey interactions between functional groups follow the Aydin
version of the foraging arena functional response. The exact form
of the functional response varies based on whether the predator and
prey groups are planktonic or nektonic. For interactions between two
planktonic groups, the flux (referred to as grazing in \figref{circleflux})
is resolved with depth for both the predator and prey group, and the
uptake rates are temperature-dependent:

\begin{equation}
Con_{ij}=\frac{{Q'_{ij}}}{\exp(K_{Gra,i}\cdot T_{avg})}\exp(K_{Gra,i}\cdot T)\left(\frac{X_{ij}\cdot\frac{{B_{j}}}{B'_{j}}}{X_{ij}-1+\frac{{B_{j}}}{B'_{j}}}\right)\left(\frac{D_{ij}\cdot\left(\frac{{B_{i}}}{B'_{i}}\right)^{\theta_{ij}}}{D_{ij}-1+\left(\frac{{B_{i}}}{B'_{i}}\right)^{\theta_{ij}}}\right)\label{eq:plankCon}
\end{equation}
where here, the subscripts $i$ and $j$ represent the prey and predator
groups, respectively. The biomass and consumption rate parameters
are derived from the Ecopath mass balance: $Q'=\frac{{Q^{*}}}{MLD}$
and $B'=\frac{{B^{*}}}{MLD}$, where $Q^{*}$ and $B^{*}$ are the
per-area mass-balanced quantities returned directly from Ecopath.
The parameters $MLD$ and $T_{avg}$ describe the yearly averaged
mixed layer depth and mixed layer temperature, respectively, as simulated
by the one-dimensional physical model.

For interactions between two nektonic groups, the functional response
follows the same form, but in units of biomass integrated over depth
(termed predation in \figref{circleflux})\textbf{. }For shorthand,\textbf{
$Bint$} is used for depth-integrated biomass. Nektonic consumption
does not vary with temperature.

\begin{equation}
Con_{ij}=Q_{ij}^{*}\left(\frac{X_{ij}\cdot\frac{{Bint_{j}}}{B_{j}^{*}}}{X_{ij}-1+\frac{{Bint_{j}}}{B_{j}^{*}}}\right)\left(\frac{D_{ij}\cdot\left(\frac{{Bint_{i}}}{B_{i}^{*}}\right)^{\theta_{ij}}}{D_{ij}-1+\left(\frac{{Bint_{i}}}{B_{i}^{*}}\right)^{\theta_{ij}}}\right)
\end{equation}


When a nektonic group preys upon a planktonic group, the total flux
is calculated in depth-integrated units. However, the loss on the
plankton side is resolved with depth and distributed proportionally
to the prey biomass at each depth (\eqref{preZoo}), while the flow
to the predator remains in depth-integrated units (\eqref{preNek}):

\begin{equation}
Con_{ij}=Q_{ij}^{*}\left(\frac{X_{ij}\cdot\frac{{Bint_{j}}}{B_{j}^{*}}}{X_{ij}-1+\frac{{Bint_{j}}}{B_{j}^{*}}}\right)\left(\frac{D_{ij}\cdot\left(\frac{{\int_{0}^{z_{max}}B_{i}dz}}{B_{i}^{*}}\right)^{\theta_{ij}}}{D_{ij}-1+\left(\frac{{\int_{0}^{z_{max}}B_{i}dz}}{B_{i}^{*}}\right)^{\theta_{ij}}}\right)\cdot\frac{1}{\Delta z}\cdot\frac{B_{i}\Delta z}{\int_{0}^{z_{max}}{B_{i}}dz}\label{eq:preZoo}
\end{equation}
\begin{equation}
ConI_{ij}=Q_{ij}^{*}\left(\frac{X_{ij}\cdot\frac{{Bint_{j}}}{B_{j}^{*}}}{X_{ij}-1+\frac{{Bint_{j}}}{B_{j}^{*}}}\right)\left(\frac{D_{ij}\cdot\left(\frac{{Bint_{i}}}{B_{i}^{*}}\right)^{\theta_{ij}}}{D_{ij}-1+\left(\frac{{Bint_{i}}}{B_{i}^{*}}\right)^{\theta_{ij}}}\right)\label{eq:preNek}
\end{equation}


In the most recent version of the model code, an additional parameter
is assigned to each predator, indicating prey visibility (as a fraction
from 0-1) versus depth. It is used to adjust $B_{i}$ and $Bint_{i}$
values to each predator based on the foraging/hunting behavior of
the predator. This allows us to, for example, limit nektonic predation
to only the upper portion of the water column for pelagic species. 

See \tabref{nemparamgra}, \tabref{ecopathderivedState}, and \tabref{ecopathderivedLink}
for parameter values.


\subsubsection{Excretion (Exc) and egestion (Ege)}

Egestion and excretion are proportional to the total consumption of
prey by a predator. Egestion flows from the predator to the $PON$
group. Excretion flows from the predator group to the $NH_{4}$ group.
All excretion and egestion by nektonic groups is assumed to take place
in the surface layer:

\begin{align}
Ege_{i}= & GS_{i}\cdot\left(\sum_{k=plank}{Con_{ki}}+\frac{\sum_{\ell=nek}{ConI_{\ell i}}}{\Delta z_{1}}\right) & z=1\\
Ege_{i}= & GS_{i}\cdot\sum_{k}{Con_{ki}} & z\neq1\\
Exc_{i}= & \left(1-GE_{i}-GS_{i}\right)\cdot\left(\sum_{k=plank}{Con_{ki}}+\frac{\sum_{\ell=nek}{ConI_{\ell i}}}{\Delta z_{1}}\right) & z=1\\
Exc_{i}= & (1-GE_{i}-GS_{i})\cdot\sum_{k}{Con_{ki}} & z\neq1
\end{align}


See \tabref{ecopathderivedState} for parameter values.


\subsubsection{Non-predatory mortality (Mor)}

The non-predatory loss process is used to represent the net effect
of a diversity of loss processes, including natural mortality (i.e.
death due to old age), loss to disease and viruses, unresolved intra-group
mortality (such as egg cannibalism and predation on juveniles of similar
species), aggregation and sinking out of the modeled system (primarily
of large phytoplankton), and metabolic costs. It flows from each living
functional group to the $PON$ group, and is allowed to vary in functional
form between groups, based on the exponent parameter $c_{i}$. A value
of $c_{i}=1$ leads to linear mortality, while $c_{i}=2$ indicates
a quadratic loss term. See \citet{Kearney2013} for a detailed sensitivity
study related to this exponent term. For planktonic groups, the non-predatory
flux is in units of mass per volume:

\begin{equation}
Mor_{i}=\left(\frac{M_{0i}}{B_{i}'^{(c_{i}-1)}}\right)\cdot B_{i}^{c_{i}}
\end{equation}


while for nektonic groups it is in units of mass per area:

\begin{equation}
MorI_{i}=\left(\frac{M_{0i}}{B_{i}^{*(c_{i}-1)}}\right)\cdot Bint_{i}^{c_{i}}
\end{equation}


As with egestion and excretion by nektonic groups, non-predatory mortality
of nektonic groups is assumed to occur in the surface layer, such
that in the surface layer, 

\begin{equation}
Mor_{i}=\frac{{MorI_{i}}}{\Delta z_{1}}
\end{equation}


when $i$ = nekton.

See \tabref{nemparammor} and \tabref{ecopathderivedState} for parameters.


\subsubsection{Proportional-to-nitrogen fluxes of silica and iron}

The majority of fluxes between iron- and silica-based state variables
occur in proportion to nitrogenous fluxes. Silica fluxes due to gross
primary production ($Gpp$), extracellular excretion ($Ex$), and
respiration ($Res$) between phytoplankton groups and $SiOH_{4}$
occur in a constant proportion to the respective fluxes in nitrogen
between phytoplankton groups and all dissolved nitrogen pools ($NO_{3}$,
$NH_{4}$, and $DON$). Similarly, fluxes due to non-predatory mortality
($Mor$) from phytoplankton groups to the $PON$ group are accompanied
by proportional fluxes of silica from the large phytoplankton to particulate
opal group. Silica is assumed to be completely egested by phytoplankton
grazers, so the proportional flux due to predator consumption ($Con$)
of phytoplankton silica is routed entirely to the particulate opal
group, rather than being split between predator, egestion, and excretion
as is the case for nitrogenous consumption.

Iron fluxes between the two phytoplankton groups and the dissolved
and particulate iron groups also occur proportionally to nitrogen
fluxes, though the ratio between the two elements varies over time
(see \subref{Iron-uptake-(Qup)}).


\subsubsection{Decomposition (Dec)}

Decomposition fluxes follow the model of \citet{Kishi2007a}, with
a decay rate related to temperature:

\begin{equation}
Dec_{ij}=V_{Dec,ij}\exp(K_{Dec,ij}T)\cdot B_{i}
\end{equation}


where the subscripts $i$ and $j$ represent the source and sink groups,
respectively, and $B_{i}$ the concentration of the source group. 

See \tabref{nemparamdec} for parameters.


\subsubsection{Iron uptake (Qup)\label{sub:Iron-uptake-(Qup)}}

This iron model is based closely on one developed for the Carbon,
Ocean Biogeochemistry and Lower Trophics (COBALT) marine ecosystem
model \citep{Stock}, with a few adjustments to parameter values in
order to tune the dynamics to a one-dimensional water column.

Iron uptake, flowing from the $Fe$ group to each phytoplankton group,
is based on an internal cell quota of the phytoplankton, accounting
for both requirements and additional luxury uptake. Iron's contribution
to overall nutrient limitation, which regulates the uptake of macronutrients,
is termed iron deficiency ($D_{Fe,i}$) and is calculated based on
the internal ratio of iron to nitrogen. However, uptake of iron is
not proportional to uptake of nitrogen, but instead based on a separate
limitation term ($L_{Fe,i}$), allowing phytoplankton to increase
their internal Fe:N ratios to a preset limit:

\begin{equation}
Qup_{i}=\begin{cases}
V_{max,i}\exp({K_{gpp,i}T})\cdot B_{Fe,i}\cdot L_{Fe,i}\cdot\mu_{Fe:N,i}, & \text{if }R_{Fe:N,i}<R_{Fe:Nmax,i}\\
0, & \text{otherwise}
\end{cases}
\end{equation}


Iron is not tracked beyond the level of phytoplankton, but upon loss
to predation is recycled to the dissolved and particulate iron state
variables proportionate to the nitrogenous excretion and egestion
fluxes of their predators.

See \tabref{nemparamquo} for parameters.


\subsubsection{Iron scavenging (Ads)}

Scavenging of dissolved iron onto particles ($Fe$ to $POFe$) follows
a single ligand model, where only non-ligand-bound iron is available
for adsorption onto particles. Light is assumed to greatly reduce
the effectiveness of ligand binding through the production of oxygen
free radicals \citep{Fan2008}. This impact is assumed to decay at
light levels below 10 W m$-^{2}$ in a manner consistent with the
observed decline of hydrogen peroxide in the water column \citep{Yuan2001}.
The free unbound iron, $Fe_{free}$, is calculated via:

\begin{equation}
K_{Lig}\cdot Fe_{free}^{2}+\left(1+K_{Lig}\cdot(Lig_{bkg}-Fe)\right)\cdot Fe_{free}-B_{Fe}=0
\end{equation}
Adsorption onto particles is directly proportional to this free iron.
Iron scavenging rates have been observed to be lower above dissolved
iron concentrations of 0.6 nM, possibly due to less complexation with
ligands at these concentrations\citep{Johnson1997}, so the final
scavenging equations allow for this:

\begin{equation}
Ads=\left\{ \begin{array}{ll}
\alpha_{scav}\cdot Fe_{free} & \quad\text{\ensuremath{Fe_{free}\leq}0.6}\\
0.08\cdot(Fe_{free}-0.6)\cdot\alpha_{scav}\cdot Fe_{free} & \quad\text{\ensuremath{Fe_{free}>}0.6}
\end{array}\right.
\end{equation}


Finally, a fraction of particulate iron is remineralized to the dissolved
iron pool, proportional to remineralization of particulate nitrogen
to ammonium.

\begin{equation}
Dec_{POFe,Fe}=Dec_{PON,NH_{4}}\cdot\frac{B_{PON}}{B_{POFe}}\cdot r_{eff}
\end{equation}


See \tabref{nemparamquo} for parameters.


\subsubsection{Rerouting of fluxes}

By default, the flux processes move biomass from specific source and
sink groups as described in the sections above. But they can also
be redirected in order to tune the model a bit to specific behaviors
of individual functional groups. 

In the Eastern Subarctic Gyre ecosystem model, the following rerouting
adjustments were made:
\begin{enumerate}
\item Small phytoplankton non-predatory mortality: 25\% is redirected to
$DON$ and 25\% is redirected to $NH_{4}$. The small size of this
group means that their loss contributes little to the heavy, sinking
particles of the $PON$ group.
\item Microzooplankton egestion: 25\% is redirected to $DON$ and 25\% is
redirected to $NH_{4}$. They prey on small phytoplankton, and again,
this flux behaves more like dissolved material than sinking particulates.
\item Non-predatory mortality of birds, mammals, and sharks (albatross,
sperm whales, toothed whales, elephant seals, seals/dolphins, fulmars,
skuas/jaegers, puffins/shearwaters/storm petrels, kittiwakes, sharks):
90\% redirected out of the system. Loss of these groups typically
occur outside of the modeled system, either out of the water or sinking
quickly to deep water, and therefore little of their biomass is recycled
to the nutrient pools of the modeled system.
\item Non-predatory mortality of fisheries target species (pomfret, chum
salmon, chinook/coho/steelhead, sockeye/pink, saury, sergestid shrimp):
90\% is redirected to fishing loss. Our initial Ecopath model does
not explicitly resolve fisheries loss, instead including this loss
with other non-predatory losses. This redirection removes these losses
from the flux back into the nutrient pools, and allows us to quantify
a rough estimate of fisheries landings.
\end{enumerate}

\subsection{Input data for the Eastern Subarctic Gyre}


\subsubsection{The Ecopath Model}

The coupled model relies on the Ecopath mass-balance modeling concept
\citep{christensen2004} to calculate a large number of the ecosystem
parameters related to upper trophic level processes: specifically,
grazing, predation, excretion and egestion, and non-predatory mortality.
The majority of the Ecopath data used to construct our food web model
came from a previously-published model, developed through a series
of PICES-sponsored workshops to look at similarities and differences
between the eastern and western subarctic gyres of the north Pacific
Ocean \citep{aydin2003}; I refer to this model as the Aydin-48 model
because it included 48 functional groups. A few modifications were
made to this Ecopath model in preparation for using it in this framework.

The first modification made was to eliminate the bacteria functional
group from the Aydin-48 model. In the original model, this functional
group was included as a prey item for microzooplankton, ``preying''
itself on the two detrital functional groups. However, in their time-dynamic
simulations, \citet{aydin2003} found that their results were highly
sensitive to this representation of the microbial loop, and it led
to some lower trophic level dynamics that disagreed with accepted
biogeochemical models from the region; they concluded that it was
sufficient to assume that bacterial processes occurred within the
detrital pools and removed the bacteria group from some of their later
simulations. Because we intended to link the Ecopath-derived food
web model directly to a biogeochemical model that already included
parameterizations for the microbial loop, we decided to eliminate
the bacteria group from the food web dynamics entirely, and we replaced
the microzooplankton diet with one of 100\% small phytoplankton.

\tabref{aydin48critterlist} provides descriptions of the 47 functional
groups that remained in the model, including the most common species
classified under each functional group. Pertinent information regarding
each group's lifecycle is also provided.

After examining the sources of zooplankton data for the Aydin-48 model,
we made several adjustments to the data for those groups. \citet{aydin2003}
resolved the zooplankton community into eleven different functional
groups: microzooplankton, copepods, euphausiids, pteropods, amphipods,
sergestidae (shrimp), chaetognaths, salps, ptenohores, large jellyfish,
and a miscellaneous group (mainly larvaceans and polychaetes). Much
of the data for these groups was derived from a simulation of the
NEMURO biogeochemical model. We found several points of disagreement
with the assumptions used to translate the NEMURO output data into
Ecopath input data. 

First, the version of NEMURO used by \citet{aydin2003} was an early
realization of that model \citep{eslinger2000,Megrey2000}. For this
study, we developed our own biogeochemical model, based very closely
on NEMURO but with the addition of iron dynamics, and substituted
the values from a simulation of our model in place of those detailed
in the \citet{aydin2003} report.

Second, \citet{aydin2003} interpreted NEMURO's predatory zooplankton
group (ZP) as representing only non-gelatinous omnivores, i.e. euphausiids,
pteropods, and amphipods. They gathered data for the gelatinous zooplankton
groups (large jellyfish, chaetognaths, salps, and ctenophores) from
other sources, and estimated the population of the carnivorous shrimp
and miscellaneous groups each as 10\% of the ZP value. Overall, this
led to a community with a very large mesozooplankton community, more
than twice that of the copepod population, despite the fact that copepods
should be the dominant mesozooplankton genera at this location \citep{Goldblatt1999,harrison2004nutrient}.
While NEMURO's ZP state variable does have an omnivorous diet, in
our opinion it was intended to represent all unresolved predators
of the two smaller zooplankton state variables (which correspond to
the microzooplankton and copepod populations); in this Ecopath model,
this includes not only the nine remaining zooplankton groups but also
all other non-planktonic groups. Based on descriptions of the mesozooplankton
community in the subarctic gyre \citep{Goldblatt1999}, we decided
to distribute 50\% of the ZP biomass across the omnivorous, non-gelatinous
zooplankton groups (euphausiids, pteropods, and amphipods), 40\% across
the omnivorous, gelatinous groups (salps and ctenophores), and the
remaining 10\% across the carnivorous groups (shrimp, ctenophores,
and miscellaneous).

Another modification we made to the NEMURO-derived zooplankton biomass
data in the \citet{aydin2003} report involved the conversion of units
between NEMURO, which tracks state variables through their nitrogen
content, and Ecopath, which uses total wet weight. While \citet{aydin2003}
detailed the assumptions used to convert the NEMURO data from mmol
N m$^{-2}$ to g C m$^{-2}$, including elemental ratios and mixed
layer depth values, no explanation was given for the 0.01 g C/g wet
weight conversion factor that was then used to calculate wet weight
of both phyto- and zooplankton groups. While this order of magnitude
estimate is common in conversions of fish wet weight to carbon content,
it is much lower than most measurements for plankton. For example,
crustacean zooplankton wet mass to carbon ratios range from 0.06-0.12
g wet mass/gC \citep{Harris2000}. We compromised with a conversion
factor of 0.03 gC/g wet weight, which we applied throughout this study
whenever converting between element-based and weight-based units.

The final adjustment we made to the Aydin-48 data concerned the growth
efficiency value applied to the ctenophore group. The value of 0.03
used for this group appeared extremely low, even for a gelatinous
group. \citet{aydin2003} cited \citet{pauly1996mass} as the source
of this number. However, \citet{pauly1996mass} derived their carnivorous
jelly data from measurements of the cnidarian \emph{Aglantha}, not
a ctenophore, and even for this species they commented that the consumption
rate they were using was ``very high, perhaps excessively so.''
Measured growth efficiencies for ctenophores vary from less than 10\%
to 45\% \citep{Reeve1978,Reeve1989}; we settled on a value of 0.3,
in line with that of the other zooplankton groups, in order to resolve
Ecopath balance issues that arose as a result of the NEMURO-derived
adjustments detailed above.


\subsubsection{Food web clustering}

In theory, the food web described in the previous section could be
used to derive the necessary parameters to run a simulation of our
model. However, for purely practical reasons (computation time, ease
of analysis, ability to create plots that included all results and
were still readable without a microscope, etc.), we preferred to decrease
the number of functional groups to the smallest number that would
still capture the processes of interest. 

The simplification was performed via agglomerative hierarchical clustering.
Our similarity metric for the clustering process included diet composition,
using shared prey and predator groups as the main descriptive parameters,
along with level of primary production and trophic level to maintain
the basic trophic hierarchy in the clustered results (\figref{Dendrogram-for-food}).
During initial development of the model, and in \citet{Kearney2012}
and \citet{Kearney2013}, I used the food web resulting from a similarity
cutoff value of 1.5, which produces a 24-group model (23 living groups
plus one detrital group). In more recent studies, I have used a lower
cutoff value of 0.5, resulting in a 33-group model (32 living groups
plus one detrital group).

\begin{figure}
\includegraphics{/Users/kakearney/Documents/Research/Working/fishBgcCoupling/ensembles/analysis/eecho_dendrogram}

\protect\caption{Dendrogram for food web clustering. \label{fig:Dendrogram-for-food}}


\end{figure}


The Ecopath input parameters for the 33-group model can be found in
\tabref{ecopathIn33} and \tabref{diet33}.


\subsubsection{Ensemble generation}

The construction of a typical Ecopath model involves the compilation
of a large amount of population-related data, including biomass, production
rates, consumption rates, diet fractions, growth efficiencies, and
assimilation efficiencies for each functional group included in the
model. These data typically come from a wide variety of sources, ranging
from high-quality scientific surveys to fisheries landing data, empirical
relationships, and other models. The uncertainty values on these numbers
can be very high, up to or beyond an order of magnitude from the point
estimates, and accurate measurement of these uncertainties is rare.

Although the inclusion of pedigree values, i.e. estimates of the quality
of each parameter based on its underlying source, is relatively common
when documenting Ecopath models, the uncertainty information is rarely
incorporated into the predicted results deriving from these models
\citep{eweusersguide}. However, the wide range of uncertainty associated
with some input values means that often a single simulation based
on the mean inputs does not capture the true range of possible outcomes;
in some cases even the direction of change in one functional group
as a result of a perturbation to a different group cannot be fully
ascertained when accounting for the entire input range. 

In previous work, when coupling an Ecopath-derived predator/prey model
to a seasonally-varying biogeochemical model \citep{Kearney2012,Kearney2013},
we found that small numerical differences in initial parameters could
occasionally lead to outlier-type results. Rather than constantly
readjusting the hundreds of parameters to find a single set of perfectly-tuned
numbers that were both representative of the entire ecosystem and
numerically well-behaved, we instead decided to use an ensemble of
Ecopath models to derive our parameters and thereby encompass as much
of the potential parameter space as possible. The process is as follows.

\emph{Step 1: For each user-input parameter in the model, choose $n$
random values based on the prescribed probability density functions
for each parameter. }Currently, the code allows uncertainty to be
applied to biomass ($B_{i}$), production per unit biomass ($\frac{{P}}{B}_{i}$),
consumption per unit biomass ($\frac{{Q}}{B}_{i}$), ecotrophic efficiency
($EE_{i}$), growth efficiency ($GE_{i}$, or $\frac{{P}}{Q}_{i}$),
and each component of the diet composition ($DC_{ij}$). Each parameter
is assigned a lognormal distribution function with mean $m$ and variance
$\left(\frac{{mp}}{2}\right)^{2}$, where $m$ is the initial point
estimate for the parameter and $p$ is the respective pedigree value;
in terms of the lognormal probability density function

\begin{equation}
P(x,\mu,\sigma)=\frac{1}{x\sigma\sqrt{2\pi}}e^{-\frac{(\ln{x}-\mu)^{2}}{2\sigma^{2}}}
\end{equation}
this leads to values of

\begin{equation}
\mu=\ln\left({\frac{m^{2}}{\sqrt{\left(\frac{mp}{2}\right)^{2}+m^{2}}}}\right)
\end{equation}


\begin{equation}
\sigma=\sqrt{\ln{\left(\frac{p^{2}}{4}+1\right)}}
\end{equation}


The lognormal distribution was chosen so that the resulting values
would always remain non-negative without the need for hard limits
on the parameter ranges. 

The remaining input variables, including assimilation efficiencies
($GS_{i}$), fishing loss due to landings and discards ($Y_{i}$),
biomass accumulation ($BA_{i}$), and net migration due to immigration
and emigration ($E_{i}$) remain as point estimates. In theory, the
technique could easily be extended to include these parameters as
well, but pedigree analyses tend to ignore these, likely for a variety
of reasons ($BA_{i}$ and $E_{i}$ are typically non-zero for only
a small number of groups in a model; $Y_{i}$ only applies to fished
species; $GS_{i}$ is rarely modified from its default value of 0.2-0.3).

The sampling of parameter values from the defined distributions can
be performed using any number of sampling schemes. Currently, the
code provides two options: Monte Carlo sampling and latin hypercube
sampling. Ecopath models typically include dozens of functional groups,
connected to each other by dozens to hundreds of predator-prey linkages.
This means that full multidimensional parameter space encompassed
by the distribution functions of all input variables is nearly impossible
to sample uniformly across the high number of dimensions, even with
thousands of samples. The latin hypercube option chooses samples using
a stratified design across each single parameter, such that the univariate
parameter space is evenly sampled; it also incorporates a maximin
criteria \citep{Johnson1990} to maximize the minimum multidimensional
distance between sample sets. While the sampling space of an Ecopath
model will always be pretty sparsely sampled, the latin hypercube
method allows for slightly better coverage of this multidimensional
parameter space with a smaller number of ensemble members than simple
Monte Carlo sampling.

\begin{figure}
\includegraphics[width=9cm]{/Users/kakearney/Documents/Research/Working/fishBgcCoupling/ensembles/analysis/lognormalpdf}

\protect\caption{Parameter values were chosen from lognormal probability distribution
functions. Shown here are the PDFs for a parameter with mean 1 and
pedigrees of 0.2-0.8; the error bars indicate the mean and standard
deviation of each PDF.}
\end{figure}


\emph{Step 2: Substitute each set of parameters into the Ecopath model,
and make the necessary adjustments to diet composition and multi-stanza
group biomass and consumption rates. }Some of the parameters in an
Ecopath model must by definition covary with other input variables.
First, diet composition components for a single predator must sum
to 1. This normalization of diet will result in the final set of $DC$
components falling along a normal distribution rather than lognormal. 

Multi-stanza groups also require special treatment at this point.
In an Ecopath model, biomass groups can represent different life history
stages of the same functional group. Ecopath assumes that multi-stanza
functional groups represent a stable age-size distribution, and that
body size (and by extension biomass and consumption rates) follow
a von Bertalanffy growth curve \citep{christensen2004}. The stable
age assumption implies that the survivorship $l_{a}$ at age $a$
is a function of mortality rates (note that in Ecopath, mortality
rate $Z_{i}$ is assumed to be equal to production rate $\frac{{P}}{B}_{i}$)
and biomass accumulation rate:

\begin{equation}
l_{a}=\exp\left({-\int_{0}^{a}{Z_{a}}-a\frac{BA_{i}}{B_{i}}}\right)
\end{equation}
and the von Bertalanffy assumption states that weight $w_{a}$ is
related to age by

\begin{equation}
w_{a}=(1-\exp(-K_{i}a))^{3}
\end{equation}
and that consumption is proportional to $w_{a}^{2/3}$; the growth
rate $K_{i}$ is provided as input by the user for all multi-stanza
groups in Ecopath. Based on these assumptions, the proportion of biomass
at age $a$ is 

\begin{equation}
b_{a}=\frac{l_{a}w_{a}}{\int_{0}^{a_{max}}l_{a}w_{a}}
\end{equation}
and by extension, the proportion of biomass within a stanza defined
between ages $a_{smin}$ and $a_{smax}$ is

\begin{equation}
b_{s}=\frac{\int_{a_{smin}}^{a_{smax}}l_{a}w_{a}}{\int_{0}^{a_{max}}l_{a}w_{a}}
\end{equation}


Likewise, consumption within a stanza is

\begin{equation}
q_{s}=\frac{\int_{a_{smin}}^{a_{smax}}l_{a}w_{a}^{2/3}}{\int_{0}^{a_{max}}l_{a}w_{a}^{2/3}}
\end{equation}


Therefore, biomass and consumption rates are only set for a single
leading stanza group, typically the oldest group, and values for the
other stanzas are calculated such that

\begin{equation}
B_{tot}=\frac{B_{s=\textrm{lead}}}{b_{s=\textrm{lead}}}
\end{equation}


\begin{equation}
B_{s}=B_{tot}b_{s}
\end{equation}


\begin{equation}
Q_{tot}=\frac{B_{s=\textrm{lead}}\frac{Q}{B}_{s=\textrm{lead}}}{q_{s=\textrm{lead}}}
\end{equation}


\begin{equation}
\left(\frac{Q}{B}\right)_{s}=\frac{Q_{tot}q_{s}}{B_{s}}
\end{equation}


\emph{Step 3: Calculate ecotrophic efficiency values for each of the
$n$ Ecopath models. Keep only the models that satisfy the mass balance
criteria.} Balance within an Ecopath model is diagnosed via the ecotrophic
efficiency (EE) values for each group, defined as the fraction of
net group production that is passed up the food chain to predators.
An ecotrophic efficiency value outside the range of 0-1 implies that
the system requires an outside sink or source in order to account
for all biomass fluxes, and results when input data contradict each
other (for example, a predator has a consumption rate that cannot
be sustained by the known biomass of its prey). At this point, we
keep only the parameter sets that result in a balanced model. For
most Ecopath models, this will only be a small fraction of $n$.

\emph{Step 4: Repeat steps 1-3 until $n$ balanced models have been
created.} By repeating the process of sampling parameters, testing
for balance, and keeping only those models that satisfied the balance
criteria, we are able to construct an ensemble of Ecopath food web
models that incorporate the potential measurement uncertainty.

%\clearpage
\FloatBarrier


\section{Tables of parameters}



% \newcolumntype{x}[1]{>{\raggedleft\hspace{0pt}}p{#1}}     % p, but right-justified 
% \newcolumntype{y}[1]{>{\raggedright\hspace{0pt}}p{#1}}    % p, but left-justified

\newcolumntype{L}[1]{>{\RaggedRight}p{#1}} % p, but left-justified
\newcolumntype{R}[1]{>{\RaggedLeft}p{#1}}  % p, but right-justified

% Biogeochemical parameters tables

\newcommand{\bgctable}[3]{
\begin{table}[h]
	\footnotesize
	\caption{Biogeochemical process-related parameters: #2}
	\begin{tabular}{p{2in}p{0.5in}p{0.75in}>{\raggedleft}p{0.5in}@{ }p{1.25in}}
		\toprule
	  	Parameter & Symbol & Group &  Value & \\ % \multicolumn{2}{l}{Value} \\
		\midrule
		#1
		\bottomrule
   	\end{tabular}
	\label{tab:nemparam#3}
\end{table}
}

% Description of critters table

\newcommand{\crittertable}[2]{
\begin{table}
	\footnotesize   
	\caption{A description of the 47 functional groups included in the 
	         unsimplified version of the food web model. Species listed in the 
	         Includes column are not exhaustive, but represent the dominant 
	         members of each functional group.}
	\renewcommand{\arraystretch}{1.5}
	\begin{tabular}{L{1.25in}L{2.0in}L{3.00in}}
		\toprule
		Group & Includes & Details \tabularnewline
		\midrule 
        #1
		\bottomrule 
	\end{tabular}
	\label{tab:aydin48critterlist#2}
\end{table}
}

\newcommand{\crittertablelong}[1]{
\footnotesize
\renewcommand{\arraystretch}{1.5}
\begin{longtable}{L{1.25in}L{2.0in}L{2.75in}}
	\caption{A description of the 47 functional groups included in the 
	         unsimplified version of the food web model. Species listed in the 
	         Includes column are not exhaustive, but represent the dominant 
	         members of each functional group.} \\
	\toprule
	Group & Includes & Details \\
	\midrule
	\endfirsthead
	\caption*{A description of the 47 functional groups included in the 
	         unsimplified version of the food web model (continued).} \\
	\toprule
	Group & Includes & Details \\
	\midrule
	\endhead
	\bottomrule
	\endfoot
    #1
	\label{tab:aydin48critterlist}
\end{longtable}
\renewcommand{\arraystretch}{1.0}
\normalsize
}


% Time-varying parameters

\newcommand{\timevarytable}[1]{
\begin{table}
	\footnotesize  
	\caption{Derived parameters. These parameters vary over time as a function 
	         of the state variables from both the physical and biological
	         models.}   
	\renewcommand{\arraystretch}{2.5} 
	\begin{tabular}{lll}
		\toprule
		Parameter Name & Symbol & Definition\tabularnewline 
		\midrule 
		#1
		\bottomrule 
	\end{tabular}
	\label{tab:wcenemDerivedParams}
\end{table}
}

% Ecopath basic input

\newcommand{\ecopathbasictable}[1]{
\begin{table}
	\footnotesize
	\caption{Ecopath basic input variables for the 33-group simplified food 
	         web, including biomass (B, tons wet weight m$^{-2}$), 
	         production/biomass (PB, yr$^{-1}$), consumption/biomass 
	         (QB, yr$^{-1}$), ecotrophic efficiency (EE), growth efficiency 
	         (GE), and fraction unassimilated (GS)}
	\begin{tabular}{L{1.50in}*{11}{r}}
		\toprule
	  	Group 	& \multicolumn{2}{c}{B}  &  \multicolumn{2}{c}{PB} & \multicolumn{2}{c}{QB} & \multicolumn{2}{c}{EE} & \multicolumn{2}{c}{GE}  & GS  \\     
	    \cmidrule(lr){2-3}\cmidrule(lr){4-5}\cmidrule(lr){6-7}\cmidrule(lr){8-9}\cmidrule(lr){10-11}
		        & Value     & Ped        & Value    & Ped          & Value    & Ped         & Value    & Ped         &  Value    & Ped         &     \\ 
		\midrule
		#1
		\bottomrule
   	\end{tabular}
	\label{tab:ecopathIn33}
\end{table}
}

% Ecopath diet input

\newcommand{\diettable}[2]{
\begin{table}
	\footnotesize 
	\caption{Ecopath diet fraction input for the 33-group food web model.}    
	\begin{tabular}{L{0.3\textwidth}L{0.3\textwidth}R{0.1\textwidth}R{0.1\textwidth}}
		\toprule
		Predator & Prey & Diet percentage & Pedigree \tabularnewline
		\midrule 
		#1
		\bottomrule 
	\end{tabular}
	\label{tab:diet33_#2}
\end{table}
}


\newcommand{\diettablelong}[1]{
\footnotesize 
\begin{longtable}[l]{L{0.3\textwidth}L{0.3\textwidth}R{0.1\textwidth}R{0.1\textwidth}}
	\caption{Ecopath diet fraction input for the 33-group food web model.} \\    
	\toprule
	Predator & Prey & \mbox{Diet percentage} & Pedigree \tabularnewline
	\midrule 
	\endfirsthead
	\caption*{Ecopath diet fraction input for the 33-group food web model (continued).} \\    
	\toprule
	Predator & Prey & \mbox{Diet percentage} & Pedigree \tabularnewline
	\midrule 
	\endhead
	\bottomrule 
	\endfoot
	#1
	\label{tab:diet33}
\end{longtable}
\normalsize
}

% Ecopath-derived, state-variable-related

\newcommand{\epderivedstatetable}[1]{
\begin{table}
    \footnotesize
    \caption{Ecopath-derived parameters for living state variables, including 
             mass-balanced biomass ($B^*$, mol N m$^{-2}$), mass-balanced mortality 
             flux per unit biomass ($M_0$, s$^{-1}$), growth efficiency ($GE$), 
             and unassimilation fraction ($GS$).  The histogram columns indicate 
             the distribution of values across ensemble members, with the column 
             width ranging from 0 to 3.5 times the mean value.}
    \begin{tabular}{llllllll}
        \toprule
        Group & \multicolumn{2}{c}{B*} & \multicolumn{2}{c}{M0} & \multicolumn{2}{c}{GE} & GS \\
        \cmidrule(r){2-3} \cmidrule(r){4-5} \cmidrule(r){6-7}
              & mean & histogram       & mean & histogram       & mean & histogram       & \\
        \midrule
		#1
        \bottomrule
    \end{tabular}
    \label{tab:ecopathderivedState}
\end{table}
}

% Ecopath-derived, predator-prey link-related

\newcommand{\epderivedlinktable}[2]{
\begin{table}
    \footnotesize
    \caption{Ecopath-derived parameters for predator-prey processes, including mass-balanced consumption rate ($Q^*$, mol N m$^{-2}$ s$^{-1}$), top-down control parameter ($X$), bottom-up control parameter ($D$), and functional response exponent ($\theta$).  The histogram column indicates the distribution of values across ensemble members, with the column width ranging from 0 to 8 times the mean value.}
    \begin{tabular}{L{0.25\textwidth}L{0.25\textwidth}llrrr}
        \toprule
        Predator & Prey & \multicolumn{2}{c}{Q*} & X & D & $\theta$ \\
        \cmidrule(r){3-4}
                 &      & mean & histogram       &   &   &          \\
        \midrule
		#1
        \bottomrule
    \end{tabular}
    \label{tab:ecopathderivedLink_#2}
\end{table}
}

\newcommand{\epderivedlinklong}[1]{
\footnotesize
\begin{longtable}[l]{L{0.28\textwidth}L{0.28\textwidth}llrrr}
	\caption{Ecopath-derived parameters for predator-prey processes, 
	         including mass-balanced consumption rate ($Q^*$, mol N 
	         m$^{-2}$ s$^{-1}$), top-down control parameter ($X$), 
	         bottom-up control parameter ($D$), and functional response 
	         exponent ($\theta$).  The histogram column indicates the 
	         distribution of values across ensemble members, with the 
	         column width ranging from 0 to 3.5 times the mean value.} \\
	\toprule
    Predator & Prey & \multicolumn{2}{c}{Q*} & X & D & $\theta$ \\
    \cmidrule(r){3-4}
             &      & mean & histogram       &   &   &          \\
    \midrule
	\endfirsthead
	\caption*{Ecopath-derived parameters for predator-prey processes (continued).} \\
	\toprule
    Predator & Prey & \multicolumn{2}{c}{Q*} & X & D & $\theta$ \\
    \cmidrule(r){3-4}
             &      & mean & histogram       &   &   &          \\
    \midrule
	\endhead
	\bottomrule
	\endfoot
	#1
	\label{tab:ecopathderivedLink}
\end{longtable}
\normalsize
}
		






%---------------------------
% List of critters (this one 
% was typed up manually)
%---------------------------

% \crittertable{
% Sperm whales		 			   		    &   sperm whales (\emph{Physeter macrocephalus}) & a very large toothed whale, only mature males are found in the subarctic gyre region, and only during the summer months. \tabularnewline
% Toothed whales                              &   orcas (\emph{Orcinus orca}) & includes the mammal-eating transient subpopulation and some portion of the piscivororous (and typically more coastal) resident subpopulation \tabularnewline
% Fin whales                                  &   fin whales (\emph{Balaenoptera physalus}) & a baleen whale, migrates to the gyre during summer months to feed \tabularnewline
% Sei whales                                  &   sei whales (\emph{Balaenoptera boealis})  & a baleen whale, migrates to the gyre during summer months to feed \tabularnewline
% Northern fur seals                          &   northern fur seals (\emph{Callorhinus ursinus}) & a large fur seal.    \tabularnewline
% Elephant seals                              &   northern elephant seals (\emph{Mirounga angustirostris}) & large seal, migrates biannually between the Alaska Gyre and California breeding beaches. \tabularnewline
% Dall's porpoises                            &   Dall's porpoises (\emph{Phocoenoides dalli}) & a porpoise \tabularnewline
% Pacific white-sided dolphins                &   Pacific white-sided dolphins (\emph{Lagenorhynchus obliquidens}) & a dolphin\tabularnewline
% Northern right whale dolphins               &   Northern right whale dolphins (\emph{Lissodelphis borealis}) & a small dolphin\tabularnewline
% Albatross                                   &   primarily Black-footed albatross (\emph{Phoebastria nigripes}) and Laysan albatross (\emph{Phoebastria immutabilis}) & large seabirds  \tabularnewline
% Shearwaters                                 &   primarily sooty shearwaters (\emph{Puffinus griseus}) and short-tailed shearwaters (\emph{Puffinus tenuirostris})  & medium-sized seabirds, the dominant seabird in the Gulf of Alaska region \tabularnewline
% Storm Petrels                               &   primarily fork-tailed storm petrels (\emph{Oceanodroma furcata}) and Leach's storm petrels (\emph{Oceanodroma leucorhoa}) &  small seabirds  \tabularnewline
% Kittiwakes                                  &   primarily black-legged kittiwakes (\emph{Rissa tridactyla})  &  seabirds in the gull family    \tabularnewline
% Fulmars                                     &   northern fulmar (\emph{Fulmarus glacialis})  &  a seabird  \tabularnewline
% Puffins                                     &   primarily tufted puffins (\emph{Fratercula cirrhata})  &  a medium-sized seabird \tabularnewline
% Skuas                                       &   primarily south-polar skuas (\emph{Stercorarius maccormicki})  & a large seabird  \tabularnewline
% Jaegers                                     &   primarily Pomarine jaegars  &  seabird, in the skua family  \tabularnewline
% Sharks                                      &   salmon sharks (\emph{Lamna ditropis}) & small shark, approximately 2m long, homeothermic \tabularnewline
% Large gonatid squid                         &   armhook squid (family Gonatidae)  & medium-sized squid  \tabularnewline}{1}
% 
% \crittertable{
% Boreal clubhook squid                       &   boreal clubhook squid (\emph{Onychoteuthis borealijaponica})  & a medium-sized squid \tabularnewline
% Neon flying squid                           &   neon flying squid (\emph{Ommastrephes bartramii}) & a slightly larger squid  \tabularnewline      
% Sockeye salmon                              &   sockeye salmon (\emph{Oncorhynchus nerka})  & the most abundant salmon species in the Eastern Gyre, anadromous \tabularnewline
% Chum salmon                                 &   chum salmon (\emph{Oncorhynchus keta})  & the second-most abundant salmon species in the Easterb Gyre, anadromous  \tabularnewline
% Pink salmon                                 &   pink salmon (\emph{Oncorhynchus gorbuscha})  & smallest Pacific salmon, anadromous, have a two-year breeding cycle, with even- and odd-year populations not interbreeding \tabularnewline
% Coho salmon                                 &   coho salmon (\emph{Oncorhynchus kisutch})  & a salmon, anadromous     \tabularnewline
% Chinook salmon                              &   Chinook salmon (\emph{Oncorhynchus tshawytscha})  &  the largest Pacific salmon, anadromous \tabularnewline
% Steelhead                                   &   steelhead, or rainbow trout (\emph{Oncorhynchus mykiss}) & salmonid, anadromous \tabularnewline
% Pomfret                                     &   Pacific pomfret (\emph{Brama japonica})  & a large perciform fish \tabularnewline
% Saury                                       &   Pacific saury (\emph{Cololabis saira}) & medium-sized (30-40 cm), highly migratory fish, important commercial fish, especially in Asia \tabularnewline
% Pelagic forage fish                         &   primarily sticklebacks (\emph{Gasterosteus aculeatus})  &  small (4 cm) schooling forage fish    \tabularnewline
% Micronektonic squid                         &   primarily gonatids such as \emph{Berryteuthis anonychus} and \emph{Gonatus onyx} & juvenile squid, few data measurements available  \tabularnewline
% Mesopelagic fish                            &   myctophids, or lanternfishes (family Myctophidae), particularly \emph{Stenobrachius leucopsarus}  & small mesopelagic fish  \tabularnewline
% Large jellyfish                             &   phylum Cnidaria  &  small jellyfish  \tabularnewline          
% Ctenophores                                 &   phylum Ctenophora  &  comb jellies, gelatinous  \tabularnewline
% Salps                                       &   family Salpidae  &  planktonic tunicates, gelatinous    \tabularnewline
% Chaetognaths                                &   phylum Chaetognatha & marine worms, gelatinous     \tabularnewline
% Sergestid shrimp                            &   family Sergestidae  & shrimp \tabularnewline
% Miscellaneous predatory zooplankton         &   mainly Larvaceans and Polychaetes  & planktonic tunicates and annelid worms \tabularnewline
% Amphipods                                   &   order Amphipoda  &  crustacean zooplankton \tabularnewline
% Pteropods                                   &   Thecosomata  & planktonic gastropods     \tabularnewline
% Euphausiids                                 &   krill (order Euphausiacea)  &  crustacean zooplankton    \tabularnewline
% Copepods                                    &   subclass Copepoda  & small crustacean zooplankton     \tabularnewline
% Microzooplankton                            &   any <200 $\mu$m  & mainly meroplanktonic larva and copepod nauplii  \tabularnewline
% Large phytoplankton                         &   any <5 $\mu$m & includes prasinophytes, prymnesiophytes (coccolithophorids), cryptophytes, and cyanobacteria  \tabularnewline
% Small phytoplankton                         &   primarily diatoms  &  represent large size class, includes silica cycle  \tabularnewline
% DNH3                                        &   detritus, dissolved  &  detritus pool  \tabularnewline
% POM                                         &   detritus, particulate  & detritus pool  \tabularnewline}{2}


\crittertablelong{
Sperm whales		 			   		    &   sperm whales (\emph{Physeter macrocephalus}) & a very large toothed whale, only mature males are found in the subarctic gyre region, and only during the summer months. \\
Toothed whales                              &   orcas (\emph{Orcinus orca}) & includes the mammal-eating transient subpopulation and some portion of the piscivororous (and typically more coastal) resident subpopulation \\
Fin whales                                  &   fin whales (\emph{Balaenoptera physalus}) & a baleen whale, migrates to the gyre during summer months to feed \\
Sei whales                                  &   sei whales (\emph{Balaenoptera boealis})  & a baleen whale, migrates to the gyre during summer months to feed \\
Northern fur seals                          &   northern fur seals (\emph{Callorhinus ursinus}) & a large fur seal.    \\
Elephant seals                              &   northern elephant seals (\emph{Mirounga angustirostris}) & large seal, migrates biannually between the Alaska Gyre and California breeding beaches. \\
Dall' s porpoises            &   Dall' s porpoises (\emph{Phocoenoides dalli}) & a porpoise \\
Pacific white-sided dolphins                &   Pacific white-sided dolphins (\emph{Lagenorhynchus obliquidens}) & a dolphin\\
Northern right whale dolphins               &   Northern right whale dolphins (\emph{Lissodelphis borealis}) & a small dolphin\\
Albatross                                   &   primarily Black-footed albatross (\emph{Phoebastria nigripes}) and Laysan albatross (\emph{Phoebastria immutabilis}) & large seabirds  \\
Shearwaters                                 &   primarily sooty shearwaters (\emph{Puffinus griseus}) and short-tailed shearwaters (\emph{Puffinus tenuirostris})  & medium-sized seabirds, the dominant seabird in the Gulf of Alaska region \\
Storm Petrels                               &   primarily fork-tailed storm petrels (\emph{Oceanodroma furcata}) and Leach' s storm petrels (\emph{Oceanodroma leucorhoa}) &  small seabirds  \\
Kittiwakes                                  &   primarily black-legged kittiwakes (\emph{Rissa tridactyla})  &  seabirds in the gull family    \\
Fulmars                                     &   northern fulmar (\emph{Fulmarus glacialis})  &  a seabird  \\
Puffins                                     &   primarily tufted puffins (\emph{Fratercula cirrhata})  &  a medium-sized seabird \\
Skuas                                       &   primarily south-polar skuas (\emph{Stercorarius maccormicki})  & a large seabird  \\
Jaegers                                     &   primarily Pomarine jaegars  &  seabird, in the skua family  \\
Sharks                                      &   salmon sharks (\emph{Lamna ditropis}) & small shark, approximately 2m long, homeothermic \\
Large gonatid squid                         &   armhook squid (family Gonatidae)  & medium-sized squid  \\
Boreal clubhook squid                       &   boreal clubhook squid (\emph{Onychoteuthis borealijaponica})  & a medium-sized squid \\
Neon flying squid                           &   neon flying squid (\emph{Ommastrephes bartramii}) & a slightly larger squid  \\      
Sockeye salmon                              &   sockeye salmon (\emph{Oncorhynchus nerka})  & the most abundant salmon species in the Eastern Gyre, anadromous \\
Chum salmon                                 &   chum salmon (\emph{Oncorhynchus keta})  & the second-most abundant salmon species in the Easterb Gyre, anadromous  \\
Pink salmon                                 &   pink salmon (\emph{Oncorhynchus gorbuscha})  & smallest Pacific salmon, anadromous, have a two-year breeding cycle, with even- and odd-year populations not interbreeding \\
Coho salmon                                 &   coho salmon (\emph{Oncorhynchus kisutch})  & a salmon, anadromous     \\
Chinook salmon                              &   Chinook salmon (\emph{Oncorhynchus tshawytscha})  &  the largest Pacific salmon, anadromous \\
Steelhead                                   &   steelhead, or rainbow trout (\emph{Oncorhynchus mykiss}) & salmonid, anadromous \\
Pomfret                                     &   Pacific pomfret (\emph{Brama japonica})  & a large perciform fish \\
Saury                                       &   Pacific saury (\emph{Cololabis saira}) & medium-sized (30-40 cm), highly migratory fish, important commercial fish, especially in Asia \\
Pelagic forage fish                         &   primarily sticklebacks (\emph{Gasterosteus aculeatus})  &  small (4 cm) schooling forage fish    \\
Micronektonic squid                         &   primarily gonatids such as \emph{Berryteuthis anonychus} and \emph{Gonatus onyx} & juvenile squid, few data measurements available  \\
Mesopelagic fish                            &   myctophids, or lanternfishes (family Myctophidae), particularly \emph{Stenobrachius leucopsarus}  & small mesopelagic fish  \\
Large jellyfish                             &   phylum Cnidaria  &  small jellyfish  \\          
Ctenophores                                 &   phylum Ctenophora  &  comb jellies, gelatinous  \\
Salps                                       &   family Salpidae  &  planktonic tunicates, gelatinous    \\
Chaetognaths                                &   phylum Chaetognatha & marine worms, gelatinous     \\
Sergestid shrimp                            &   family Sergestidae  & shrimp \\
Miscellaneous predatory zooplankton         &   mainly Larvaceans and Polychaetes  & planktonic tunicates and annelid worms \\
Amphipods                                   &   order Amphipoda  &  crustacean zooplankton \\
Pteropods                                   &   Thecosomata  & planktonic gastropods     \\
Euphausiids                                 &   krill (order Euphausiacea)  &  crustacean zooplankton    \\
Copepods                                    &   subclass Copepoda  & small crustacean zooplankton     \\
Microzooplankton                            &   any <200 $\mu$m  & mainly meroplanktonic larva and copepod nauplii  \\
Large phytoplankton                         &   any <5 $\mu$m & includes prasinophytes, prymnesiophytes (coccolithophorids), cryptophytes, and cyanobacteria  \\
Small phytoplankton                         &   primarily diatoms  &  represent large size class, includes silica cycle  \\
DNH3                                        &   detritus, dissolved  &  detritus pool  \\
POM                                         &   detritus, particulate  & detritus pool  \\
}



%---------------------------
% Biogeochemical parameters 
% (taken from thesis tables, 
% with a few manual 
% adjustments)
%---------------------------

% Primary production (TODO: add new iron)

\bgctable{
	  	Ammonium inhibition constant 					& $\psi$ 		& PS 	& 1.5 		& (mmol N m$^{-3}$)$^{-1}$				\\
	  	 												&  				& PL 	& 1.5 		& (mmol N m$^{-3}$)$^{-1}$              \\
	  	Half-saturation constant for ammonium			& $K_{NH_{4}}$ 	& PS 	& 0.1 		& mmol N m$^{-3}$                       \\
	  	 												&  				& PL 	& 0.3 		& mmol N m$^{-3}$                       \\
	  	Half-saturation constant for nitrate 			& $K_{NO_{3}}$ 	& PS 	& 1 		& mmol N m$^{-3}$                       \\
	  	 												&  				& PL 	& 3 		& mmol N m$^{-3}$                       \\
	  	Half-saturation constant for silica				& $K_{Si}$ 		& PL    & 6 		& mmol Si m$^{-3}$                      \\
	  	Initial slope of P-I curve 						& $\alpha$ 		& PS    & 0.017 	& (W m$^{-2}$)$^{-1}$ d$^{-1}$          \\
	  	 												&  				& PL    & 0.016 	& (W m$^{-2}$)$^{-1}$ d$^{-1}$          \\
	  	Light dissipation coefficient of seawater		& $\alpha_{1}$ 	&       & 0.04 		& m$^{-1}$                              \\
	  	Maximum uptake rate at 0 deg C 					& $V_{max}$ 	& PS    & 0.4 		& d$^{-1}$                              \\
	  	 												&  				& PL    & 0.8 		& d$^{-1}$                              \\
	  	Phytoplankton self-shading coefficient 			& $\alpha_{2}$ 	&  		& 0.04 		& m$^{-1}$ (mmol N m$^{-3}$)$^{-1}$     \\
	  	Silica to nitrogen ratio 						& $R_{Si:N}$ 	&  		& 2 		& mmol Si (mmol N)$^{-1}$               \\
	  	Carbon to nitrogen ratio 						& $R_{C:N}$ 	&  		& 6.625 	& mol C (mol N)$^{-1}$                  \\
	  	Temperature coefficient for photosynthesis 		& $K_{gpp}$ 	& PS 	& 0.0693 	& (deg C)$^{-1}$                        \\
	  	 												&  				& PL 	& 0.0693 	& (deg C)$^{-1}$                        \\
	  	Empirical Fe:C function coefficient 			& $b_{Fe}$ 		& PS 	& 28.5 		& (mol C m$^{-3}$)$^{-1}$               \\
	  	 												&  				& PL 	& 42.6 		& (mol C m$^{-3}$)$^{-1}$               \\
	  	Empirical Fe:C function power 					& $\alpha_{Fe}$ & PS 	& 0.21 		&                                       \\
	  	 												&  				& PL 	& 0.46 		&                                       \\
	  	Fraction of iron remineralized 					& $f_{rem}$ 	& PS 	& 0.5 		&                                       \\
	  	 												&  				& PL 	& 0.5 		&                                       \\
	  	Half-saturation constant for Fe:C 				& $K_{Fe:C}$ 	& PS 	& 12 		& $\mu$mol Fe (mol C)$^{-1}$            \\
	  	 												&  				& PL 	& 16.9 		& $\mu$mol Fe (mol C)$^{-1}$            \\
	  	Timescale for iron uptake 						& $t_{Fe}$ 		& PS 	& 1 		& d                                     \\
	  	 												&  				& PL 	& 1 		& d                                     \\}{Primary production}{gpp}

\bgctable{
		Maximum Fe:N ratio												& $R_{Fe:Nmax}$ 		    & PS	& 331.25  	    & $\mu$mol Fe (mol N)$^{-1}$      \\
																		&							& PL    & 3312.5   	    & $\mu$mol Fe (mol N)$^{-1}$      \\
		Half-saturation constant for iron								& $K_{Fe}$                  & PS    & 0.6		  	& $\mu$mol Fe m$^{-3}$ 	     \\
		                                                                &                           & PL    & 3.0           & $\mu$mol Fe m$^{-3}$ 	     \\  
		Half-saturation constant for internal Fe:N ratio				& $K_{Fe:N}$                & PS    & 66.25         & $\mu$mol Fe (mol N)$^{-1}$      \\
		                                                                &                           & PL    &  132.5     	& $\mu$mol Fe (mol N)$^{-1}$      \\
		Iron uptake factor												& $\mu_{Fe:N}$              &       & 100			& $\mu$mol Fe (mol N)$^{-1}$ \\
		Background ligand concentration									& $Lig_bkg$                 &       & 1.0 			& $\mu$mol m$^{-3}$ 		 \\
		Half saturation constant for light effect on ligand-binding 	& $K_{Iscav}$   			&       & 1.0			& W m$^{-2}$ 				 \\
		Lower limit of ligand binding under low-light conditions  		& $K_{LigLo}$               &       & 300			& m$^3$ ($\mu$mol)$^{-1}$ 	 \\
		Upper limit of ligand binding under high-light conditions  		& $K_{LigHi}$               &       & 0.1           & m$^3$ ($\mu$mol)$^{-1}$ 	 \\ 
		Iron scavenging coefficient										& $\alpha_{scav}$           &       & 50			& yr$^{-1}$ 				 \\
		Fraction of iron remineralized, relative to organic nitrogen 	& $r_{eff}$    				&       & 0.25	   		& 							 \\}{Iron quota model}{quo}


\bgctable{                                                                                                       
		Respiration rate at 0 deg C 					& $Res_{0}$ 	& PS   	& 0.03 		& d$^{-1}$                              \\
														&  				& PL 	& 0.03 		& d$^{-1}$                              \\
		Temperature coefficient for respiration 		& $K_{Res}$ 	& PS 	& 0.0519 	& d$^{-1}$                              \\
														&  				& PL 	& 0.0519 	& d$^{-1}$                              \\}{Respiration}{res}

\bgctable{
	Ratio of extracellular excretion to photosynthesis  & $\gamma$ 		& PS 	& 0.135 	&                                       \\
 														&  				& PL 	& 0.135 	&                                       \\}{Extracellular excretion}{exx}
 

\bgctable{
        Grazing inhibition coefficient 					& $\psi_{gr}$ 	& ZP on PL 		& 4.605		& (mmol N m$^{-3}$)$^{-1}$		\\
 														&  				& ZP on ZS 		& 3.01 		& (mmol N m$^{-3}$)$^{-1}$      \\
	    Grazing threshhold 								& $B_{thresh}$ 	& ZS on PS 		& 0.04 		& mmol N m$^{-3}$               \\
 														&  				& ZL on PS 		& 0.04 		& mmol N m$^{-3}$               \\
 														&  				& ZL on PL 		& 0.04 		& mmol N m$^{-3}$               \\
 														&  				& ZP on PL 		& 0.04 		& mmol N m$^{-3}$               \\
 														&  				& ZL on ZS 		& 0.04 		& mmol N m$^{-3}$               \\
 														&  				& ZP on ZS 		& 0.04 		& mmol N m$^{-3}$               \\
 														&  				& ZP on ZL 		& 0.04 		& mmol N m$^{-3}$               \\
  		Ivlev constant 									& $\lambda$ 	& ZS 			& 1.4 		& (mmol N m$^{-3}$)$^{-1}$      \\
 														&  				& ZL 			& 1.4 		& (mmol N m$^{-3}$)$^{-1}$      \\
 														&  				& ZP 			& 1.4 		& (mmol N m$^{-3}$)$^{-1}$      \\
		Maximum grazing rate at 0 deg C 				& $g_{max}$ 	& ZS on PS 		& 0.8 		& d$^{-1}$                      \\
 														&  				& ZL on PS 		& 0.1 		& d$^{-1}$                      \\
 														&  				& ZL on PL 		& 0.4 		& d$^{-1}$                      \\
 														&  				& ZP on PL 		& 0.2 		& d$^{-1}$                      \\
 														&  				& ZL on ZS 		& 0.4 		& d$^{-1}$                      \\
 														&  				& ZP on ZS 		& 0.2 		& d$^{-1}$                      \\
 														&  				& ZP on ZL 		& 0.2 		& d$^{-1}$                      \\
        Temperature coefficient for grazing 			& $K_{Gra}$ 	& ZS 			& 0.0693	& (deg C)$^{-1}$                \\
 														&  				& ZL 			& 0.0693 	& (deg C)$^{-1}$                \\
 														&  				& ZP 			& 0.0693 	& (deg C)$^{-1}$                \\
 														&  				& gel. zoo. 	& 0.0693 	& (deg C)$^{-1}$                \\
 														&  				& pred. zoo. 	& 0.0693 	& (deg C)$^{-1}$                \\
		Mixed layer depth, annual average 				& $MLD$ 		&  				& 80 		& m                             \\
		Mixed layer temperature, annual average 		& $T_{avg}$ 	&  				& 8.26 		& deg C                         \\}{Grazing}{gra}           
		
		
\bgctable{ 
		Assimilation efficiency 						& $\alpha_{eg}$ & ZS 			& 0.7 		&								\\ 
		  												& 				& ZL 			& 0.7 		&                               \\
		  												&  				& ZP 			& 0.7 		&                               \\
		Growth efficiency 								& $\beta_{eg}$ 	& ZS 			& 0.3 		&                               \\
		  												&  				& ZL 			& 0.3 		&                               \\
		  												&  				& ZP 			& 0.3 		&                               \\}{Egestion and excretion}{ege}
 

% \clearpage % Don't know why it chokes here, but 18 tables in a row seem to drive latex a little crazy
\bgctable{   
		Decomposition (or nitrification) rate 			& $V_{Dec}$ 	& NH$_{4}$ to NO$_{3}$ 	& 0.03 		& d$^{-1}$     			\\
		  												&  				& PON to NH$_{4}$ 		& 0.1 		& d$^{-1}$              \\
		  												&  				& PON to DON 			& 0.1 		& d$^{-1}$              \\
		  												&  				& DON to NH$_{4}$ 		& 0.02 		& d$^{-1}$              \\
		  												&  				& Opal to SiOH$_{4}$ 	& 0.04 		& d$^{-1}$              \\
		Temperature coefficient for decomposition 		& $K_{Dec}$ 	& NH$_{4}$ to NO$_{3}$ 	& 0.0693 	& (deg C)$^{-1}$        \\
		  												&  				& PON to NH$_{4}$ 		& 0.0693 	& (deg C)$^{-1}$        \\
		  												&  				& PON to DON 			& 0.0693 	& (deg C)$^{-1}$        \\
		  												&  				& DON to NH$_{4}$ 		& 0.0693 	& (deg C)$^{-1}$        \\
    	  												&  				& Opal to SiOH$_{4}$ 	& 0.0693 	& (deg C)$^{-1}$        \\}{Decomposition}{dec} 
                                      
\bgctable{
		Mortality rate at 0 deg C 						& $Mor_{0}$ 	& PS 					& 0.0585 	& d$^{-1}$				\\
		  												&  				& PL 					& 0.029 	& d$^{-1}$              \\
		  												&  				& ZS 					& 0.0585 	& d$^{-1}$              \\
		  												&  				& ZL 					& 0.0585 	& d$^{-1}$              \\
		  												&  				& ZP 					& 0.0585 	& d$^{-1}$              \\
		Temperature coefficient for mortality 			& $K_{Mor}$ 	& PS 					& 0.0693 	& (deg C)$^{-1}$        \\
		  												&  				& PL 					& 0.0693 	& (deg C)$^{-1}$        \\
		  												&  				& ZS 					& 0.0693 	& (deg C)$^{-1}$        \\
		  												&  				& ZL 					& 0.0693 	& (deg C)$^{-1}$        \\
		  												&  				& ZP 					& 0.0693 	& (deg C)$^{-1}$        \\
		Mortality exponent								& $c$			& phytoplankton			& 2.0		&                       \\
														& 				& zooplankton			& 1.5		&                       \\
														& 				& nekton				& 1.0		&                       \\}{Mortality}{mor}

%---------------------------
% Time-varying parameters
% (entered manually)
%---------------------------

\timevarytable{
	Nitrogen limitation	 						& $L_{N}$ 		& $\frac{{NO_{3}}}{K_{NO_{3}}+NO_{3}}\cdot\exp\left(-\psi NH_{4}\right)+\frac{{NH_{4}}}{K_{NH_{4}}+NH_{4}}$\tabularnewline
	Silica limitation 							& $L_{Si}$ 		& $\frac{{SiOH_{4}}}{K_{SiOH_{4}}+SiOH_{4}}$\tabularnewline
	Iron limitation 							& $L_{Fe}$ 		& $\frac{{R_{Fe:C}^{2}}}{K_{Fe:C}^{2}+R_{Fe:C}^{2}}$\tabularnewline
	Iron limitation (quota model) 				& $L_{Fe}$ 		& $\frac{B_{Fe}}{K_{Fe}+B_{Fe}}$\tabularnewline
	Iron deficiency 							& $D_{Fe}$ 		& $\frac{R_{Fe:N}^2}{K_{Fe:N}^2+R_{Fe:N}^2}$\tabularnewline
	f-ratio 									& $f$ 			& $\frac{{\frac{{NO_{3}}}{K_{NO_{3}}+NO_{3}}\cdot\exp\left(-\psi NH_{4}\right)}}{{\frac{{NO_{3}}}{K_{NO_{3}}+NO_{3}}\cdot\exp\left(-\psi NH_{4}\right)+\frac{{NH_{4}}}{K_{NH_{4}}+NH_{4}}}}$\tabularnewline
	Total nutrient limitation  				    & $L_{nut}$ 	& $\min\left(L_{N},L_{Si},L_{Fe}\right)$\tabularnewline
	Total nutrient limitation (quota model) 	& $L_{nut}$ 	& $\min\left(L_{N},L_{Si},D_{Fe}\right)$\tabularnewline      
	Light limitation 							& $L_{light}$ 	& $1-\exp\left(\frac{{\alpha I_{z}}}{V_{max}}\right)$\tabularnewline
	Empirical Fe:C ratio 						& $R_{0i}$ 		& $b_{Fe,i}Fe_{z}^{a_{Fe,i}}$\tabularnewline
	Realized Fe:C ratio 						& $R_{i}$ 		& $\frac{B_{Fe,i}}{B_{i}\cdot R_{C:N}}$\tabularnewline
	Realized Fe:N ratio         				& $R_{Fe:N}$ 	&  $\frac{B_{Fe,i}}{B_{i}}$\tabularnewline   
	Ligand-binding parameter					& $K_{Lig}$ 	& $10^{\left(\log_{10}(K_{LigLo}) - \frac{I_z}{K_{Iscav} + I_z}\right)\left(\log_{10}(K_{LigLo}) - \log_{10}(K_{LigHi})\right)}$\tabularnewline}
	
	
%---------------------------
% Ecopath input
% (eechoanalysis)
%---------------------------

\ecopathbasictable{
Albatross                           &      4e-05 & 0.50 &       0.05 & 0.40 &            & 0.20 &            & 0.05 &  0.0006128 & 0.20 &        0.2 \\ 
Sperm whales                        &   0.000929 & 0.50 &     0.0596 & 0.40 &            & 0.20 &            & 0.05 &   0.009017 & 0.20 &        0.2 \\ 
Sharks                              &       0.05 & 0.80 &        0.2 & 0.60 &            & 0.40 &            & 0.05 &    0.01826 & 0.40 &        0.2 \\ 
Neon flying squid                   &       0.45 & 0.80 &      2.555 & 0.60 &            & 0.60 &            & 0.05 &     0.4118 & 0.60 &        0.2 \\ 
Toothed whales                      &    2.8e-05 & 0.50 &     0.0252 & 0.40 &            & 0.20 &            & 0.05 &   0.002258 & 0.20 &        0.2 \\ 
Elephant seals                      &    0.00043 & 0.50 &      0.368 & 0.40 &            & 0.20 &            & 0.05 &    0.03321 & 0.20 &        0.2 \\ 
Seals,dolphins                      &    0.01409 & 0.50 &     0.1302 & 0.40 &            & 0.20 &            & 0.05 &   0.004952 & 0.20 &        0.2 \\ 
Boreal clubhook squid               &      0.012 & 0.80 &      2.555 & 0.60 &            & 0.60 &            & 0.05 &       0.35 & 0.60 &        0.2 \\ 
Fulmars                             &    7.4e-05 & 0.50 &        0.1 & 0.40 &            & 0.20 &            & 0.05 &  0.0009974 & 0.20 &        0.2 \\ 
Chinook,coho,steelhead              &    0.02306 & 0.50 &      1.123 & 0.26 &        7.5 & 0.26 &            & 0.05 &     0.1499 & 0.26 &        0.2 \\ 
Skuas,Jaegers                       &    9.2e-05 & 0.50 &      0.075 & 0.40 &            & 0.20 &            & 0.05 &  0.0007764 & 0.20 &        0.2 \\ 
Pomfret                             &       0.21 & 0.80 &       0.75 & 0.40 &            & 0.40 &            & 0.05 &        0.2 & 0.40 &        0.2 \\ 
Puffins,Shearwaters,Storm Petrels   &   0.000514 & 0.50 &        0.1 & 0.40 &            & 0.20 &            & 0.05 &  0.0009411 & 0.20 &        0.2 \\ 
Kittiwakes                          &    5.2e-05 & 0.50 &        0.1 & 0.40 &            & 0.20 &            & 0.05 &   0.000813 & 0.20 &        0.2 \\ 
Large gonatid squid                 &       0.03 & 0.80 &      2.555 & 0.60 &            & 0.60 &            & 0.05 &       0.35 & 0.60 &        0.2 \\ 
Sockeye,Pink                        &     0.1129 & 0.50 &      1.703 & 0.10 &            & 0.10 &            & 0.05 &     0.1437 & 0.10 &        0.2 \\ 
Fin,sei whales                      &    0.03379 & 0.50 &       0.02 & 0.40 &            & 0.20 &            & 0.05 &   0.004134 & 0.20 &        0.2 \\ 
Micronektonic squid                 &            & 0.80 &          3 & 0.60 &            & 0.70 &        0.9 & 0.05 &        0.2 & 0.70 &        0.2 \\ 
Mesopelagic fish                    &        4.5 & 0.80 &        0.9 & 0.60 &            & 0.70 &            & 0.05 &        0.3 & 0.70 &        0.2 \\ 
Pelagic forage fish                 &            & 0.80 &        1.5 & 0.60 &            & 0.70 &        0.9 & 0.05 &        0.3 & 0.70 &        0.2 \\ 
Saury                               &       0.45 & 0.80 &        1.6 & 0.60 &            & 0.70 &            & 0.05 &     0.2025 & 0.70 &        0.2 \\ 
Chum salmon                         &    0.05414 & 0.50 &       1.93 & 0.10 &            & 0.10 &            & 0.05 &      0.133 & 0.10 &        0.2 \\ 
Large jellyfish                     &          4 & 0.80 &          3 & 0.70 &            & 0.70 &            & 0.05 &        0.3 & 0.70 &        0.2 \\ 
Chaetognaths                        &      4.129 & 0.50 &      7.642 & 0.70 &            & 0.70 &            & 0.05 &        0.3 & 0.70 &        0.3 \\ 
Predatory zooplankton               &      4.129 & 0.80 &      7.642 & 0.70 &            & 0.70 &            & 0.05 &        0.3 & 0.70 &        0.3 \\ 
Sergestid shrimp                    &      4.129 & 0.80 &      7.642 & 0.70 &            & 0.70 &            & 0.05 &        0.3 & 0.70 &        0.3 \\ 
Mesozooplankton                     &      15.48 & 0.80 &      7.642 & 0.60 &            & 0.60 &            & 0.05 &        0.3 & 0.60 &        0.3 \\ 
Gelatinous zooplankton              &      3.097 & 0.50 &        6.5 & 0.70 &            & 0.70 &            & 0.05 &        0.3 & 0.70 &        0.3 \\ 
Copepods                            &      20.35 & 0.10 &      28.07 & 0.10 &            & 0.40 &            & 0.05 &        0.3 & 0.40 &        0.3 \\ 
Microzooplankton                    &      18.41 & 0.10 &       46.1 & 0.10 &            & 0.40 &            & 0.05 &        0.3 & 0.40 &        0.3 \\ 
Small phytoplankton                 &      45.54 & 0.50 &      85.83 & 0.10 &          0 & 0.40 &            & 0.05 &          0 & 0.40 &          0 \\ 
Large phytoplankton                 &      26.11 & 0.80 &      44.27 & 0.10 &          0 & 0.40 &            & 0.05 &          0 & 0.40 &          0 \\ 
DON                                 &        100 & 0.10 &          0 & 0.10 &          0 & 0.40 &            & 0.05 &          0 & 0.40 &          0 \\ 
}

% \diettable{
% Albatross                           & Neon flying squid                   &      68.60 &  0.70 \tabularnewline
%                                     & Boreal clubhook squid               &       1.83 &       \tabularnewline
%                                     & Large gonatid squid                 &       4.57 &       \tabularnewline
%                                     & Micronektonic squid                 &       5.00 &       \tabularnewline
%                                     & Pelagic forage fish                 &      10.00 &       \tabularnewline
%                                     & Saury                               &      10.00 &       \tabularnewline
% Sperm whales                        & Neon flying squid                   &      34.30 &  0.70 \tabularnewline
%                                     & Boreal clubhook squid               &       0.92 &       \tabularnewline
%                                     & Chinook,coho,steelhead              &       0.05 &       \tabularnewline
%                                     & Pomfret                             &       0.49 &       \tabularnewline
%                                     & Large gonatid squid                 &       2.29 &       \tabularnewline
%                                     & Sockeye,Pink                        &       0.26 &       \tabularnewline
%                                     & Micronektonic squid                 &      37.50 &       \tabularnewline
%                                     & Mesopelagic fish                    &      10.42 &       \tabularnewline
%                                     & Pelagic forage fish                 &      12.60 &       \tabularnewline
%                                     & Saury                               &       1.04 &       \tabularnewline
%                                     & Chum salmon                         &       0.12 &       \tabularnewline
% Sharks                              & Neon flying squid                   &      26.72 &  0.80 \tabularnewline
%                                     & Boreal clubhook squid               &       0.71 &       \tabularnewline
%                                     & Chinook,coho,steelhead              &       1.37 &       \tabularnewline
%                                     & Pomfret                             &      12.47 &       \tabularnewline
%                                     & Large gonatid squid                 &       1.78 &       \tabularnewline
%                                     & Sockeye,Pink                        &       6.71 &       \tabularnewline
%                                     & Micronektonic squid                 &      10.00 &       \tabularnewline
%                                     & Pelagic forage fish                 &      10.30 &       \tabularnewline
%                                     & Saury                               &      26.72 &       \tabularnewline
%                                     & Chum salmon                         &       3.21 &       \tabularnewline
% Neon flying squid                   & Neon flying squid                   &      29.50 &  0.60 \tabularnewline
%                                     & Micronektonic squid                 &      22.30 &       \tabularnewline
%                                     & Mesopelagic fish                    &      10.50 &       \tabularnewline
%                                     & Pelagic forage fish                 &      31.90 &       \tabularnewline
%                                     & Saury                               &       5.80 &       \tabularnewline
% Toothed whales                      & Albatross                           &       0.03 &  0.70 \tabularnewline
%                                     & Neon flying squid                   &       4.57 &       \tabularnewline
%                                     & Elephant seals                      &       0.35 &       \tabularnewline
%                                     & Seals,dolphins                      &      11.48 &       \tabularnewline
%                                     & Boreal clubhook squid               &       0.12 &       \tabularnewline
%                                     & Fulmars                             &       0.06 &       \tabularnewline
%                                     & Chinook,coho,steelhead              &       0.64 &       \tabularnewline
%                                     & Skuas,Jaegers                       &       0.07 &       \tabularnewline
%                                     & Pomfret                             &       5.83 &       \tabularnewline
%                                     & Puffins,Shearwaters,Storm Petrels   &       0.42 &       \tabularnewline
%                                     & Kittiwakes                          &       0.04 &       \tabularnewline
%                                     & Large gonatid squid                 &       0.30 &       \tabularnewline
%                                     & Sockeye,Pink                        &       3.14 &       \tabularnewline
%                                     & Fin,sei whales                      &      27.54 &       \tabularnewline
%                                     & Micronektonic squid                 &       5.00 &       \tabularnewline
%                                     & Pelagic forage fish                 &      26.40 &       \tabularnewline
%                                     & Saury                               &      12.50 &       \tabularnewline
%                                     & Chum salmon                         &       1.50 &       \tabularnewline
% Elephant seals                      & Neon flying squid                   &      18.29 &  0.70 \tabularnewline
%                                     & Boreal clubhook squid               &       0.49 &       \tabularnewline
%                                     & Chinook,coho,steelhead              &       0.29 &       \tabularnewline
%                                     & Pomfret                             &       5.60 &       \tabularnewline
%                                     & Large gonatid squid                 &       1.22 &       \tabularnewline
%                                     & Sockeye,Pink                        &       1.43 &       \tabularnewline
%                                     & Micronektonic squid                 &      40.00 &       \tabularnewline
%                                     & Pelagic forage fish                 &      20.00 &       \tabularnewline
%                                     & Saury                               &      12.00 &       \tabularnewline
%                                     & Chum salmon                         &       0.68 &       \tabularnewline}{1}
% 
% \diettable{
% Seals,dolphins                      & Neon flying squid                   &      18.52 &  0.70 \tabularnewline
%                                     & Boreal clubhook squid               &       0.49 &       \tabularnewline
%                                     & Chinook,coho,steelhead              &       0.50 &       \tabularnewline
%                                     & Pomfret                             &       4.56 &       \tabularnewline
%                                     & Large gonatid squid                 &       1.23 &       \tabularnewline
%                                     & Sockeye,Pink                        &       2.45 &       \tabularnewline
%                                     & Micronektonic squid                 &      23.06 &       \tabularnewline
%                                     & Mesopelagic fish                    &      22.63 &       \tabularnewline
%                                     & Pelagic forage fish                 &      15.60 &       \tabularnewline
%                                     & Saury                               &       9.76 &       \tabularnewline
%                                     & Chum salmon                         &       1.17 &       \tabularnewline
% Boreal clubhook squid               & Micronektonic squid                 &      99.00 &  0.60 \tabularnewline
%                                     & Pelagic forage fish                 &       1.00 &       \tabularnewline
% Fulmars                             & Micronektonic squid                 &      96.00 &  0.70 \tabularnewline
%                                     & Pelagic forage fish                 &       4.00 &       \tabularnewline
% Chinook,coho,steelhead              & Micronektonic squid                 &      20.57 &  0.10 \tabularnewline
%                                     & Mesopelagic fish                    &      36.72 &       \tabularnewline
%                                     & Pelagic forage fish                 &      36.72 &       \tabularnewline
%                                     & Mesozooplankton                     &       5.70 &       \tabularnewline
%                                     & Gelatinous zooplankton              &       0.01 &       \tabularnewline
%                                     & Copepods                            &       0.27 &       \tabularnewline
% Skuas,Jaegers                       & Pelagic forage fish                 &      50.00 &  0.70 \tabularnewline
%                                     & Saury                               &      50.00 &       \tabularnewline
% Pomfret                             & Micronektonic squid                 &      75.00 &  0.50 \tabularnewline
%                                     & Mesopelagic fish                    &       8.00 &       \tabularnewline
%                                     & Saury                               &       4.00 &       \tabularnewline
%                                     & Chaetognaths                        &       1.00 &       \tabularnewline
%                                     & Predatory zooplankton               &       1.00 &       \tabularnewline
%                                     & Sergestid shrimp                    &       1.00 &       \tabularnewline
%                                     & Mesozooplankton                     &       9.00 &       \tabularnewline
%                                     & Copepods                            &       1.00 &       \tabularnewline
% Puffins,Shearwaters,Storm Petrels   & Micronektonic squid                 &      31.01 &  0.70 \tabularnewline
%                                     & Pelagic forage fish                 &      26.46 &       \tabularnewline
%                                     & Saury                               &      26.46 &       \tabularnewline
%                                     & Mesozooplankton                     &       9.11 &       \tabularnewline
%                                     & Copepods                            &       6.96 &       \tabularnewline
% Kittiwakes                          & Pelagic forage fish                 &      40.00 &  0.70 \tabularnewline
%                                     & Saury                               &      40.00 &       \tabularnewline
%                                     & Mesozooplankton                     &      11.34 &       \tabularnewline
%                                     & Copepods                            &       8.66 &       \tabularnewline
% Large gonatid squid                 & Micronektonic squid                 &      33.00 &  0.60 \tabularnewline
%                                     & Pelagic forage fish                 &       1.00 &       \tabularnewline
%                                     & Chaetognaths                        &       4.48 &       \tabularnewline
%                                     & Predatory zooplankton               &       3.44 &       \tabularnewline
%                                     & Sergestid shrimp                    &       3.40 &       \tabularnewline
%                                     & Mesozooplankton                     &      31.00 &       \tabularnewline
%                                     & Copepods                            &      23.68 &       \tabularnewline
% Sockeye,Pink                        & Micronektonic squid                 &       7.04 &  0.10 \tabularnewline
%                                     & Mesopelagic fish                    &      10.12 &       \tabularnewline
%                                     & Pelagic forage fish                 &      10.12 &       \tabularnewline
%                                     & Predatory zooplankton               &       0.14 &       \tabularnewline
%                                     & Mesozooplankton                     &      66.82 &       \tabularnewline
%                                     & Gelatinous zooplankton              &       2.71 &       \tabularnewline
%                                     & Copepods                            &       3.04 &       \tabularnewline}{2}
% 
% \diettable{
% Fin,sei whales                      & Neon flying squid                   &       2.29 &  0.70 \tabularnewline
%                                     & Boreal clubhook squid               &       0.06 &       \tabularnewline
%                                     & Chinook,coho,steelhead              &       0.03 &       \tabularnewline
%                                     & Pomfret                             &       0.29 &       \tabularnewline
%                                     & Large gonatid squid                 &       0.15 &       \tabularnewline
%                                     & Sockeye,Pink                        &       0.16 &       \tabularnewline
%                                     & Micronektonic squid                 &       2.50 &       \tabularnewline
%                                     & Mesopelagic fish                    &       6.25 &       \tabularnewline
%                                     & Pelagic forage fish                 &       7.60 &       \tabularnewline
%                                     & Saury                               &       0.62 &       \tabularnewline
%                                     & Chum salmon                         &       0.08 &       \tabularnewline
%                                     & Chaetognaths                        &       5.44 &       \tabularnewline
%                                     & Predatory zooplankton               &       4.17 &       \tabularnewline
%                                     & Sergestid shrimp                    &       4.12 &       \tabularnewline
%                                     & Mesozooplankton                     &      37.57 &       \tabularnewline
%                                     & Copepods                            &      28.70 &       \tabularnewline
% Micronektonic squid                 & Micronektonic squid                 &       5.00 &  0.70 \tabularnewline
%                                     & Chaetognaths                        &       6.46 &       \tabularnewline
%                                     & Predatory zooplankton               &       4.96 &       \tabularnewline
%                                     & Sergestid shrimp                    &       4.89 &       \tabularnewline
%                                     & Mesozooplankton                     &      44.62 &       \tabularnewline
%                                     & Copepods                            &      34.08 &       \tabularnewline
% Mesopelagic fish                    & Chaetognaths                        &      15.00 &  0.70 \tabularnewline
%                                     & Predatory zooplankton               &       3.00 &       \tabularnewline
%                                     & Sergestid shrimp                    &       3.00 &       \tabularnewline
%                                     & Mesozooplankton                     &      44.20 &       \tabularnewline
%                                     & Copepods                            &      34.80 &       \tabularnewline
% Pelagic forage fish                 & Chaetognaths                        &       6.79 &  0.70 \tabularnewline
%                                     & Predatory zooplankton               &       5.22 &       \tabularnewline
%                                     & Sergestid shrimp                    &       5.15 &       \tabularnewline
%                                     & Mesozooplankton                     &      46.97 &       \tabularnewline
%                                     & Copepods                            &      35.88 &       \tabularnewline
% Saury                               & Chaetognaths                        &       5.30 &  0.70 \tabularnewline
%                                     & Predatory zooplankton               &       4.07 &       \tabularnewline
%                                     & Sergestid shrimp                    &       4.01 &       \tabularnewline
%                                     & Mesozooplankton                     &      36.62 &       \tabularnewline
%                                     & Copepods                            &      50.00 &       \tabularnewline
% Chum salmon                         & Micronektonic squid                 &       3.93 &  0.10 \tabularnewline
%                                     & Mesopelagic fish                    &       0.80 &       \tabularnewline
%                                     & Pelagic forage fish                 &       0.80 &       \tabularnewline
%                                     & Chaetognaths                        &       0.04 &       \tabularnewline
%                                     & Predatory zooplankton               &       1.47 &       \tabularnewline
%                                     & Mesozooplankton                     &      22.57 &       \tabularnewline
%                                     & Gelatinous zooplankton              &      40.57 &       \tabularnewline
%                                     & Copepods                            &      29.81 &       \tabularnewline
% Large jellyfish                     & Chaetognaths                        &       3.16 &  0.70 \tabularnewline
%                                     & Predatory zooplankton               &       2.43 &       \tabularnewline
%                                     & Sergestid shrimp                    &       2.39 &       \tabularnewline
%                                     & Mesozooplankton                     &      21.83 &       \tabularnewline
%                                     & Gelatinous zooplankton              &       8.19 &       \tabularnewline
%                                     & Copepods                            &      62.00 &       \tabularnewline
% Chaetognaths                        & Mesozooplankton                     &      20.00 &  0.70 \tabularnewline
%                                     & Copepods                            &      80.00 &       \tabularnewline
% Predatory zooplankton               & Mesozooplankton                     &      20.00 &  0.70 \tabularnewline
%                                     & Copepods                            &      80.00 &       \tabularnewline
% Sergestid shrimp                    & Mesozooplankton                     &      20.00 &  0.70 \tabularnewline
%                                     & Copepods                            &      80.00 &       \tabularnewline
% Mesozooplankton                     & Copepods                            &      40.00 &  0.57 \tabularnewline
%                                     & Microzooplankton                    &      40.00 &       \tabularnewline
%                                     & Large phytoplankton                 &      20.00 &       \tabularnewline}{3}
% 
% \diettable{
% Gelatinous zooplankton              & Copepods                            &      25.00 &  0.70 \tabularnewline
%                                     & Microzooplankton                    &      25.00 &       \tabularnewline
%                                     & Large phytoplankton                 &      50.00 &       \tabularnewline
% Copepods                            & Microzooplankton                    &      30.00 &  0.30 \tabularnewline
%                                     & Small phytoplankton                 &      30.00 &       \tabularnewline
%                                     & Large phytoplankton                 &      40.00 &       \tabularnewline
% Microzooplankton                    & Small phytoplankton                 &     100.00 &  0.30 \tabularnewline}{4}

\FloatBarrier
\diettablelong{
Albatross                           & Neon flying squid                   &      68.60 &  0.70 \tabularnewline
                                    & Boreal clubhook squid               &       1.83 &       \tabularnewline
                                    & Large gonatid squid                 &       4.57 &       \tabularnewline
                                    & Micronektonic squid                 &       5.00 &       \tabularnewline
                                    & Pelagic forage fish                 &      10.00 &       \tabularnewline
                                    & Saury                               &      10.00 &       \tabularnewline
Sperm whales                        & Neon flying squid                   &      34.30 &  0.70 \tabularnewline
                                    & Boreal clubhook squid               &       0.92 &       \tabularnewline
                                    & Chinook,coho,steelhead              &       0.05 &       \tabularnewline
                                    & Pomfret                             &       0.49 &       \tabularnewline
                                    & Large gonatid squid                 &       2.29 &       \tabularnewline
                                    & Sockeye,Pink                        &       0.26 &       \tabularnewline
                                    & Micronektonic squid                 &      37.50 &       \tabularnewline
                                    & Mesopelagic fish                    &      10.42 &       \tabularnewline
                                    & Pelagic forage fish                 &      12.60 &       \tabularnewline
                                    & Saury                               &       1.04 &       \tabularnewline
                                    & Chum salmon                         &       0.12 &       \tabularnewline
Sharks                              & Neon flying squid                   &      26.72 &  0.80 \tabularnewline
                                    & Boreal clubhook squid               &       0.71 &       \tabularnewline
                                    & Chinook,coho,steelhead              &       1.37 &       \tabularnewline
                                    & Pomfret                             &      12.47 &       \tabularnewline
                                    & Large gonatid squid                 &       1.78 &       \tabularnewline
                                    & Sockeye,Pink                        &       6.71 &       \tabularnewline
                                    & Micronektonic squid                 &      10.00 &       \tabularnewline
                                    & Pelagic forage fish                 &      10.30 &       \tabularnewline
                                    & Saury                               &      26.72 &       \tabularnewline
                                    & Chum salmon                         &       3.21 &       \tabularnewline
Neon flying squid                   & Neon flying squid                   &      29.50 &  0.60 \tabularnewline
                                    & Micronektonic squid                 &      22.30 &       \tabularnewline
                                    & Mesopelagic fish                    &      10.50 &       \tabularnewline
                                    & Pelagic forage fish                 &      31.90 &       \tabularnewline
                                    & Saury                               &       5.80 &       \tabularnewline
Toothed whales                      & Albatross                           &       0.03 &  0.70 \tabularnewline
                                    & Neon flying squid                   &       4.57 &       \tabularnewline
                                    & Elephant seals                      &       0.35 &       \tabularnewline
                                    & Seals,dolphins                      &      11.48 &       \tabularnewline
                                    & Boreal clubhook squid               &       0.12 &       \tabularnewline
                                    & Fulmars                             &       0.06 &       \tabularnewline
                                    & Chinook,coho,steelhead              &       0.64 &       \tabularnewline
                                    & Skuas,Jaegers                       &       0.07 &       \tabularnewline
                                    & Pomfret                             &       5.83 &       \tabularnewline
                                    & Puffins,Shearwaters,Storm Petrels   &       0.42 &       \tabularnewline
                                    & Kittiwakes                          &       0.04 &       \tabularnewline
                                    & Large gonatid squid                 &       0.30 &       \tabularnewline
                                    & Sockeye,Pink                        &       3.14 &       \tabularnewline
                                    & Fin,sei whales                      &      27.54 &       \tabularnewline
                                    & Micronektonic squid                 &       5.00 &       \tabularnewline
                                    & Pelagic forage fish                 &      26.40 &       \tabularnewline
                                    & Saury                               &      12.50 &       \tabularnewline
                                    & Chum salmon                         &       1.50 &       \tabularnewline
Elephant seals                      & Neon flying squid                   &      18.29 &  0.70 \tabularnewline
                                    & Boreal clubhook squid               &       0.49 &       \tabularnewline
                                    & Chinook,coho,steelhead              &       0.29 &       \tabularnewline
                                    & Pomfret                             &       5.60 &       \tabularnewline
                                    & Large gonatid squid                 &       1.22 &       \tabularnewline
                                    & Sockeye,Pink                        &       1.43 &       \tabularnewline
                                    & Micronektonic squid                 &      40.00 &       \tabularnewline
                                    & Pelagic forage fish                 &      20.00 &       \tabularnewline
                                    & Saury                               &      12.00 &       \tabularnewline
                                    & Chum salmon                         &       0.68 &       \tabularnewline
Seals,dolphins                      & Neon flying squid                   &      18.52 &  0.70 \tabularnewline
                                    & Boreal clubhook squid               &       0.49 &       \tabularnewline
                                    & Chinook,coho,steelhead              &       0.50 &       \tabularnewline
                                    & Pomfret                             &       4.56 &       \tabularnewline
                                    & Large gonatid squid                 &       1.23 &       \tabularnewline
                                    & Sockeye,Pink                        &       2.45 &       \tabularnewline
                                    & Micronektonic squid                 &      23.06 &       \tabularnewline
                                    & Mesopelagic fish                    &      22.63 &       \tabularnewline
                                    & Pelagic forage fish                 &      15.60 &       \tabularnewline
                                    & Saury                               &       9.76 &       \tabularnewline
                                    & Chum salmon                         &       1.17 &       \tabularnewline
Boreal clubhook squid               & Micronektonic squid                 &      99.00 &  0.60 \tabularnewline
                                    & Pelagic forage fish                 &       1.00 &       \tabularnewline
Fulmars                             & Micronektonic squid                 &      96.00 &  0.70 \tabularnewline
                                    & Pelagic forage fish                 &       4.00 &       \tabularnewline
Chinook,coho,steelhead              & Micronektonic squid                 &      20.57 &  0.10 \tabularnewline
                                    & Mesopelagic fish                    &      36.72 &       \tabularnewline
                                    & Pelagic forage fish                 &      36.72 &       \tabularnewline
                                    & Mesozooplankton                     &       5.70 &       \tabularnewline
                                    & Gelatinous zooplankton              &       0.01 &       \tabularnewline
                                    & Copepods                            &       0.27 &       \tabularnewline
Skuas,Jaegers                       & Pelagic forage fish                 &      50.00 &  0.70 \tabularnewline
                                    & Saury                               &      50.00 &       \tabularnewline
Pomfret                             & Micronektonic squid                 &      75.00 &  0.50 \tabularnewline
                                    & Mesopelagic fish                    &       8.00 &       \tabularnewline
                                    & Saury                               &       4.00 &       \tabularnewline
                                    & Chaetognaths                        &       1.00 &       \tabularnewline
                                    & Predatory zooplankton               &       1.00 &       \tabularnewline
                                    & Sergestid shrimp                    &       1.00 &       \tabularnewline
                                    & Mesozooplankton                     &       9.00 &       \tabularnewline
                                    & Copepods                            &       1.00 &       \tabularnewline
Puffins,Shearwaters,Storm Petrels   & Micronektonic squid                 &      31.01 &  0.70 \tabularnewline
                                    & Pelagic forage fish                 &      26.46 &       \tabularnewline
                                    & Saury                               &      26.46 &       \tabularnewline
                                    & Mesozooplankton                     &       9.11 &       \tabularnewline
                                    & Copepods                            &       6.96 &       \tabularnewline
Kittiwakes                          & Pelagic forage fish                 &      40.00 &  0.70 \tabularnewline
                                    & Saury                               &      40.00 &       \tabularnewline
                                    & Mesozooplankton                     &      11.34 &       \tabularnewline
                                    & Copepods                            &       8.66 &       \tabularnewline
Large gonatid squid                 & Micronektonic squid                 &      33.00 &  0.60 \tabularnewline
                                    & Pelagic forage fish                 &       1.00 &       \tabularnewline
                                    & Chaetognaths                        &       4.48 &       \tabularnewline
                                    & Predatory zooplankton               &       3.44 &       \tabularnewline
                                    & Sergestid shrimp                    &       3.40 &       \tabularnewline
                                    & Mesozooplankton                     &      31.00 &       \tabularnewline
                                    & Copepods                            &      23.68 &       \tabularnewline
Sockeye,Pink                        & Micronektonic squid                 &       7.04 &  0.10 \tabularnewline
                                    & Mesopelagic fish                    &      10.12 &       \tabularnewline
                                    & Pelagic forage fish                 &      10.12 &       \tabularnewline
                                    & Predatory zooplankton               &       0.14 &       \tabularnewline
                                    & Mesozooplankton                     &      66.82 &       \tabularnewline
                                    & Gelatinous zooplankton              &       2.71 &       \tabularnewline
                                    & Copepods                            &       3.04 &       \tabularnewline
Fin,sei whales                      & Neon flying squid                   &       2.29 &  0.70 \tabularnewline
                                    & Boreal clubhook squid               &       0.06 &       \tabularnewline
                                    & Chinook,coho,steelhead              &       0.03 &       \tabularnewline
                                    & Pomfret                             &       0.29 &       \tabularnewline
                                    & Large gonatid squid                 &       0.15 &       \tabularnewline
                                    & Sockeye,Pink                        &       0.16 &       \tabularnewline
                                    & Micronektonic squid                 &       2.50 &       \tabularnewline
                                    & Mesopelagic fish                    &       6.25 &       \tabularnewline
                                    & Pelagic forage fish                 &       7.60 &       \tabularnewline
                                    & Saury                               &       0.62 &       \tabularnewline
                                    & Chum salmon                         &       0.08 &       \tabularnewline
                                    & Chaetognaths                        &       5.44 &       \tabularnewline
                                    & Predatory zooplankton               &       4.17 &       \tabularnewline
                                    & Sergestid shrimp                    &       4.12 &       \tabularnewline
                                    & Mesozooplankton                     &      37.57 &       \tabularnewline
                                    & Copepods                            &      28.70 &       \tabularnewline
Micronektonic squid                 & Micronektonic squid                 &       5.00 &  0.70 \tabularnewline
                                    & Chaetognaths                        &       6.46 &       \tabularnewline
                                    & Predatory zooplankton               &       4.96 &       \tabularnewline
                                    & Sergestid shrimp                    &       4.89 &       \tabularnewline
                                    & Mesozooplankton                     &      44.62 &       \tabularnewline
                                    & Copepods                            &      34.08 &       \tabularnewline
Mesopelagic fish                    & Chaetognaths                        &      15.00 &  0.70 \tabularnewline
                                    & Predatory zooplankton               &       3.00 &       \tabularnewline
                                    & Sergestid shrimp                    &       3.00 &       \tabularnewline
                                    & Mesozooplankton                     &      44.20 &       \tabularnewline
                                    & Copepods                            &      34.80 &       \tabularnewline
Pelagic forage fish                 & Chaetognaths                        &       6.79 &  0.70 \tabularnewline
                                    & Predatory zooplankton               &       5.22 &       \tabularnewline
                                    & Sergestid shrimp                    &       5.15 &       \tabularnewline
                                    & Mesozooplankton                     &      46.97 &       \tabularnewline
                                    & Copepods                            &      35.88 &       \tabularnewline
Saury                               & Chaetognaths                        &       5.30 &  0.70 \tabularnewline
                                    & Predatory zooplankton               &       4.07 &       \tabularnewline
                                    & Sergestid shrimp                    &       4.01 &       \tabularnewline
                                    & Mesozooplankton                     &      36.62 &       \tabularnewline
                                    & Copepods                            &      50.00 &       \tabularnewline
Chum salmon                         & Micronektonic squid                 &       3.93 &  0.10 \tabularnewline
                                    & Mesopelagic fish                    &       0.80 &       \tabularnewline
                                    & Pelagic forage fish                 &       0.80 &       \tabularnewline
                                    & Chaetognaths                        &       0.04 &       \tabularnewline
                                    & Predatory zooplankton               &       1.47 &       \tabularnewline
                                    & Mesozooplankton                     &      22.57 &       \tabularnewline
                                    & Gelatinous zooplankton              &      40.57 &       \tabularnewline
                                    & Copepods                            &      29.81 &       \tabularnewline
Large jellyfish                     & Chaetognaths                        &       3.16 &  0.70 \tabularnewline
                                    & Predatory zooplankton               &       2.43 &       \tabularnewline
                                    & Sergestid shrimp                    &       2.39 &       \tabularnewline
                                    & Mesozooplankton                     &      21.83 &       \tabularnewline
                                    & Gelatinous zooplankton              &       8.19 &       \tabularnewline
                                    & Copepods                            &      62.00 &       \tabularnewline
Chaetognaths                        & Mesozooplankton                     &      20.00 &  0.70 \tabularnewline
                                    & Copepods                            &      80.00 &       \tabularnewline
Predatory zooplankton               & Mesozooplankton                     &      20.00 &  0.70 \tabularnewline
                                    & Copepods                            &      80.00 &       \tabularnewline
Sergestid shrimp                    & Mesozooplankton                     &      20.00 &  0.70 \tabularnewline
                                    & Copepods                            &      80.00 &       \tabularnewline
Mesozooplankton                     & Copepods                            &      40.00 &  0.57 \tabularnewline
                                    & Microzooplankton                    &      40.00 &       \tabularnewline
                                    & Large phytoplankton                 &      20.00 &       \tabularnewline
Gelatinous zooplankton              & Copepods                            &      25.00 &  0.70 \tabularnewline
                                    & Microzooplankton                    &      25.00 &       \tabularnewline
                                    & Large phytoplankton                 &      50.00 &       \tabularnewline
Copepods                            & Microzooplankton                    &      30.00 &  0.30 \tabularnewline
                                    & Small phytoplankton                 &      30.00 &       \tabularnewline
                                    & Large phytoplankton                 &      40.00 &       \tabularnewline
Microzooplankton                    & Small phytoplankton                 &     100.00 &  0.30 \tabularnewline
}


\epderivedstatetable{
Albatross                           &   1.54e-08 & 
\begin{sparkline}{10}
\spark 0.15 0.00 0.15 0.14 0.19 0.14 0.19 0.60 0.23 0.60 0.23 0.66 0.27 0.66 0.27 0.63 0.31 0.63 0.31 0.29 0.35 0.29 0.35 0.26 0.39 0.26 0.39 0.11 0.43 0.11 0.43 0.09 0.47 0.09 0.47 0.00 0.51 0.00 0.51 0.06 0.55 0.06 0.55 0.00 0.59 0.00 0.59 0.03 0.63 0.03 0.63 0.00 /
\end{sparkline}
 &   1.51e-09 & 
\begin{sparkline}{10}
\spark 0.16 0.00 0.16 0.06 0.19 0.06 0.19 0.34 0.22 0.34 0.22 0.57 0.26 0.57 0.26 0.69 0.29 0.69 0.29 0.40 0.32 0.40 0.32 0.40 0.36 0.40 0.36 0.31 0.39 0.31 0.39 0.03 0.42 0.03 0.42 0.03 0.46 0.03 0.46 0.03 0.49 0.03 0.49 0.00 /
\end{sparkline}
 &      0.001 & 
\begin{sparkline}{10}
\spark 0.22 0.00 0.22 0.14 0.24 0.14 0.24 0.26 0.25 0.26 0.25 0.34 0.27 0.34 0.27 0.57 0.28 0.57 0.28 0.71 0.30 0.71 0.30 0.46 0.31 0.46 0.31 0.20 0.33 0.20 0.33 0.11 0.35 0.11 0.35 0.06 0.36 0.06 0.36 0.00 /
\end{sparkline}
 &   0.2 \\ 
Sperm whales                        &   3.36e-07 & 
\begin{sparkline}{10}
\spark 0.16 0.00 0.16 0.34 0.20 0.34 0.20 0.40 0.24 0.40 0.24 0.60 0.27 0.60 0.27 0.60 0.31 0.60 0.31 0.31 0.35 0.31 0.35 0.31 0.38 0.31 0.38 0.14 0.42 0.14 0.42 0.06 0.45 0.06 0.45 0.06 0.49 0.06 0.49 0.00 0.53 0.00 0.53 0.03 0.56 0.03 0.56 0.00 /
\end{sparkline}
 &   1.89e-09 & 
\begin{sparkline}{10}
\spark 0.19 0.00 0.19 0.46 0.23 0.46 0.23 1.00 0.27 1.00 0.27 0.43 0.30 0.43 0.30 0.37 0.34 0.37 0.34 0.23 0.37 0.23 0.37 0.23 0.41 0.23 0.41 0.14 0.44 0.14 0.44 0.00 /
\end{sparkline}
 &      0.009 & 
\begin{sparkline}{10}
\spark 0.23 0.00 0.23 0.26 0.24 0.26 0.24 0.17 0.26 0.17 0.26 0.40 0.27 0.40 0.27 0.60 0.28 0.60 0.28 0.63 0.30 0.63 0.30 0.31 0.31 0.31 0.31 0.20 0.32 0.20 0.32 0.14 0.34 0.14 0.34 0.06 0.35 0.06 0.35 0.09 0.36 0.09 0.36 0.00 /
\end{sparkline}
 &   0.2 \\ 
Sharks                              &   1.74e-05 & 
\begin{sparkline}{10}
\spark 0.12 0.00 0.12 0.29 0.17 0.29 0.17 0.40 0.22 0.40 0.22 0.74 0.27 0.74 0.27 0.57 0.32 0.57 0.32 0.31 0.37 0.31 0.37 0.31 0.42 0.31 0.42 0.11 0.47 0.11 0.47 0.03 0.52 0.03 0.52 0.06 0.56 0.06 0.56 0.03 0.61 0.03 0.61 0.00 /
\end{sparkline}
 &   5.98e-09 & 
\begin{sparkline}{10}
\spark 0.12 0.00 0.12 0.06 0.16 0.06 0.16 0.43 0.21 0.43 0.21 0.80 0.26 0.80 0.26 0.46 0.31 0.46 0.31 0.57 0.36 0.57 0.36 0.29 0.41 0.29 0.41 0.26 0.46 0.26 0.46 0.00 /
\end{sparkline}
 &      0.019 & 
\begin{sparkline}{10}
\spark 0.18 0.00 0.18 0.17 0.21 0.17 0.21 0.40 0.24 0.40 0.24 0.57 0.27 0.57 0.27 0.57 0.30 0.57 0.30 0.60 0.33 0.60 0.33 0.37 0.36 0.37 0.36 0.11 0.39 0.11 0.39 0.06 0.42 0.06 0.42 0.00 /
\end{sparkline}
 &   0.2 \\ 
Neon flying squid                   &   1.69e-04 & 
\begin{sparkline}{10}
\spark 0.13 0.00 0.13 0.31 0.18 0.31 0.18 0.49 0.22 0.49 0.22 0.77 0.27 0.77 0.27 0.43 0.32 0.43 0.32 0.37 0.37 0.37 0.37 0.17 0.42 0.17 0.42 0.20 0.46 0.20 0.46 0.03 0.51 0.03 0.51 0.03 0.56 0.03 0.56 0.00 0.61 0.00 0.61 0.03 0.66 0.03 0.66 0.03 0.71 0.03 0.71 0.00 /
\end{sparkline}
 &   1.95e-08 & 
\begin{sparkline}{10}
\spark 0.00 0.00 0.00 0.49 0.09 0.49 0.09 0.26 0.17 0.26 0.17 0.54 0.25 0.54 0.25 0.60 0.33 0.60 0.33 0.46 0.42 0.46 0.42 0.17 0.50 0.17 0.50 0.14 0.58 0.14 0.58 0.06 0.67 0.06 0.67 0.00 0.75 0.00 0.75 0.06 0.83 0.06 0.83 0.00 0.92 0.00 0.92 0.09 1.00 0.09 1.00 0.00 /
\end{sparkline}
 &      0.504 & 
\begin{sparkline}{10}
\spark 0.17 0.00 0.17 0.17 0.20 0.17 0.20 0.40 0.23 0.40 0.23 0.60 0.26 0.60 0.26 0.51 0.29 0.51 0.29 0.54 0.32 0.54 0.32 0.20 0.35 0.20 0.35 0.17 0.38 0.17 0.38 0.11 0.41 0.11 0.41 0.06 0.44 0.06 0.44 0.06 0.47 0.06 0.47 0.03 0.50 0.03 0.50 0.00 /
\end{sparkline}
 &   0.2 \\ 
Toothed whales                      &   1.08e-08 & 
\begin{sparkline}{10}
\spark 0.14 0.00 0.14 0.11 0.18 0.11 0.18 0.26 0.22 0.26 0.22 0.69 0.25 0.69 0.25 0.57 0.29 0.57 0.29 0.49 0.33 0.49 0.33 0.43 0.37 0.43 0.37 0.17 0.40 0.17 0.40 0.06 0.44 0.06 0.44 0.03 0.48 0.03 0.48 0.06 0.51 0.06 0.51 0.00 /
\end{sparkline}
 &   7.80e-10 & 
\begin{sparkline}{10}
\spark 0.13 0.00 0.13 0.03 0.16 0.03 0.16 0.17 0.19 0.17 0.19 0.23 0.22 0.23 0.22 0.37 0.25 0.37 0.25 0.57 0.28 0.57 0.28 0.60 0.31 0.60 0.31 0.49 0.35 0.49 0.35 0.20 0.38 0.20 0.38 0.11 0.41 0.11 0.41 0.06 0.44 0.06 0.44 0.03 0.47 0.03 0.47 0.00 /
\end{sparkline}
 &      0.002 & 
\begin{sparkline}{10}
\spark 0.21 0.00 0.21 0.09 0.23 0.09 0.23 0.23 0.25 0.23 0.25 0.51 0.27 0.51 0.27 0.57 0.29 0.57 0.29 0.60 0.30 0.60 0.30 0.51 0.32 0.51 0.32 0.34 0.34 0.34 0.34 0.00 /
\end{sparkline}
 &   0.2 \\ 
Elephant seals                      &   1.64e-07 & 
\begin{sparkline}{10}
\spark 0.16 0.00 0.16 0.14 0.19 0.14 0.19 0.51 0.22 0.51 0.22 0.37 0.26 0.37 0.26 0.69 0.29 0.69 0.29 0.51 0.33 0.51 0.33 0.29 0.36 0.29 0.36 0.11 0.40 0.11 0.40 0.06 0.43 0.06 0.43 0.09 0.47 0.09 0.47 0.03 0.50 0.03 0.50 0.06 0.54 0.06 0.54 0.00 /
\end{sparkline}
 &   1.13e-08 & 
\begin{sparkline}{10}
\spark 0.15 0.00 0.15 0.06 0.17 0.06 0.17 0.09 0.20 0.09 0.20 0.31 0.23 0.31 0.23 0.34 0.26 0.34 0.26 0.91 0.29 0.91 0.29 0.40 0.31 0.40 0.31 0.20 0.34 0.20 0.34 0.23 0.37 0.23 0.37 0.23 0.40 0.23 0.40 0.03 0.42 0.03 0.42 0.00 0.45 0.00 0.45 0.06 0.48 0.06 0.48 0.00 /
\end{sparkline}
 &      0.033 & 
\begin{sparkline}{10}
\spark 0.22 0.00 0.22 0.23 0.24 0.23 0.24 0.43 0.26 0.43 0.26 0.46 0.28 0.46 0.28 0.51 0.29 0.51 0.29 0.71 0.31 0.71 0.31 0.29 0.33 0.29 0.33 0.20 0.35 0.20 0.35 0.00 0.37 0.00 0.37 0.03 0.39 0.03 0.39 0.00 /
\end{sparkline}
 &   0.2 \\ 
Seals,dolphins                      &   5.36e-06 & 
\begin{sparkline}{10}
\spark 0.17 0.00 0.17 0.20 0.20 0.20 0.20 0.43 0.23 0.43 0.23 0.66 0.27 0.66 0.27 0.51 0.30 0.51 0.30 0.46 0.33 0.46 0.33 0.31 0.37 0.31 0.37 0.11 0.40 0.11 0.40 0.00 0.43 0.00 0.43 0.17 0.47 0.17 0.47 0.00 /
\end{sparkline}
 &   4.04e-09 & 
\begin{sparkline}{10}
\spark 0.15 0.00 0.15 0.06 0.18 0.06 0.18 0.06 0.21 0.06 0.21 0.63 0.25 0.63 0.25 0.77 0.28 0.77 0.28 0.49 0.31 0.49 0.31 0.43 0.34 0.43 0.34 0.20 0.38 0.20 0.38 0.14 0.41 0.14 0.41 0.03 0.44 0.03 0.44 0.06 0.47 0.06 0.47 0.00 /
\end{sparkline}
 &      0.005 & 
\begin{sparkline}{10}
\spark 0.20 0.00 0.20 0.03 0.22 0.03 0.22 0.17 0.24 0.17 0.24 0.37 0.25 0.37 0.25 0.46 0.27 0.46 0.27 0.60 0.29 0.60 0.29 0.51 0.31 0.51 0.31 0.37 0.33 0.37 0.33 0.26 0.35 0.26 0.35 0.09 0.36 0.09 0.36 0.00 /
\end{sparkline}
 &   0.2 \\ 
Boreal clubhook squid               &   4.40e-06 & 
\begin{sparkline}{10}
\spark 0.10 0.00 0.10 0.37 0.16 0.37 0.16 0.51 0.22 0.51 0.22 0.74 0.29 0.74 0.29 0.49 0.35 0.49 0.35 0.37 0.41 0.37 0.41 0.20 0.47 0.20 0.47 0.09 0.54 0.09 0.54 0.00 0.60 0.00 0.60 0.03 0.66 0.03 0.66 0.00 0.72 0.00 0.72 0.03 0.79 0.03 0.79 0.03 0.85 0.03 0.85 0.00 /
\end{sparkline}
 &   6.54e-08 & 
\begin{sparkline}{10}
\spark 0.06 0.00 0.06 0.20 0.12 0.20 0.12 0.11 0.17 0.11 0.17 0.57 0.23 0.57 0.23 0.63 0.29 0.63 0.29 0.69 0.34 0.69 0.34 0.26 0.40 0.26 0.40 0.20 0.46 0.20 0.46 0.11 0.51 0.11 0.51 0.03 0.57 0.03 0.57 0.06 0.62 0.06 0.62 0.00 /
\end{sparkline}
 &      0.342 & 
\begin{sparkline}{10}
\spark 0.16 0.00 0.16 0.31 0.20 0.31 0.20 0.49 0.24 0.49 0.24 0.71 0.27 0.71 0.27 0.54 0.31 0.54 0.31 0.26 0.35 0.26 0.35 0.20 0.39 0.20 0.39 0.17 0.42 0.17 0.42 0.09 0.46 0.09 0.46 0.03 0.50 0.03 0.50 0.03 0.54 0.03 0.54 0.00 0.57 0.00 0.57 0.03 0.61 0.03 0.61 0.00 /
\end{sparkline}
 &   0.2 \\ 
Fulmars                             &   2.76e-08 & 
\begin{sparkline}{10}
\spark 0.15 0.00 0.15 0.20 0.19 0.20 0.19 0.29 0.23 0.29 0.23 0.97 0.27 0.97 0.27 0.46 0.31 0.46 0.31 0.43 0.35 0.43 0.35 0.37 0.39 0.37 0.39 0.06 0.43 0.06 0.43 0.03 0.47 0.03 0.47 0.06 0.51 0.06 0.51 0.00 /
\end{sparkline}
 &   3.06e-09 & 
\begin{sparkline}{10}
\spark 0.17 0.00 0.17 0.11 0.20 0.11 0.20 0.46 0.24 0.46 0.24 0.66 0.27 0.66 0.27 0.63 0.30 0.63 0.30 0.46 0.33 0.46 0.33 0.40 0.37 0.40 0.37 0.06 0.40 0.06 0.40 0.03 0.43 0.03 0.43 0.06 0.46 0.06 0.46 0.00 /
\end{sparkline}
 &      0.001 & 
\begin{sparkline}{10}
\spark 0.22 0.00 0.22 0.11 0.24 0.11 0.24 0.20 0.25 0.20 0.25 0.54 0.27 0.54 0.27 0.63 0.29 0.63 0.29 0.60 0.30 0.60 0.30 0.37 0.32 0.37 0.32 0.20 0.33 0.20 0.33 0.17 0.35 0.17 0.35 0.00 0.36 0.00 0.36 0.03 0.38 0.03 0.38 0.00 /
\end{sparkline}
 &   0.2 \\ 
Chinook,coho,steelhead              &   8.67e-06 & 
\begin{sparkline}{10}
\spark 0.16 0.00 0.16 0.26 0.19 0.26 0.19 0.20 0.22 0.20 0.22 0.57 0.25 0.57 0.25 0.66 0.28 0.66 0.28 0.40 0.32 0.40 0.32 0.40 0.35 0.40 0.35 0.06 0.38 0.06 0.38 0.06 0.41 0.06 0.41 0.06 0.45 0.06 0.45 0.09 0.48 0.09 0.48 0.09 0.51 0.09 0.51 0.03 0.54 0.03 0.54 0.00 /
\end{sparkline}
 &   2.29e-08 & 
\begin{sparkline}{10}
\spark 0.01 0.00 0.01 0.03 0.06 0.03 0.06 0.09 0.11 0.09 0.11 0.06 0.16 0.06 0.16 0.40 0.21 0.40 0.21 0.60 0.27 0.60 0.27 0.66 0.32 0.66 0.32 0.54 0.37 0.54 0.37 0.29 0.42 0.29 0.42 0.20 0.47 0.20 0.47 0.00 /
\end{sparkline}
 &      0.155 & 
\begin{sparkline}{10}
\spark 0.19 0.00 0.19 0.20 0.22 0.20 0.22 0.46 0.25 0.46 0.25 0.74 0.27 0.74 0.27 0.43 0.30 0.43 0.30 0.49 0.33 0.49 0.33 0.20 0.36 0.20 0.36 0.26 0.39 0.26 0.39 0.06 0.42 0.06 0.42 0.03 0.44 0.03 0.44 0.00 /
\end{sparkline}
 &   0.2 \\ 
Skuas,Jaegers                       &   3.46e-08 & 
\begin{sparkline}{10}
\spark 0.16 0.00 0.16 0.14 0.20 0.14 0.20 0.57 0.23 0.57 0.23 0.54 0.27 0.54 0.27 0.66 0.31 0.66 0.31 0.49 0.34 0.49 0.34 0.23 0.38 0.23 0.38 0.14 0.42 0.14 0.42 0.06 0.45 0.06 0.45 0.00 0.49 0.00 0.49 0.00 0.53 0.00 0.53 0.00 0.56 0.00 0.56 0.03 0.60 0.03 0.60 0.00 /
\end{sparkline}
 &   2.28e-09 & 
\begin{sparkline}{10}
\spark 0.18 0.00 0.18 0.20 0.21 0.20 0.21 0.43 0.24 0.43 0.24 0.60 0.27 0.60 0.27 0.74 0.30 0.74 0.30 0.26 0.33 0.26 0.33 0.34 0.36 0.34 0.36 0.17 0.39 0.17 0.39 0.06 0.42 0.06 0.42 0.06 0.45 0.06 0.45 0.00 /
\end{sparkline}
 &      0.001 & 
\begin{sparkline}{10}
\spark 0.22 0.00 0.22 0.14 0.24 0.14 0.24 0.11 0.25 0.11 0.25 0.40 0.27 0.40 0.27 0.71 0.28 0.71 0.28 0.43 0.30 0.43 0.30 0.54 0.31 0.54 0.31 0.34 0.33 0.34 0.33 0.11 0.34 0.11 0.34 0.06 0.36 0.06 0.36 0.00 /
\end{sparkline}
 &   0.2 \\ 
Pomfret                             &   8.60e-05 & 
\begin{sparkline}{10}
\spark 0.10 0.00 0.10 0.09 0.14 0.09 0.14 0.40 0.19 0.40 0.19 0.63 0.24 0.63 0.24 0.74 0.28 0.74 0.28 0.37 0.33 0.37 0.33 0.17 0.38 0.17 0.38 0.20 0.42 0.20 0.42 0.09 0.47 0.09 0.47 0.06 0.51 0.06 0.51 0.00 0.56 0.00 0.56 0.00 0.61 0.00 0.61 0.03 0.65 0.03 0.65 0.06 0.70 0.06 0.70 0.00 0.74 0.00 0.74 0.00 0.79 0.00 0.79 0.00 0.84 0.00 0.84 0.00 0.88 0.00 0.88 0.00 0.93 0.00 0.93 0.03 0.98 0.03 0.98 0.00 /
\end{sparkline}
 &   1.28e-08 & 
\begin{sparkline}{10}
\spark 0.01 0.00 0.01 0.23 0.09 0.23 0.09 0.43 0.17 0.43 0.17 0.57 0.26 0.57 0.26 0.66 0.34 0.66 0.34 0.60 0.43 0.60 0.43 0.17 0.51 0.17 0.51 0.20 0.59 0.20 0.59 0.00 /
\end{sparkline}
 &      0.205 & 
\begin{sparkline}{10}
\spark 0.17 0.00 0.17 0.06 0.20 0.06 0.20 0.17 0.23 0.17 0.23 0.54 0.25 0.54 0.25 0.63 0.28 0.63 0.28 0.66 0.31 0.66 0.31 0.31 0.33 0.31 0.33 0.26 0.36 0.26 0.36 0.17 0.39 0.17 0.39 0.06 0.41 0.06 0.41 0.00 /
\end{sparkline}
 &   0.2 \\ 
Puffins,Shearwaters,Storm Petrels   &   1.90e-07 & 
\begin{sparkline}{10}
\spark 0.15 0.00 0.15 0.09 0.18 0.09 0.18 0.20 0.22 0.20 0.22 0.49 0.25 0.49 0.25 0.74 0.28 0.74 0.28 0.49 0.32 0.49 0.32 0.51 0.35 0.51 0.35 0.20 0.38 0.20 0.38 0.09 0.41 0.09 0.41 0.00 0.45 0.00 0.45 0.03 0.48 0.03 0.48 0.00 0.51 0.00 0.51 0.03 0.55 0.03 0.55 0.00 /
\end{sparkline}
 &   3.08e-09 & 
\begin{sparkline}{10}
\spark 0.18 0.00 0.18 0.26 0.21 0.26 0.21 0.46 0.24 0.46 0.24 0.60 0.27 0.60 0.27 0.57 0.31 0.57 0.31 0.49 0.34 0.49 0.34 0.26 0.37 0.26 0.37 0.03 0.40 0.03 0.40 0.14 0.44 0.14 0.44 0.06 0.47 0.06 0.47 0.00 /
\end{sparkline}
 &      0.001 & 
\begin{sparkline}{10}
\spark 0.22 0.00 0.22 0.14 0.24 0.14 0.24 0.34 0.25 0.34 0.25 0.43 0.27 0.43 0.27 0.69 0.29 0.69 0.29 0.60 0.31 0.60 0.31 0.34 0.33 0.34 0.33 0.17 0.34 0.17 0.34 0.09 0.36 0.09 0.36 0.06 0.38 0.06 0.38 0.00 /
\end{sparkline}
 &   0.2 \\ 
Kittiwakes                          &   1.84e-08 & 
\begin{sparkline}{10}
\spark 0.15 0.00 0.15 0.20 0.19 0.20 0.19 0.37 0.23 0.37 0.23 0.63 0.26 0.63 0.26 0.69 0.30 0.69 0.30 0.40 0.34 0.40 0.34 0.20 0.37 0.20 0.37 0.14 0.41 0.14 0.41 0.11 0.45 0.11 0.45 0.09 0.48 0.09 0.48 0.03 0.52 0.03 0.52 0.00 /
\end{sparkline}
 &   3.18e-09 & 
\begin{sparkline}{10}
\spark 0.17 0.00 0.17 0.17 0.21 0.17 0.21 0.57 0.24 0.57 0.24 0.54 0.27 0.54 0.27 0.60 0.30 0.60 0.30 0.37 0.34 0.37 0.34 0.31 0.37 0.31 0.37 0.11 0.40 0.11 0.40 0.09 0.43 0.09 0.43 0.09 0.46 0.09 0.46 0.00 /
\end{sparkline}
 &      0.001 & 
\begin{sparkline}{10}
\spark 0.22 0.00 0.22 0.17 0.24 0.17 0.24 0.54 0.26 0.54 0.26 0.51 0.28 0.51 0.28 0.69 0.30 0.69 0.30 0.43 0.32 0.43 0.32 0.34 0.34 0.34 0.34 0.17 0.36 0.17 0.36 0.00 /
\end{sparkline}
 &   0.2 \\ 
Large gonatid squid                 &   1.12e-05 & 
\begin{sparkline}{10}
\spark 0.08 0.00 0.08 0.14 0.14 0.14 0.14 0.49 0.19 0.49 0.19 0.66 0.25 0.66 0.25 0.54 0.30 0.54 0.30 0.54 0.36 0.54 0.36 0.11 0.41 0.11 0.41 0.11 0.47 0.11 0.47 0.06 0.52 0.06 0.52 0.11 0.58 0.11 0.58 0.06 0.63 0.06 0.63 0.03 0.69 0.03 0.69 0.00 /
\end{sparkline}
 &   5.71e-08 & 
\begin{sparkline}{10}
\spark 0.08 0.00 0.08 0.17 0.14 0.17 0.14 0.31 0.20 0.31 0.20 0.66 0.25 0.66 0.25 0.63 0.31 0.63 0.31 0.60 0.36 0.60 0.36 0.20 0.42 0.20 0.42 0.14 0.48 0.14 0.48 0.06 0.53 0.06 0.53 0.06 0.59 0.06 0.59 0.00 0.64 0.00 0.64 0.03 0.70 0.03 0.70 0.00 /
\end{sparkline}
 &      0.336 & 
\begin{sparkline}{10}
\spark 0.12 0.00 0.12 0.11 0.16 0.11 0.16 0.34 0.21 0.34 0.21 0.66 0.25 0.66 0.25 0.51 0.30 0.51 0.30 0.63 0.34 0.63 0.34 0.31 0.39 0.31 0.39 0.14 0.44 0.14 0.44 0.11 0.48 0.11 0.48 0.03 0.53 0.03 0.53 0.00 /
\end{sparkline}
 &   0.2 \\ 
Sockeye,Pink                        &   4.15e-05 & 
\begin{sparkline}{10}
\spark 0.16 0.00 0.16 0.29 0.20 0.29 0.20 0.49 0.24 0.49 0.24 0.63 0.27 0.63 0.27 0.60 0.31 0.60 0.31 0.34 0.34 0.34 0.34 0.20 0.38 0.20 0.38 0.20 0.42 0.20 0.42 0.09 0.45 0.09 0.45 0.00 0.49 0.00 0.49 0.00 0.53 0.00 0.53 0.00 0.56 0.00 0.56 0.00 0.60 0.00 0.60 0.03 0.63 0.03 0.63 0.00 /
\end{sparkline}
 &   4.12e-08 & 
\begin{sparkline}{10}
\spark 0.15 0.00 0.15 0.06 0.17 0.06 0.17 0.03 0.19 0.03 0.19 0.06 0.21 0.06 0.21 0.11 0.23 0.11 0.23 0.20 0.25 0.20 0.25 0.49 0.27 0.49 0.27 0.49 0.29 0.49 0.29 0.57 0.31 0.57 0.31 0.43 0.33 0.43 0.33 0.31 0.35 0.31 0.35 0.11 0.37 0.11 0.37 0.00 /
\end{sparkline}
 &      0.145 & 
\begin{sparkline}{10}
\spark 0.25 0.00 0.25 0.03 0.26 0.03 0.26 0.06 0.26 0.06 0.26 0.31 0.27 0.31 0.27 0.31 0.28 0.31 0.28 0.43 0.28 0.43 0.28 0.51 0.29 0.51 0.29 0.54 0.29 0.54 0.29 0.23 0.30 0.23 0.30 0.23 0.31 0.23 0.31 0.20 0.31 0.20 0.31 0.00 /
\end{sparkline}
 &   0.2 \\ 
Fin,sei whales                      &   1.25e-05 & 
\begin{sparkline}{10}
\spark 0.16 0.00 0.16 0.17 0.19 0.17 0.19 0.31 0.23 0.31 0.23 0.66 0.26 0.66 0.26 0.66 0.30 0.66 0.30 0.43 0.33 0.43 0.33 0.23 0.36 0.23 0.36 0.20 0.40 0.20 0.40 0.09 0.43 0.09 0.43 0.11 0.47 0.11 0.47 0.00 /
\end{sparkline}
 &   5.57e-10 & 
\begin{sparkline}{10}
\spark 0.11 0.00 0.11 0.03 0.15 0.03 0.15 0.11 0.19 0.11 0.19 0.49 0.23 0.49 0.23 0.63 0.27 0.63 0.27 0.51 0.31 0.51 0.31 0.63 0.35 0.63 0.35 0.26 0.39 0.26 0.39 0.11 0.42 0.11 0.42 0.03 0.46 0.03 0.46 0.06 0.50 0.06 0.50 0.00 /
\end{sparkline}
 &      0.004 & 
\begin{sparkline}{10}
\spark 0.21 0.00 0.21 0.09 0.23 0.09 0.23 0.14 0.25 0.14 0.25 0.46 0.26 0.46 0.26 0.60 0.28 0.60 0.28 0.71 0.30 0.71 0.30 0.46 0.32 0.46 0.32 0.26 0.33 0.26 0.33 0.11 0.35 0.11 0.35 0.03 0.37 0.03 0.37 0.00 /
\end{sparkline}
 &   0.2 \\ 
Micronektonic squid                 &   3.99e-04 & 
\begin{sparkline}{10}
\spark 0.09 0.00 0.09 0.54 0.16 0.54 0.16 0.77 0.23 0.77 0.23 0.66 0.30 0.66 0.30 0.26 0.37 0.26 0.37 0.23 0.43 0.23 0.43 0.14 0.50 0.14 0.50 0.03 0.57 0.03 0.57 0.09 0.64 0.09 0.64 0.03 0.71 0.03 0.71 0.06 0.77 0.06 0.77 0.06 0.84 0.06 0.84 0.00 /
\end{sparkline}
 &   9.34e-09 & 
\begin{sparkline}{10}
\spark 0.09 0.00 0.09 0.11 0.14 0.11 0.14 0.34 0.19 0.34 0.19 0.40 0.24 0.40 0.24 0.69 0.28 0.69 0.28 0.54 0.33 0.54 0.33 0.29 0.38 0.29 0.38 0.29 0.43 0.29 0.43 0.06 0.47 0.06 0.47 0.06 0.52 0.06 0.52 0.06 0.57 0.06 0.57 0.03 0.62 0.03 0.62 0.00 /
\end{sparkline}
 &      0.200 & 
\begin{sparkline}{10}
\spark 0.09 0.00 0.09 0.03 0.14 0.03 0.14 0.23 0.18 0.23 0.18 0.51 0.22 0.51 0.22 0.51 0.27 0.51 0.27 0.54 0.31 0.54 0.31 0.46 0.35 0.46 0.35 0.09 0.39 0.09 0.39 0.26 0.44 0.26 0.44 0.14 0.48 0.14 0.48 0.03 0.52 0.03 0.52 0.06 0.57 0.06 0.57 0.00 /
\end{sparkline}
 &   0.2 \\ 
Mesopelagic fish                    &   1.63e-03 & 
\begin{sparkline}{10}
\spark 0.07 0.00 0.07 0.09 0.14 0.09 0.14 0.43 0.20 0.43 0.20 0.89 0.26 0.89 0.26 0.51 0.33 0.51 0.33 0.43 0.39 0.43 0.39 0.40 0.45 0.40 0.45 0.06 0.52 0.06 0.52 0.03 0.58 0.03 0.58 0.00 0.64 0.00 0.64 0.00 0.70 0.00 0.70 0.03 0.77 0.03 0.77 0.00 /
\end{sparkline}
 &   2.35e-08 & 
\begin{sparkline}{10}
\spark 0.06 0.00 0.06 0.09 0.11 0.09 0.11 0.09 0.15 0.09 0.15 0.46 0.20 0.46 0.20 0.31 0.24 0.31 0.24 0.57 0.29 0.57 0.29 0.66 0.33 0.66 0.33 0.23 0.38 0.23 0.38 0.17 0.42 0.17 0.42 0.03 0.47 0.03 0.47 0.11 0.51 0.11 0.51 0.06 0.56 0.06 0.56 0.09 0.60 0.09 0.60 0.00 /
\end{sparkline}
 &      0.291 & 
\begin{sparkline}{10}
\spark 0.13 0.00 0.13 0.11 0.17 0.11 0.17 0.60 0.22 0.60 0.22 0.63 0.26 0.63 0.26 0.60 0.30 0.60 0.30 0.23 0.35 0.23 0.35 0.23 0.39 0.23 0.39 0.14 0.43 0.14 0.43 0.26 0.48 0.26 0.48 0.00 0.52 0.00 0.52 0.00 0.56 0.00 0.56 0.00 0.61 0.00 0.61 0.06 0.65 0.06 0.65 0.00 /
\end{sparkline}
 &   0.2 \\ 
Pelagic forage fish                 &   3.60e-04 & 
\begin{sparkline}{10}
\spark 0.07 0.00 0.07 0.43 0.15 0.43 0.15 0.97 0.23 0.97 0.23 0.57 0.30 0.57 0.30 0.31 0.38 0.31 0.38 0.23 0.45 0.23 0.45 0.11 0.53 0.11 0.53 0.03 0.61 0.03 0.61 0.06 0.68 0.06 0.68 0.03 0.76 0.03 0.76 0.00 0.84 0.00 0.84 0.11 0.91 0.11 0.91 0.00 /
\end{sparkline}
 &   4.99e-09 & 
\begin{sparkline}{10}
\spark 0.10 0.00 0.10 0.09 0.14 0.09 0.14 0.20 0.19 0.20 0.19 0.51 0.23 0.51 0.23 0.69 0.28 0.69 0.28 0.40 0.32 0.40 0.32 0.46 0.36 0.46 0.36 0.17 0.41 0.17 0.41 0.23 0.45 0.23 0.45 0.09 0.50 0.09 0.50 0.03 0.54 0.03 0.54 0.00 /
\end{sparkline}
 &      0.301 & 
\begin{sparkline}{10}
\spark 0.12 0.00 0.12 0.37 0.18 0.37 0.18 0.83 0.24 0.83 0.24 0.66 0.31 0.66 0.31 0.40 0.37 0.40 0.37 0.29 0.43 0.29 0.43 0.14 0.49 0.14 0.49 0.14 0.55 0.14 0.55 0.00 0.61 0.00 0.61 0.00 0.67 0.00 0.67 0.03 0.73 0.03 0.73 0.00 /
\end{sparkline}
 &   0.2 \\ 
Saury                               &   1.81e-04 & 
\begin{sparkline}{10}
\spark 0.11 0.00 0.11 0.34 0.17 0.34 0.17 0.60 0.23 0.60 0.23 0.66 0.28 0.66 0.28 0.51 0.34 0.51 0.34 0.29 0.40 0.29 0.40 0.29 0.45 0.29 0.45 0.06 0.51 0.06 0.51 0.00 0.56 0.00 0.56 0.03 0.62 0.03 0.62 0.06 0.68 0.06 0.68 0.03 0.73 0.03 0.73 0.00 /
\end{sparkline}
 &   2.65e-08 & 
\begin{sparkline}{10}
\spark 0.02 0.00 0.02 0.51 0.12 0.51 0.12 0.80 0.22 0.80 0.22 0.43 0.32 0.43 0.32 0.57 0.43 0.57 0.43 0.23 0.53 0.23 0.53 0.17 0.63 0.17 0.63 0.06 0.73 0.06 0.73 0.09 0.84 0.09 0.84 0.00 /
\end{sparkline}
 &      0.203 & 
\begin{sparkline}{10}
\spark 0.08 0.00 0.08 0.06 0.13 0.06 0.13 0.31 0.18 0.31 0.18 0.66 0.23 0.66 0.23 0.49 0.28 0.49 0.28 0.54 0.33 0.54 0.33 0.40 0.38 0.40 0.38 0.14 0.43 0.14 0.43 0.09 0.48 0.09 0.48 0.03 0.54 0.03 0.54 0.09 0.59 0.09 0.59 0.06 0.64 0.06 0.64 0.00 /
\end{sparkline}
 &   0.2 \\ 
Chum salmon                         &   2.04e-05 & 
\begin{sparkline}{10}
\spark 0.14 0.00 0.14 0.23 0.18 0.23 0.18 0.40 0.23 0.40 0.23 0.71 0.27 0.71 0.27 0.69 0.31 0.69 0.31 0.29 0.36 0.29 0.36 0.26 0.40 0.26 0.40 0.23 0.44 0.23 0.44 0.06 0.49 0.06 0.49 0.00 /
\end{sparkline}
 &   4.85e-08 & 
\begin{sparkline}{10}
\spark 0.11 0.00 0.11 0.03 0.13 0.03 0.13 0.03 0.15 0.03 0.15 0.06 0.18 0.06 0.18 0.03 0.20 0.03 0.20 0.06 0.22 0.06 0.22 0.26 0.24 0.26 0.24 0.34 0.27 0.34 0.27 0.60 0.29 0.60 0.29 0.66 0.31 0.66 0.31 0.46 0.34 0.46 0.34 0.26 0.36 0.26 0.36 0.09 0.38 0.09 0.38 0.00 /
\end{sparkline}
 &      0.133 & 
\begin{sparkline}{10}
\spark 0.25 0.00 0.25 0.06 0.25 0.06 0.25 0.14 0.26 0.14 0.26 0.40 0.27 0.40 0.27 0.46 0.28 0.46 0.28 0.69 0.29 0.69 0.29 0.54 0.30 0.54 0.30 0.34 0.31 0.34 0.31 0.14 0.32 0.14 0.32 0.03 0.32 0.03 0.32 0.03 0.33 0.03 0.33 0.03 0.34 0.03 0.34 0.00 /
\end{sparkline}
 &   0.2 \\ 
Large jellyfish                     &   1.44e-03 & 
\begin{sparkline}{10}
\spark 0.10 0.00 0.10 0.14 0.14 0.14 0.14 0.43 0.19 0.43 0.19 0.54 0.24 0.54 0.24 0.51 0.29 0.51 0.29 0.57 0.34 0.57 0.34 0.17 0.39 0.17 0.39 0.23 0.44 0.23 0.44 0.09 0.48 0.09 0.48 0.06 0.53 0.06 0.53 0.06 0.58 0.06 0.58 0.06 0.63 0.06 0.63 0.00 /
\end{sparkline}
 &   8.94e-08 & 
\begin{sparkline}{10}
\spark 0.14 0.00 0.14 0.60 0.19 0.60 0.19 0.71 0.25 0.71 0.25 0.54 0.31 0.54 0.31 0.49 0.37 0.49 0.37 0.23 0.42 0.23 0.42 0.17 0.48 0.17 0.48 0.06 0.54 0.06 0.54 0.06 0.59 0.06 0.59 0.00 /
\end{sparkline}
 &      0.306 & 
\begin{sparkline}{10}
\spark 0.13 0.00 0.13 0.23 0.17 0.23 0.17 0.49 0.22 0.49 0.22 0.51 0.26 0.51 0.26 0.74 0.31 0.74 0.31 0.31 0.35 0.31 0.35 0.26 0.40 0.26 0.40 0.11 0.44 0.11 0.44 0.03 0.49 0.03 0.49 0.17 0.53 0.17 0.53 0.00 /
\end{sparkline}
 &   0.2 \\ 
Chaetognaths                        &   1.48e-03 & 
\begin{sparkline}{10}
\spark 0.13 0.00 0.13 0.03 0.16 0.03 0.16 0.11 0.19 0.11 0.19 0.20 0.22 0.20 0.22 0.40 0.25 0.40 0.25 0.71 0.28 0.71 0.28 0.40 0.31 0.40 0.31 0.51 0.34 0.51 0.34 0.17 0.37 0.17 0.37 0.09 0.40 0.09 0.40 0.14 0.43 0.14 0.43 0.06 0.46 0.06 0.46 0.00 0.49 0.00 0.49 0.00 0.52 0.00 0.52 0.03 0.55 0.03 0.55 0.00 /
\end{sparkline}
 &   1.77e-07 & 
\begin{sparkline}{10}
\spark 0.01 0.00 0.01 0.03 0.07 0.03 0.07 0.06 0.14 0.06 0.14 0.60 0.20 0.60 0.20 0.86 0.27 0.86 0.27 0.46 0.33 0.46 0.33 0.29 0.39 0.29 0.39 0.34 0.46 0.34 0.46 0.17 0.52 0.17 0.52 0.03 0.59 0.03 0.59 0.03 0.65 0.03 0.65 0.00 /
\end{sparkline}
 &      0.325 & 
\begin{sparkline}{10}
\spark 0.14 0.00 0.14 0.43 0.19 0.43 0.19 0.57 0.23 0.57 0.23 0.54 0.28 0.54 0.28 0.54 0.33 0.54 0.33 0.37 0.38 0.37 0.38 0.11 0.43 0.11 0.43 0.14 0.48 0.14 0.48 0.06 0.53 0.06 0.53 0.09 0.58 0.09 0.58 0.00 /
\end{sparkline}
 &   0.3 \\ 
Predatory zooplankton               &   1.38e-03 & 
\begin{sparkline}{10}
\spark 0.13 0.00 0.13 0.40 0.19 0.40 0.19 0.71 0.24 0.71 0.24 0.69 0.30 0.69 0.30 0.37 0.36 0.37 0.36 0.49 0.41 0.49 0.41 0.00 0.47 0.00 0.47 0.09 0.53 0.09 0.53 0.09 0.59 0.09 0.59 0.03 0.64 0.03 0.64 0.00 /
\end{sparkline}
 &   1.92e-07 & 
\begin{sparkline}{10}
\spark 0.09 0.00 0.09 0.26 0.16 0.26 0.16 0.63 0.22 0.63 0.22 0.71 0.29 0.71 0.29 0.49 0.36 0.49 0.36 0.46 0.43 0.46 0.43 0.14 0.49 0.14 0.49 0.11 0.56 0.11 0.56 0.06 0.63 0.06 0.63 0.00 /
\end{sparkline}
 &      0.320 & 
\begin{sparkline}{10}
\spark 0.11 0.00 0.11 0.14 0.15 0.14 0.15 0.29 0.20 0.29 0.20 0.57 0.24 0.57 0.24 0.43 0.28 0.43 0.28 0.54 0.33 0.54 0.33 0.43 0.37 0.43 0.37 0.17 0.41 0.17 0.41 0.23 0.46 0.23 0.46 0.00 0.50 0.00 0.50 0.06 0.54 0.06 0.54 0.00 /
\end{sparkline}
 &   0.3 \\ 
Sergestid shrimp                    &   1.46e-03 & 
\begin{sparkline}{10}
\spark 0.11 0.00 0.11 0.31 0.17 0.31 0.17 0.74 0.23 0.74 0.23 0.74 0.29 0.74 0.29 0.34 0.35 0.34 0.35 0.34 0.41 0.34 0.41 0.20 0.47 0.20 0.47 0.09 0.53 0.09 0.53 0.06 0.59 0.06 0.59 0.03 0.65 0.03 0.65 0.00 /
\end{sparkline}
 &   2.18e-07 & 
\begin{sparkline}{10}
\spark 0.05 0.00 0.05 0.03 0.10 0.03 0.10 0.14 0.15 0.14 0.15 0.43 0.20 0.43 0.20 0.51 0.24 0.51 0.24 0.54 0.29 0.54 0.29 0.51 0.34 0.51 0.34 0.23 0.39 0.23 0.39 0.23 0.43 0.23 0.43 0.00 0.48 0.00 0.48 0.09 0.53 0.09 0.53 0.06 0.58 0.06 0.58 0.09 0.62 0.09 0.62 0.00 /
\end{sparkline}
 &      0.330 & 
\begin{sparkline}{10}
\spark 0.11 0.00 0.11 0.11 0.15 0.11 0.15 0.40 0.20 0.40 0.20 0.54 0.24 0.54 0.24 0.60 0.28 0.60 0.28 0.43 0.32 0.43 0.32 0.31 0.37 0.31 0.37 0.14 0.41 0.14 0.41 0.11 0.45 0.11 0.45 0.03 0.49 0.03 0.49 0.06 0.54 0.06 0.54 0.03 0.58 0.03 0.58 0.09 0.62 0.09 0.62 0.00 /
\end{sparkline}
 &   0.3 \\ 
Mesozooplankton                     &   5.93e-03 & 
\begin{sparkline}{10}
\spark 0.14 0.00 0.14 0.20 0.18 0.20 0.18 0.54 0.23 0.54 0.23 0.69 0.27 0.69 0.27 0.49 0.31 0.49 0.31 0.40 0.36 0.40 0.36 0.29 0.40 0.29 0.40 0.11 0.44 0.11 0.44 0.06 0.49 0.06 0.49 0.00 0.53 0.00 0.53 0.06 0.57 0.06 0.57 0.03 0.62 0.03 0.62 0.00 /
\end{sparkline}
 &   7.51e-08 & 
\begin{sparkline}{10}
\spark 0.00 0.00 0.00 0.94 0.13 0.94 0.13 0.71 0.26 0.71 0.26 0.34 0.39 0.34 0.39 0.34 0.52 0.34 0.52 0.26 0.65 0.26 0.65 0.17 0.78 0.17 0.78 0.09 0.92 0.09 0.92 0.00 /
\end{sparkline}
 &      0.320 & 
\begin{sparkline}{10}
\spark 0.17 0.00 0.17 0.46 0.21 0.46 0.21 0.77 0.25 0.77 0.25 0.66 0.29 0.66 0.29 0.26 0.33 0.26 0.33 0.31 0.37 0.31 0.37 0.20 0.41 0.20 0.41 0.11 0.45 0.11 0.45 0.09 0.50 0.09 0.50 0.00 /
\end{sparkline}
 &   0.3 \\ 
Gelatinous zooplankton              &   1.18e-03 & 
\begin{sparkline}{10}
\spark 0.13 0.00 0.13 0.06 0.17 0.06 0.17 0.20 0.21 0.20 0.21 0.71 0.25 0.71 0.25 0.57 0.29 0.57 0.29 0.57 0.33 0.57 0.33 0.43 0.37 0.43 0.37 0.20 0.41 0.20 0.41 0.11 0.46 0.11 0.46 0.00 /
\end{sparkline}
 &   1.60e-07 & 
\begin{sparkline}{10}
\spark 0.09 0.00 0.09 0.46 0.15 0.46 0.15 0.37 0.22 0.37 0.22 0.77 0.29 0.77 0.29 0.51 0.35 0.51 0.35 0.23 0.42 0.23 0.42 0.34 0.49 0.34 0.49 0.17 0.56 0.17 0.56 0.00 /
\end{sparkline}
 &      0.313 & 
\begin{sparkline}{10}
\spark 0.09 0.00 0.09 0.20 0.15 0.20 0.15 0.49 0.21 0.49 0.21 0.71 0.26 0.71 0.26 0.43 0.32 0.43 0.32 0.60 0.38 0.60 0.38 0.20 0.44 0.20 0.44 0.14 0.50 0.14 0.50 0.09 0.56 0.09 0.56 0.00 /
\end{sparkline}
 &   0.3 \\ 
Copepods                            &   7.62e-03 & 
\begin{sparkline}{10}
\spark 0.25 0.00 0.25 0.09 0.26 0.09 0.26 0.17 0.27 0.17 0.27 0.37 0.27 0.37 0.27 0.46 0.28 0.46 0.28 0.54 0.29 0.54 0.29 0.60 0.30 0.60 0.30 0.29 0.30 0.29 0.30 0.20 0.31 0.20 0.31 0.09 0.32 0.09 0.32 0.06 0.33 0.06 0.33 0.00 /
\end{sparkline}
 &   2.12e-07 & 
\begin{sparkline}{10}
\spark 0.00 0.00 0.00 0.49 0.11 0.49 0.11 0.46 0.22 0.46 0.22 0.83 0.32 0.83 0.32 0.54 0.43 0.54 0.43 0.29 0.54 0.29 0.54 0.26 0.65 0.26 0.65 0.00 /
\end{sparkline}
 &      0.330 & 
\begin{sparkline}{10}
\spark 0.20 0.00 0.20 0.23 0.22 0.23 0.22 0.40 0.25 0.40 0.25 0.54 0.27 0.54 0.27 0.69 0.30 0.69 0.30 0.49 0.32 0.49 0.32 0.17 0.35 0.17 0.35 0.26 0.38 0.26 0.38 0.03 0.40 0.03 0.40 0.03 0.43 0.03 0.43 0.03 0.45 0.03 0.45 0.00 /
\end{sparkline}
 &   0.3 \\ 
Microzooplankton                    &   6.95e-03 & 
\begin{sparkline}{10}
\spark 0.26 0.00 0.26 0.11 0.26 0.11 0.26 0.17 0.27 0.17 0.27 0.34 0.28 0.34 0.28 0.60 0.28 0.60 0.28 0.63 0.29 0.63 0.29 0.37 0.30 0.37 0.30 0.34 0.30 0.34 0.30 0.26 0.31 0.26 0.31 0.00 0.32 0.00 0.32 0.03 0.32 0.03 0.32 0.00 /
\end{sparkline}
 &   2.80e-07 & 
\begin{sparkline}{10}
\spark 0.01 0.00 0.01 0.94 0.14 0.94 0.14 0.46 0.27 0.46 0.27 0.66 0.40 0.66 0.40 0.37 0.53 0.37 0.53 0.37 0.67 0.37 0.67 0.06 0.80 0.06 0.80 0.00 /
\end{sparkline}
 &      0.308 & 
\begin{sparkline}{10}
\spark 0.17 0.00 0.17 0.03 0.19 0.03 0.19 0.14 0.22 0.14 0.22 0.46 0.25 0.46 0.25 0.60 0.27 0.60 0.27 0.51 0.30 0.51 0.30 0.57 0.33 0.57 0.33 0.29 0.35 0.29 0.35 0.20 0.38 0.20 0.38 0.03 0.41 0.03 0.41 0.03 0.43 0.03 0.43 0.00 /
\end{sparkline}
 &   0.3 \\ 
Small phytoplankton                 &   1.92e-02 & 
\begin{sparkline}{10}
\spark 0.17 0.00 0.17 0.06 0.20 0.06 0.20 0.26 0.23 0.26 0.23 0.66 0.26 0.66 0.26 0.54 0.29 0.54 0.29 0.63 0.32 0.63 0.32 0.37 0.35 0.37 0.35 0.17 0.38 0.17 0.38 0.11 0.41 0.11 0.41 0.06 0.44 0.06 0.44 0.00 /
\end{sparkline}
 &   5.79e-07 & 
\begin{sparkline}{10}
\spark 0.00 0.00 0.00 0.51 0.10 0.51 0.10 0.54 0.21 0.54 0.21 0.43 0.31 0.43 0.31 0.71 0.41 0.71 0.41 0.54 0.52 0.54 0.52 0.00 0.62 0.00 0.62 0.11 0.72 0.11 0.72 0.00 /
\end{sparkline}
 &      0.000 & 
  &   0.0 \\ 
Large phytoplankton                 &   1.12e-02 & 
\begin{sparkline}{10}
\spark 0.14 0.00 0.14 0.23 0.19 0.23 0.19 0.69 0.24 0.69 0.24 0.74 0.28 0.74 0.28 0.46 0.33 0.46 0.33 0.46 0.37 0.46 0.37 0.09 0.42 0.09 0.42 0.09 0.46 0.09 0.46 0.00 0.51 0.00 0.51 0.03 0.55 0.03 0.55 0.06 0.60 0.06 0.60 0.00 0.64 0.00 0.64 0.03 0.69 0.03 0.69 0.00 /
\end{sparkline}
 &   4.71e-07 & 
\begin{sparkline}{10}
\spark 0.01 0.00 0.01 0.43 0.09 0.43 0.09 0.26 0.18 0.26 0.18 0.66 0.26 0.66 0.26 0.60 0.34 0.60 0.34 0.34 0.43 0.34 0.43 0.29 0.51 0.29 0.51 0.09 0.59 0.09 0.59 0.20 0.68 0.20 0.68 0.00 /
\end{sparkline}
 &      0.000 & 
  &   0.0 \\}

% \clearpage % Floats backing up

% \epderivedlinktable{
% Albatross                           & Neon flying squid                   &   2.64e-14 & 
% \begin{sparkline}{10}
% \spark 0.05 0.00 0.05 0.19 0.07 0.19 0.07 0.38 0.09 0.38 0.09 0.54 0.11 0.54 0.11 0.65 0.13 0.65 0.13 0.38 0.15 0.38 0.15 0.27 0.17 0.27 0.17 0.08 0.19 0.08 0.19 0.14 0.22 0.14 0.22 0.00 0.24 0.00 0.24 0.03 0.26 0.03 0.26 0.00 0.28 0.00 0.28 0.03 0.30 0.03 0.30 0.00 0.32 0.00 0.32 0.03 0.34 0.03 0.34 0.00 /
% \end{sparkline}
%  &   2.0 & 1000.0 &   2.0 \\ 
% Albatross                           & Boreal clubhook squid               &   7.52e-16 & 
% \begin{sparkline}{10}
% \spark 0.04 0.00 0.04 0.43 0.07 0.43 0.07 0.62 0.10 0.62 0.10 0.65 0.13 0.65 0.13 0.35 0.16 0.35 0.16 0.35 0.19 0.35 0.19 0.08 0.21 0.08 0.21 0.05 0.24 0.05 0.24 0.11 0.27 0.11 0.27 0.00 0.30 0.00 0.30 0.03 0.33 0.03 0.33 0.00 0.36 0.00 0.36 0.03 0.39 0.03 0.39 0.00 /
% \end{sparkline}
%  &   2.0 & 1000.0 &   2.0 \\ 
% Albatross                           & Large gonatid squid                 &   1.98e-15 & 
% \begin{sparkline}{10}
% \spark 0.04 0.00 0.04 0.70 0.07 0.70 0.07 0.68 0.11 0.68 0.11 0.54 0.14 0.54 0.14 0.24 0.18 0.24 0.18 0.14 0.21 0.14 0.21 0.22 0.24 0.22 0.24 0.05 0.28 0.05 0.28 0.03 0.31 0.03 0.31 0.08 0.35 0.08 0.35 0.03 0.38 0.03 0.38 0.00 /
% \end{sparkline}
%  &   2.0 & 1000.0 &   2.0 \\ 
% Albatross                           & Micronektonic squid                 &   2.10e-15 & 
% \begin{sparkline}{10}
% \spark 0.03 0.00 0.03 0.41 0.06 0.41 0.06 0.65 0.09 0.65 0.09 0.62 0.12 0.62 0.12 0.38 0.16 0.38 0.16 0.16 0.19 0.16 0.19 0.19 0.22 0.19 0.22 0.11 0.25 0.11 0.25 0.05 0.28 0.05 0.28 0.14 0.31 0.14 0.31 0.00 /
% \end{sparkline}
%  &   2.0 & 1000.0 &   2.0 \\ 
% Albatross                           & Pelagic forage fish                 &   4.55e-15 & 
% \begin{sparkline}{10}
% \spark 0.02 0.00 0.02 0.32 0.05 0.32 0.05 0.70 0.09 0.70 0.09 0.81 0.12 0.81 0.12 0.30 0.15 0.30 0.15 0.16 0.19 0.16 0.19 0.11 0.22 0.11 0.22 0.11 0.25 0.11 0.25 0.03 0.29 0.03 0.29 0.08 0.32 0.08 0.32 0.00 0.35 0.00 0.35 0.00 0.39 0.00 0.39 0.03 0.42 0.03 0.42 0.00 0.45 0.00 0.45 0.05 0.49 0.05 0.49 0.00 /
% \end{sparkline}
%  &   2.0 & 1000.0 &   2.0 \\ 
% Albatross                           & Saury                               &   4.35e-15 & 
% \begin{sparkline}{10}
% \spark 0.03 0.00 0.03 0.65 0.07 0.65 0.07 0.70 0.10 0.70 0.10 0.49 0.14 0.49 0.14 0.27 0.17 0.27 0.17 0.22 0.21 0.22 0.21 0.08 0.25 0.08 0.25 0.14 0.28 0.14 0.28 0.05 0.32 0.05 0.32 0.05 0.35 0.05 0.35 0.03 0.39 0.03 0.39 0.00 0.42 0.00 0.42 0.03 0.46 0.03 0.46 0.00 /
% \end{sparkline}
%  &   2.0 & 1000.0 &   2.0 \\ 
% Sperm whales                        & Neon flying squid                   &   2.53e-14 & 
% \begin{sparkline}{10}
% \spark 0.04 0.00 0.04 0.35 0.07 0.35 0.07 0.68 0.10 0.68 0.10 0.57 0.13 0.57 0.13 0.41 0.16 0.41 0.16 0.22 0.18 0.22 0.18 0.32 0.21 0.32 0.21 0.11 0.24 0.11 0.24 0.00 0.27 0.00 0.27 0.03 0.30 0.03 0.30 0.03 0.33 0.03 0.33 0.00 /
% \end{sparkline}
%  &   2.0 & 1000.0 &   2.0 \\ 
% Sperm whales                        & Boreal clubhook squid               &   6.56e-16 & 
% \begin{sparkline}{10}
% \spark 0.04 0.00 0.04 0.11 0.06 0.11 0.06 0.62 0.09 0.62 0.09 0.65 0.11 0.65 0.11 0.49 0.14 0.49 0.14 0.30 0.16 0.30 0.16 0.19 0.18 0.19 0.18 0.08 0.21 0.08 0.21 0.08 0.23 0.08 0.23 0.11 0.26 0.11 0.26 0.00 0.28 0.00 0.28 0.03 0.30 0.03 0.30 0.03 0.33 0.03 0.33 0.00 0.35 0.00 0.35 0.03 0.38 0.03 0.38 0.00 /
% \end{sparkline}
%  &   2.0 & 1000.0 &   2.0 \\ 
% Sperm whales                        & Chinook,coho,steelhead              &   3.92e-17 & 
% \begin{sparkline}{10}
% \spark 0.05 0.00 0.05 0.22 0.07 0.22 0.07 0.32 0.09 0.32 0.09 0.35 0.11 0.35 0.11 0.57 0.13 0.57 0.13 0.73 0.15 0.73 0.15 0.16 0.17 0.16 0.17 0.16 0.19 0.16 0.19 0.05 0.21 0.05 0.21 0.03 0.23 0.03 0.23 0.03 0.25 0.03 0.25 0.03 0.27 0.03 0.27 0.05 0.29 0.05 0.29 0.00 /
% \end{sparkline}
%  &   2.0 & 1000.0 &   2.0 \\ 
% Sperm whales                        & Pomfret                             &   3.54e-16 & 
% \begin{sparkline}{10}
% \spark 0.03 0.00 0.03 0.27 0.06 0.27 0.06 0.57 0.09 0.57 0.09 0.62 0.12 0.62 0.12 0.46 0.15 0.46 0.15 0.32 0.17 0.32 0.17 0.19 0.20 0.19 0.20 0.14 0.23 0.14 0.23 0.03 0.26 0.03 0.26 0.03 0.29 0.03 0.29 0.05 0.31 0.05 0.31 0.03 0.34 0.03 0.34 0.00 /
% \end{sparkline}
%  &   2.0 & 1000.0 &   2.0 \\ 
% Sperm whales                        & Large gonatid squid                 &   1.70e-15 & 
% \begin{sparkline}{10}
% \spark 0.03 0.00 0.03 0.38 0.06 0.38 0.06 0.35 0.09 0.35 0.09 0.49 0.11 0.49 0.11 0.68 0.14 0.68 0.14 0.38 0.17 0.38 0.17 0.08 0.19 0.08 0.19 0.14 0.22 0.14 0.22 0.11 0.25 0.11 0.25 0.03 0.27 0.03 0.27 0.03 0.30 0.03 0.30 0.05 0.33 0.05 0.33 0.00 /
% \end{sparkline}
%  &   2.0 & 1000.0 &   2.0 \\ 
% Sperm whales                        & Sockeye,Pink                        &   1.89e-16 & 
% \begin{sparkline}{10}
% \spark 0.05 0.00 0.05 0.19 0.07 0.19 0.07 0.38 0.09 0.38 0.09 0.51 0.11 0.51 0.11 0.70 0.13 0.70 0.13 0.46 0.16 0.46 0.16 0.22 0.18 0.22 0.18 0.05 0.20 0.05 0.20 0.08 0.22 0.08 0.22 0.03 0.24 0.03 0.24 0.05 0.27 0.05 0.27 0.03 0.29 0.03 0.29 0.00 /
% \end{sparkline}
%  &   2.0 & 1000.0 &   2.0 \\ 
% Sperm whales                        & Micronektonic squid                 &   2.52e-14 & 
% \begin{sparkline}{10}
% \spark 0.05 0.00 0.05 0.32 0.07 0.32 0.07 0.46 0.10 0.46 0.10 0.65 0.12 0.65 0.12 0.46 0.15 0.46 0.15 0.35 0.17 0.35 0.17 0.27 0.19 0.27 0.19 0.11 0.22 0.11 0.22 0.05 0.24 0.05 0.24 0.00 0.26 0.00 0.26 0.00 0.29 0.00 0.29 0.03 0.31 0.03 0.31 0.00 /
% \end{sparkline}
%  &   2.0 & 1000.0 &   2.0 \\ 
% Sperm whales                        & Mesopelagic fish                    &   7.67e-15 & 
% \begin{sparkline}{10}
% \spark 0.03 0.00 0.03 0.19 0.06 0.19 0.06 0.54 0.09 0.54 0.09 0.73 0.12 0.73 0.12 0.43 0.15 0.43 0.15 0.43 0.18 0.43 0.18 0.05 0.20 0.05 0.20 0.19 0.23 0.19 0.23 0.14 0.26 0.14 0.26 0.00 /
% \end{sparkline}
%  &   2.0 & 1000.0 &   2.0 \\ 
% Sperm whales                        & Pelagic forage fish                 &   8.47e-15 & 
% \begin{sparkline}{10}
% \spark 0.03 0.00 0.03 0.30 0.06 0.30 0.06 0.59 0.09 0.59 0.09 0.62 0.12 0.62 0.12 0.38 0.15 0.38 0.15 0.38 0.18 0.38 0.18 0.19 0.21 0.19 0.21 0.08 0.24 0.08 0.24 0.05 0.27 0.05 0.27 0.11 0.29 0.11 0.29 0.00 /
% \end{sparkline}
%  &   2.0 & 1000.0 &   2.0 \\ 
% Sperm whales                        & Saury                               &   7.41e-16 & 
% \begin{sparkline}{10}
% \spark 0.03 0.00 0.03 0.49 0.07 0.49 0.07 0.73 0.10 0.73 0.10 0.59 0.13 0.59 0.13 0.32 0.17 0.32 0.17 0.24 0.20 0.24 0.20 0.14 0.23 0.14 0.23 0.08 0.27 0.08 0.27 0.08 0.30 0.08 0.30 0.00 0.33 0.00 0.33 0.00 0.37 0.00 0.37 0.00 0.40 0.00 0.40 0.00 0.44 0.00 0.44 0.00 0.47 0.00 0.47 0.00 0.50 0.00 0.50 0.00 0.54 0.00 0.54 0.00 0.57 0.00 0.57 0.00 0.60 0.00 0.60 0.00 0.64 0.00 0.64 0.03 0.67 0.03 0.67 0.00 /
% \end{sparkline}
%  &   2.0 & 1000.0 &   2.0 \\ 
% Sperm whales                        & Chum salmon                         &   9.11e-17 & 
% \begin{sparkline}{10}
% \spark 0.04 0.00 0.04 0.11 0.06 0.11 0.06 0.38 0.08 0.38 0.08 0.41 0.10 0.41 0.10 0.41 0.12 0.41 0.12 0.73 0.14 0.73 0.14 0.32 0.16 0.32 0.16 0.14 0.18 0.14 0.18 0.03 0.20 0.03 0.20 0.05 0.22 0.05 0.22 0.05 0.24 0.05 0.24 0.05 0.26 0.05 0.26 0.00 0.28 0.00 0.28 0.03 0.30 0.03 0.30 0.00 /
% \end{sparkline}
%  &   2.0 & 1000.0 &   2.0 \\ 
% Sharks                              & Neon flying squid                   &   1.48e-12 & 
% \begin{sparkline}{10}
% \spark 0.04 0.00 0.04 0.46 0.07 0.46 0.07 0.70 0.11 0.70 0.11 0.49 0.14 0.49 0.14 0.57 0.17 0.57 0.17 0.30 0.21 0.30 0.21 0.08 0.24 0.08 0.24 0.08 0.28 0.08 0.28 0.03 0.31 0.03 0.31 0.00 /
% \end{sparkline}
%  &   2.0 & 1000.0 &   2.0 \\ 
% Sharks                              & Boreal clubhook squid               &   4.06e-14 & 
% \begin{sparkline}{10}
% \spark 0.04 0.00 0.04 0.46 0.07 0.46 0.07 0.78 0.10 0.78 0.10 0.32 0.13 0.32 0.13 0.51 0.16 0.51 0.16 0.32 0.19 0.32 0.19 0.08 0.23 0.08 0.23 0.11 0.26 0.11 0.26 0.03 0.29 0.03 0.29 0.00 0.32 0.00 0.32 0.03 0.35 0.03 0.35 0.05 0.38 0.05 0.38 0.00 /
% \end{sparkline}
%  &   2.0 & 1000.0 &   2.0 \\ 
% Sharks                              & Chinook,coho,steelhead              &   7.94e-14 & 
% \begin{sparkline}{10}
% \spark 0.04 0.00 0.04 0.35 0.06 0.35 0.06 0.24 0.09 0.24 0.09 0.68 0.12 0.68 0.12 0.65 0.14 0.65 0.14 0.30 0.17 0.30 0.17 0.19 0.19 0.19 0.19 0.11 0.22 0.11 0.22 0.11 0.24 0.11 0.24 0.03 0.27 0.03 0.27 0.05 0.29 0.05 0.29 0.00 /
% \end{sparkline}
%  &   2.0 & 1000.0 &   2.0 \\ 
% Sharks                              & Pomfret                             &   6.83e-13 & 
% \begin{sparkline}{10}
% \spark 0.04 0.00 0.04 0.41 0.07 0.41 0.07 0.57 0.10 0.57 0.10 0.70 0.13 0.70 0.13 0.38 0.16 0.38 0.16 0.32 0.19 0.32 0.19 0.11 0.22 0.11 0.22 0.08 0.25 0.08 0.25 0.08 0.28 0.08 0.28 0.05 0.31 0.05 0.31 0.00 /
% \end{sparkline}
%  &   2.0 & 1000.0 &   2.0 \\ 
% Sharks                              & Large gonatid squid                 &   1.00e-13 & 
% \begin{sparkline}{10}
% \spark 0.03 0.00 0.03 0.49 0.06 0.49 0.06 0.57 0.10 0.57 0.10 0.46 0.13 0.46 0.13 0.57 0.16 0.57 0.16 0.22 0.19 0.22 0.19 0.30 0.23 0.30 0.23 0.05 0.26 0.05 0.26 0.00 0.29 0.00 0.29 0.03 0.33 0.03 0.33 0.00 0.36 0.00 0.36 0.03 0.39 0.03 0.39 0.00 /
% \end{sparkline}
%  &   2.0 & 1000.0 &   2.0 \\ 
% Sharks                              & Sockeye,Pink                        &   3.83e-13 & 
% \begin{sparkline}{10}
% \spark 0.04 0.00 0.04 0.32 0.06 0.32 0.06 0.30 0.09 0.30 0.09 0.57 0.11 0.57 0.11 0.65 0.14 0.65 0.14 0.38 0.16 0.38 0.16 0.16 0.19 0.16 0.19 0.14 0.22 0.14 0.22 0.08 0.24 0.08 0.24 0.05 0.27 0.05 0.27 0.05 0.29 0.05 0.29 0.00 /
% \end{sparkline}
%  &   2.0 & 1000.0 &   2.0 \\ 
% Sharks                              & Micronektonic squid                 &   6.04e-13 & 
% \begin{sparkline}{10}
% \spark 0.03 0.00 0.03 0.43 0.06 0.43 0.06 0.59 0.10 0.59 0.10 0.70 0.14 0.70 0.14 0.46 0.17 0.46 0.17 0.27 0.21 0.27 0.21 0.08 0.25 0.08 0.25 0.08 0.28 0.08 0.28 0.05 0.32 0.05 0.32 0.00 0.36 0.00 0.36 0.00 0.39 0.00 0.39 0.00 0.43 0.00 0.43 0.03 0.47 0.03 0.47 0.00 /
% \end{sparkline}
%  &   2.0 & 1000.0 &   2.0 \\ 
% Sharks                              & Pelagic forage fish                 &   5.95e-13 & 
% \begin{sparkline}{10}
% \spark 0.02 0.00 0.02 0.35 0.06 0.35 0.06 0.54 0.09 0.54 0.09 0.54 0.12 0.54 0.12 0.54 0.15 0.54 0.15 0.38 0.19 0.38 0.19 0.14 0.22 0.14 0.22 0.08 0.25 0.08 0.25 0.03 0.28 0.03 0.28 0.05 0.32 0.05 0.32 0.05 0.35 0.05 0.35 0.00 /
% \end{sparkline}
%  &   2.0 & 1000.0 &   2.0 \\ 
% Sharks                              & Saury                               &   1.51e-12 & 
% \begin{sparkline}{10}
% \spark 0.02 0.00 0.02 0.35 0.06 0.35 0.06 0.57 0.09 0.57 0.09 0.62 0.12 0.62 0.12 0.51 0.16 0.51 0.16 0.24 0.19 0.24 0.19 0.16 0.22 0.16 0.22 0.11 0.26 0.11 0.26 0.03 0.29 0.03 0.29 0.05 0.32 0.05 0.32 0.05 0.35 0.05 0.35 0.00 /
% \end{sparkline}
%  &   2.0 & 1000.0 &   2.0 \\ 
% Sharks                              & Chum salmon                         &   1.87e-13 & 
% \begin{sparkline}{10}
% \spark 0.04 0.00 0.04 0.41 0.07 0.41 0.07 0.43 0.10 0.43 0.10 0.68 0.12 0.68 0.12 0.46 0.15 0.46 0.15 0.32 0.18 0.32 0.18 0.24 0.21 0.24 0.21 0.08 0.24 0.08 0.24 0.05 0.27 0.05 0.27 0.03 0.29 0.03 0.29 0.00 /
% \end{sparkline}
%  &   2.0 & 1000.0 &   2.0 \\ 
% Neon flying squid                   & Neon flying squid                   &   8.19e-12 & 
% \begin{sparkline}{10}
% \spark 0.03 0.00 0.03 0.46 0.06 0.46 0.06 0.49 0.09 0.49 0.09 0.46 0.11 0.46 0.11 0.62 0.14 0.62 0.14 0.19 0.17 0.19 0.17 0.08 0.20 0.08 0.20 0.24 0.23 0.24 0.23 0.03 0.25 0.03 0.25 0.03 0.28 0.03 0.28 0.03 0.31 0.03 0.31 0.00 0.34 0.00 0.34 0.05 0.36 0.05 0.36 0.00 0.39 0.00 0.39 0.00 0.42 0.00 0.42 0.00 0.45 0.00 0.45 0.00 0.47 0.00 0.47 0.00 0.50 0.00 0.50 0.00 0.53 0.00 0.53 0.00 0.56 0.00 0.56 0.00 0.59 0.00 0.59 0.00 0.61 0.00 0.61 0.00 0.64 0.00 0.64 0.03 0.67 0.03 0.67 0.00 /
% \end{sparkline}
%  &   2.0 & 1000.0 &   2.0 \\ 
% Neon flying squid                   & Micronektonic squid                 &   7.06e-12 & 
% \begin{sparkline}{10}
% \spark 0.01 0.00 0.01 0.51 0.06 0.51 0.06 0.81 0.10 0.81 0.10 0.59 0.15 0.59 0.15 0.30 0.19 0.30 0.19 0.19 0.23 0.19 0.23 0.08 0.28 0.08 0.28 0.11 0.32 0.11 0.32 0.08 0.37 0.08 0.37 0.00 0.41 0.00 0.41 0.03 0.45 0.03 0.45 0.00 /
% \end{sparkline}
%  &   2.0 & 1000.0 &   2.0 \\ 
% Neon flying squid                   & Mesopelagic fish                    &   3.33e-12 & 
% \begin{sparkline}{10}
% \spark 0.02 0.00 0.02 0.59 0.06 0.59 0.06 0.68 0.10 0.68 0.10 0.49 0.14 0.49 0.14 0.59 0.18 0.59 0.18 0.19 0.22 0.19 0.22 0.05 0.26 0.05 0.26 0.05 0.30 0.05 0.30 0.00 0.35 0.00 0.35 0.00 0.39 0.00 0.39 0.00 0.43 0.00 0.43 0.00 0.47 0.00 0.47 0.00 0.51 0.00 0.51 0.00 0.55 0.00 0.55 0.03 0.59 0.03 0.59 0.00 0.63 0.00 0.63 0.00 0.67 0.00 0.67 0.00 0.71 0.00 0.71 0.00 0.75 0.00 0.75 0.03 0.79 0.03 0.79 0.00 /
% \end{sparkline}
%  &   2.0 & 1000.0 &   2.0 \\ 
% Neon flying squid                   & Pelagic forage fish                 &   9.59e-12 & 
% \begin{sparkline}{10}
% \spark 0.02 0.00 0.02 0.24 0.06 0.24 0.06 0.81 0.09 0.81 0.09 0.76 0.12 0.76 0.12 0.24 0.16 0.24 0.16 0.24 0.19 0.24 0.19 0.08 0.22 0.08 0.22 0.08 0.25 0.08 0.25 0.08 0.29 0.08 0.29 0.11 0.32 0.11 0.32 0.05 0.35 0.05 0.35 0.00 /
% \end{sparkline}
%  &   2.0 & 1000.0 &   2.0 \\ 
% Neon flying squid                   & Saury                               &   1.62e-12 & 
% \begin{sparkline}{10}
% \spark 0.03 0.00 0.03 0.59 0.07 0.59 0.07 0.81 0.11 0.81 0.11 0.46 0.15 0.46 0.15 0.35 0.19 0.35 0.19 0.24 0.23 0.24 0.23 0.08 0.27 0.08 0.27 0.08 0.31 0.08 0.31 0.03 0.35 0.03 0.35 0.03 0.39 0.03 0.39 0.00 0.43 0.00 0.43 0.00 0.47 0.00 0.47 0.03 0.51 0.03 0.51 0.00 /
% \end{sparkline}
%  &   2.0 & 1000.0 &   2.0 \\ 
% Toothed whales                      & Albatross                           &   1.13e-18 & 
% \begin{sparkline}{10}
% \spark 0.03 0.00 0.03 0.30 0.06 0.30 0.06 0.78 0.10 0.78 0.10 0.46 0.13 0.46 0.13 0.49 0.16 0.49 0.16 0.43 0.19 0.43 0.19 0.08 0.23 0.08 0.23 0.03 0.26 0.03 0.26 0.14 0.29 0.14 0.29 0.00 /
% \end{sparkline}
%  &   2.0 & 1000.0 &   2.0 \\ 
% Toothed whales                      & Neon flying squid                   &   1.67e-16 & 
% \begin{sparkline}{10}
% \spark 0.03 0.00 0.03 0.19 0.06 0.19 0.06 0.78 0.09 0.78 0.09 0.59 0.13 0.59 0.13 0.49 0.16 0.49 0.16 0.32 0.20 0.32 0.20 0.16 0.23 0.16 0.23 0.16 0.27 0.16 0.27 0.00 /
% \end{sparkline}
%  &   2.0 & 1000.0 &   2.0 \\ 
% Toothed whales                      & Elephant seals                      &   1.33e-17 & 
% \begin{sparkline}{10}
% \spark 0.02 0.00 0.02 0.30 0.06 0.30 0.06 0.59 0.09 0.59 0.09 0.68 0.13 0.68 0.13 0.43 0.16 0.43 0.16 0.32 0.19 0.32 0.19 0.19 0.23 0.19 0.23 0.08 0.26 0.08 0.26 0.11 0.30 0.11 0.30 0.00 /
% \end{sparkline}
%  &   2.0 & 1000.0 &   2.0 \\ 
% Toothed whales                      & Seals,dolphins                      &   4.21e-16 & 
% \begin{sparkline}{10}
% \spark 0.04 0.00 0.04 0.62 0.08 0.62 0.08 0.59 0.11 0.59 0.11 0.59 0.15 0.59 0.15 0.49 0.18 0.49 0.18 0.24 0.21 0.24 0.21 0.08 0.25 0.08 0.25 0.03 0.28 0.03 0.28 0.05 0.32 0.05 0.32 0.00 /
% \end{sparkline}
%  &   2.0 & 1000.0 &   2.0 \\ 
% Toothed whales                      & Boreal clubhook squid               &   4.53e-18 & 
% \begin{sparkline}{10}
% \spark 0.02 0.00 0.02 0.27 0.05 0.27 0.05 0.27 0.08 0.27 0.08 0.46 0.10 0.46 0.10 0.51 0.13 0.51 0.13 0.43 0.15 0.43 0.15 0.27 0.18 0.27 0.18 0.30 0.20 0.30 0.20 0.08 0.23 0.08 0.23 0.05 0.25 0.05 0.25 0.00 0.28 0.00 0.28 0.00 0.30 0.00 0.30 0.00 0.33 0.00 0.33 0.05 0.35 0.05 0.35 0.00 /
% \end{sparkline}
%  &   2.0 & 1000.0 &   2.0 \\ 
% Toothed whales                      & Fulmars                             &   2.24e-18 & 
% \begin{sparkline}{10}
% \spark 0.03 0.00 0.03 0.32 0.06 0.32 0.06 0.57 0.09 0.57 0.09 0.70 0.12 0.70 0.12 0.38 0.15 0.38 0.15 0.27 0.18 0.27 0.18 0.14 0.21 0.14 0.21 0.16 0.24 0.16 0.24 0.11 0.27 0.11 0.27 0.05 0.30 0.05 0.30 0.00 /
% \end{sparkline}
%  &   2.0 & 1000.0 &   2.0 \\ 
% Toothed whales                      & Chinook,coho,steelhead              &   2.42e-17 & 
% \begin{sparkline}{10}
% \spark 0.03 0.00 0.03 0.11 0.06 0.11 0.06 0.59 0.09 0.59 0.09 0.51 0.12 0.51 0.12 0.65 0.15 0.65 0.15 0.49 0.18 0.49 0.18 0.22 0.21 0.22 0.21 0.14 0.24 0.14 0.24 0.00 /
% \end{sparkline}
%  &   2.0 & 1000.0 &   2.0 \\ 
% Toothed whales                      & Skuas,Jaegers                       &   2.86e-18 & 
% \begin{sparkline}{10}
% \spark 0.03 0.00 0.03 0.43 0.06 0.43 0.06 0.57 0.10 0.57 0.10 0.70 0.13 0.70 0.13 0.32 0.16 0.32 0.16 0.43 0.20 0.43 0.20 0.08 0.23 0.08 0.23 0.08 0.27 0.08 0.27 0.00 0.30 0.00 0.30 0.05 0.33 0.05 0.33 0.00 0.37 0.00 0.37 0.03 0.40 0.03 0.40 0.00 /
% \end{sparkline}
%  &   2.0 & 1000.0 &   2.0 \\ 
% Toothed whales                      & Pomfret                             &   2.21e-16 & 
% \begin{sparkline}{10}
% \spark 0.03 0.00 0.03 0.19 0.06 0.19 0.06 0.73 0.10 0.73 0.10 0.59 0.13 0.59 0.13 0.49 0.16 0.49 0.16 0.38 0.19 0.38 0.19 0.19 0.22 0.19 0.22 0.08 0.25 0.08 0.25 0.05 0.29 0.05 0.29 0.00 /
% \end{sparkline}
%  &   2.0 & 1000.0 &   2.0 \\ 
% Toothed whales                      & Puffins,Shearwaters,Storm Petrels   &   1.62e-17 & 
% \begin{sparkline}{10}
% \spark 0.04 0.00 0.04 0.27 0.07 0.27 0.07 0.65 0.09 0.65 0.09 0.57 0.12 0.57 0.12 0.57 0.15 0.57 0.15 0.27 0.17 0.27 0.17 0.14 0.20 0.14 0.20 0.05 0.23 0.05 0.23 0.03 0.26 0.03 0.26 0.11 0.28 0.11 0.28 0.00 0.31 0.00 0.31 0.00 0.34 0.00 0.34 0.00 0.36 0.00 0.36 0.03 0.39 0.03 0.39 0.00 0.42 0.00 0.42 0.03 0.45 0.03 0.45 0.00 /
% \end{sparkline}
%  &   2.0 & 1000.0 &   2.0 \\ 
% Toothed whales                      & Kittiwakes                          &   1.62e-18 & 
% \begin{sparkline}{10}
% \spark 0.02 0.00 0.02 0.24 0.06 0.24 0.06 0.76 0.09 0.76 0.09 0.68 0.13 0.68 0.13 0.32 0.17 0.32 0.17 0.49 0.20 0.49 0.20 0.03 0.24 0.03 0.24 0.16 0.28 0.16 0.28 0.00 0.31 0.00 0.31 0.03 0.35 0.03 0.35 0.00 /
% \end{sparkline}
%  &   2.0 & 1000.0 &   2.0 \\ 
% Toothed whales                      & Large gonatid squid                 &   1.11e-17 & 
% \begin{sparkline}{10}
% \spark 0.02 0.00 0.02 0.14 0.05 0.14 0.05 0.22 0.07 0.22 0.07 0.41 0.09 0.41 0.09 0.65 0.11 0.65 0.11 0.54 0.14 0.54 0.14 0.24 0.16 0.24 0.16 0.05 0.18 0.05 0.18 0.14 0.21 0.14 0.21 0.11 0.23 0.11 0.23 0.08 0.25 0.08 0.25 0.05 0.28 0.05 0.28 0.05 0.30 0.05 0.30 0.03 0.32 0.03 0.32 0.00 /
% \end{sparkline}
%  &   2.0 & 1000.0 &   2.0 \\ 
% Toothed whales                      & Sockeye,Pink                        &   1.18e-16 & 
% \begin{sparkline}{10}
% \spark 0.03 0.00 0.03 0.08 0.05 0.08 0.05 0.43 0.08 0.43 0.08 0.54 0.11 0.54 0.11 0.54 0.13 0.54 0.13 0.51 0.16 0.51 0.16 0.38 0.19 0.38 0.19 0.16 0.21 0.16 0.21 0.05 0.24 0.05 0.24 0.00 /
% \end{sparkline}
%  &   2.0 & 1000.0 &   2.0 \\ 
% Toothed whales                      & Fin,sei whales                      &   1.03e-15 & 
% \begin{sparkline}{10}
% \spark 0.03 0.00 0.03 0.11 0.06 0.11 0.06 0.43 0.08 0.43 0.08 0.70 0.11 0.70 0.11 0.46 0.14 0.46 0.14 0.49 0.16 0.49 0.16 0.22 0.19 0.22 0.19 0.16 0.22 0.16 0.22 0.14 0.24 0.14 0.24 0.00 /
% \end{sparkline}
%  &   2.0 & 1000.0 &   2.0 \\ 
% Toothed whales                      & Micronektonic squid                 &   1.83e-16 & 
% \begin{sparkline}{10}
% \spark 0.03 0.00 0.03 0.16 0.06 0.16 0.06 0.81 0.09 0.81 0.09 0.49 0.12 0.49 0.12 0.46 0.15 0.46 0.15 0.38 0.17 0.38 0.17 0.22 0.20 0.22 0.20 0.05 0.23 0.05 0.23 0.05 0.26 0.05 0.26 0.00 0.29 0.00 0.29 0.00 0.31 0.00 0.31 0.05 0.34 0.05 0.34 0.00 0.37 0.00 0.37 0.00 0.40 0.00 0.40 0.00 0.43 0.00 0.43 0.00 0.45 0.00 0.45 0.00 0.48 0.00 0.48 0.03 0.51 0.03 0.51 0.00 /
% \end{sparkline}
%  &   2.0 & 1000.0 &   2.0 \\ 
% Toothed whales                      & Pelagic forage fish                 &   1.03e-15 & 
% \begin{sparkline}{10}
% \spark 0.04 0.00 0.04 0.65 0.08 0.65 0.08 0.62 0.11 0.62 0.11 0.54 0.15 0.54 0.15 0.59 0.19 0.59 0.19 0.16 0.22 0.16 0.22 0.05 0.26 0.05 0.26 0.08 0.29 0.08 0.29 0.00 /
% \end{sparkline}
%  &   2.0 & 1000.0 &   2.0 \\ 
% Toothed whales                      & Saury                               &   4.38e-16 & 
% \begin{sparkline}{10}
% \spark 0.03 0.00 0.03 0.22 0.06 0.22 0.06 0.62 0.09 0.62 0.09 0.57 0.12 0.57 0.12 0.62 0.16 0.62 0.16 0.38 0.19 0.38 0.19 0.11 0.22 0.11 0.22 0.14 0.25 0.14 0.25 0.03 0.28 0.03 0.28 0.03 0.31 0.03 0.31 0.00 /
% \end{sparkline}
%  &   2.0 & 1000.0 &   2.0 \\ 
% Toothed whales                      & Chum salmon                         &   5.73e-17 & 
% \begin{sparkline}{10}
% \spark 0.03 0.00 0.03 0.14 0.06 0.14 0.06 0.49 0.08 0.49 0.08 0.49 0.11 0.49 0.11 0.68 0.14 0.68 0.14 0.43 0.17 0.43 0.17 0.35 0.20 0.35 0.20 0.08 0.22 0.08 0.22 0.05 0.25 0.05 0.25 0.00 /
% \end{sparkline}
%  &   2.0 & 1000.0 &   2.0 \\ 
% Elephant seals                      & Neon flying squid                   &   1.02e-14 & 
% \begin{sparkline}{10}
% \spark 0.05 0.00 0.05 0.46 0.07 0.46 0.07 0.62 0.10 0.62 0.10 0.51 0.13 0.51 0.13 0.51 0.16 0.51 0.16 0.30 0.18 0.30 0.18 0.05 0.21 0.05 0.21 0.14 0.24 0.14 0.24 0.03 0.26 0.03 0.26 0.08 0.29 0.08 0.29 0.00 /
% \end{sparkline}
%  &   2.0 & 1000.0 &   2.0 \\ 
% Elephant seals                      & Boreal clubhook squid               &   2.69e-16 & 
% \begin{sparkline}{10}
% \spark 0.04 0.00 0.04 0.14 0.06 0.14 0.06 0.30 0.08 0.30 0.08 0.65 0.10 0.65 0.10 0.54 0.13 0.54 0.13 0.49 0.15 0.49 0.15 0.30 0.17 0.30 0.17 0.11 0.19 0.11 0.19 0.03 0.22 0.03 0.22 0.03 0.24 0.03 0.24 0.03 0.26 0.03 0.26 0.05 0.28 0.05 0.28 0.00 0.31 0.00 0.31 0.00 0.33 0.00 0.33 0.03 0.35 0.03 0.35 0.00 0.37 0.00 0.37 0.00 0.40 0.00 0.40 0.00 0.42 0.00 0.42 0.00 0.44 0.00 0.44 0.00 0.47 0.00 0.47 0.03 0.49 0.03 0.49 0.00 /
% \end{sparkline}
%  &   2.0 & 1000.0 &   2.0 \\ 
% Elephant seals                      & Chinook,coho,steelhead              &   1.61e-16 & 
% \begin{sparkline}{10}
% \spark 0.05 0.00 0.05 0.19 0.07 0.19 0.07 0.38 0.09 0.38 0.09 0.68 0.11 0.68 0.11 0.49 0.13 0.49 0.13 0.43 0.16 0.43 0.16 0.24 0.18 0.24 0.18 0.19 0.20 0.19 0.20 0.00 0.22 0.00 0.22 0.05 0.24 0.05 0.24 0.03 0.26 0.03 0.26 0.03 0.28 0.03 0.28 0.00 /
% \end{sparkline}
%  &   2.0 & 1000.0 &   2.0 \\ 
% Elephant seals                      & Pomfret                             &   3.11e-15 & 
% \begin{sparkline}{10}
% \spark 0.04 0.00 0.04 0.27 0.07 0.27 0.07 0.43 0.09 0.43 0.09 0.59 0.12 0.59 0.12 0.68 0.14 0.68 0.14 0.22 0.17 0.22 0.17 0.22 0.19 0.22 0.19 0.11 0.22 0.11 0.22 0.11 0.24 0.11 0.24 0.03 0.27 0.03 0.27 0.03 0.30 0.03 0.30 0.00 0.32 0.00 0.32 0.03 0.35 0.03 0.35 0.00 /
% \end{sparkline}
%  &   2.0 & 1000.0 &   2.0 \\ 
% Elephant seals                      & Large gonatid squid                 &   6.87e-16 & 
% \begin{sparkline}{10}
% \spark 0.04 0.00 0.04 0.51 0.07 0.51 0.07 0.49 0.10 0.49 0.10 0.62 0.13 0.62 0.13 0.43 0.16 0.43 0.16 0.19 0.19 0.19 0.19 0.32 0.22 0.32 0.22 0.03 0.25 0.03 0.25 0.05 0.28 0.05 0.28 0.05 0.31 0.05 0.31 0.00 /
% \end{sparkline}
%  &   2.0 & 1000.0 &   2.0 \\}{1}
% 
% \epderivedlinktable{
% Elephant seals                      & Sockeye,Pink                        &   8.02e-16 & 
% \begin{sparkline}{10}
% \spark 0.05 0.00 0.05 0.19 0.07 0.19 0.07 0.43 0.09 0.43 0.09 0.62 0.11 0.62 0.11 0.54 0.14 0.54 0.14 0.38 0.16 0.38 0.16 0.16 0.18 0.16 0.18 0.14 0.20 0.14 0.20 0.14 0.22 0.14 0.22 0.11 0.24 0.11 0.24 0.00 /
% \end{sparkline}
%  &   2.0 & 1000.0 &   2.0 \\ 
% Elephant seals                      & Micronektonic squid                 &   2.38e-14 & 
% \begin{sparkline}{10}
% \spark 0.05 0.00 0.05 0.32 0.07 0.32 0.07 0.41 0.09 0.41 0.09 0.46 0.11 0.46 0.11 0.70 0.14 0.70 0.14 0.24 0.16 0.24 0.16 0.11 0.18 0.11 0.18 0.14 0.20 0.14 0.20 0.16 0.23 0.16 0.23 0.11 0.25 0.11 0.25 0.05 0.27 0.05 0.27 0.00 /
% \end{sparkline}
%  &   2.0 & 1000.0 &   2.0 \\ 
% Elephant seals                      & Pelagic forage fish                 &   9.99e-15 & 
% \begin{sparkline}{10}
% \spark 0.03 0.00 0.03 0.14 0.06 0.14 0.06 0.68 0.09 0.68 0.09 0.49 0.12 0.49 0.12 0.70 0.15 0.70 0.15 0.35 0.18 0.35 0.18 0.22 0.21 0.22 0.21 0.05 0.24 0.05 0.24 0.03 0.27 0.03 0.27 0.05 0.29 0.05 0.29 0.00 /
% \end{sparkline}
%  &   2.0 & 1000.0 &   2.0 \\ 
% Elephant seals                      & Saury                               &   6.75e-15 & 
% \begin{sparkline}{10}
% \spark 0.04 0.00 0.04 0.57 0.08 0.57 0.08 0.73 0.11 0.73 0.11 0.59 0.15 0.59 0.15 0.35 0.18 0.35 0.18 0.27 0.22 0.27 0.22 0.05 0.25 0.05 0.25 0.08 0.29 0.08 0.29 0.05 0.32 0.05 0.32 0.00 /
% \end{sparkline}
%  &   2.0 & 1000.0 &   2.0 \\ 
% Elephant seals                      & Chum salmon                         &   3.82e-16 & 
% \begin{sparkline}{10}
% \spark 0.05 0.00 0.05 0.32 0.08 0.32 0.08 0.49 0.10 0.49 0.10 0.65 0.12 0.65 0.12 0.41 0.14 0.41 0.14 0.38 0.16 0.38 0.16 0.22 0.18 0.22 0.18 0.11 0.21 0.11 0.21 0.05 0.23 0.05 0.23 0.08 0.25 0.08 0.25 0.00 /
% \end{sparkline}
%  &   2.0 & 1000.0 &   2.0 \\ 
% Seals,dolphins                      & Neon flying squid                   &   8.66e-13 & 
% \begin{sparkline}{10}
% \spark 0.05 0.00 0.05 0.49 0.08 0.49 0.08 0.73 0.11 0.73 0.11 0.59 0.14 0.59 0.14 0.27 0.17 0.27 0.17 0.46 0.20 0.46 0.20 0.03 0.23 0.03 0.23 0.11 0.26 0.11 0.26 0.03 0.29 0.03 0.29 0.00 /
% \end{sparkline}
%  &   2.0 & 1000.0 &   2.0 \\ 
% Seals,dolphins                      & Boreal clubhook squid               &   2.34e-14 & 
% \begin{sparkline}{10}
% \spark 0.03 0.00 0.03 0.24 0.06 0.24 0.06 0.65 0.09 0.65 0.09 0.59 0.12 0.59 0.12 0.38 0.15 0.38 0.15 0.32 0.18 0.32 0.18 0.19 0.21 0.19 0.21 0.16 0.23 0.16 0.23 0.08 0.26 0.08 0.26 0.03 0.29 0.03 0.29 0.03 0.32 0.03 0.32 0.00 0.35 0.00 0.35 0.03 0.38 0.03 0.38 0.00 /
% \end{sparkline}
%  &   2.0 & 1000.0 &   2.0 \\ 
% Seals,dolphins                      & Chinook,coho,steelhead              &   2.31e-14 & 
% \begin{sparkline}{10}
% \spark 0.05 0.00 0.05 0.22 0.07 0.22 0.07 0.38 0.09 0.38 0.09 0.59 0.11 0.59 0.11 0.57 0.14 0.57 0.14 0.41 0.16 0.41 0.16 0.24 0.18 0.24 0.18 0.14 0.20 0.14 0.20 0.08 0.22 0.08 0.22 0.05 0.24 0.05 0.24 0.00 0.26 0.00 0.26 0.00 0.29 0.00 0.29 0.03 0.31 0.03 0.31 0.00 /
% \end{sparkline}
%  &   2.0 & 1000.0 &   2.0 \\ 
% Seals,dolphins                      & Pomfret                             &   2.06e-13 & 
% \begin{sparkline}{10}
% \spark 0.04 0.00 0.04 0.27 0.07 0.27 0.07 0.35 0.09 0.35 0.09 0.70 0.11 0.70 0.11 0.57 0.14 0.57 0.14 0.30 0.16 0.30 0.16 0.24 0.19 0.24 0.19 0.03 0.21 0.03 0.21 0.11 0.24 0.11 0.24 0.14 0.26 0.14 0.26 0.00 /
% \end{sparkline}
%  &   2.0 & 1000.0 &   2.0 \\ 
% Seals,dolphins                      & Large gonatid squid                 &   5.28e-14 & 
% \begin{sparkline}{10}
% \spark 0.04 0.00 0.04 0.16 0.07 0.16 0.07 0.68 0.09 0.68 0.09 0.65 0.12 0.65 0.12 0.38 0.15 0.38 0.15 0.41 0.18 0.41 0.18 0.24 0.20 0.24 0.20 0.14 0.23 0.14 0.23 0.05 0.26 0.05 0.26 0.00 /
% \end{sparkline}
%  &   2.0 & 1000.0 &   2.0 \\ 
% Seals,dolphins                      & Sockeye,Pink                        &   1.13e-13 & 
% \begin{sparkline}{10}
% \spark 0.05 0.00 0.05 0.22 0.07 0.22 0.07 0.35 0.09 0.35 0.09 0.73 0.12 0.73 0.12 0.57 0.14 0.57 0.14 0.30 0.16 0.30 0.16 0.27 0.19 0.27 0.19 0.16 0.21 0.16 0.21 0.08 0.23 0.08 0.23 0.00 0.26 0.00 0.26 0.00 0.28 0.00 0.28 0.03 0.30 0.03 0.30 0.00 /
% \end{sparkline}
%  &   2.0 & 1000.0 &   2.0 \\ 
% Seals,dolphins                      & Micronektonic squid                 &   1.02e-12 & 
% \begin{sparkline}{10}
% \spark 0.03 0.00 0.03 0.19 0.06 0.19 0.06 0.78 0.09 0.78 0.09 0.59 0.12 0.59 0.12 0.41 0.15 0.41 0.15 0.30 0.18 0.30 0.18 0.16 0.21 0.16 0.21 0.16 0.25 0.16 0.25 0.05 0.28 0.05 0.28 0.03 0.31 0.03 0.31 0.00 0.34 0.00 0.34 0.00 0.37 0.00 0.37 0.00 0.40 0.00 0.40 0.03 0.43 0.03 0.43 0.00 /
% \end{sparkline}
%  &   2.0 & 1000.0 &   2.0 \\ 
% Seals,dolphins                      & Mesopelagic fish                    &   1.05e-12 & 
% \begin{sparkline}{10}
% \spark 0.03 0.00 0.03 0.27 0.06 0.27 0.06 0.68 0.09 0.68 0.09 0.65 0.12 0.65 0.12 0.46 0.15 0.46 0.15 0.16 0.18 0.16 0.18 0.24 0.21 0.24 0.21 0.16 0.24 0.16 0.24 0.03 0.27 0.03 0.27 0.00 0.30 0.00 0.30 0.03 0.33 0.03 0.33 0.03 0.36 0.03 0.36 0.00 /
% \end{sparkline}
%  &   2.0 & 1000.0 &   2.0 \\ 
% Seals,dolphins                      & Pelagic forage fish                 &   6.77e-13 & 
% \begin{sparkline}{10}
% \spark 0.03 0.00 0.03 0.22 0.06 0.22 0.06 0.38 0.08 0.38 0.08 0.68 0.11 0.68 0.11 0.49 0.14 0.49 0.14 0.35 0.16 0.35 0.16 0.27 0.19 0.27 0.19 0.16 0.21 0.16 0.21 0.03 0.24 0.03 0.24 0.08 0.27 0.08 0.27 0.03 0.29 0.03 0.29 0.03 0.32 0.03 0.32 0.00 /
% \end{sparkline}
%  &   2.0 & 1000.0 &   2.0 \\ 
% Seals,dolphins                      & Saury                               &   4.55e-13 & 
% \begin{sparkline}{10}
% \spark 0.04 0.00 0.04 0.46 0.07 0.46 0.07 0.84 0.10 0.84 0.10 0.35 0.14 0.35 0.14 0.46 0.17 0.46 0.17 0.38 0.20 0.38 0.20 0.08 0.23 0.08 0.23 0.03 0.27 0.03 0.27 0.05 0.30 0.05 0.30 0.03 0.33 0.03 0.33 0.03 0.36 0.03 0.36 0.00 /
% \end{sparkline}
%  &   2.0 & 1000.0 &   2.0 \\ 
% Seals,dolphins                      & Chum salmon                         &   5.51e-14 & 
% \begin{sparkline}{10}
% \spark 0.04 0.00 0.04 0.22 0.07 0.22 0.07 0.32 0.09 0.32 0.09 0.57 0.11 0.57 0.11 0.54 0.13 0.54 0.13 0.54 0.15 0.54 0.15 0.19 0.18 0.19 0.18 0.14 0.20 0.14 0.20 0.11 0.22 0.11 0.22 0.05 0.24 0.05 0.24 0.00 0.27 0.00 0.27 0.00 0.29 0.00 0.29 0.03 0.31 0.03 0.31 0.00 /
% \end{sparkline}
%  &   2.0 & 1000.0 &   2.0 \\ 
% Boreal clubhook squid               & Micronektonic squid                 &   1.10e-12 & 
% \begin{sparkline}{10}
% \spark 0.03 0.00 0.03 0.49 0.07 0.49 0.07 0.70 0.10 0.70 0.10 0.46 0.14 0.46 0.14 0.54 0.17 0.54 0.17 0.22 0.20 0.22 0.20 0.19 0.24 0.19 0.24 0.03 0.27 0.03 0.27 0.05 0.30 0.05 0.30 0.00 0.34 0.00 0.34 0.00 0.37 0.00 0.37 0.00 0.41 0.00 0.41 0.03 0.44 0.03 0.44 0.00 /
% \end{sparkline}
%  &   2.0 & 1000.0 &   2.0 \\ 
% Boreal clubhook squid               & Pelagic forage fish                 &   1.34e-14 & 
% \begin{sparkline}{10}
% \spark 0.02 0.00 0.02 0.78 0.06 0.78 0.06 0.76 0.11 0.76 0.11 0.46 0.15 0.46 0.15 0.27 0.19 0.27 0.19 0.11 0.24 0.11 0.24 0.14 0.28 0.14 0.28 0.05 0.32 0.05 0.32 0.03 0.36 0.03 0.36 0.03 0.41 0.03 0.41 0.05 0.45 0.05 0.45 0.00 0.49 0.00 0.49 0.00 0.54 0.00 0.54 0.00 0.58 0.00 0.58 0.00 0.62 0.00 0.62 0.00 0.67 0.00 0.67 0.03 0.71 0.03 0.71 0.00 /
% \end{sparkline}
%  &   2.0 & 1000.0 &   2.0 \\ 
% Fulmars                             & Micronektonic squid                 &   8.55e-14 & 
% \begin{sparkline}{10}
% \spark 0.06 0.00 0.06 0.30 0.08 0.30 0.08 0.70 0.10 0.70 0.10 0.57 0.13 0.57 0.13 0.41 0.15 0.41 0.15 0.41 0.17 0.41 0.17 0.14 0.19 0.14 0.19 0.11 0.22 0.11 0.22 0.00 0.24 0.00 0.24 0.03 0.26 0.03 0.26 0.03 0.29 0.03 0.29 0.00 0.31 0.00 0.31 0.03 0.33 0.03 0.33 0.00 /
% \end{sparkline}
%  &   2.0 & 1000.0 &   2.0 \\ 
% Fulmars                             & Pelagic forage fish                 &   3.88e-15 & 
% \begin{sparkline}{10}
% \spark 0.03 0.00 0.03 0.54 0.07 0.54 0.07 0.59 0.10 0.59 0.10 0.57 0.13 0.57 0.13 0.38 0.16 0.38 0.16 0.19 0.20 0.19 0.20 0.19 0.23 0.19 0.23 0.08 0.26 0.08 0.26 0.08 0.30 0.08 0.30 0.00 0.33 0.00 0.33 0.05 0.36 0.05 0.36 0.03 0.39 0.03 0.39 0.00 /
% \end{sparkline}
%  &   2.0 & 1000.0 &   2.0 \\ 
% Chinook,coho,steelhead              & Micronektonic squid                 &   4.05e-13 & 
% \begin{sparkline}{10}
% \spark 0.04 0.00 0.04 0.24 0.07 0.24 0.07 0.78 0.10 0.78 0.10 0.65 0.13 0.65 0.13 0.35 0.16 0.35 0.16 0.43 0.19 0.43 0.19 0.08 0.22 0.08 0.22 0.05 0.25 0.05 0.25 0.05 0.28 0.05 0.28 0.00 0.31 0.00 0.31 0.00 0.34 0.00 0.34 0.00 0.37 0.00 0.37 0.03 0.40 0.03 0.40 0.03 0.43 0.03 0.43 0.00 /
% \end{sparkline}
%  &   2.0 & 1000.0 &   2.0 \\ 
% Chinook,coho,steelhead              & Mesopelagic fish                    &   7.61e-13 & 
% \begin{sparkline}{10}
% \spark 0.04 0.00 0.04 0.32 0.07 0.32 0.07 0.32 0.09 0.32 0.09 0.57 0.11 0.57 0.11 0.57 0.13 0.57 0.13 0.43 0.16 0.43 0.16 0.11 0.18 0.11 0.18 0.14 0.20 0.14 0.20 0.05 0.22 0.05 0.22 0.14 0.25 0.14 0.25 0.00 0.27 0.00 0.27 0.00 0.29 0.00 0.29 0.05 0.31 0.05 0.31 0.00 /
% \end{sparkline}
%  &   2.0 & 1000.0 &   2.0 \\ 
% Chinook,coho,steelhead              & Pelagic forage fish                 &   7.49e-13 & 
% \begin{sparkline}{10}
% \spark 0.05 0.00 0.05 0.49 0.08 0.49 0.08 0.70 0.11 0.70 0.11 0.62 0.14 0.62 0.14 0.38 0.17 0.38 0.17 0.32 0.20 0.32 0.20 0.14 0.23 0.14 0.23 0.00 0.26 0.00 0.26 0.00 0.29 0.00 0.29 0.03 0.32 0.03 0.32 0.00 0.36 0.00 0.36 0.00 0.39 0.00 0.39 0.03 0.42 0.03 0.42 0.00 /
% \end{sparkline}
%  &   2.0 & 1000.0 &   2.0 \\ 
% Chinook,coho,steelhead              & Mesozooplankton                     &   1.09e-13 & 
% \begin{sparkline}{10}
% \spark 0.05 0.00 0.05 0.24 0.07 0.24 0.07 0.54 0.10 0.54 0.10 0.57 0.12 0.57 0.12 0.62 0.14 0.62 0.14 0.27 0.17 0.27 0.17 0.24 0.19 0.24 0.19 0.08 0.21 0.08 0.21 0.05 0.24 0.05 0.24 0.05 0.26 0.05 0.26 0.00 0.29 0.00 0.29 0.03 0.31 0.03 0.31 0.00 /
% \end{sparkline}
%  &   2.0 & 1000.0 &   2.0 \\ 
% Chinook,coho,steelhead              & Gelatinous zooplankton              &   2.86e-16 & 
% \begin{sparkline}{10}
% \spark 0.04 0.00 0.04 0.38 0.07 0.38 0.07 0.84 0.10 0.84 0.10 0.54 0.14 0.54 0.14 0.49 0.17 0.49 0.17 0.19 0.20 0.19 0.20 0.14 0.23 0.14 0.23 0.05 0.27 0.05 0.27 0.05 0.30 0.05 0.30 0.03 0.33 0.03 0.33 0.00 /
% \end{sparkline}
%  &   2.0 & 1000.0 &   2.0 \\ 
% Chinook,coho,steelhead              & Copepods                            &   5.65e-15 & 
% \begin{sparkline}{10}
% \spark 0.05 0.00 0.05 0.24 0.07 0.24 0.07 0.30 0.09 0.30 0.09 0.41 0.11 0.41 0.11 0.57 0.13 0.57 0.13 0.54 0.15 0.54 0.15 0.32 0.17 0.32 0.17 0.08 0.19 0.08 0.19 0.16 0.21 0.16 0.21 0.03 0.23 0.03 0.23 0.00 0.25 0.00 0.25 0.03 0.27 0.03 0.27 0.03 0.29 0.03 0.29 0.00 /
% \end{sparkline}
%  &   2.0 & 1000.0 &   2.0 \\ 
% Skuas,Jaegers                       & Pelagic forage fish                 &   5.31e-14 & 
% \begin{sparkline}{10}
% \spark 0.04 0.00 0.04 0.14 0.06 0.14 0.06 0.38 0.09 0.38 0.09 0.59 0.11 0.59 0.11 0.70 0.13 0.70 0.13 0.32 0.15 0.32 0.15 0.24 0.18 0.24 0.18 0.08 0.20 0.08 0.20 0.03 0.22 0.03 0.22 0.14 0.25 0.14 0.25 0.08 0.27 0.08 0.27 0.00 /
% \end{sparkline}
%  &   2.0 & 1000.0 &   2.0 \\ 
% Skuas,Jaegers                       & Saury                               &   5.20e-14 & 
% \begin{sparkline}{10}
% \spark 0.04 0.00 0.04 0.11 0.06 0.11 0.06 0.24 0.08 0.24 0.08 0.30 0.10 0.30 0.10 0.70 0.11 0.70 0.11 0.54 0.13 0.54 0.13 0.27 0.15 0.27 0.15 0.11 0.17 0.11 0.17 0.22 0.19 0.22 0.19 0.11 0.21 0.11 0.21 0.03 0.23 0.03 0.23 0.00 0.25 0.00 0.25 0.00 0.27 0.00 0.27 0.03 0.29 0.03 0.29 0.00 0.31 0.00 0.31 0.00 0.33 0.00 0.33 0.05 0.35 0.05 0.35 0.00 /
% \end{sparkline}
%  &   2.0 & 1000.0 &   2.0 \\ 
% Pomfret                             & Micronektonic squid                 &   7.85e-12 & 
% \begin{sparkline}{10}
% \spark 0.03 0.00 0.03 0.32 0.06 0.32 0.06 0.41 0.09 0.41 0.09 0.76 0.12 0.76 0.12 0.41 0.14 0.41 0.14 0.38 0.17 0.38 0.17 0.19 0.20 0.19 0.20 0.08 0.22 0.08 0.22 0.05 0.25 0.05 0.25 0.00 0.28 0.00 0.28 0.05 0.31 0.05 0.31 0.03 0.33 0.03 0.33 0.00 0.36 0.00 0.36 0.00 0.39 0.00 0.39 0.00 0.41 0.00 0.41 0.00 0.44 0.00 0.44 0.00 0.47 0.00 0.47 0.03 0.49 0.03 0.49 0.00 /
% \end{sparkline}
%  &   2.0 & 1000.0 &   2.0 \\ 
% Pomfret                             & Mesopelagic fish                    &   8.35e-13 & 
% \begin{sparkline}{10}
% \spark 0.02 0.00 0.02 0.35 0.06 0.35 0.06 0.62 0.09 0.62 0.09 0.76 0.13 0.76 0.13 0.30 0.16 0.30 0.16 0.30 0.19 0.30 0.19 0.16 0.23 0.16 0.23 0.08 0.26 0.08 0.26 0.08 0.30 0.08 0.30 0.03 0.33 0.03 0.33 0.00 0.36 0.00 0.36 0.00 0.40 0.00 0.40 0.00 0.43 0.00 0.43 0.00 0.47 0.00 0.47 0.00 0.50 0.00 0.50 0.00 0.54 0.00 0.54 0.00 0.57 0.00 0.57 0.03 0.60 0.03 0.60 0.00 /
% \end{sparkline}
%  &   2.0 & 1000.0 &   2.0 \\ 
% Pomfret                             & Saury                               &   3.99e-13 & 
% \begin{sparkline}{10}
% \spark 0.02 0.00 0.02 0.27 0.05 0.27 0.05 0.68 0.08 0.68 0.08 0.57 0.12 0.57 0.12 0.49 0.15 0.49 0.15 0.22 0.18 0.22 0.18 0.24 0.22 0.24 0.22 0.11 0.25 0.11 0.25 0.03 0.28 0.03 0.28 0.05 0.32 0.05 0.32 0.00 0.35 0.00 0.35 0.00 0.38 0.00 0.38 0.00 0.42 0.00 0.42 0.03 0.45 0.03 0.45 0.00 0.48 0.00 0.48 0.00 0.52 0.00 0.52 0.00 0.55 0.00 0.55 0.03 0.58 0.03 0.58 0.00 /
% \end{sparkline}
%  &   2.0 & 1000.0 &   2.0 \\ 
% Pomfret                             & Chaetognaths                        &   1.06e-13 & 
% \begin{sparkline}{10}
% \spark 0.02 0.00 0.02 0.22 0.05 0.22 0.05 0.65 0.08 0.65 0.08 0.68 0.12 0.68 0.12 0.43 0.15 0.43 0.15 0.38 0.18 0.38 0.18 0.11 0.22 0.11 0.22 0.11 0.25 0.11 0.25 0.03 0.28 0.03 0.28 0.00 0.31 0.00 0.31 0.00 0.35 0.00 0.35 0.03 0.38 0.03 0.38 0.03 0.41 0.03 0.41 0.03 0.44 0.03 0.44 0.00 0.48 0.00 0.48 0.03 0.51 0.03 0.51 0.00 /
% \end{sparkline}
%  &   2.0 & 1000.0 &   2.0 \\ 
% Pomfret                             & Predatory zooplankton               &   1.04e-13 & 
% \begin{sparkline}{10}
% \spark 0.03 0.00 0.03 0.38 0.06 0.38 0.06 0.65 0.09 0.65 0.09 0.62 0.12 0.62 0.12 0.41 0.16 0.41 0.16 0.27 0.19 0.27 0.19 0.11 0.22 0.11 0.22 0.14 0.25 0.14 0.25 0.03 0.29 0.03 0.29 0.03 0.32 0.03 0.32 0.00 0.35 0.00 0.35 0.03 0.38 0.03 0.38 0.03 0.42 0.03 0.42 0.03 0.45 0.03 0.45 0.00 /
% \end{sparkline}
%  &   2.0 & 1000.0 &   2.0 \\ 
% Pomfret                             & Sergestid shrimp                    &   1.04e-13 & 
% \begin{sparkline}{10}
% \spark 0.02 0.00 0.02 0.35 0.05 0.35 0.05 0.81 0.09 0.81 0.09 0.57 0.12 0.57 0.12 0.30 0.16 0.30 0.16 0.27 0.20 0.27 0.20 0.14 0.23 0.14 0.23 0.08 0.27 0.08 0.27 0.08 0.31 0.08 0.31 0.05 0.34 0.05 0.34 0.00 0.38 0.00 0.38 0.03 0.42 0.03 0.42 0.00 0.45 0.00 0.45 0.00 0.49 0.00 0.49 0.03 0.53 0.03 0.53 0.00 /
% \end{sparkline}
%  &   2.0 & 1000.0 &   2.0 \\ 
% Pomfret                             & Mesozooplankton                     &   9.59e-13 & 
% \begin{sparkline}{10}
% \spark 0.02 0.00 0.02 0.38 0.05 0.38 0.05 0.46 0.08 0.46 0.08 0.68 0.11 0.68 0.11 0.35 0.14 0.35 0.14 0.35 0.17 0.35 0.17 0.19 0.20 0.19 0.20 0.03 0.23 0.03 0.23 0.11 0.26 0.11 0.26 0.05 0.29 0.05 0.29 0.00 0.32 0.00 0.32 0.05 0.35 0.05 0.35 0.00 0.38 0.00 0.38 0.05 0.41 0.05 0.41 0.00 /
% \end{sparkline}
%  &   2.0 & 1000.0 &   2.0 \\ 
% Pomfret                             & Copepods                            &   1.06e-13 & 
% \begin{sparkline}{10}
% \spark 0.03 0.00 0.03 0.38 0.06 0.38 0.06 0.51 0.09 0.51 0.09 0.57 0.12 0.57 0.12 0.54 0.14 0.54 0.14 0.22 0.17 0.22 0.17 0.19 0.20 0.19 0.20 0.08 0.23 0.08 0.23 0.05 0.25 0.05 0.25 0.08 0.28 0.08 0.28 0.00 0.31 0.00 0.31 0.05 0.34 0.05 0.34 0.00 0.37 0.00 0.37 0.00 0.39 0.00 0.39 0.00 0.42 0.00 0.42 0.00 0.45 0.00 0.45 0.00 0.48 0.00 0.48 0.03 0.50 0.03 0.50 0.00 /
% \end{sparkline}
%  &   2.0 & 1000.0 &   2.0 \\ 
% Puffins,Shearwaters,Storm Petrels   & Micronektonic squid                 &   2.06e-13 & 
% \begin{sparkline}{10}
% \spark 0.04 0.00 0.04 0.59 0.08 0.59 0.08 0.73 0.11 0.73 0.11 0.54 0.14 0.54 0.14 0.35 0.18 0.35 0.18 0.30 0.21 0.30 0.21 0.11 0.25 0.11 0.25 0.00 0.28 0.00 0.28 0.08 0.31 0.08 0.31 0.00 /
% \end{sparkline}
%  &   2.0 & 1000.0 &   2.0 \\ 
% Puffins,Shearwaters,Storm Petrels   & Pelagic forage fish                 &   1.78e-13 & 
% \begin{sparkline}{10}
% \spark 0.04 0.00 0.04 0.22 0.06 0.22 0.06 0.38 0.08 0.38 0.08 0.68 0.11 0.68 0.11 0.51 0.13 0.51 0.13 0.30 0.15 0.30 0.15 0.22 0.17 0.22 0.17 0.14 0.20 0.14 0.20 0.00 0.22 0.00 0.22 0.08 0.24 0.08 0.24 0.08 0.27 0.08 0.27 0.05 0.29 0.05 0.29 0.03 0.31 0.03 0.31 0.03 0.33 0.03 0.33 0.00 /
% \end{sparkline}
%  &   2.0 & 1000.0 &   2.0 \\ 
% Puffins,Shearwaters,Storm Petrels   & Saury                               &   1.65e-13 & 
% \begin{sparkline}{10}
% \spark 0.04 0.00 0.04 0.16 0.06 0.16 0.06 0.51 0.09 0.51 0.09 0.62 0.11 0.62 0.11 0.51 0.14 0.51 0.14 0.49 0.16 0.49 0.16 0.22 0.19 0.22 0.19 0.03 0.21 0.03 0.21 0.03 0.24 0.03 0.24 0.08 0.27 0.08 0.27 0.00 0.29 0.00 0.29 0.05 0.32 0.05 0.32 0.00 /
% \end{sparkline}
%  &   2.0 & 1000.0 &   2.0 \\ 
% Puffins,Shearwaters,Storm Petrels   & Mesozooplankton                     &   5.68e-14 & 
% \begin{sparkline}{10}
% \spark 0.04 0.00 0.04 0.14 0.07 0.14 0.07 0.59 0.09 0.59 0.09 0.78 0.12 0.78 0.12 0.43 0.15 0.43 0.15 0.35 0.17 0.35 0.17 0.16 0.20 0.16 0.20 0.08 0.23 0.08 0.23 0.14 0.25 0.14 0.25 0.03 0.28 0.03 0.28 0.00 /
% \end{sparkline}
%  &   2.0 & 1000.0 &   2.0 \\ 
% Puffins,Shearwaters,Storm Petrels   & Copepods                            &   4.50e-14 & 
% \begin{sparkline}{10}
% \spark 0.05 0.00 0.05 0.27 0.07 0.27 0.07 0.35 0.09 0.35 0.09 0.65 0.11 0.65 0.11 0.51 0.13 0.51 0.13 0.35 0.15 0.35 0.15 0.19 0.17 0.19 0.17 0.14 0.19 0.14 0.19 0.05 0.21 0.05 0.21 0.14 0.23 0.14 0.23 0.00 0.25 0.00 0.25 0.05 0.27 0.05 0.27 0.00 /
% \end{sparkline}
%  &   2.0 & 1000.0 &   2.0 \\ 
% Kittiwakes                          & Pelagic forage fish                 &   3.02e-14 & 
% \begin{sparkline}{10}
% \spark 0.04 0.00 0.04 0.32 0.07 0.32 0.07 0.68 0.10 0.68 0.10 0.65 0.13 0.65 0.13 0.43 0.16 0.43 0.16 0.30 0.19 0.30 0.19 0.19 0.22 0.19 0.22 0.05 0.25 0.05 0.25 0.03 0.28 0.03 0.28 0.05 0.31 0.05 0.31 0.00 /
% \end{sparkline}
%  &   2.0 & 1000.0 &   2.0 \\ 
% Kittiwakes                          & Saury                               &   3.05e-14 & 
% \begin{sparkline}{10}
% \spark 0.05 0.00 0.05 0.51 0.08 0.51 0.08 0.51 0.11 0.51 0.11 0.81 0.13 0.81 0.13 0.41 0.16 0.41 0.16 0.22 0.19 0.22 0.19 0.05 0.22 0.05 0.22 0.05 0.25 0.05 0.25 0.05 0.28 0.05 0.28 0.08 0.30 0.08 0.30 0.00 /
% \end{sparkline}
%  &   2.0 & 1000.0 &   2.0 \\ 
% Kittiwakes                          & Mesozooplankton                     &   8.43e-15 & 
% \begin{sparkline}{10}
% \spark 0.04 0.00 0.04 0.22 0.06 0.22 0.06 0.62 0.09 0.62 0.09 0.51 0.12 0.51 0.12 0.62 0.14 0.62 0.14 0.27 0.17 0.27 0.17 0.19 0.20 0.19 0.20 0.14 0.22 0.14 0.22 0.05 0.25 0.05 0.25 0.00 0.28 0.00 0.28 0.03 0.30 0.03 0.30 0.05 0.33 0.05 0.33 0.00 /
% \end{sparkline}
%  &   2.0 & 1000.0 &   2.0 \\ 
% Kittiwakes                          & Copepods                            &   6.57e-15 & 
% \begin{sparkline}{10}
% \spark 0.05 0.00 0.05 0.43 0.08 0.43 0.08 0.73 0.10 0.73 0.10 0.57 0.13 0.57 0.13 0.41 0.16 0.41 0.16 0.16 0.18 0.16 0.18 0.24 0.21 0.24 0.21 0.03 0.23 0.03 0.23 0.00 0.26 0.00 0.26 0.08 0.28 0.08 0.28 0.03 0.31 0.03 0.31 0.03 0.34 0.03 0.34 0.00 /
% \end{sparkline}
%  &   2.0 & 1000.0 &   2.0 \\ 
% Large gonatid squid                 & Micronektonic squid                 &   8.74e-13 & 
% \begin{sparkline}{10}
% \spark 0.03 0.00 0.03 0.76 0.07 0.76 0.07 0.81 0.10 0.81 0.10 0.41 0.14 0.41 0.14 0.30 0.18 0.30 0.18 0.08 0.22 0.08 0.22 0.11 0.25 0.11 0.25 0.08 0.29 0.08 0.29 0.05 0.33 0.05 0.33 0.00 0.37 0.00 0.37 0.00 0.40 0.00 0.40 0.00 0.44 0.00 0.44 0.03 0.48 0.03 0.48 0.03 0.52 0.03 0.52 0.03 0.55 0.03 0.55 0.03 0.59 0.03 0.59 0.00 /
% \end{sparkline}
%  &   2.0 & 1000.0 &   2.0 \\ 
% Large gonatid squid                 & Pelagic forage fish                 &   2.83e-14 & 
% \begin{sparkline}{10}
% \spark 0.02 0.00 0.02 0.65 0.06 0.65 0.06 0.76 0.09 0.76 0.09 0.41 0.13 0.41 0.13 0.35 0.17 0.35 0.17 0.16 0.20 0.16 0.20 0.08 0.24 0.08 0.24 0.08 0.27 0.08 0.27 0.05 0.31 0.05 0.31 0.03 0.34 0.03 0.34 0.05 0.38 0.05 0.38 0.00 0.42 0.00 0.42 0.00 0.45 0.00 0.45 0.00 0.49 0.00 0.49 0.03 0.52 0.03 0.52 0.05 0.56 0.05 0.56 0.00 /
% \end{sparkline}
%  &   2.0 & 1000.0 &   2.0 \\ 
% Large gonatid squid                 & Chaetognaths                        &   1.25e-13 & 
% \begin{sparkline}{10}
% \spark 0.03 0.00 0.03 0.70 0.07 0.70 0.07 0.70 0.11 0.70 0.11 0.49 0.15 0.49 0.15 0.35 0.19 0.35 0.19 0.22 0.23 0.22 0.23 0.08 0.27 0.08 0.27 0.00 0.31 0.00 0.31 0.08 0.35 0.08 0.35 0.03 0.39 0.03 0.39 0.03 0.43 0.03 0.43 0.00 0.47 0.00 0.47 0.00 0.51 0.00 0.51 0.03 0.55 0.03 0.55 0.00 /
% \end{sparkline}
%  &   2.0 & 1000.0 &   2.0 \\ 
% Large gonatid squid                 & Predatory zooplankton               &   1.04e-13 & 
% \begin{sparkline}{10}
% \spark 0.02 0.00 0.02 0.65 0.06 0.65 0.06 0.84 0.10 0.84 0.10 0.46 0.14 0.46 0.14 0.24 0.18 0.24 0.18 0.22 0.22 0.22 0.22 0.05 0.26 0.05 0.26 0.08 0.30 0.08 0.30 0.08 0.34 0.08 0.34 0.03 0.38 0.03 0.38 0.00 0.42 0.00 0.42 0.00 0.46 0.00 0.46 0.00 0.50 0.00 0.50 0.00 0.54 0.00 0.54 0.00 0.58 0.00 0.58 0.00 0.62 0.00 0.62 0.00 0.66 0.00 0.66 0.03 0.70 0.03 0.70 0.00 0.74 0.00 0.74 0.00 0.78 0.00 0.78 0.00 0.82 0.00 0.82 0.00 0.86 0.00 0.86 0.00 0.90 0.00 0.90 0.00 0.94 0.00 0.94 0.03 0.98 0.03 0.98 0.00 /
% \end{sparkline}
%  &   2.0 & 1000.0 &   2.0 \\ 
% Large gonatid squid                 & Sergestid shrimp                    &   9.54e-14 & 
% \begin{sparkline}{10}
% \spark 0.01 0.00 0.01 0.54 0.06 0.54 0.06 0.92 0.10 0.92 0.10 0.49 0.14 0.49 0.14 0.19 0.18 0.19 0.18 0.22 0.22 0.22 0.22 0.14 0.26 0.14 0.26 0.05 0.30 0.05 0.30 0.05 0.34 0.05 0.34 0.05 0.39 0.05 0.39 0.00 0.43 0.00 0.43 0.03 0.47 0.03 0.47 0.00 0.51 0.00 0.51 0.00 0.55 0.00 0.55 0.00 0.59 0.00 0.59 0.00 0.63 0.00 0.63 0.00 0.67 0.00 0.67 0.03 0.72 0.03 0.72 0.00 /
% \end{sparkline}
%  &   2.0 & 1000.0 &   2.0 \\ 
% Large gonatid squid                 & Mesozooplankton                     &   8.34e-13 & 
% \begin{sparkline}{10}
% \spark 0.03 0.00 0.03 0.43 0.06 0.43 0.06 0.78 0.09 0.78 0.09 0.57 0.12 0.57 0.12 0.27 0.16 0.27 0.16 0.16 0.19 0.16 0.19 0.19 0.22 0.19 0.22 0.08 0.25 0.08 0.25 0.03 0.29 0.03 0.29 0.08 0.32 0.08 0.32 0.03 0.35 0.03 0.35 0.05 0.39 0.05 0.39 0.00 0.42 0.00 0.42 0.03 0.45 0.03 0.45 0.00 /
% \end{sparkline}
%  &   2.0 & 1000.0 &   2.0 \\ 
% Large gonatid squid                 & Copepods                            &   6.58e-13 & 
% \begin{sparkline}{10}
% \spark 0.02 0.00 0.02 0.54 0.06 0.54 0.06 1.00 0.11 1.00 0.11 0.38 0.15 0.38 0.15 0.41 0.19 0.41 0.19 0.19 0.23 0.19 0.23 0.00 0.28 0.00 0.28 0.03 0.32 0.03 0.32 0.05 0.36 0.05 0.36 0.05 0.40 0.05 0.40 0.03 0.45 0.03 0.45 0.03 0.49 0.03 0.49 0.00 /
% \end{sparkline}
%  &   2.0 & 1000.0 &   2.0 \\}{2}
% 
% \epderivedlinktable{
% Sockeye,Pink                        & Micronektonic squid                 &   1.16e-12 & 
% \begin{sparkline}{10}
% \spark 0.05 0.00 0.05 0.30 0.07 0.30 0.07 0.41 0.09 0.41 0.09 0.59 0.11 0.59 0.11 0.41 0.14 0.41 0.14 0.51 0.16 0.51 0.16 0.16 0.18 0.16 0.18 0.14 0.21 0.14 0.21 0.11 0.23 0.11 0.23 0.00 0.25 0.00 0.25 0.00 0.27 0.00 0.27 0.00 0.30 0.00 0.30 0.08 0.32 0.08 0.32 0.00 /
% \end{sparkline}
%  &   2.0 & 1000.0 &   2.0 \\ 
% Sockeye,Pink                        & Mesopelagic fish                    &   1.73e-12 & 
% \begin{sparkline}{10}
% \spark 0.04 0.00 0.04 0.54 0.07 0.54 0.07 0.76 0.11 0.76 0.11 0.54 0.14 0.54 0.14 0.30 0.17 0.30 0.17 0.19 0.20 0.19 0.20 0.24 0.24 0.24 0.24 0.03 0.27 0.03 0.27 0.08 0.30 0.08 0.30 0.00 0.34 0.00 0.34 0.00 0.37 0.00 0.37 0.00 0.40 0.00 0.40 0.03 0.43 0.03 0.43 0.00 /
% \end{sparkline}
%  &   2.0 & 1000.0 &   2.0 \\ 
% Sockeye,Pink                        & Pelagic forage fish                 &   1.60e-12 & 
% \begin{sparkline}{10}
% \spark 0.06 0.00 0.06 0.54 0.08 0.54 0.08 0.76 0.11 0.76 0.11 0.51 0.14 0.51 0.14 0.32 0.16 0.32 0.16 0.24 0.19 0.24 0.19 0.08 0.22 0.08 0.22 0.11 0.24 0.11 0.24 0.08 0.27 0.08 0.27 0.05 0.30 0.05 0.30 0.00 /
% \end{sparkline}
%  &   2.0 & 1000.0 &   2.0 \\ 
% Sockeye,Pink                        & Predatory zooplankton               &   2.09e-14 & 
% \begin{sparkline}{10}
% \spark 0.03 0.00 0.03 0.16 0.06 0.16 0.06 0.73 0.09 0.73 0.09 0.51 0.12 0.51 0.12 0.54 0.15 0.54 0.15 0.41 0.18 0.41 0.18 0.08 0.21 0.08 0.21 0.19 0.24 0.19 0.24 0.03 0.27 0.03 0.27 0.03 0.30 0.03 0.30 0.00 0.33 0.00 0.33 0.03 0.36 0.03 0.36 0.00 /
% \end{sparkline}
%  &   2.0 & 1000.0 &   2.0 \\ 
% Sockeye,Pink                        & Mesozooplankton                     &   1.01e-11 & 
% \begin{sparkline}{10}
% \spark 0.06 0.00 0.06 0.24 0.08 0.24 0.08 0.38 0.10 0.38 0.10 0.68 0.12 0.68 0.12 0.57 0.14 0.57 0.14 0.30 0.16 0.30 0.16 0.38 0.18 0.38 0.18 0.08 0.19 0.08 0.19 0.05 0.21 0.05 0.21 0.00 0.23 0.00 0.23 0.00 0.25 0.00 0.25 0.00 0.27 0.00 0.27 0.00 0.29 0.00 0.29 0.03 0.31 0.03 0.31 0.00 /
% \end{sparkline}
%  &   2.0 & 1000.0 &   2.0 \\ 
% Sockeye,Pink                        & Gelatinous zooplankton              &   4.17e-13 & 
% \begin{sparkline}{10}
% \spark 0.04 0.00 0.04 0.22 0.06 0.22 0.06 0.41 0.09 0.41 0.09 0.73 0.11 0.73 0.11 0.46 0.14 0.46 0.14 0.30 0.16 0.30 0.16 0.27 0.18 0.27 0.18 0.14 0.21 0.14 0.21 0.08 0.23 0.08 0.23 0.00 0.25 0.00 0.25 0.05 0.28 0.05 0.28 0.00 0.30 0.00 0.30 0.00 0.32 0.00 0.32 0.00 0.35 0.00 0.35 0.05 0.37 0.05 0.37 0.00 /
% \end{sparkline}
%  &   2.0 & 1000.0 &   2.0 \\ 
% Sockeye,Pink                        & Copepods                            &   4.87e-13 & 
% \begin{sparkline}{10}
% \spark 0.05 0.00 0.05 0.22 0.07 0.22 0.07 0.41 0.09 0.41 0.09 0.73 0.12 0.73 0.12 0.46 0.14 0.46 0.14 0.35 0.16 0.35 0.16 0.27 0.18 0.27 0.18 0.11 0.20 0.11 0.20 0.05 0.23 0.05 0.23 0.03 0.25 0.03 0.25 0.08 0.27 0.08 0.27 0.00 /
% \end{sparkline}
%  &   2.0 & 1000.0 &   2.0 \\ 
% Fin,sei whales                      & Neon flying squid                   &   4.63e-14 & 
% \begin{sparkline}{10}
% \spark 0.04 0.00 0.04 0.49 0.07 0.49 0.07 0.54 0.10 0.54 0.10 0.46 0.13 0.46 0.13 0.68 0.15 0.68 0.15 0.08 0.18 0.08 0.18 0.19 0.21 0.19 0.21 0.08 0.24 0.08 0.24 0.05 0.26 0.05 0.26 0.08 0.29 0.08 0.29 0.05 0.32 0.05 0.32 0.00 /
% \end{sparkline}
%  &   2.0 & 1000.0 &   2.0 \\ 
% Fin,sei whales                      & Boreal clubhook squid               &   1.20e-15 & 
% \begin{sparkline}{10}
% \spark 0.03 0.00 0.03 0.41 0.06 0.41 0.06 0.59 0.10 0.59 0.10 0.62 0.13 0.62 0.13 0.46 0.16 0.46 0.16 0.30 0.20 0.30 0.20 0.14 0.23 0.14 0.23 0.14 0.26 0.14 0.26 0.05 0.30 0.05 0.30 0.00 /
% \end{sparkline}
%  &   2.0 & 1000.0 &   2.0 \\ 
% Fin,sei whales                      & Chinook,coho,steelhead              &   6.25e-16 & 
% \begin{sparkline}{10}
% \spark 0.05 0.00 0.05 0.30 0.08 0.30 0.08 0.43 0.10 0.43 0.10 0.70 0.12 0.70 0.12 0.49 0.14 0.49 0.14 0.35 0.17 0.35 0.17 0.14 0.19 0.14 0.19 0.14 0.21 0.14 0.21 0.16 0.23 0.16 0.23 0.00 /
% \end{sparkline}
%  &   2.0 & 1000.0 &   2.0 \\ 
% Fin,sei whales                      & Pomfret                             &   5.86e-15 & 
% \begin{sparkline}{10}
% \spark 0.05 0.00 0.05 0.32 0.07 0.32 0.07 0.65 0.10 0.65 0.10 0.59 0.13 0.59 0.13 0.46 0.16 0.46 0.16 0.35 0.18 0.35 0.18 0.27 0.21 0.27 0.21 0.00 0.24 0.00 0.24 0.03 0.26 0.03 0.26 0.00 0.29 0.00 0.29 0.03 0.32 0.03 0.32 0.00 /
% \end{sparkline}
%  &   2.0 & 1000.0 &   2.0 \\ 
% Fin,sei whales                      & Large gonatid squid                 &   2.98e-15 & 
% \begin{sparkline}{10}
% \spark 0.04 0.00 0.04 0.32 0.07 0.32 0.07 0.62 0.09 0.62 0.09 0.62 0.12 0.62 0.12 0.41 0.15 0.41 0.15 0.38 0.17 0.38 0.17 0.11 0.20 0.11 0.20 0.05 0.23 0.05 0.23 0.05 0.25 0.05 0.25 0.03 0.28 0.03 0.28 0.00 0.31 0.00 0.31 0.05 0.33 0.05 0.33 0.00 0.36 0.00 0.36 0.03 0.39 0.03 0.39 0.00 0.41 0.00 0.41 0.00 0.44 0.00 0.44 0.03 0.47 0.03 0.47 0.00 /
% \end{sparkline}
%  &   2.0 & 1000.0 &   2.0 \\ 
% Fin,sei whales                      & Sockeye,Pink                        &   3.04e-15 & 
% \begin{sparkline}{10}
% \spark 0.05 0.00 0.05 0.14 0.07 0.14 0.07 0.38 0.09 0.38 0.09 0.62 0.11 0.62 0.11 0.51 0.13 0.51 0.13 0.46 0.15 0.46 0.15 0.27 0.17 0.27 0.17 0.05 0.19 0.05 0.19 0.14 0.21 0.14 0.21 0.14 0.24 0.14 0.24 0.00 /
% \end{sparkline}
%  &   2.0 & 1000.0 &   2.0 \\ 
% Fin,sei whales                      & Micronektonic squid                 &   5.00e-14 & 
% \begin{sparkline}{10}
% \spark 0.04 0.00 0.04 0.41 0.07 0.41 0.07 0.54 0.10 0.54 0.10 0.70 0.13 0.70 0.13 0.35 0.16 0.35 0.16 0.41 0.20 0.41 0.20 0.16 0.23 0.16 0.23 0.08 0.26 0.08 0.26 0.03 0.29 0.03 0.29 0.03 0.32 0.03 0.32 0.00 /
% \end{sparkline}
%  &   2.0 & 1000.0 &   2.0 \\ 
% Fin,sei whales                      & Mesopelagic fish                    &   1.22e-13 & 
% \begin{sparkline}{10}
% \spark 0.03 0.00 0.03 0.27 0.06 0.27 0.06 0.62 0.09 0.62 0.09 0.62 0.12 0.62 0.12 0.51 0.15 0.51 0.15 0.22 0.18 0.22 0.18 0.22 0.21 0.22 0.21 0.03 0.24 0.03 0.24 0.11 0.26 0.11 0.26 0.05 0.29 0.05 0.29 0.00 0.32 0.00 0.32 0.00 0.35 0.00 0.35 0.00 0.38 0.00 0.38 0.03 0.41 0.03 0.41 0.00 0.44 0.00 0.44 0.00 0.47 0.00 0.47 0.03 0.50 0.03 0.50 0.00 /
% \end{sparkline}
%  &   2.0 & 1000.0 &   2.0 \\ 
% Fin,sei whales                      & Pelagic forage fish                 &   1.46e-13 & 
% \begin{sparkline}{10}
% \spark 0.03 0.00 0.03 0.38 0.07 0.38 0.07 0.68 0.10 0.68 0.10 0.57 0.13 0.57 0.13 0.41 0.16 0.41 0.16 0.32 0.19 0.32 0.19 0.19 0.22 0.19 0.22 0.08 0.26 0.08 0.26 0.05 0.29 0.05 0.29 0.03 0.32 0.03 0.32 0.00 /
% \end{sparkline}
%  &   2.0 & 1000.0 &   2.0 \\ 
% Fin,sei whales                      & Saury                               &   1.35e-14 & 
% \begin{sparkline}{10}
% \spark 0.03 0.00 0.03 0.22 0.06 0.22 0.06 0.51 0.09 0.51 0.09 0.68 0.12 0.68 0.12 0.49 0.15 0.49 0.15 0.27 0.17 0.27 0.17 0.27 0.20 0.27 0.20 0.11 0.23 0.11 0.23 0.03 0.26 0.03 0.26 0.14 0.29 0.14 0.29 0.00 /
% \end{sparkline}
%  &   2.0 & 1000.0 &   2.0 \\ 
% Fin,sei whales                      & Chum salmon                         &   1.46e-15 & 
% \begin{sparkline}{10}
% \spark 0.05 0.00 0.05 0.14 0.06 0.14 0.06 0.27 0.08 0.27 0.08 0.49 0.10 0.49 0.10 0.46 0.12 0.46 0.12 0.49 0.14 0.49 0.14 0.43 0.16 0.43 0.16 0.08 0.18 0.08 0.18 0.16 0.20 0.16 0.20 0.08 0.22 0.08 0.22 0.05 0.24 0.05 0.24 0.05 0.25 0.05 0.25 0.00 /
% \end{sparkline}
%  &   2.0 & 1000.0 &   2.0 \\ 
% Fin,sei whales                      & Chaetognaths                        &   1.02e-13 & 
% \begin{sparkline}{10}
% \spark 0.03 0.00 0.03 0.16 0.06 0.16 0.06 0.43 0.08 0.43 0.08 0.65 0.11 0.65 0.11 0.54 0.13 0.54 0.13 0.32 0.16 0.32 0.16 0.22 0.19 0.22 0.19 0.16 0.21 0.16 0.21 0.08 0.24 0.08 0.24 0.03 0.27 0.03 0.27 0.11 0.29 0.11 0.29 0.00 /
% \end{sparkline}
%  &   2.0 & 1000.0 &   2.0 \\ 
% Fin,sei whales                      & Predatory zooplankton               &   8.39e-14 & 
% \begin{sparkline}{10}
% \spark 0.03 0.00 0.03 0.19 0.06 0.19 0.06 0.51 0.09 0.51 0.09 0.68 0.12 0.68 0.12 0.51 0.15 0.51 0.15 0.30 0.17 0.30 0.17 0.32 0.20 0.32 0.20 0.05 0.23 0.05 0.23 0.03 0.26 0.03 0.26 0.11 0.29 0.11 0.29 0.00 /
% \end{sparkline}
%  &   2.0 & 1000.0 &   2.0 \\ 
% Fin,sei whales                      & Sergestid shrimp                    &   7.50e-14 & 
% \begin{sparkline}{10}
% \spark 0.03 0.00 0.03 0.22 0.06 0.22 0.06 0.59 0.09 0.59 0.09 0.51 0.12 0.51 0.12 0.51 0.15 0.51 0.15 0.46 0.18 0.46 0.18 0.19 0.21 0.19 0.21 0.11 0.23 0.11 0.23 0.08 0.26 0.08 0.26 0.03 0.29 0.03 0.29 0.00 /
% \end{sparkline}
%  &   2.0 & 1000.0 &   2.0 \\ 
% Fin,sei whales                      & Mesozooplankton                     &   7.16e-13 & 
% \begin{sparkline}{10}
% \spark 0.04 0.00 0.04 0.08 0.06 0.08 0.06 0.41 0.08 0.41 0.08 0.59 0.11 0.59 0.11 0.65 0.13 0.65 0.13 0.46 0.16 0.46 0.16 0.19 0.18 0.19 0.18 0.19 0.20 0.19 0.20 0.05 0.23 0.05 0.23 0.05 0.25 0.05 0.25 0.00 0.27 0.00 0.27 0.03 0.30 0.03 0.30 0.00 /
% \end{sparkline}
%  &   2.0 & 1000.0 &   2.0 \\ 
% Fin,sei whales                      & Copepods                            &   5.52e-13 & 
% \begin{sparkline}{10}
% \spark 0.05 0.00 0.05 0.32 0.07 0.32 0.07 0.35 0.09 0.35 0.09 0.73 0.12 0.73 0.12 0.43 0.14 0.43 0.14 0.32 0.16 0.32 0.16 0.19 0.18 0.19 0.18 0.11 0.21 0.11 0.21 0.11 0.23 0.11 0.23 0.14 0.25 0.14 0.25 0.00 /
% \end{sparkline}
%  &   2.0 & 1000.0 &   2.0 \\ 
% Micronektonic squid                 & Micronektonic squid                 &   1.12e-11 & 
% \begin{sparkline}{10}
% \spark 0.01 0.00 0.01 0.43 0.05 0.43 0.05 0.70 0.08 0.70 0.08 0.70 0.12 0.70 0.12 0.24 0.15 0.24 0.15 0.14 0.19 0.14 0.19 0.19 0.22 0.19 0.22 0.08 0.25 0.08 0.25 0.03 0.29 0.03 0.29 0.03 0.32 0.03 0.32 0.05 0.36 0.05 0.36 0.00 0.39 0.00 0.39 0.00 0.43 0.00 0.43 0.05 0.46 0.05 0.46 0.00 0.50 0.00 0.50 0.00 0.53 0.00 0.53 0.00 0.56 0.00 0.56 0.03 0.60 0.03 0.60 0.00 0.63 0.00 0.63 0.00 0.67 0.00 0.67 0.00 0.70 0.00 0.70 0.00 0.74 0.00 0.74 0.00 0.77 0.00 0.77 0.00 0.81 0.00 0.81 0.00 0.84 0.00 0.84 0.00 0.88 0.00 0.88 0.00 0.91 0.00 0.91 0.03 0.94 0.03 0.94 0.00 /
% \end{sparkline}
%  &   2.0 & 1000.0 &   2.0 \\ 
% Micronektonic squid                 & Chaetognaths                        &   1.32e-11 & 
% \begin{sparkline}{10}
% \spark 0.02 0.00 0.02 0.32 0.05 0.32 0.05 0.81 0.09 0.81 0.09 0.68 0.12 0.68 0.12 0.27 0.16 0.27 0.16 0.22 0.19 0.22 0.19 0.27 0.22 0.27 0.22 0.00 0.26 0.00 0.26 0.03 0.29 0.03 0.29 0.00 0.33 0.00 0.33 0.03 0.36 0.03 0.36 0.03 0.39 0.03 0.39 0.00 0.43 0.00 0.43 0.00 0.46 0.00 0.46 0.00 0.50 0.00 0.50 0.00 0.53 0.00 0.53 0.00 0.56 0.00 0.56 0.00 0.60 0.00 0.60 0.00 0.63 0.00 0.63 0.00 0.67 0.00 0.67 0.03 0.70 0.03 0.70 0.00 0.73 0.00 0.73 0.03 0.77 0.03 0.77 0.00 /
% \end{sparkline}
%  &   2.0 & 1000.0 &   2.0 \\ 
% Micronektonic squid                 & Predatory zooplankton               &   1.08e-11 & 
% \begin{sparkline}{10}
% \spark 0.01 0.00 0.01 0.38 0.05 0.38 0.05 0.92 0.09 0.92 0.09 0.51 0.13 0.51 0.13 0.41 0.17 0.41 0.17 0.30 0.21 0.30 0.21 0.08 0.24 0.08 0.24 0.00 0.28 0.00 0.28 0.00 0.32 0.00 0.32 0.00 0.36 0.00 0.36 0.03 0.40 0.03 0.40 0.00 0.44 0.00 0.44 0.00 0.48 0.00 0.48 0.03 0.52 0.03 0.52 0.00 0.55 0.00 0.55 0.00 0.59 0.00 0.59 0.00 0.63 0.00 0.63 0.00 0.67 0.00 0.67 0.00 0.71 0.00 0.71 0.00 0.75 0.00 0.75 0.03 0.79 0.03 0.79 0.00 0.83 0.00 0.83 0.00 0.86 0.00 0.86 0.03 0.90 0.03 0.90 0.00 /
% \end{sparkline}
%  &   2.0 & 1000.0 &   2.0 \\ 
% Micronektonic squid                 & Sergestid shrimp                    &   1.01e-11 & 
% \begin{sparkline}{10}
% \spark 0.03 0.00 0.03 0.73 0.06 0.73 0.06 0.59 0.10 0.59 0.10 0.57 0.14 0.57 0.14 0.35 0.18 0.35 0.18 0.11 0.22 0.11 0.22 0.11 0.26 0.11 0.26 0.11 0.30 0.11 0.30 0.08 0.34 0.08 0.34 0.00 0.38 0.00 0.38 0.00 0.42 0.00 0.42 0.00 0.46 0.00 0.46 0.00 0.49 0.00 0.49 0.00 0.53 0.00 0.53 0.05 0.57 0.05 0.57 0.00 /
% \end{sparkline}
%  &   2.0 & 1000.0 &   2.0 \\ 
% Micronektonic squid                 & Mesozooplankton                     &   9.18e-11 & 
% \begin{sparkline}{10}
% \spark 0.03 0.00 0.03 0.49 0.06 0.49 0.06 0.68 0.09 0.68 0.09 0.54 0.13 0.54 0.13 0.43 0.16 0.43 0.16 0.30 0.19 0.30 0.19 0.08 0.23 0.08 0.23 0.05 0.26 0.05 0.26 0.00 0.29 0.00 0.29 0.05 0.32 0.05 0.32 0.00 0.36 0.00 0.36 0.00 0.39 0.00 0.39 0.00 0.42 0.00 0.42 0.03 0.46 0.03 0.46 0.00 0.49 0.00 0.49 0.03 0.52 0.03 0.52 0.00 0.55 0.00 0.55 0.00 0.59 0.00 0.59 0.00 0.62 0.00 0.62 0.00 0.65 0.00 0.65 0.00 0.69 0.00 0.69 0.00 0.72 0.00 0.72 0.03 0.75 0.03 0.75 0.00 /
% \end{sparkline}
%  &   2.0 & 1000.0 &   2.0 \\ 
% Micronektonic squid                 & Copepods                            &   7.17e-11 & 
% \begin{sparkline}{10}
% \spark 0.03 0.00 0.03 0.59 0.06 0.59 0.06 0.68 0.10 0.68 0.10 0.54 0.14 0.54 0.14 0.49 0.18 0.49 0.18 0.16 0.21 0.16 0.21 0.11 0.25 0.11 0.25 0.00 0.29 0.00 0.29 0.03 0.32 0.03 0.32 0.00 0.36 0.00 0.36 0.00 0.40 0.00 0.40 0.03 0.44 0.03 0.44 0.03 0.47 0.03 0.47 0.00 0.51 0.00 0.51 0.03 0.55 0.03 0.55 0.03 0.59 0.03 0.59 0.00 /
% \end{sparkline}
%  &   2.0 & 1000.0 &   2.0 \\ 
% Mesopelagic fish                    & Chaetognaths                        &   2.58e-11 & 
% \begin{sparkline}{10}
% \spark 0.02 0.00 0.02 0.38 0.06 0.38 0.06 0.84 0.10 0.84 0.10 0.57 0.14 0.57 0.14 0.32 0.18 0.32 0.18 0.27 0.22 0.27 0.22 0.19 0.26 0.19 0.26 0.08 0.29 0.08 0.29 0.03 0.33 0.03 0.33 0.03 0.37 0.03 0.37 0.00 /
% \end{sparkline}
%  &   2.0 & 1000.0 &   2.0 \\ 
% Mesopelagic fish                    & Predatory zooplankton               &   5.22e-12 & 
% \begin{sparkline}{10}
% \spark 0.02 0.00 0.02 0.46 0.05 0.46 0.05 0.62 0.09 0.62 0.09 0.65 0.13 0.65 0.13 0.38 0.17 0.38 0.17 0.24 0.21 0.24 0.21 0.08 0.25 0.08 0.25 0.14 0.28 0.14 0.28 0.05 0.32 0.05 0.32 0.03 0.36 0.03 0.36 0.05 0.40 0.05 0.40 0.00 /
% \end{sparkline}
%  &   2.0 & 1000.0 &   2.0 \\ 
% Mesopelagic fish                    & Sergestid shrimp                    &   5.40e-12 & 
% \begin{sparkline}{10}
% \spark 0.02 0.00 0.02 0.62 0.06 0.62 0.06 0.62 0.10 0.62 0.10 0.41 0.14 0.41 0.14 0.57 0.18 0.57 0.18 0.19 0.22 0.19 0.22 0.08 0.26 0.08 0.26 0.05 0.30 0.05 0.30 0.05 0.34 0.05 0.34 0.05 0.38 0.05 0.38 0.03 0.42 0.03 0.42 0.00 0.46 0.00 0.46 0.03 0.50 0.03 0.50 0.00 /
% \end{sparkline}
%  &   2.0 & 1000.0 &   2.0 \\ 
% Mesopelagic fish                    & Mesozooplankton                     &   7.85e-11 & 
% \begin{sparkline}{10}
% \spark 0.03 0.00 0.03 0.46 0.06 0.46 0.06 0.57 0.09 0.57 0.09 0.68 0.13 0.68 0.13 0.38 0.16 0.38 0.16 0.19 0.20 0.19 0.20 0.24 0.23 0.24 0.23 0.03 0.26 0.03 0.26 0.03 0.30 0.03 0.30 0.03 0.33 0.03 0.33 0.08 0.36 0.08 0.36 0.03 0.40 0.03 0.40 0.00 /
% \end{sparkline}
%  &   2.0 & 1000.0 &   2.0 \\ 
% Mesopelagic fish                    & Copepods                            &   6.19e-11 & 
% \begin{sparkline}{10}
% \spark 0.02 0.00 0.02 0.38 0.06 0.38 0.06 0.51 0.09 0.51 0.09 0.57 0.12 0.57 0.12 0.51 0.15 0.51 0.15 0.32 0.18 0.32 0.18 0.19 0.22 0.19 0.22 0.08 0.25 0.08 0.25 0.05 0.28 0.05 0.28 0.03 0.31 0.03 0.31 0.05 0.35 0.05 0.35 0.00 /
% \end{sparkline}
%  &   2.0 & 1000.0 &   2.0 \\ 
% Pelagic forage fish                 & Chaetognaths                        &   4.06e-12 & 
% \begin{sparkline}{10}
% \spark 0.01 0.00 0.01 0.35 0.05 0.35 0.05 0.62 0.09 0.62 0.09 0.81 0.13 0.81 0.13 0.27 0.17 0.27 0.17 0.24 0.20 0.24 0.20 0.14 0.24 0.14 0.24 0.08 0.28 0.08 0.28 0.05 0.32 0.05 0.32 0.05 0.36 0.05 0.36 0.08 0.40 0.08 0.40 0.00 /
% \end{sparkline}
%  &   2.0 & 1000.0 &   2.0 \\ 
% Pelagic forage fish                 & Predatory zooplankton               &   2.99e-12 & 
% \begin{sparkline}{10}
% \spark 0.02 0.00 0.02 0.62 0.06 0.62 0.06 0.51 0.09 0.51 0.09 0.57 0.13 0.57 0.13 0.43 0.16 0.43 0.16 0.16 0.20 0.16 0.20 0.14 0.23 0.14 0.23 0.05 0.27 0.05 0.27 0.05 0.30 0.05 0.30 0.05 0.34 0.05 0.34 0.03 0.37 0.03 0.37 0.08 0.41 0.08 0.41 0.00 /
% \end{sparkline}
%  &   2.0 & 1000.0 &   2.0 \\ 
% Pelagic forage fish                 & Sergestid shrimp                    &   2.83e-12 & 
% \begin{sparkline}{10}
% \spark 0.02 0.00 0.02 0.57 0.06 0.57 0.06 0.65 0.10 0.65 0.10 0.65 0.14 0.65 0.14 0.35 0.18 0.35 0.18 0.11 0.22 0.11 0.22 0.22 0.25 0.22 0.25 0.08 0.29 0.08 0.29 0.00 0.33 0.00 0.33 0.08 0.37 0.08 0.37 0.00 /
% \end{sparkline}
%  &   2.0 & 1000.0 &   2.0 \\ 
% Pelagic forage fish                 & Mesozooplankton                     &   2.80e-11 & 
% \begin{sparkline}{10}
% \spark 0.03 0.00 0.03 0.49 0.06 0.49 0.06 0.54 0.10 0.54 0.10 0.65 0.13 0.65 0.13 0.41 0.16 0.41 0.16 0.19 0.20 0.19 0.20 0.16 0.23 0.16 0.23 0.14 0.26 0.14 0.26 0.03 0.30 0.03 0.30 0.11 0.33 0.11 0.33 0.00 /
% \end{sparkline}
%  &   2.0 & 1000.0 &   2.0 \\ 
% Pelagic forage fish                 & Copepods                            &   1.99e-11 & 
% \begin{sparkline}{10}
% \spark 0.03 0.00 0.03 0.38 0.06 0.38 0.06 0.62 0.09 0.62 0.09 0.51 0.12 0.51 0.12 0.62 0.15 0.62 0.15 0.11 0.19 0.11 0.19 0.24 0.22 0.24 0.22 0.05 0.25 0.05 0.25 0.00 0.28 0.00 0.28 0.08 0.31 0.08 0.31 0.03 0.34 0.03 0.34 0.05 0.37 0.05 0.37 0.00 /
% \end{sparkline}
%  &   2.0 & 1000.0 &   2.0 \\ 
% Saury                               & Chaetognaths                        &   2.89e-12 & 
% \begin{sparkline}{10}
% \spark 0.02 0.00 0.02 0.51 0.05 0.51 0.05 0.68 0.09 0.68 0.09 0.49 0.12 0.49 0.12 0.51 0.15 0.51 0.15 0.08 0.19 0.08 0.19 0.05 0.22 0.05 0.22 0.08 0.25 0.08 0.25 0.08 0.29 0.08 0.29 0.08 0.32 0.08 0.32 0.05 0.35 0.05 0.35 0.03 0.38 0.03 0.38 0.03 0.42 0.03 0.42 0.00 0.45 0.00 0.45 0.00 0.48 0.00 0.48 0.00 0.52 0.00 0.52 0.00 0.55 0.00 0.55 0.00 0.58 0.00 0.58 0.00 0.62 0.00 0.62 0.03 0.65 0.03 0.65 0.00 /
% \end{sparkline}
%  &   2.0 & 1000.0 &   2.0 \\ 
% Saury                               & Predatory zooplankton               &   2.18e-12 & 
% \begin{sparkline}{10}
% \spark 0.03 0.00 0.03 0.46 0.06 0.46 0.06 0.73 0.10 0.73 0.10 0.70 0.13 0.70 0.13 0.19 0.17 0.19 0.17 0.22 0.20 0.22 0.20 0.11 0.23 0.11 0.23 0.08 0.27 0.08 0.27 0.05 0.30 0.05 0.30 0.03 0.34 0.03 0.34 0.05 0.37 0.05 0.37 0.03 0.41 0.03 0.41 0.05 0.44 0.05 0.44 0.00 /
% \end{sparkline}
%  &   2.0 & 1000.0 &   2.0 \\ 
% Saury                               & Sergestid shrimp                    &   2.34e-12 & 
% \begin{sparkline}{10}
% \spark 0.01 0.00 0.01 0.54 0.05 0.54 0.05 0.76 0.08 0.76 0.08 0.51 0.12 0.51 0.12 0.35 0.15 0.35 0.15 0.11 0.18 0.11 0.18 0.08 0.22 0.08 0.22 0.03 0.25 0.03 0.25 0.05 0.28 0.05 0.28 0.03 0.32 0.03 0.32 0.05 0.35 0.05 0.35 0.00 0.38 0.00 0.38 0.05 0.42 0.05 0.42 0.00 0.45 0.00 0.45 0.03 0.48 0.03 0.48 0.08 0.52 0.08 0.52 0.00 0.55 0.00 0.55 0.00 0.59 0.00 0.59 0.00 0.62 0.00 0.62 0.00 0.65 0.00 0.65 0.03 0.69 0.03 0.69 0.00 /
% \end{sparkline}
%  &   2.0 & 1000.0 &   2.0 \\ 
% Saury                               & Mesozooplankton                     &   2.00e-11 & 
% \begin{sparkline}{10}
% \spark 0.02 0.00 0.02 0.62 0.06 0.62 0.06 0.65 0.10 0.65 0.10 0.59 0.13 0.59 0.13 0.27 0.17 0.27 0.17 0.14 0.21 0.14 0.21 0.16 0.24 0.16 0.24 0.14 0.28 0.14 0.28 0.03 0.32 0.03 0.32 0.03 0.35 0.03 0.35 0.03 0.39 0.03 0.39 0.03 0.43 0.03 0.43 0.00 0.46 0.00 0.46 0.00 0.50 0.00 0.50 0.00 0.54 0.00 0.54 0.00 0.57 0.00 0.57 0.03 0.61 0.03 0.61 0.00 /
% \end{sparkline}
%  &   2.0 & 1000.0 &   2.0 \\ 
% Saury                               & Copepods                            &   2.76e-11 & 
% \begin{sparkline}{10}
% \spark 0.03 0.00 0.03 0.59 0.06 0.59 0.06 0.86 0.10 0.86 0.10 0.54 0.14 0.54 0.14 0.16 0.17 0.16 0.17 0.19 0.21 0.19 0.21 0.08 0.25 0.08 0.25 0.08 0.29 0.08 0.29 0.03 0.32 0.03 0.32 0.08 0.36 0.08 0.36 0.05 0.40 0.05 0.40 0.00 0.43 0.00 0.43 0.00 0.47 0.00 0.47 0.00 0.51 0.00 0.51 0.00 0.54 0.00 0.54 0.03 0.58 0.03 0.58 0.00 /
% \end{sparkline}
%  &   2.0 & 1000.0 &   2.0 \\ 
% Chum salmon                         & Micronektonic squid                 &   3.78e-13 & 
% \begin{sparkline}{10}
% \spark 0.02 0.00 0.02 0.11 0.05 0.11 0.05 0.38 0.08 0.38 0.08 0.76 0.11 0.76 0.11 0.59 0.14 0.59 0.14 0.32 0.17 0.32 0.17 0.19 0.19 0.19 0.19 0.19 0.22 0.19 0.22 0.08 0.25 0.08 0.25 0.03 0.28 0.03 0.28 0.03 0.31 0.03 0.31 0.03 0.34 0.03 0.34 0.00 /
% \end{sparkline}
%  &   2.0 & 1000.0 &   2.0 \\ 
% Chum salmon                         & Mesopelagic fish                    &   8.61e-14 & 
% \begin{sparkline}{10}
% \spark 0.04 0.00 0.04 0.51 0.07 0.51 0.07 0.68 0.10 0.68 0.10 0.59 0.13 0.59 0.13 0.38 0.16 0.38 0.16 0.19 0.19 0.19 0.19 0.14 0.23 0.14 0.23 0.05 0.26 0.05 0.26 0.11 0.29 0.11 0.29 0.00 0.32 0.00 0.32 0.05 0.35 0.05 0.35 0.00 /
% \end{sparkline}
%  &   2.0 & 1000.0 &   2.0 \\ 
% Chum salmon                         & Pelagic forage fish                 &   7.60e-14 & 
% \begin{sparkline}{10}
% \spark 0.03 0.00 0.03 0.38 0.06 0.38 0.06 0.49 0.09 0.49 0.09 0.59 0.12 0.59 0.12 0.43 0.15 0.43 0.15 0.46 0.18 0.46 0.18 0.16 0.21 0.16 0.21 0.08 0.24 0.08 0.24 0.08 0.28 0.08 0.28 0.00 0.31 0.00 0.31 0.00 0.34 0.00 0.34 0.03 0.37 0.03 0.37 0.00 /
% \end{sparkline}
%  &   2.0 & 1000.0 &   2.0 \\ 
% Chum salmon                         & Chaetognaths                        &   3.89e-15 & 
% \begin{sparkline}{10}
% \spark 0.04 0.00 0.04 0.30 0.07 0.30 0.07 0.70 0.09 0.70 0.09 0.57 0.12 0.57 0.12 0.46 0.15 0.46 0.15 0.30 0.18 0.30 0.18 0.05 0.21 0.05 0.21 0.14 0.24 0.14 0.24 0.08 0.26 0.08 0.26 0.03 0.29 0.03 0.29 0.08 0.32 0.08 0.32 0.00 /
% \end{sparkline}
%  &   2.0 & 1000.0 &   2.0 \\ 
% Chum salmon                         & Predatory zooplankton               &   1.45e-13 & 
% \begin{sparkline}{10}
% \spark 0.04 0.00 0.04 0.49 0.07 0.49 0.07 0.76 0.10 0.76 0.10 0.41 0.14 0.41 0.14 0.49 0.17 0.49 0.17 0.32 0.20 0.32 0.20 0.11 0.23 0.11 0.23 0.05 0.27 0.05 0.27 0.08 0.30 0.08 0.30 0.00 /
% \end{sparkline}
%  &   2.0 & 1000.0 &   2.0 \\ 
% Chum salmon                         & Mesozooplankton                     &   2.10e-12 & 
% \begin{sparkline}{10}
% \spark 0.05 0.00 0.05 0.22 0.07 0.22 0.07 0.65 0.10 0.65 0.10 0.59 0.12 0.59 0.12 0.54 0.14 0.54 0.14 0.16 0.17 0.16 0.17 0.24 0.19 0.24 0.19 0.14 0.21 0.14 0.21 0.08 0.24 0.08 0.24 0.03 0.26 0.03 0.26 0.00 0.28 0.00 0.28 0.03 0.31 0.03 0.31 0.03 0.33 0.03 0.33 0.00 /
% \end{sparkline}
%  &   2.0 & 1000.0 &   2.0 \\ 
% Chum salmon                         & Gelatinous zooplankton              &   3.79e-12 & 
% \begin{sparkline}{10}
% \spark 0.04 0.00 0.04 0.38 0.07 0.38 0.07 0.73 0.10 0.73 0.10 0.46 0.14 0.46 0.14 0.54 0.17 0.54 0.17 0.41 0.20 0.41 0.20 0.16 0.23 0.16 0.23 0.03 0.26 0.03 0.26 0.00 /
% \end{sparkline}
%  &   2.0 & 1000.0 &   2.0 \\ 
% Chum salmon                         & Copepods                            &   2.87e-12 & 
% \begin{sparkline}{10}
% \spark 0.05 0.00 0.05 0.16 0.07 0.16 0.07 0.51 0.09 0.51 0.09 0.57 0.12 0.57 0.12 0.62 0.14 0.62 0.14 0.35 0.16 0.35 0.16 0.27 0.18 0.27 0.18 0.08 0.21 0.08 0.21 0.08 0.23 0.08 0.23 0.05 0.25 0.05 0.25 0.00 /
% \end{sparkline}
%  &   2.0 & 1000.0 &   2.0 \\ 
% Large jellyfish                     & Chaetognaths                        &   1.33e-11 & 
% \begin{sparkline}{10}
% \spark 0.03 0.00 0.03 0.68 0.07 0.68 0.07 0.62 0.10 0.62 0.10 0.51 0.14 0.51 0.14 0.35 0.17 0.35 0.17 0.11 0.21 0.11 0.21 0.11 0.25 0.11 0.25 0.14 0.28 0.14 0.28 0.08 0.32 0.08 0.32 0.05 0.35 0.05 0.35 0.05 0.39 0.05 0.39 0.00 /
% \end{sparkline}
%  &   2.0 & 1000.0 &   2.0 \\ 
% Large jellyfish                     & Predatory zooplankton               &   1.15e-11 & 
% \begin{sparkline}{10}
% \spark 0.02 0.00 0.02 0.92 0.07 0.92 0.07 0.73 0.12 0.73 0.12 0.38 0.18 0.38 0.18 0.24 0.23 0.24 0.23 0.27 0.28 0.27 0.28 0.05 0.33 0.05 0.33 0.05 0.39 0.05 0.39 0.05 0.44 0.05 0.44 0.00 /
% \end{sparkline}
%  &   2.0 & 1000.0 &   2.0 \\ 
% Large jellyfish                     & Sergestid shrimp                    &   1.10e-11 & 
% \begin{sparkline}{10}
% \spark 0.02 0.00 0.02 0.81 0.07 0.81 0.07 0.59 0.11 0.59 0.11 0.51 0.15 0.51 0.15 0.30 0.20 0.30 0.20 0.24 0.24 0.24 0.24 0.08 0.29 0.08 0.29 0.08 0.33 0.08 0.33 0.05 0.38 0.05 0.38 0.03 0.42 0.03 0.42 0.00 /
% \end{sparkline}
%  &   2.0 & 1000.0 &   2.0 \\ 
% Large jellyfish                     & Mesozooplankton                     &   9.25e-11 & 
% \begin{sparkline}{10}
% \spark 0.03 0.00 0.03 0.46 0.06 0.46 0.06 0.62 0.09 0.62 0.09 0.51 0.13 0.51 0.13 0.46 0.16 0.46 0.16 0.16 0.19 0.16 0.19 0.19 0.23 0.19 0.23 0.16 0.26 0.16 0.26 0.05 0.30 0.05 0.30 0.05 0.33 0.05 0.33 0.03 0.36 0.03 0.36 0.00 /
% \end{sparkline}
%  &   2.0 & 1000.0 &   2.0 \\ 
% Large jellyfish                     & Gelatinous zooplankton              &   3.80e-11 & 
% \begin{sparkline}{10}
% \spark 0.03 0.00 0.03 0.54 0.06 0.54 0.06 0.57 0.10 0.57 0.10 0.65 0.13 0.65 0.13 0.41 0.16 0.41 0.16 0.16 0.20 0.16 0.20 0.14 0.23 0.14 0.23 0.08 0.26 0.08 0.26 0.05 0.30 0.05 0.30 0.03 0.33 0.03 0.33 0.03 0.37 0.03 0.37 0.00 0.40 0.00 0.40 0.03 0.43 0.03 0.43 0.00 0.47 0.00 0.47 0.00 0.50 0.00 0.50 0.00 0.53 0.00 0.53 0.03 0.57 0.03 0.57 0.00 /
% \end{sparkline}
%  &   2.0 & 1000.0 &   2.0 \\}{3}
% 
% \epderivedlinktable{ 
% Large jellyfish                     & Copepods                            &   2.85e-10 & 
% \begin{sparkline}{10}
% \spark 0.03 0.00 0.03 0.65 0.07 0.65 0.07 0.68 0.10 0.68 0.10 0.51 0.14 0.51 0.14 0.32 0.18 0.32 0.18 0.19 0.22 0.19 0.22 0.22 0.25 0.22 0.25 0.08 0.29 0.08 0.29 0.00 0.33 0.00 0.33 0.03 0.36 0.03 0.36 0.00 0.40 0.00 0.40 0.03 0.44 0.03 0.44 0.00 /
% \end{sparkline}
%  &   2.0 & 1000.0 &   2.0 \\ 
% Chaetognaths                        & Mesozooplankton                     &   2.08e-10 & 
% \begin{sparkline}{10}
% \spark 0.03 0.00 0.03 0.32 0.07 0.32 0.07 0.70 0.10 0.70 0.10 0.86 0.14 0.86 0.14 0.19 0.17 0.19 0.17 0.32 0.21 0.32 0.21 0.16 0.24 0.16 0.24 0.11 0.28 0.11 0.28 0.00 0.31 0.00 0.31 0.03 0.35 0.03 0.35 0.00 /
% \end{sparkline}
%  & 1000.0 &  14.7 &   2.0 \\ 
% Chaetognaths                        & Copepods                            &   8.71e-10 & 
% \begin{sparkline}{10}
% \spark 0.04 0.00 0.04 0.24 0.06 0.24 0.06 0.51 0.09 0.51 0.09 0.73 0.12 0.73 0.12 0.43 0.15 0.43 0.15 0.35 0.17 0.35 0.17 0.19 0.20 0.19 0.20 0.08 0.23 0.08 0.23 0.05 0.25 0.05 0.25 0.05 0.28 0.05 0.28 0.03 0.31 0.03 0.31 0.00 0.34 0.00 0.34 0.00 0.36 0.00 0.36 0.03 0.39 0.03 0.39 0.00 /
% \end{sparkline}
%  & 1000.0 &   3.3 &   2.0 \\ 
% Predatory zooplankton               & Mesozooplankton                     &   1.93e-10 & 
% \begin{sparkline}{10}
% \spark 0.02 0.00 0.02 0.49 0.06 0.49 0.06 0.70 0.10 0.70 0.10 0.59 0.14 0.59 0.14 0.35 0.17 0.35 0.17 0.24 0.21 0.24 0.21 0.16 0.25 0.16 0.25 0.03 0.28 0.03 0.28 0.08 0.32 0.08 0.32 0.03 0.36 0.03 0.36 0.00 0.40 0.00 0.40 0.03 0.43 0.03 0.43 0.00 /
% \end{sparkline}
%  & 1000.0 &  19.6 &   2.0 \\ 
% Predatory zooplankton               & Copepods                            &   8.24e-10 & 
% \begin{sparkline}{10}
% \spark 0.02 0.00 0.02 0.24 0.05 0.24 0.05 0.59 0.09 0.59 0.09 0.73 0.12 0.73 0.12 0.43 0.15 0.43 0.15 0.27 0.18 0.27 0.18 0.11 0.21 0.11 0.21 0.08 0.24 0.08 0.24 0.08 0.27 0.08 0.27 0.05 0.31 0.05 0.31 0.03 0.34 0.03 0.34 0.03 0.37 0.03 0.37 0.03 0.40 0.03 0.40 0.03 0.43 0.03 0.43 0.00 /
% \end{sparkline}
%  & 1000.0 &   7.8 &   2.0 \\ 
% Sergestid shrimp                    & Mesozooplankton                     &   2.32e-10 & 
% \begin{sparkline}{10}
% \spark 0.02 0.00 0.02 0.30 0.05 0.30 0.05 0.54 0.08 0.54 0.08 0.62 0.12 0.62 0.12 0.54 0.15 0.54 0.15 0.27 0.18 0.27 0.18 0.11 0.22 0.11 0.22 0.14 0.25 0.14 0.25 0.08 0.28 0.08 0.28 0.05 0.32 0.05 0.32 0.05 0.35 0.05 0.35 0.00 /
% \end{sparkline}
%  & 1000.0 &  13.3 &   2.0 \\ 
% Sergestid shrimp                    & Copepods                            &   9.38e-10 & 
% \begin{sparkline}{10}
% \spark 0.02 0.00 0.02 0.27 0.05 0.27 0.05 0.43 0.08 0.43 0.08 0.65 0.11 0.65 0.11 0.54 0.14 0.54 0.14 0.35 0.18 0.35 0.18 0.11 0.21 0.11 0.21 0.16 0.24 0.16 0.24 0.08 0.27 0.08 0.27 0.03 0.30 0.03 0.30 0.08 0.34 0.08 0.34 0.00 /
% \end{sparkline}
%  & 1000.0 &   3.7 &   2.0 \\ 
% Mesozooplankton                     & Copepods                            &   1.86e-09 & 
% \begin{sparkline}{10}
% \spark 0.04 0.00 0.04 0.22 0.07 0.22 0.07 0.49 0.09 0.49 0.09 0.59 0.12 0.59 0.12 0.49 0.14 0.49 0.14 0.49 0.17 0.49 0.17 0.22 0.19 0.22 0.19 0.14 0.22 0.14 0.22 0.03 0.24 0.03 0.24 0.05 0.27 0.05 0.27 0.00 /
% \end{sparkline}
%  & 1000.0 &  22.2 &   2.0 \\ 
% Mesozooplankton                     & Microzooplankton                    &   1.88e-09 & 
% \begin{sparkline}{10}
% \spark 0.04 0.00 0.04 0.16 0.06 0.16 0.06 0.41 0.09 0.41 0.09 0.49 0.11 0.49 0.11 0.65 0.13 0.65 0.13 0.38 0.15 0.38 0.15 0.22 0.18 0.22 0.18 0.27 0.20 0.27 0.20 0.08 0.22 0.08 0.22 0.00 0.24 0.00 0.24 0.03 0.27 0.03 0.27 0.00 0.29 0.00 0.29 0.03 0.31 0.03 0.31 0.00 /
% \end{sparkline}
%  & 1000.0 &  19.8 &   2.0 \\ 
% Mesozooplankton                     & Large phytoplankton                 &   9.30e-10 & 
% \begin{sparkline}{10}
% \spark 0.04 0.00 0.04 0.30 0.07 0.30 0.07 0.32 0.09 0.32 0.09 0.49 0.11 0.49 0.11 0.57 0.14 0.57 0.14 0.43 0.16 0.43 0.16 0.32 0.18 0.32 0.18 0.22 0.21 0.22 0.21 0.03 0.23 0.03 0.23 0.03 0.25 0.03 0.25 0.00 /
% \end{sparkline}
%  & 1000.0 &  32.2 &   2.0 \\ 
% Gelatinous zooplankton              & Copepods                            &   2.11e-10 & 
% \begin{sparkline}{10}
% \spark 0.05 0.00 0.05 0.81 0.08 0.81 0.08 0.65 0.11 0.65 0.11 0.35 0.14 0.35 0.14 0.38 0.18 0.38 0.18 0.22 0.21 0.22 0.21 0.11 0.24 0.11 0.24 0.08 0.28 0.08 0.28 0.05 0.31 0.05 0.31 0.03 0.34 0.03 0.34 0.00 0.38 0.00 0.38 0.00 0.41 0.00 0.41 0.00 0.44 0.00 0.44 0.03 0.48 0.03 0.48 0.00 /
% \end{sparkline}
%  & 1000.0 &  12.4 &   2.0 \\ 
% Gelatinous zooplankton              & Microzooplankton                    &   2.18e-10 & 
% \begin{sparkline}{10}
% \spark 0.03 0.00 0.03 0.62 0.07 0.62 0.07 0.78 0.10 0.78 0.10 0.38 0.13 0.38 0.13 0.46 0.17 0.46 0.17 0.14 0.20 0.14 0.20 0.14 0.23 0.14 0.23 0.00 0.26 0.00 0.26 0.08 0.30 0.08 0.30 0.03 0.33 0.03 0.33 0.00 0.36 0.00 0.36 0.00 0.39 0.00 0.39 0.03 0.43 0.03 0.43 0.00 0.46 0.00 0.46 0.03 0.49 0.03 0.49 0.00 0.52 0.00 0.52 0.00 0.56 0.00 0.56 0.00 0.59 0.00 0.59 0.00 0.62 0.00 0.62 0.03 0.65 0.03 0.65 0.00 /
% \end{sparkline}
%  & 1000.0 &  10.8 &   2.0 \\ 
% Gelatinous zooplankton              & Large phytoplankton                 &   4.20e-10 & 
% \begin{sparkline}{10}
% \spark 0.04 0.00 0.04 0.76 0.07 0.76 0.07 0.78 0.11 0.78 0.11 0.38 0.15 0.38 0.15 0.38 0.18 0.38 0.18 0.14 0.22 0.14 0.22 0.16 0.26 0.16 0.26 0.00 0.29 0.00 0.29 0.05 0.33 0.05 0.33 0.00 0.36 0.00 0.36 0.03 0.40 0.03 0.40 0.00 0.44 0.00 0.44 0.00 0.47 0.00 0.47 0.00 0.51 0.00 0.51 0.00 0.54 0.00 0.54 0.00 0.58 0.00 0.58 0.00 0.62 0.00 0.62 0.03 0.65 0.03 0.65 0.00 /
% \end{sparkline}
%  & 1000.0 &   8.6 &   2.0 \\ 
% Copepods                            & Microzooplankton                    &   6.13e-09 & 
% \begin{sparkline}{10}
% \spark 0.07 0.00 0.07 0.14 0.09 0.14 0.09 0.65 0.11 0.65 0.11 0.62 0.12 0.62 0.12 0.51 0.14 0.51 0.14 0.57 0.16 0.57 0.16 0.22 0.17 0.22 0.17 0.00 /
% \end{sparkline}
%  & 1000.0 &  10.7 &   2.0 \\ 
% Copepods                            & Small phytoplankton                 &   6.57e-09 & 
% \begin{sparkline}{10}
% \spark 0.06 0.00 0.06 0.03 0.08 0.03 0.08 0.16 0.09 0.16 0.09 0.51 0.10 0.51 0.10 0.49 0.12 0.49 0.12 0.65 0.13 0.65 0.13 0.35 0.14 0.35 0.14 0.14 0.16 0.14 0.16 0.14 0.17 0.14 0.17 0.11 0.19 0.11 0.19 0.05 0.20 0.05 0.20 0.05 0.21 0.05 0.21 0.00 0.23 0.00 0.23 0.03 0.24 0.03 0.24 0.00 /
% \end{sparkline}
%  & 1000.0 &   1.8 &   2.0 \\ 
% Copepods                            & Large phytoplankton                 &   8.46e-09 & 
% \begin{sparkline}{10}
% \spark 0.06 0.00 0.06 0.08 0.08 0.08 0.08 0.38 0.10 0.38 0.10 0.70 0.11 0.70 0.11 0.57 0.13 0.57 0.13 0.32 0.15 0.32 0.15 0.35 0.16 0.35 0.16 0.19 0.18 0.19 0.18 0.08 0.20 0.08 0.20 0.03 0.22 0.03 0.22 0.00 /
% \end{sparkline}
%  & 1000.0 &   7.4 &   2.0 \\ 
% Microzooplankton                    & Small phytoplankton                 &   3.39e-08 & 
% \begin{sparkline}{10}
% \spark 0.08 0.00 0.08 0.16 0.09 0.16 0.09 0.27 0.10 0.27 0.10 0.49 0.11 0.49 0.11 0.57 0.13 0.57 0.13 0.65 0.14 0.65 0.14 0.19 0.15 0.19 0.15 0.22 0.16 0.22 0.16 0.08 0.17 0.08 0.17 0.05 0.18 0.05 0.18 0.03 0.20 0.03 0.20 0.00 /
% \end{sparkline}
%  & 1000.0 &   2.9 &   2.0 \\}{4}

\FloatBarrier
\epderivedlinklong{
Albatross                           & Neon flying squid                   &   2.64e-14 & 
\begin{sparkline}{10}
\spark 0.11 0.00 0.11 0.21 0.15 0.21 0.15 0.41 0.20 0.41 0.20 0.59 0.25 0.59 0.25 0.71 0.30 0.71 0.30 0.41 0.35 0.41 0.35 0.29 0.40 0.29 0.40 0.09 0.44 0.09 0.44 0.15 0.49 0.15 0.49 0.00 0.54 0.00 0.54 0.03 0.59 0.03 0.59 0.00 0.64 0.00 0.64 0.03 0.69 0.03 0.69 0.00 0.74 0.00 0.74 0.03 0.78 0.03 0.78 0.00 /
\end{sparkline}
 &   2.0 & 1000.0 &   2.0 \\ 
Albatross                           & Boreal clubhook squid               &   7.52e-16 & 
\begin{sparkline}{10}
\spark 0.10 0.00 0.10 0.47 0.16 0.47 0.16 0.68 0.23 0.68 0.23 0.71 0.29 0.71 0.29 0.38 0.36 0.38 0.36 0.38 0.42 0.38 0.42 0.09 0.49 0.09 0.49 0.06 0.56 0.06 0.56 0.12 0.62 0.12 0.62 0.00 0.69 0.00 0.69 0.03 0.75 0.03 0.75 0.00 0.82 0.00 0.82 0.03 0.88 0.03 0.88 0.00 /
\end{sparkline}
 &   2.0 & 1000.0 &   2.0 \\ 
Albatross                           & Large gonatid squid                 &   1.98e-15 & 
\begin{sparkline}{10}
\spark 0.09 0.00 0.09 0.76 0.16 0.76 0.16 0.74 0.24 0.74 0.24 0.59 0.32 0.59 0.32 0.26 0.40 0.26 0.40 0.15 0.48 0.15 0.48 0.24 0.56 0.24 0.56 0.06 0.64 0.06 0.64 0.03 0.72 0.03 0.72 0.09 0.80 0.09 0.80 0.03 0.88 0.03 0.88 0.00 /
\end{sparkline}
 &   2.0 & 1000.0 &   2.0 \\ 
Albatross                           & Micronektonic squid                 &   2.10e-15 & 
\begin{sparkline}{10}
\spark 0.06 0.00 0.06 0.44 0.14 0.44 0.14 0.71 0.21 0.71 0.21 0.68 0.28 0.68 0.28 0.41 0.36 0.41 0.36 0.18 0.43 0.18 0.43 0.21 0.50 0.21 0.50 0.12 0.57 0.12 0.57 0.06 0.65 0.06 0.65 0.15 0.72 0.15 0.72 0.00 /
\end{sparkline}
 &   2.0 & 1000.0 &   2.0 \\ 
Albatross                           & Pelagic forage fish                 &   4.55e-15 & 
\begin{sparkline}{10}
\spark 0.05 0.00 0.05 0.24 0.12 0.24 0.12 0.68 0.18 0.68 0.18 0.71 0.25 0.71 0.25 0.53 0.32 0.53 0.32 0.26 0.39 0.26 0.39 0.15 0.46 0.15 0.46 0.09 0.52 0.09 0.52 0.09 0.59 0.09 0.59 0.03 0.66 0.03 0.66 0.09 0.73 0.09 0.73 0.00 0.80 0.00 0.80 0.00 0.86 0.00 0.86 0.03 0.93 0.03 0.93 0.00 1.00 0.00 1.00 0.00 /
\end{sparkline}
 &   2.0 & 1000.0 &   2.0 \\ 
Albatross                           & Saury                               &   4.35e-15 & 
\begin{sparkline}{10}
\spark 0.07 0.00 0.07 0.65 0.15 0.65 0.15 0.74 0.22 0.74 0.22 0.59 0.30 0.59 0.30 0.29 0.38 0.29 0.38 0.18 0.46 0.18 0.46 0.12 0.53 0.12 0.53 0.18 0.61 0.18 0.61 0.06 0.69 0.06 0.69 0.03 0.77 0.03 0.77 0.09 0.84 0.09 0.84 0.00 0.92 0.00 0.92 0.00 1.00 0.00 1.00 0.00 /
\end{sparkline}
 &   2.0 & 1000.0 &   2.0 \\ 
Sperm whales                        & Neon flying squid                   &   2.53e-14 & 
\begin{sparkline}{10}
\spark 0.09 0.00 0.09 0.38 0.16 0.38 0.16 0.74 0.22 0.74 0.22 0.62 0.29 0.62 0.29 0.44 0.35 0.44 0.35 0.24 0.42 0.24 0.42 0.35 0.48 0.35 0.48 0.12 0.55 0.12 0.55 0.00 0.61 0.00 0.61 0.03 0.68 0.03 0.68 0.03 0.74 0.03 0.74 0.00 /
\end{sparkline}
 &   2.0 & 1000.0 &   2.0 \\ 
Sperm whales                        & Boreal clubhook squid               &   6.56e-16 & 
\begin{sparkline}{10}
\spark 0.09 0.00 0.09 0.12 0.15 0.12 0.15 0.68 0.20 0.68 0.20 0.71 0.26 0.71 0.26 0.53 0.31 0.53 0.31 0.32 0.37 0.32 0.37 0.21 0.42 0.21 0.42 0.09 0.48 0.09 0.48 0.09 0.53 0.09 0.53 0.12 0.59 0.12 0.59 0.00 0.64 0.00 0.64 0.03 0.70 0.03 0.70 0.03 0.75 0.03 0.75 0.00 0.81 0.00 0.81 0.03 0.86 0.03 0.86 0.00 /
\end{sparkline}
 &   2.0 & 1000.0 &   2.0 \\ 
Sperm whales                        & Chinook,coho,steelhead              &   3.92e-17 & 
\begin{sparkline}{10}
\spark 0.11 0.00 0.11 0.24 0.15 0.24 0.15 0.35 0.20 0.35 0.20 0.38 0.25 0.38 0.25 0.62 0.29 0.62 0.29 0.79 0.34 0.79 0.34 0.18 0.38 0.18 0.38 0.18 0.43 0.18 0.43 0.06 0.48 0.06 0.48 0.03 0.52 0.03 0.52 0.03 0.57 0.03 0.57 0.03 0.62 0.03 0.62 0.06 0.66 0.06 0.66 0.00 /
\end{sparkline}
 &   2.0 & 1000.0 &   2.0 \\ 
Sperm whales                        & Pomfret                             &   3.54e-16 & 
\begin{sparkline}{10}
\spark 0.08 0.00 0.08 0.29 0.14 0.29 0.14 0.62 0.21 0.62 0.21 0.68 0.27 0.68 0.27 0.50 0.33 0.50 0.33 0.35 0.40 0.35 0.40 0.21 0.46 0.21 0.46 0.15 0.53 0.15 0.53 0.03 0.59 0.03 0.59 0.03 0.65 0.03 0.65 0.06 0.72 0.06 0.72 0.03 0.78 0.03 0.78 0.00 /
\end{sparkline}
 &   2.0 & 1000.0 &   2.0 \\ 
Sperm whales                        & Large gonatid squid                 &   1.70e-15 & 
\begin{sparkline}{10}
\spark 0.08 0.00 0.08 0.41 0.14 0.41 0.14 0.38 0.20 0.38 0.20 0.53 0.26 0.53 0.26 0.74 0.32 0.74 0.32 0.41 0.38 0.41 0.38 0.09 0.44 0.09 0.44 0.15 0.50 0.15 0.50 0.12 0.56 0.12 0.56 0.03 0.62 0.03 0.62 0.03 0.68 0.03 0.68 0.06 0.75 0.06 0.75 0.00 /
\end{sparkline}
 &   2.0 & 1000.0 &   2.0 \\ 
Sperm whales                        & Sockeye,Pink                        &   1.89e-16 & 
\begin{sparkline}{10}
\spark 0.11 0.00 0.11 0.21 0.16 0.21 0.16 0.41 0.21 0.41 0.21 0.56 0.26 0.56 0.26 0.76 0.31 0.76 0.31 0.50 0.36 0.50 0.36 0.24 0.41 0.24 0.41 0.06 0.46 0.06 0.46 0.09 0.51 0.09 0.51 0.03 0.56 0.03 0.56 0.06 0.61 0.06 0.61 0.03 0.66 0.03 0.66 0.00 /
\end{sparkline}
 &   2.0 & 1000.0 &   2.0 \\ 
Sperm whales                        & Micronektonic squid                 &   2.52e-14 & 
\begin{sparkline}{10}
\spark 0.12 0.00 0.12 0.35 0.17 0.35 0.17 0.50 0.23 0.50 0.23 0.71 0.28 0.71 0.28 0.50 0.33 0.50 0.33 0.38 0.39 0.38 0.39 0.29 0.44 0.29 0.44 0.12 0.49 0.12 0.49 0.06 0.55 0.06 0.55 0.00 0.60 0.00 0.60 0.00 0.65 0.00 0.65 0.03 0.71 0.03 0.71 0.00 /
\end{sparkline}
 &   2.0 & 1000.0 &   2.0 \\ 
Sperm whales                        & Mesopelagic fish                    &   7.67e-15 & 
\begin{sparkline}{10}
\spark 0.07 0.00 0.07 0.21 0.14 0.21 0.14 0.59 0.20 0.59 0.20 0.79 0.27 0.79 0.27 0.47 0.33 0.47 0.33 0.47 0.40 0.47 0.40 0.06 0.47 0.06 0.47 0.21 0.53 0.21 0.53 0.15 0.60 0.15 0.60 0.00 /
\end{sparkline}
 &   2.0 & 1000.0 &   2.0 \\ 
Sperm whales                        & Pelagic forage fish                 &   8.47e-15 & 
\begin{sparkline}{10}
\spark 0.07 0.00 0.07 0.32 0.14 0.32 0.14 0.65 0.21 0.65 0.21 0.68 0.27 0.68 0.27 0.41 0.34 0.41 0.34 0.41 0.41 0.41 0.41 0.21 0.47 0.21 0.47 0.09 0.54 0.09 0.54 0.06 0.61 0.06 0.61 0.12 0.67 0.12 0.67 0.00 /
\end{sparkline}
 &   2.0 & 1000.0 &   2.0 \\ 
Sperm whales                        & Saury                               &   7.41e-16 & 
\begin{sparkline}{10}
\spark 0.07 0.00 0.07 0.29 0.12 0.29 0.12 0.44 0.17 0.44 0.17 0.53 0.22 0.53 0.22 0.47 0.27 0.47 0.27 0.26 0.32 0.26 0.32 0.29 0.37 0.29 0.37 0.18 0.42 0.18 0.42 0.15 0.46 0.15 0.46 0.12 0.51 0.12 0.51 0.03 0.56 0.03 0.56 0.06 0.61 0.06 0.61 0.06 0.66 0.06 0.66 0.03 0.71 0.03 0.71 0.00 0.76 0.00 0.76 0.00 0.81 0.00 0.81 0.00 0.85 0.00 0.85 0.00 0.90 0.00 0.90 0.00 0.95 0.00 0.95 0.00 1.00 0.00 1.00 0.00 /
\end{sparkline}
 &   2.0 & 1000.0 &   2.0 \\ 
Sperm whales                        & Chum salmon                         &   9.11e-17 & 
\begin{sparkline}{10}
\spark 0.10 0.00 0.10 0.12 0.15 0.12 0.15 0.41 0.19 0.41 0.19 0.44 0.24 0.44 0.24 0.44 0.28 0.44 0.28 0.79 0.33 0.79 0.33 0.35 0.37 0.35 0.37 0.15 0.42 0.15 0.42 0.03 0.46 0.03 0.46 0.06 0.51 0.06 0.51 0.06 0.55 0.06 0.55 0.06 0.60 0.06 0.60 0.00 0.65 0.00 0.65 0.03 0.69 0.03 0.69 0.00 /
\end{sparkline}
 &   2.0 & 1000.0 &   2.0 \\ 
Sharks                              & Neon flying squid                   &   1.48e-12 & 
\begin{sparkline}{10}
\spark 0.08 0.00 0.08 0.50 0.16 0.50 0.16 0.76 0.24 0.76 0.24 0.53 0.32 0.53 0.32 0.62 0.40 0.62 0.40 0.32 0.48 0.32 0.48 0.09 0.55 0.09 0.55 0.09 0.63 0.09 0.63 0.03 0.71 0.03 0.71 0.00 /
\end{sparkline}
 &   2.0 & 1000.0 &   2.0 \\ 
Sharks                              & Boreal clubhook squid               &   4.06e-14 & 
\begin{sparkline}{10}
\spark 0.08 0.00 0.08 0.50 0.15 0.50 0.15 0.85 0.23 0.85 0.23 0.35 0.30 0.35 0.30 0.56 0.37 0.56 0.37 0.35 0.44 0.35 0.44 0.09 0.52 0.09 0.52 0.12 0.59 0.12 0.59 0.03 0.66 0.03 0.66 0.00 0.73 0.00 0.73 0.03 0.81 0.03 0.81 0.06 0.88 0.06 0.88 0.00 /
\end{sparkline}
 &   2.0 & 1000.0 &   2.0 \\ 
Sharks                              & Chinook,coho,steelhead              &   7.94e-14 & 
\begin{sparkline}{10}
\spark 0.09 0.00 0.09 0.38 0.15 0.38 0.15 0.26 0.21 0.26 0.21 0.74 0.26 0.74 0.26 0.71 0.32 0.71 0.32 0.32 0.38 0.32 0.38 0.21 0.44 0.21 0.44 0.12 0.50 0.12 0.50 0.12 0.56 0.12 0.56 0.03 0.61 0.03 0.61 0.06 0.67 0.06 0.67 0.00 /
\end{sparkline}
 &   2.0 & 1000.0 &   2.0 \\ 
Sharks                              & Pomfret                             &   6.83e-13 & 
\begin{sparkline}{10}
\spark 0.09 0.00 0.09 0.44 0.16 0.44 0.16 0.62 0.23 0.62 0.23 0.76 0.30 0.76 0.30 0.41 0.36 0.41 0.36 0.35 0.43 0.35 0.43 0.12 0.50 0.12 0.50 0.09 0.57 0.09 0.57 0.09 0.64 0.09 0.64 0.06 0.71 0.06 0.71 0.00 /
\end{sparkline}
 &   2.0 & 1000.0 &   2.0 \\ 
Sharks                              & Large gonatid squid                 &   1.00e-13 & 
\begin{sparkline}{10}
\spark 0.07 0.00 0.07 0.53 0.14 0.53 0.14 0.62 0.22 0.62 0.22 0.50 0.29 0.50 0.29 0.62 0.37 0.62 0.37 0.24 0.44 0.24 0.44 0.32 0.52 0.32 0.52 0.06 0.59 0.06 0.59 0.00 0.67 0.00 0.67 0.03 0.74 0.03 0.74 0.00 0.82 0.00 0.82 0.03 0.89 0.03 0.89 0.00 /
\end{sparkline}
 &   2.0 & 1000.0 &   2.0 \\ 
Sharks                              & Sockeye,Pink                        &   3.83e-13 & 
\begin{sparkline}{10}
\spark 0.08 0.00 0.08 0.35 0.14 0.35 0.14 0.32 0.20 0.32 0.20 0.62 0.26 0.62 0.26 0.71 0.32 0.71 0.32 0.41 0.38 0.41 0.38 0.18 0.43 0.18 0.43 0.15 0.49 0.15 0.49 0.09 0.55 0.09 0.55 0.06 0.61 0.06 0.61 0.06 0.67 0.06 0.67 0.00 /
\end{sparkline}
 &   2.0 & 1000.0 &   2.0 \\ 
Sharks                              & Micronektonic squid                 &   6.04e-13 & 
\begin{sparkline}{10}
\spark 0.06 0.00 0.06 0.44 0.14 0.44 0.14 0.62 0.22 0.62 0.22 0.79 0.29 0.79 0.29 0.35 0.37 0.35 0.37 0.41 0.45 0.41 0.45 0.12 0.53 0.12 0.53 0.06 0.61 0.06 0.61 0.09 0.69 0.09 0.69 0.03 0.76 0.03 0.76 0.00 0.84 0.00 0.84 0.00 0.92 0.00 0.92 0.00 1.00 0.00 1.00 0.00 /
\end{sparkline}
 &   2.0 & 1000.0 &   2.0 \\ 
Sharks                              & Pelagic forage fish                 &   5.95e-13 & 
\begin{sparkline}{10}
\spark 0.05 0.00 0.05 0.38 0.13 0.38 0.13 0.59 0.20 0.59 0.20 0.59 0.27 0.59 0.27 0.59 0.35 0.59 0.35 0.41 0.42 0.41 0.42 0.15 0.50 0.15 0.50 0.09 0.57 0.09 0.57 0.03 0.65 0.03 0.65 0.06 0.72 0.06 0.72 0.06 0.79 0.06 0.79 0.00 /
\end{sparkline}
 &   2.0 & 1000.0 &   2.0 \\ 
Sharks                              & Saury                               &   1.51e-12 & 
\begin{sparkline}{10}
\spark 0.06 0.00 0.06 0.38 0.13 0.38 0.13 0.62 0.21 0.62 0.21 0.68 0.28 0.68 0.28 0.56 0.36 0.56 0.36 0.26 0.43 0.26 0.43 0.18 0.51 0.18 0.51 0.12 0.58 0.12 0.58 0.03 0.66 0.03 0.66 0.06 0.73 0.06 0.73 0.06 0.81 0.06 0.81 0.00 /
\end{sparkline}
 &   2.0 & 1000.0 &   2.0 \\ 
Sharks                              & Chum salmon                         &   1.87e-13 & 
\begin{sparkline}{10}
\spark 0.09 0.00 0.09 0.44 0.16 0.44 0.16 0.47 0.22 0.47 0.22 0.74 0.29 0.74 0.29 0.50 0.35 0.50 0.35 0.35 0.41 0.35 0.41 0.26 0.48 0.26 0.48 0.09 0.54 0.09 0.54 0.06 0.61 0.06 0.61 0.03 0.67 0.03 0.67 0.00 /
\end{sparkline}
 &   2.0 & 1000.0 &   2.0 \\ 
Neon flying squid                   & Neon flying squid                   &   8.19e-12 & 
\begin{sparkline}{10}
\spark 0.07 0.00 0.07 0.26 0.11 0.26 0.11 0.32 0.15 0.32 0.15 0.38 0.19 0.38 0.19 0.32 0.23 0.32 0.23 0.44 0.27 0.44 0.27 0.32 0.31 0.32 0.31 0.24 0.35 0.24 0.35 0.15 0.39 0.15 0.39 0.03 0.43 0.03 0.43 0.12 0.48 0.12 0.48 0.18 0.52 0.18 0.52 0.03 0.56 0.03 0.56 0.00 0.60 0.00 0.60 0.03 0.64 0.03 0.64 0.03 0.68 0.03 0.68 0.00 0.72 0.00 0.72 0.00 0.76 0.00 0.76 0.06 0.80 0.06 0.80 0.00 0.84 0.00 0.84 0.00 0.88 0.00 0.88 0.00 0.92 0.00 0.92 0.00 0.96 0.00 0.96 0.00 1.00 0.00 1.00 0.00 /
\end{sparkline}
 &   2.0 & 1000.0 &   2.0 \\ 
Neon flying squid                   & Micronektonic squid                 &   7.06e-12 & 
\begin{sparkline}{10}
\spark 0.03 0.00 0.03 0.56 0.13 0.56 0.13 0.82 0.22 0.82 0.22 0.65 0.32 0.65 0.32 0.29 0.42 0.29 0.42 0.29 0.51 0.29 0.51 0.09 0.61 0.09 0.61 0.12 0.71 0.12 0.71 0.06 0.81 0.06 0.81 0.03 0.90 0.03 0.90 0.00 1.00 0.00 1.00 0.00 /
\end{sparkline}
 &   2.0 & 1000.0 &   2.0 \\ 
Neon flying squid                   & Mesopelagic fish                    &   3.33e-12 & 
\begin{sparkline}{10}
\spark 0.04 0.00 0.04 0.26 0.09 0.26 0.09 0.50 0.14 0.50 0.14 0.35 0.19 0.35 0.19 0.35 0.24 0.35 0.24 0.24 0.29 0.24 0.29 0.32 0.34 0.32 0.34 0.50 0.39 0.50 0.39 0.06 0.45 0.06 0.45 0.09 0.50 0.09 0.50 0.12 0.55 0.12 0.55 0.03 0.60 0.03 0.60 0.03 0.65 0.03 0.65 0.03 0.70 0.03 0.70 0.00 0.75 0.00 0.75 0.00 0.80 0.00 0.80 0.00 0.85 0.00 0.85 0.00 0.90 0.00 0.90 0.00 0.95 0.00 0.95 0.00 1.00 0.00 1.00 0.00 /
\end{sparkline}
 &   2.0 & 1000.0 &   2.0 \\ 
Neon flying squid                   & Pelagic forage fish                 &   9.59e-12 & 
\begin{sparkline}{10}
\spark 0.06 0.00 0.06 0.26 0.13 0.26 0.13 0.88 0.21 0.88 0.21 0.82 0.28 0.82 0.28 0.26 0.36 0.26 0.36 0.26 0.43 0.26 0.43 0.09 0.51 0.09 0.51 0.09 0.58 0.09 0.58 0.09 0.66 0.09 0.66 0.12 0.73 0.12 0.73 0.06 0.81 0.06 0.81 0.00 /
\end{sparkline}
 &   2.0 & 1000.0 &   2.0 \\ 
Neon flying squid                   & Saury                               &   1.62e-12 & 
\begin{sparkline}{10}
\spark 0.06 0.00 0.06 0.56 0.14 0.56 0.14 0.76 0.22 0.76 0.22 0.50 0.29 0.50 0.29 0.26 0.37 0.26 0.37 0.47 0.45 0.47 0.45 0.15 0.53 0.15 0.53 0.06 0.61 0.06 0.61 0.09 0.69 0.09 0.69 0.03 0.76 0.03 0.76 0.03 0.84 0.03 0.84 0.00 0.92 0.00 0.92 0.00 1.00 0.00 1.00 0.00 /
\end{sparkline}
 &   2.0 & 1000.0 &   2.0 \\ 
Toothed whales                      & Albatross                           &   1.13e-18 & 
\begin{sparkline}{10}
\spark 0.07 0.00 0.07 0.32 0.15 0.32 0.15 0.85 0.22 0.85 0.22 0.50 0.29 0.50 0.29 0.53 0.37 0.53 0.37 0.47 0.44 0.47 0.44 0.09 0.52 0.09 0.52 0.03 0.59 0.03 0.59 0.15 0.67 0.15 0.67 0.00 /
\end{sparkline}
 &   2.0 & 1000.0 &   2.0 \\ 
Toothed whales                      & Neon flying squid                   &   1.67e-16 & 
\begin{sparkline}{10}
\spark 0.06 0.00 0.06 0.21 0.14 0.21 0.14 0.85 0.22 0.85 0.22 0.65 0.29 0.65 0.29 0.53 0.37 0.53 0.37 0.35 0.45 0.35 0.45 0.18 0.53 0.18 0.53 0.18 0.61 0.18 0.61 0.00 /
\end{sparkline}
 &   2.0 & 1000.0 &   2.0 \\ 
Toothed whales                      & Elephant seals                      &   1.33e-17 & 
\begin{sparkline}{10}
\spark 0.06 0.00 0.06 0.32 0.13 0.32 0.13 0.65 0.21 0.65 0.21 0.74 0.29 0.74 0.29 0.47 0.37 0.47 0.37 0.35 0.44 0.35 0.44 0.21 0.52 0.21 0.52 0.09 0.60 0.09 0.60 0.12 0.68 0.12 0.68 0.00 /
\end{sparkline}
 &   2.0 & 1000.0 &   2.0 \\ 
Toothed whales                      & Seals,dolphins                      &   4.21e-16 & 
\begin{sparkline}{10}
\spark 0.10 0.00 0.10 0.68 0.18 0.68 0.18 0.65 0.26 0.65 0.26 0.65 0.33 0.65 0.33 0.53 0.41 0.53 0.41 0.26 0.49 0.26 0.49 0.09 0.57 0.09 0.57 0.03 0.64 0.03 0.64 0.06 0.72 0.06 0.72 0.00 /
\end{sparkline}
 &   2.0 & 1000.0 &   2.0 \\ 
Toothed whales                      & Boreal clubhook squid               &   4.53e-18 & 
\begin{sparkline}{10}
\spark 0.06 0.00 0.06 0.29 0.11 0.29 0.11 0.29 0.17 0.29 0.17 0.50 0.23 0.50 0.23 0.56 0.29 0.56 0.29 0.47 0.35 0.47 0.35 0.29 0.40 0.29 0.40 0.32 0.46 0.32 0.46 0.09 0.52 0.09 0.52 0.06 0.58 0.06 0.58 0.00 0.63 0.00 0.63 0.00 0.69 0.00 0.69 0.00 0.75 0.00 0.75 0.06 0.81 0.06 0.81 0.00 /
\end{sparkline}
 &   2.0 & 1000.0 &   2.0 \\ 
Toothed whales                      & Fulmars                             &   2.24e-18 & 
\begin{sparkline}{10}
\spark 0.07 0.00 0.07 0.35 0.14 0.35 0.14 0.62 0.21 0.62 0.21 0.76 0.28 0.76 0.28 0.41 0.35 0.41 0.35 0.29 0.42 0.29 0.42 0.15 0.48 0.15 0.48 0.18 0.55 0.18 0.55 0.12 0.62 0.12 0.62 0.06 0.69 0.06 0.69 0.00 /
\end{sparkline}
 &   2.0 & 1000.0 &   2.0 \\ 
Toothed whales                      & Chinook,coho,steelhead              &   2.42e-17 & 
\begin{sparkline}{10}
\spark 0.06 0.00 0.06 0.12 0.13 0.12 0.13 0.65 0.20 0.65 0.20 0.56 0.27 0.56 0.27 0.71 0.34 0.71 0.34 0.53 0.40 0.53 0.40 0.24 0.47 0.24 0.47 0.15 0.54 0.15 0.54 0.00 /
\end{sparkline}
 &   2.0 & 1000.0 &   2.0 \\ 
Toothed whales                      & Skuas,Jaegers                       &   2.86e-18 & 
\begin{sparkline}{10}
\spark 0.07 0.00 0.07 0.47 0.15 0.47 0.15 0.62 0.22 0.62 0.22 0.76 0.30 0.76 0.30 0.35 0.38 0.35 0.38 0.47 0.45 0.47 0.45 0.09 0.53 0.09 0.53 0.09 0.61 0.09 0.61 0.00 0.68 0.00 0.68 0.06 0.76 0.06 0.76 0.00 0.84 0.00 0.84 0.03 0.91 0.03 0.91 0.00 /
\end{sparkline}
 &   2.0 & 1000.0 &   2.0 \\ 
Toothed whales                      & Pomfret                             &   2.21e-16 & 
\begin{sparkline}{10}
\spark 0.07 0.00 0.07 0.21 0.14 0.21 0.14 0.79 0.22 0.79 0.22 0.65 0.29 0.65 0.29 0.53 0.36 0.53 0.36 0.41 0.44 0.41 0.44 0.21 0.51 0.21 0.51 0.09 0.58 0.09 0.58 0.06 0.66 0.06 0.66 0.00 /
\end{sparkline}
 &   2.0 & 1000.0 &   2.0 \\ 
Toothed whales                      & Puffins,Shearwaters,Storm Petrels   &   1.62e-17 & 
\begin{sparkline}{10}
\spark 0.09 0.00 0.09 0.29 0.15 0.29 0.15 0.68 0.21 0.68 0.21 0.62 0.27 0.62 0.27 0.62 0.33 0.62 0.33 0.29 0.39 0.29 0.39 0.18 0.45 0.18 0.45 0.03 0.51 0.03 0.51 0.06 0.57 0.06 0.57 0.09 0.63 0.09 0.63 0.03 0.70 0.03 0.70 0.00 0.76 0.00 0.76 0.00 0.82 0.00 0.82 0.03 0.88 0.03 0.88 0.00 0.94 0.00 0.94 0.00 1.00 0.00 1.00 0.00 /
\end{sparkline}
 &   2.0 & 1000.0 &   2.0 \\ 
Toothed whales                      & Kittiwakes                          &   1.62e-18 & 
\begin{sparkline}{10}
\spark 0.04 0.00 0.04 0.26 0.13 0.26 0.13 0.82 0.21 0.82 0.21 0.74 0.30 0.74 0.30 0.35 0.38 0.35 0.38 0.53 0.46 0.53 0.46 0.03 0.55 0.03 0.55 0.18 0.63 0.18 0.63 0.00 0.72 0.00 0.72 0.03 0.80 0.03 0.80 0.00 /
\end{sparkline}
 &   2.0 & 1000.0 &   2.0 \\ 
Toothed whales                      & Large gonatid squid                 &   1.11e-17 & 
\begin{sparkline}{10}
\spark 0.05 0.00 0.05 0.15 0.11 0.15 0.11 0.24 0.16 0.24 0.16 0.44 0.21 0.44 0.21 0.71 0.26 0.71 0.26 0.59 0.32 0.59 0.32 0.26 0.37 0.26 0.37 0.06 0.42 0.06 0.42 0.15 0.47 0.15 0.47 0.12 0.52 0.12 0.52 0.09 0.58 0.09 0.58 0.06 0.63 0.06 0.63 0.06 0.68 0.06 0.68 0.03 0.73 0.03 0.73 0.00 /
\end{sparkline}
 &   2.0 & 1000.0 &   2.0 \\ 
Toothed whales                      & Sockeye,Pink                        &   1.18e-16 & 
\begin{sparkline}{10}
\spark 0.06 0.00 0.06 0.09 0.12 0.09 0.12 0.47 0.19 0.47 0.19 0.59 0.25 0.59 0.25 0.59 0.31 0.59 0.31 0.56 0.37 0.56 0.37 0.41 0.43 0.41 0.43 0.18 0.49 0.18 0.49 0.06 0.55 0.06 0.55 0.00 /
\end{sparkline}
 &   2.0 & 1000.0 &   2.0 \\ 
Toothed whales                      & Fin,sei whales                      &   1.03e-15 & 
\begin{sparkline}{10}
\spark 0.07 0.00 0.07 0.12 0.13 0.12 0.13 0.47 0.19 0.47 0.19 0.76 0.25 0.76 0.25 0.50 0.31 0.50 0.31 0.53 0.37 0.53 0.37 0.24 0.43 0.24 0.43 0.18 0.49 0.18 0.49 0.15 0.56 0.15 0.56 0.00 /
\end{sparkline}
 &   2.0 & 1000.0 &   2.0 \\ 
Toothed whales                      & Micronektonic squid                 &   1.83e-16 & 
\begin{sparkline}{10}
\spark 0.08 0.00 0.08 0.15 0.13 0.15 0.13 0.59 0.19 0.59 0.19 0.68 0.24 0.68 0.24 0.38 0.29 0.38 0.29 0.44 0.35 0.44 0.35 0.29 0.40 0.29 0.40 0.15 0.46 0.15 0.46 0.12 0.51 0.12 0.51 0.03 0.57 0.03 0.57 0.03 0.62 0.03 0.62 0.00 0.67 0.00 0.67 0.03 0.73 0.03 0.73 0.03 0.78 0.03 0.78 0.00 0.84 0.00 0.84 0.00 0.89 0.00 0.89 0.00 0.95 0.00 0.95 0.00 1.00 0.00 1.00 0.00 /
\end{sparkline}
 &   2.0 & 1000.0 &   2.0 \\ 
Toothed whales                      & Pelagic forage fish                 &   1.03e-15 & 
\begin{sparkline}{10}
\spark 0.10 0.00 0.10 0.71 0.18 0.71 0.18 0.68 0.26 0.68 0.26 0.59 0.34 0.59 0.34 0.65 0.42 0.65 0.42 0.18 0.50 0.18 0.50 0.06 0.59 0.06 0.59 0.09 0.67 0.09 0.67 0.00 /
\end{sparkline}
 &   2.0 & 1000.0 &   2.0 \\ 
Toothed whales                      & Saury                               &   4.38e-16 & 
\begin{sparkline}{10}
\spark 0.07 0.00 0.07 0.24 0.14 0.24 0.14 0.68 0.21 0.68 0.21 0.62 0.28 0.62 0.28 0.68 0.36 0.68 0.36 0.41 0.43 0.41 0.43 0.12 0.50 0.12 0.50 0.15 0.57 0.15 0.57 0.03 0.64 0.03 0.64 0.03 0.72 0.03 0.72 0.00 /
\end{sparkline}
 &   2.0 & 1000.0 &   2.0 \\ 
Toothed whales                      & Chum salmon                         &   5.73e-17 & 
\begin{sparkline}{10}
\spark 0.06 0.00 0.06 0.15 0.13 0.15 0.13 0.53 0.19 0.53 0.19 0.53 0.26 0.53 0.26 0.74 0.32 0.74 0.32 0.47 0.39 0.47 0.39 0.38 0.45 0.38 0.45 0.09 0.51 0.09 0.51 0.06 0.58 0.06 0.58 0.00 /
\end{sparkline}
 &   2.0 & 1000.0 &   2.0 \\ 
Elephant seals                      & Neon flying squid                   &   1.02e-14 & 
\begin{sparkline}{10}
\spark 0.11 0.00 0.11 0.50 0.17 0.50 0.17 0.68 0.23 0.68 0.23 0.56 0.29 0.56 0.29 0.56 0.36 0.56 0.36 0.32 0.42 0.32 0.42 0.06 0.48 0.06 0.48 0.15 0.54 0.15 0.54 0.03 0.60 0.03 0.60 0.09 0.67 0.09 0.67 0.00 /
\end{sparkline}
 &   2.0 & 1000.0 &   2.0 \\ 
Elephant seals                      & Boreal clubhook squid               &   2.69e-16 & 
\begin{sparkline}{10}
\spark 0.08 0.00 0.08 0.15 0.13 0.15 0.13 0.29 0.17 0.29 0.17 0.47 0.22 0.47 0.22 0.56 0.27 0.56 0.27 0.59 0.31 0.59 0.31 0.35 0.36 0.35 0.36 0.24 0.40 0.24 0.40 0.09 0.45 0.09 0.45 0.03 0.50 0.03 0.50 0.03 0.54 0.03 0.54 0.03 0.59 0.03 0.59 0.03 0.63 0.03 0.63 0.03 0.68 0.03 0.68 0.00 0.72 0.00 0.72 0.00 0.77 0.00 0.77 0.03 0.82 0.03 0.82 0.00 0.86 0.00 0.86 0.00 0.91 0.00 0.91 0.00 0.95 0.00 0.95 0.00 1.00 0.00 1.00 0.00 /
\end{sparkline}
 &   2.0 & 1000.0 &   2.0 \\ 
Elephant seals                      & Chinook,coho,steelhead              &   1.61e-16 & 
\begin{sparkline}{10}
\spark 0.11 0.00 0.11 0.21 0.16 0.21 0.16 0.41 0.21 0.41 0.21 0.74 0.26 0.74 0.26 0.53 0.31 0.53 0.31 0.47 0.36 0.47 0.36 0.26 0.41 0.26 0.41 0.21 0.45 0.21 0.45 0.00 0.50 0.00 0.50 0.06 0.55 0.06 0.55 0.03 0.60 0.03 0.60 0.03 0.65 0.03 0.65 0.00 /
\end{sparkline}
 &   2.0 & 1000.0 &   2.0 \\ 
Elephant seals                      & Pomfret                             &   3.11e-15 & 
\begin{sparkline}{10}
\spark 0.09 0.00 0.09 0.29 0.15 0.29 0.15 0.47 0.21 0.47 0.21 0.65 0.27 0.65 0.27 0.74 0.33 0.74 0.33 0.24 0.38 0.24 0.38 0.24 0.44 0.24 0.44 0.12 0.50 0.12 0.50 0.12 0.56 0.12 0.56 0.03 0.62 0.03 0.62 0.03 0.68 0.03 0.68 0.00 0.74 0.00 0.74 0.03 0.79 0.03 0.79 0.00 /
\end{sparkline}
 &   2.0 & 1000.0 &   2.0 \\ 
Elephant seals                      & Large gonatid squid                 &   6.87e-16 & 
\begin{sparkline}{10}
\spark 0.09 0.00 0.09 0.56 0.16 0.56 0.16 0.53 0.23 0.53 0.23 0.68 0.30 0.68 0.30 0.47 0.36 0.47 0.36 0.21 0.43 0.21 0.43 0.35 0.50 0.35 0.50 0.03 0.57 0.03 0.57 0.06 0.64 0.06 0.64 0.06 0.71 0.06 0.71 0.00 /
\end{sparkline}
 &   2.0 & 1000.0 &   2.0 \\ 
Elephant seals                      & Sockeye,Pink                        &   8.02e-16 & 
\begin{sparkline}{10}
\spark 0.12 0.00 0.12 0.21 0.17 0.21 0.17 0.47 0.21 0.47 0.21 0.68 0.26 0.68 0.26 0.59 0.31 0.59 0.31 0.41 0.36 0.41 0.36 0.18 0.40 0.18 0.40 0.15 0.45 0.15 0.45 0.15 0.50 0.15 0.50 0.12 0.55 0.12 0.55 0.00 /
\end{sparkline}
 &   2.0 & 1000.0 &   2.0 \\ 
Elephant seals                      & Micronektonic squid                 &   2.38e-14 & 
\begin{sparkline}{10}
\spark 0.10 0.00 0.10 0.35 0.16 0.35 0.16 0.44 0.21 0.44 0.21 0.50 0.26 0.50 0.26 0.76 0.31 0.76 0.31 0.26 0.36 0.26 0.36 0.12 0.41 0.12 0.41 0.15 0.46 0.15 0.46 0.18 0.52 0.18 0.52 0.12 0.57 0.12 0.57 0.06 0.62 0.06 0.62 0.00 /
\end{sparkline}
 &   2.0 & 1000.0 &   2.0 \\ 
Elephant seals                      & Pelagic forage fish                 &   9.99e-15 & 
\begin{sparkline}{10}
\spark 0.07 0.00 0.07 0.15 0.14 0.15 0.14 0.74 0.21 0.74 0.21 0.53 0.27 0.53 0.27 0.76 0.34 0.76 0.34 0.38 0.41 0.38 0.41 0.24 0.47 0.24 0.47 0.06 0.54 0.06 0.54 0.03 0.61 0.03 0.61 0.06 0.67 0.06 0.67 0.00 /
\end{sparkline}
 &   2.0 & 1000.0 &   2.0 \\ 
Elephant seals                      & Saury                               &   6.75e-15 & 
\begin{sparkline}{10}
\spark 0.09 0.00 0.09 0.62 0.17 0.62 0.17 0.79 0.25 0.79 0.25 0.65 0.33 0.65 0.33 0.38 0.41 0.38 0.41 0.29 0.49 0.29 0.49 0.06 0.57 0.06 0.57 0.09 0.65 0.09 0.65 0.06 0.73 0.06 0.73 0.00 /
\end{sparkline}
 &   2.0 & 1000.0 &   2.0 \\ 
Elephant seals                      & Chum salmon                         &   3.82e-16 & 
\begin{sparkline}{10}
\spark 0.13 0.00 0.13 0.35 0.17 0.35 0.17 0.53 0.22 0.53 0.22 0.71 0.27 0.71 0.27 0.44 0.32 0.44 0.32 0.41 0.37 0.41 0.37 0.24 0.42 0.24 0.42 0.12 0.47 0.12 0.47 0.06 0.52 0.06 0.52 0.09 0.57 0.09 0.57 0.00 /
\end{sparkline}
 &   2.0 & 1000.0 &   2.0 \\ 
Seals,dolphins                      & Neon flying squid                   &   8.66e-13 & 
\begin{sparkline}{10}
\spark 0.10 0.00 0.10 0.53 0.17 0.53 0.17 0.79 0.25 0.79 0.25 0.65 0.32 0.65 0.32 0.29 0.39 0.29 0.39 0.50 0.46 0.50 0.46 0.03 0.53 0.03 0.53 0.12 0.60 0.12 0.60 0.03 0.67 0.03 0.67 0.00 /
\end{sparkline}
 &   2.0 & 1000.0 &   2.0 \\ 
Seals,dolphins                      & Boreal clubhook squid               &   2.34e-14 & 
\begin{sparkline}{10}
\spark 0.07 0.00 0.07 0.26 0.13 0.26 0.13 0.71 0.20 0.71 0.20 0.65 0.27 0.65 0.27 0.41 0.34 0.41 0.34 0.35 0.40 0.35 0.40 0.21 0.47 0.21 0.47 0.18 0.54 0.18 0.54 0.09 0.60 0.09 0.60 0.03 0.67 0.03 0.67 0.03 0.74 0.03 0.74 0.00 0.80 0.00 0.80 0.03 0.87 0.03 0.87 0.00 /
\end{sparkline}
 &   2.0 & 1000.0 &   2.0 \\ 
Seals,dolphins                      & Chinook,coho,steelhead              &   2.31e-14 & 
\begin{sparkline}{10}
\spark 0.12 0.00 0.12 0.24 0.16 0.24 0.16 0.41 0.21 0.41 0.21 0.65 0.26 0.65 0.26 0.62 0.31 0.62 0.31 0.44 0.36 0.44 0.36 0.26 0.41 0.26 0.41 0.15 0.46 0.15 0.46 0.09 0.51 0.09 0.51 0.06 0.55 0.06 0.55 0.00 0.60 0.00 0.60 0.00 0.65 0.00 0.65 0.03 0.70 0.03 0.70 0.00 /
\end{sparkline}
 &   2.0 & 1000.0 &   2.0 \\ 
Seals,dolphins                      & Pomfret                             &   2.06e-13 & 
\begin{sparkline}{10}
\spark 0.10 0.00 0.10 0.29 0.15 0.29 0.15 0.38 0.21 0.38 0.21 0.76 0.26 0.76 0.26 0.62 0.32 0.62 0.32 0.32 0.37 0.32 0.37 0.26 0.43 0.26 0.43 0.03 0.48 0.03 0.48 0.12 0.54 0.12 0.54 0.15 0.59 0.15 0.59 0.00 /
\end{sparkline}
 &   2.0 & 1000.0 &   2.0 \\ 
Seals,dolphins                      & Large gonatid squid                 &   5.28e-14 & 
\begin{sparkline}{10}
\spark 0.09 0.00 0.09 0.18 0.15 0.18 0.15 0.74 0.22 0.74 0.22 0.71 0.28 0.71 0.28 0.41 0.34 0.41 0.34 0.44 0.40 0.44 0.40 0.26 0.47 0.26 0.47 0.15 0.53 0.15 0.53 0.06 0.59 0.06 0.59 0.00 /
\end{sparkline}
 &   2.0 & 1000.0 &   2.0 \\ 
Seals,dolphins                      & Sockeye,Pink                        &   1.13e-13 & 
\begin{sparkline}{10}
\spark 0.10 0.00 0.10 0.24 0.16 0.24 0.16 0.38 0.21 0.38 0.21 0.79 0.26 0.79 0.26 0.62 0.32 0.62 0.32 0.32 0.37 0.32 0.37 0.29 0.42 0.29 0.42 0.18 0.48 0.18 0.48 0.09 0.53 0.09 0.53 0.00 0.58 0.00 0.58 0.00 0.64 0.00 0.64 0.03 0.69 0.03 0.69 0.00 /
\end{sparkline}
 &   2.0 & 1000.0 &   2.0 \\ 
Seals,dolphins                      & Micronektonic squid                 &   1.02e-12 & 
\begin{sparkline}{10}
\spark 0.06 0.00 0.06 0.21 0.13 0.21 0.13 0.85 0.20 0.85 0.20 0.65 0.27 0.65 0.27 0.44 0.35 0.44 0.35 0.32 0.42 0.32 0.42 0.18 0.49 0.18 0.49 0.18 0.56 0.18 0.56 0.06 0.63 0.06 0.63 0.03 0.70 0.03 0.70 0.00 0.77 0.00 0.77 0.00 0.85 0.00 0.85 0.00 0.92 0.00 0.92 0.03 0.99 0.03 0.99 0.00 /
\end{sparkline}
 &   2.0 & 1000.0 &   2.0 \\ 
Seals,dolphins                      & Mesopelagic fish                    &   1.05e-12 & 
\begin{sparkline}{10}
\spark 0.08 0.00 0.08 0.29 0.15 0.29 0.15 0.74 0.22 0.74 0.22 0.71 0.28 0.71 0.28 0.50 0.35 0.50 0.35 0.18 0.42 0.18 0.42 0.26 0.49 0.26 0.49 0.18 0.56 0.18 0.56 0.03 0.63 0.03 0.63 0.00 0.69 0.00 0.69 0.03 0.76 0.03 0.76 0.03 0.83 0.03 0.83 0.00 /
\end{sparkline}
 &   2.0 & 1000.0 &   2.0 \\ 
Seals,dolphins                      & Pelagic forage fish                 &   6.77e-13 & 
\begin{sparkline}{10}
\spark 0.07 0.00 0.07 0.24 0.13 0.24 0.13 0.41 0.19 0.41 0.19 0.74 0.25 0.74 0.25 0.53 0.31 0.53 0.31 0.38 0.37 0.38 0.37 0.29 0.43 0.29 0.43 0.18 0.49 0.18 0.49 0.03 0.55 0.03 0.55 0.09 0.61 0.09 0.61 0.03 0.67 0.03 0.67 0.03 0.73 0.03 0.73 0.00 /
\end{sparkline}
 &   2.0 & 1000.0 &   2.0 \\ 
Seals,dolphins                      & Saury                               &   4.55e-13 & 
\begin{sparkline}{10}
\spark 0.09 0.00 0.09 0.50 0.16 0.50 0.16 0.91 0.24 0.91 0.24 0.38 0.31 0.38 0.31 0.50 0.38 0.50 0.38 0.41 0.46 0.41 0.46 0.09 0.53 0.09 0.53 0.03 0.61 0.03 0.61 0.06 0.68 0.06 0.68 0.03 0.76 0.03 0.76 0.03 0.83 0.03 0.83 0.00 /
\end{sparkline}
 &   2.0 & 1000.0 &   2.0 \\ 
Seals,dolphins                      & Chum salmon                         &   5.51e-14 & 
\begin{sparkline}{10}
\spark 0.10 0.00 0.10 0.24 0.15 0.24 0.15 0.35 0.20 0.35 0.20 0.62 0.25 0.62 0.25 0.59 0.30 0.59 0.30 0.59 0.35 0.59 0.35 0.21 0.40 0.21 0.40 0.15 0.45 0.15 0.45 0.12 0.51 0.12 0.51 0.06 0.56 0.06 0.56 0.00 0.61 0.00 0.61 0.00 0.66 0.00 0.66 0.03 0.71 0.03 0.71 0.00 /
\end{sparkline}
 &   2.0 & 1000.0 &   2.0 \\ 
Boreal clubhook squid               & Micronektonic squid                 &   1.10e-12 & 
\begin{sparkline}{10}
\spark 0.08 0.00 0.08 0.53 0.16 0.53 0.16 0.76 0.23 0.76 0.23 0.47 0.31 0.47 0.31 0.62 0.39 0.62 0.39 0.21 0.46 0.21 0.46 0.24 0.54 0.24 0.54 0.03 0.62 0.03 0.62 0.06 0.69 0.06 0.69 0.00 0.77 0.00 0.77 0.00 0.85 0.00 0.85 0.00 0.92 0.00 0.92 0.00 1.00 0.00 1.00 0.00 /
\end{sparkline}
 &   2.0 & 1000.0 &   2.0 \\ 
Boreal clubhook squid               & Pelagic forage fish                 &   1.34e-14 & 
\begin{sparkline}{10}
\spark 0.05 0.00 0.05 0.47 0.11 0.47 0.11 0.65 0.17 0.65 0.17 0.50 0.23 0.50 0.23 0.32 0.29 0.32 0.29 0.24 0.34 0.24 0.34 0.18 0.40 0.18 0.40 0.18 0.46 0.18 0.46 0.00 0.52 0.00 0.52 0.15 0.58 0.15 0.58 0.06 0.64 0.06 0.64 0.03 0.70 0.03 0.70 0.03 0.76 0.03 0.76 0.03 0.82 0.03 0.82 0.03 0.88 0.03 0.88 0.00 0.94 0.00 0.94 0.03 1.00 0.03 1.00 0.00 /
\end{sparkline}
 &   2.0 & 1000.0 &   2.0 \\ 
Fulmars                             & Micronektonic squid                 &   8.55e-14 & 
\begin{sparkline}{10}
\spark 0.13 0.00 0.13 0.32 0.18 0.32 0.18 0.76 0.24 0.76 0.24 0.62 0.29 0.62 0.29 0.44 0.34 0.44 0.34 0.44 0.39 0.44 0.39 0.15 0.44 0.15 0.44 0.12 0.50 0.12 0.50 0.00 0.55 0.00 0.55 0.03 0.60 0.03 0.60 0.03 0.65 0.03 0.65 0.00 0.70 0.00 0.70 0.03 0.76 0.03 0.76 0.00 /
\end{sparkline}
 &   2.0 & 1000.0 &   2.0 \\ 
Fulmars                             & Pelagic forage fish                 &   3.88e-15 & 
\begin{sparkline}{10}
\spark 0.07 0.00 0.07 0.59 0.15 0.59 0.15 0.65 0.22 0.65 0.22 0.62 0.30 0.62 0.30 0.41 0.37 0.41 0.37 0.21 0.45 0.21 0.45 0.21 0.53 0.21 0.53 0.09 0.60 0.09 0.60 0.09 0.68 0.09 0.68 0.00 0.75 0.00 0.75 0.06 0.83 0.06 0.83 0.03 0.90 0.03 0.90 0.00 /
\end{sparkline}
 &   2.0 & 1000.0 &   2.0 \\ 
Chinook,coho,steelhead              & Micronektonic squid                 &   4.05e-13 & 
\begin{sparkline}{10}
\spark 0.09 0.00 0.09 0.26 0.16 0.26 0.16 0.85 0.23 0.85 0.23 0.71 0.29 0.71 0.29 0.38 0.36 0.38 0.36 0.47 0.43 0.47 0.43 0.09 0.50 0.09 0.50 0.06 0.57 0.06 0.57 0.06 0.63 0.06 0.63 0.00 0.70 0.00 0.70 0.00 0.77 0.00 0.77 0.00 0.84 0.00 0.84 0.03 0.91 0.03 0.91 0.03 0.98 0.03 0.98 0.00 /
\end{sparkline}
 &   2.0 & 1000.0 &   2.0 \\ 
Chinook,coho,steelhead              & Mesopelagic fish                    &   7.61e-13 & 
\begin{sparkline}{10}
\spark 0.10 0.00 0.10 0.35 0.15 0.35 0.15 0.35 0.21 0.35 0.21 0.62 0.26 0.62 0.26 0.62 0.31 0.62 0.31 0.47 0.36 0.47 0.36 0.12 0.41 0.12 0.41 0.15 0.46 0.15 0.46 0.06 0.51 0.06 0.51 0.15 0.56 0.15 0.56 0.00 0.62 0.00 0.62 0.00 0.67 0.00 0.67 0.06 0.72 0.06 0.72 0.00 /
\end{sparkline}
 &   2.0 & 1000.0 &   2.0 \\ 
Chinook,coho,steelhead              & Pelagic forage fish                 &   7.49e-13 & 
\begin{sparkline}{10}
\spark 0.11 0.00 0.11 0.53 0.18 0.53 0.18 0.76 0.25 0.76 0.25 0.68 0.32 0.68 0.32 0.41 0.39 0.41 0.39 0.35 0.46 0.35 0.46 0.15 0.53 0.15 0.53 0.00 0.60 0.00 0.60 0.00 0.67 0.00 0.67 0.03 0.74 0.03 0.74 0.00 0.81 0.00 0.81 0.00 0.88 0.00 0.88 0.03 0.95 0.03 0.95 0.00 /
\end{sparkline}
 &   2.0 & 1000.0 &   2.0 \\ 
Chinook,coho,steelhead              & Mesozooplankton                     &   1.09e-13 & 
\begin{sparkline}{10}
\spark 0.11 0.00 0.11 0.26 0.16 0.26 0.16 0.59 0.22 0.59 0.22 0.62 0.27 0.62 0.27 0.68 0.33 0.68 0.33 0.29 0.38 0.29 0.38 0.26 0.44 0.26 0.44 0.09 0.49 0.09 0.49 0.06 0.55 0.06 0.55 0.06 0.60 0.06 0.60 0.00 0.66 0.00 0.66 0.03 0.71 0.03 0.71 0.00 /
\end{sparkline}
 &   2.0 & 1000.0 &   2.0 \\ 
Chinook,coho,steelhead              & Gelatinous zooplankton              &   2.86e-16 & 
\begin{sparkline}{10}
\spark 0.09 0.00 0.09 0.41 0.16 0.41 0.16 0.91 0.24 0.91 0.24 0.59 0.31 0.59 0.31 0.53 0.39 0.53 0.39 0.21 0.46 0.21 0.46 0.15 0.53 0.15 0.53 0.06 0.61 0.06 0.61 0.06 0.68 0.06 0.68 0.03 0.75 0.03 0.75 0.00 /
\end{sparkline}
 &   2.0 & 1000.0 &   2.0 \\ 
Chinook,coho,steelhead              & Copepods                            &   5.65e-15 & 
\begin{sparkline}{10}
\spark 0.12 0.00 0.12 0.26 0.16 0.26 0.16 0.32 0.21 0.32 0.21 0.44 0.25 0.44 0.25 0.62 0.30 0.62 0.30 0.59 0.34 0.59 0.34 0.35 0.39 0.35 0.39 0.09 0.43 0.09 0.43 0.18 0.48 0.18 0.48 0.03 0.52 0.03 0.52 0.00 0.57 0.00 0.57 0.03 0.61 0.03 0.61 0.03 0.66 0.03 0.66 0.00 /
\end{sparkline}
 &   2.0 & 1000.0 &   2.0 \\ 
Skuas,Jaegers                       & Pelagic forage fish                 &   5.31e-14 & 
\begin{sparkline}{10}
\spark 0.09 0.00 0.09 0.15 0.14 0.15 0.14 0.41 0.19 0.41 0.19 0.65 0.25 0.65 0.25 0.76 0.30 0.76 0.30 0.35 0.35 0.35 0.35 0.26 0.40 0.26 0.40 0.09 0.46 0.09 0.46 0.03 0.51 0.03 0.51 0.15 0.56 0.15 0.56 0.09 0.61 0.09 0.61 0.00 /
\end{sparkline}
 &   2.0 & 1000.0 &   2.0 \\ 
Skuas,Jaegers                       & Saury                               &   5.20e-14 & 
\begin{sparkline}{10}
\spark 0.08 0.00 0.08 0.12 0.13 0.12 0.13 0.26 0.17 0.26 0.17 0.32 0.22 0.32 0.22 0.76 0.26 0.76 0.26 0.59 0.31 0.59 0.31 0.29 0.35 0.29 0.35 0.12 0.40 0.12 0.40 0.24 0.44 0.24 0.44 0.12 0.49 0.12 0.49 0.03 0.53 0.03 0.53 0.00 0.58 0.00 0.58 0.00 0.62 0.00 0.62 0.03 0.67 0.03 0.67 0.00 0.71 0.00 0.71 0.00 0.76 0.00 0.76 0.06 0.80 0.06 0.80 0.00 /
\end{sparkline}
 &   2.0 & 1000.0 &   2.0 \\ 
Pomfret                             & Micronektonic squid                 &   7.85e-12 & 
\begin{sparkline}{10}
\spark 0.08 0.00 0.08 0.26 0.13 0.26 0.13 0.35 0.19 0.35 0.19 0.65 0.24 0.65 0.24 0.53 0.30 0.53 0.30 0.53 0.35 0.53 0.35 0.21 0.40 0.21 0.40 0.15 0.46 0.15 0.46 0.09 0.51 0.09 0.51 0.06 0.57 0.06 0.57 0.00 0.62 0.00 0.62 0.06 0.68 0.06 0.68 0.03 0.73 0.03 0.73 0.00 0.78 0.00 0.78 0.00 0.84 0.00 0.84 0.00 0.89 0.00 0.89 0.00 0.95 0.00 0.95 0.00 1.00 0.00 1.00 0.00 /
\end{sparkline}
 &   2.0 & 1000.0 &   2.0 \\ 
Pomfret                             & Mesopelagic fish                    &   8.35e-13 & 
\begin{sparkline}{10}
\spark 0.05 0.00 0.05 0.32 0.11 0.32 0.11 0.29 0.16 0.29 0.16 0.56 0.22 0.56 0.22 0.62 0.28 0.62 0.28 0.24 0.33 0.24 0.33 0.32 0.39 0.32 0.39 0.18 0.44 0.18 0.44 0.09 0.50 0.09 0.50 0.18 0.55 0.18 0.55 0.00 0.61 0.00 0.61 0.09 0.67 0.09 0.67 0.03 0.72 0.03 0.72 0.00 0.78 0.00 0.78 0.00 0.83 0.00 0.83 0.00 0.89 0.00 0.89 0.00 0.94 0.00 0.94 0.00 1.00 0.00 1.00 0.00 /
\end{sparkline}
 &   2.0 & 1000.0 &   2.0 \\ 
Pomfret                             & Saury                               &   3.99e-13 & 
\begin{sparkline}{10}
\spark 0.04 0.00 0.04 0.18 0.10 0.18 0.10 0.50 0.15 0.50 0.15 0.53 0.21 0.53 0.21 0.44 0.27 0.44 0.27 0.41 0.32 0.41 0.32 0.26 0.38 0.26 0.38 0.12 0.44 0.12 0.44 0.21 0.49 0.21 0.49 0.12 0.55 0.12 0.55 0.03 0.61 0.03 0.61 0.09 0.66 0.09 0.66 0.00 0.72 0.00 0.72 0.00 0.77 0.00 0.77 0.00 0.83 0.00 0.83 0.00 0.89 0.00 0.89 0.00 0.94 0.00 0.94 0.03 1.00 0.03 1.00 0.00 /
\end{sparkline}
 &   2.0 & 1000.0 &   2.0 \\ 
Pomfret                             & Chaetognaths                        &   1.06e-13 & 
\begin{sparkline}{10}
\spark 0.04 0.00 0.04 0.21 0.11 0.21 0.11 0.50 0.17 0.50 0.17 0.65 0.24 0.65 0.24 0.56 0.30 0.56 0.30 0.38 0.36 0.38 0.36 0.26 0.43 0.26 0.43 0.12 0.49 0.12 0.49 0.12 0.55 0.12 0.55 0.03 0.62 0.03 0.62 0.00 0.68 0.00 0.68 0.00 0.75 0.00 0.75 0.00 0.81 0.00 0.81 0.03 0.87 0.03 0.87 0.03 0.94 0.03 0.94 0.00 1.00 0.00 1.00 0.00 /
\end{sparkline}
 &   2.0 & 1000.0 &   2.0 \\ 
Pomfret                             & Predatory zooplankton               &   1.04e-13 & 
\begin{sparkline}{10}
\spark 0.06 0.00 0.06 0.41 0.13 0.41 0.13 0.68 0.21 0.68 0.21 0.68 0.28 0.68 0.28 0.38 0.35 0.38 0.35 0.38 0.42 0.38 0.42 0.12 0.49 0.12 0.49 0.12 0.57 0.12 0.57 0.06 0.64 0.06 0.64 0.03 0.71 0.03 0.71 0.00 0.78 0.00 0.78 0.03 0.86 0.03 0.86 0.00 0.93 0.00 0.93 0.03 1.00 0.03 1.00 0.00 /
\end{sparkline}
 &   2.0 & 1000.0 &   2.0 \\ 
Pomfret                             & Sergestid shrimp                    &   1.04e-13 & 
\begin{sparkline}{10}
\spark 0.03 0.00 0.03 0.26 0.10 0.26 0.10 0.53 0.17 0.53 0.17 0.85 0.24 0.85 0.24 0.41 0.31 0.41 0.31 0.21 0.38 0.21 0.38 0.24 0.45 0.24 0.45 0.12 0.52 0.12 0.52 0.03 0.59 0.03 0.59 0.12 0.66 0.12 0.66 0.06 0.72 0.06 0.72 0.06 0.79 0.06 0.79 0.00 0.86 0.00 0.86 0.00 0.93 0.00 0.93 0.03 1.00 0.03 1.00 0.00 /
\end{sparkline}
 &   2.0 & 1000.0 &   2.0 \\ 
Pomfret                             & Mesozooplankton                     &   9.59e-13 & 
\begin{sparkline}{10}
\spark 0.06 0.00 0.06 0.41 0.12 0.41 0.12 0.50 0.19 0.50 0.19 0.74 0.26 0.74 0.26 0.38 0.33 0.38 0.33 0.38 0.40 0.38 0.40 0.21 0.47 0.21 0.47 0.03 0.53 0.03 0.53 0.12 0.60 0.12 0.60 0.06 0.67 0.06 0.67 0.00 0.74 0.00 0.74 0.06 0.81 0.06 0.81 0.00 0.88 0.00 0.88 0.06 0.94 0.06 0.94 0.00 /
\end{sparkline}
 &   2.0 & 1000.0 &   2.0 \\ 
Pomfret                             & Copepods                            &   1.06e-13 & 
\begin{sparkline}{10}
\spark 0.08 0.00 0.08 0.29 0.13 0.29 0.13 0.47 0.19 0.47 0.19 0.62 0.24 0.62 0.24 0.65 0.30 0.65 0.30 0.29 0.35 0.29 0.35 0.12 0.40 0.12 0.40 0.18 0.46 0.18 0.46 0.09 0.51 0.09 0.51 0.06 0.57 0.06 0.57 0.03 0.62 0.03 0.62 0.06 0.67 0.06 0.67 0.00 0.73 0.00 0.73 0.06 0.78 0.06 0.78 0.00 0.84 0.00 0.84 0.00 0.89 0.00 0.89 0.00 0.95 0.00 0.95 0.00 1.00 0.00 1.00 0.00 /
\end{sparkline}
 &   2.0 & 1000.0 &   2.0 \\ 
Puffins,Shearwaters,Storm Petrels   & Micronektonic squid                 &   2.06e-13 & 
\begin{sparkline}{10}
\spark 0.10 0.00 0.10 0.65 0.17 0.65 0.17 0.79 0.25 0.79 0.25 0.59 0.33 0.59 0.33 0.38 0.41 0.38 0.41 0.32 0.48 0.32 0.48 0.12 0.56 0.12 0.56 0.00 0.64 0.00 0.64 0.09 0.72 0.09 0.72 0.00 /
\end{sparkline}
 &   2.0 & 1000.0 &   2.0 \\ 
Puffins,Shearwaters,Storm Petrels   & Pelagic forage fish                 &   1.78e-13 & 
\begin{sparkline}{10}
\spark 0.09 0.00 0.09 0.24 0.14 0.24 0.14 0.41 0.19 0.41 0.19 0.74 0.24 0.74 0.24 0.56 0.29 0.56 0.29 0.32 0.35 0.32 0.35 0.24 0.40 0.24 0.40 0.15 0.45 0.15 0.45 0.00 0.50 0.00 0.50 0.09 0.55 0.09 0.55 0.09 0.61 0.09 0.61 0.06 0.66 0.06 0.66 0.03 0.71 0.03 0.71 0.03 0.76 0.03 0.76 0.00 /
\end{sparkline}
 &   2.0 & 1000.0 &   2.0 \\ 
Puffins,Shearwaters,Storm Petrels   & Saury                               &   1.65e-13 & 
\begin{sparkline}{10}
\spark 0.09 0.00 0.09 0.18 0.14 0.18 0.14 0.56 0.20 0.56 0.20 0.68 0.26 0.68 0.26 0.56 0.32 0.56 0.32 0.53 0.37 0.53 0.37 0.24 0.43 0.24 0.43 0.03 0.49 0.03 0.49 0.03 0.55 0.03 0.55 0.09 0.61 0.09 0.61 0.00 0.66 0.00 0.66 0.06 0.72 0.06 0.72 0.00 /
\end{sparkline}
 &   2.0 & 1000.0 &   2.0 \\ 
Puffins,Shearwaters,Storm Petrels   & Mesozooplankton                     &   5.68e-14 & 
\begin{sparkline}{10}
\spark 0.09 0.00 0.09 0.15 0.15 0.15 0.15 0.65 0.21 0.65 0.21 0.85 0.27 0.85 0.27 0.47 0.33 0.47 0.33 0.38 0.39 0.38 0.39 0.18 0.45 0.18 0.45 0.09 0.52 0.09 0.52 0.15 0.58 0.15 0.58 0.03 0.64 0.03 0.64 0.00 /
\end{sparkline}
 &   2.0 & 1000.0 &   2.0 \\ 
Puffins,Shearwaters,Storm Petrels   & Copepods                            &   4.50e-14 & 
\begin{sparkline}{10}
\spark 0.12 0.00 0.12 0.29 0.17 0.29 0.17 0.38 0.21 0.38 0.21 0.71 0.26 0.71 0.26 0.56 0.30 0.56 0.30 0.38 0.35 0.38 0.35 0.21 0.40 0.21 0.40 0.15 0.44 0.15 0.44 0.06 0.49 0.06 0.49 0.15 0.53 0.15 0.53 0.00 0.58 0.00 0.58 0.06 0.63 0.06 0.63 0.00 /
\end{sparkline}
 &   2.0 & 1000.0 &   2.0 \\ 
Kittiwakes                          & Pelagic forage fish                 &   3.02e-14 & 
\begin{sparkline}{10}
\spark 0.09 0.00 0.09 0.35 0.16 0.35 0.16 0.74 0.23 0.74 0.23 0.71 0.30 0.71 0.30 0.47 0.37 0.47 0.37 0.32 0.44 0.32 0.44 0.21 0.51 0.21 0.51 0.06 0.58 0.06 0.58 0.03 0.65 0.03 0.65 0.06 0.72 0.06 0.72 0.00 /
\end{sparkline}
 &   2.0 & 1000.0 &   2.0 \\ 
Kittiwakes                          & Saury                               &   3.05e-14 & 
\begin{sparkline}{10}
\spark 0.11 0.00 0.11 0.56 0.18 0.56 0.18 0.56 0.24 0.56 0.24 0.88 0.31 0.88 0.31 0.44 0.37 0.44 0.37 0.24 0.44 0.24 0.44 0.06 0.50 0.06 0.50 0.06 0.56 0.06 0.56 0.06 0.63 0.06 0.63 0.09 0.69 0.09 0.69 0.00 /
\end{sparkline}
 &   2.0 & 1000.0 &   2.0 \\ 
Kittiwakes                          & Mesozooplankton                     &   8.43e-15 & 
\begin{sparkline}{10}
\spark 0.09 0.00 0.09 0.24 0.15 0.24 0.15 0.68 0.21 0.68 0.21 0.56 0.27 0.56 0.27 0.68 0.33 0.68 0.33 0.29 0.39 0.29 0.39 0.21 0.45 0.21 0.45 0.15 0.51 0.15 0.51 0.06 0.57 0.06 0.57 0.00 0.63 0.00 0.63 0.03 0.69 0.03 0.69 0.06 0.75 0.06 0.75 0.00 /
\end{sparkline}
 &   2.0 & 1000.0 &   2.0 \\ 
Kittiwakes                          & Copepods                            &   6.57e-15 & 
\begin{sparkline}{10}
\spark 0.12 0.00 0.12 0.47 0.18 0.47 0.18 0.79 0.24 0.79 0.24 0.62 0.30 0.62 0.30 0.44 0.35 0.44 0.35 0.18 0.41 0.18 0.41 0.26 0.47 0.26 0.47 0.03 0.53 0.03 0.53 0.00 0.59 0.00 0.59 0.09 0.65 0.09 0.65 0.03 0.71 0.03 0.71 0.03 0.77 0.03 0.77 0.00 /
\end{sparkline}
 &   2.0 & 1000.0 &   2.0 \\ 
Large gonatid squid                 & Micronektonic squid                 &   8.74e-13 & 
\begin{sparkline}{10}
\spark 0.07 0.00 0.07 0.59 0.13 0.59 0.13 0.68 0.19 0.68 0.19 0.53 0.25 0.53 0.25 0.29 0.31 0.29 0.31 0.26 0.38 0.26 0.38 0.15 0.44 0.15 0.44 0.06 0.50 0.06 0.50 0.09 0.56 0.09 0.56 0.09 0.63 0.09 0.63 0.03 0.69 0.03 0.69 0.06 0.75 0.06 0.75 0.00 0.81 0.00 0.81 0.00 0.88 0.00 0.88 0.00 0.94 0.00 0.94 0.00 1.00 0.00 1.00 0.00 /
\end{sparkline}
 &   2.0 & 1000.0 &   2.0 \\ 
Large gonatid squid                 & Pelagic forage fish                 &   2.83e-14 & 
\begin{sparkline}{10}
\spark 0.05 0.00 0.05 0.41 0.12 0.41 0.12 0.74 0.18 0.74 0.18 0.56 0.24 0.56 0.24 0.32 0.31 0.32 0.31 0.29 0.37 0.29 0.37 0.18 0.43 0.18 0.43 0.09 0.50 0.09 0.50 0.06 0.56 0.06 0.56 0.06 0.62 0.06 0.62 0.06 0.68 0.06 0.68 0.03 0.75 0.03 0.75 0.03 0.81 0.03 0.81 0.03 0.87 0.03 0.87 0.00 0.94 0.00 0.94 0.00 1.00 0.00 1.00 0.00 /
\end{sparkline}
 &   2.0 & 1000.0 &   2.0 \\ 
Large gonatid squid                 & Chaetognaths                        &   1.25e-13 & 
\begin{sparkline}{10}
\spark 0.06 0.00 0.06 0.65 0.13 0.65 0.13 0.56 0.20 0.56 0.20 0.53 0.28 0.53 0.28 0.32 0.35 0.32 0.35 0.35 0.42 0.35 0.42 0.24 0.49 0.24 0.49 0.03 0.57 0.03 0.57 0.09 0.64 0.09 0.64 0.00 0.71 0.00 0.71 0.09 0.78 0.09 0.78 0.03 0.86 0.03 0.86 0.00 0.93 0.00 0.93 0.03 1.00 0.03 1.00 0.00 /
\end{sparkline}
 &   2.0 & 1000.0 &   2.0 \\ 
Large gonatid squid                 & Predatory zooplankton               &   1.04e-13 & 
\begin{sparkline}{10}
\spark 0.04 0.00 0.04 0.24 0.08 0.24 0.08 0.32 0.12 0.32 0.12 0.53 0.16 0.53 0.16 0.24 0.20 0.24 0.20 0.41 0.24 0.41 0.24 0.18 0.28 0.18 0.28 0.24 0.32 0.24 0.32 0.15 0.36 0.15 0.36 0.09 0.40 0.09 0.40 0.15 0.44 0.15 0.44 0.09 0.48 0.09 0.48 0.03 0.52 0.03 0.52 0.03 0.56 0.03 0.56 0.03 0.60 0.03 0.60 0.03 0.64 0.03 0.64 0.03 0.68 0.03 0.68 0.06 0.72 0.06 0.72 0.03 0.76 0.03 0.76 0.00 0.80 0.00 0.80 0.00 0.84 0.00 0.84 0.03 0.88 0.03 0.88 0.00 0.92 0.00 0.92 0.00 0.96 0.00 0.96 0.00 1.00 0.00 1.00 0.00 /
\end{sparkline}
 &   2.0 & 1000.0 &   2.0 \\ 
Large gonatid squid                 & Sergestid shrimp                    &   9.54e-14 & 
\begin{sparkline}{10}
\spark 0.03 0.00 0.03 0.18 0.09 0.18 0.09 0.56 0.15 0.56 0.15 0.71 0.20 0.71 0.20 0.50 0.26 0.50 0.26 0.18 0.32 0.18 0.32 0.03 0.37 0.03 0.37 0.24 0.43 0.24 0.43 0.15 0.49 0.15 0.49 0.12 0.54 0.12 0.54 0.06 0.60 0.06 0.60 0.03 0.66 0.03 0.66 0.03 0.72 0.03 0.72 0.03 0.77 0.03 0.77 0.09 0.83 0.09 0.83 0.00 0.89 0.00 0.89 0.00 0.94 0.00 0.94 0.03 1.00 0.03 1.00 0.00 /
\end{sparkline}
 &   2.0 & 1000.0 &   2.0 \\ 
Large gonatid squid                 & Mesozooplankton                     &   8.34e-13 & 
\begin{sparkline}{10}
\spark 0.06 0.00 0.06 0.44 0.13 0.44 0.13 0.85 0.20 0.85 0.20 0.59 0.28 0.59 0.28 0.35 0.35 0.35 0.35 0.18 0.42 0.18 0.42 0.12 0.49 0.12 0.49 0.18 0.57 0.18 0.57 0.03 0.64 0.03 0.64 0.09 0.71 0.09 0.71 0.00 0.78 0.00 0.78 0.06 0.86 0.06 0.86 0.03 0.93 0.03 0.93 0.00 1.00 0.00 1.00 0.00 /
\end{sparkline}
 &   2.0 & 1000.0 &   2.0 \\ 
Large gonatid squid                 & Copepods                            &   6.58e-13 & 
\begin{sparkline}{10}
\spark 0.05 0.00 0.05 0.47 0.13 0.47 0.13 0.94 0.22 0.94 0.22 0.56 0.31 0.56 0.31 0.29 0.39 0.29 0.39 0.38 0.48 0.38 0.48 0.09 0.57 0.09 0.57 0.03 0.65 0.03 0.65 0.00 0.74 0.00 0.74 0.06 0.83 0.06 0.83 0.06 0.91 0.06 0.91 0.00 1.00 0.00 1.00 0.00 /
\end{sparkline}
 &   2.0 & 1000.0 &   2.0 \\ 
Sockeye,Pink                        & Micronektonic squid                 &   1.16e-12 & 
\begin{sparkline}{10}
\spark 0.10 0.00 0.10 0.32 0.16 0.32 0.16 0.44 0.21 0.44 0.21 0.65 0.26 0.65 0.26 0.44 0.31 0.44 0.31 0.56 0.37 0.56 0.37 0.18 0.42 0.18 0.42 0.15 0.47 0.15 0.47 0.12 0.52 0.12 0.52 0.00 0.57 0.00 0.57 0.00 0.63 0.00 0.63 0.00 0.68 0.00 0.68 0.09 0.73 0.09 0.73 0.00 /
\end{sparkline}
 &   2.0 & 1000.0 &   2.0 \\ 
Sockeye,Pink                        & Mesopelagic fish                    &   1.73e-12 & 
\begin{sparkline}{10}
\spark 0.09 0.00 0.09 0.59 0.17 0.59 0.17 0.82 0.24 0.82 0.24 0.59 0.32 0.59 0.32 0.32 0.39 0.32 0.39 0.21 0.47 0.21 0.47 0.26 0.54 0.26 0.54 0.03 0.62 0.03 0.62 0.09 0.69 0.09 0.69 0.00 0.77 0.00 0.77 0.00 0.84 0.00 0.84 0.00 0.92 0.00 0.92 0.03 0.99 0.03 0.99 0.00 /
\end{sparkline}
 &   2.0 & 1000.0 &   2.0 \\ 
Sockeye,Pink                        & Pelagic forage fish                 &   1.60e-12 & 
\begin{sparkline}{10}
\spark 0.13 0.00 0.13 0.59 0.19 0.59 0.19 0.82 0.25 0.82 0.25 0.56 0.31 0.56 0.31 0.35 0.37 0.35 0.37 0.26 0.43 0.26 0.43 0.09 0.50 0.09 0.50 0.12 0.56 0.12 0.56 0.09 0.62 0.09 0.62 0.06 0.68 0.06 0.68 0.00 /
\end{sparkline}
 &   2.0 & 1000.0 &   2.0 \\ 
Sockeye,Pink                        & Predatory zooplankton               &   2.09e-14 & 
\begin{sparkline}{10}
\spark 0.08 0.00 0.08 0.18 0.14 0.18 0.14 0.79 0.21 0.79 0.21 0.56 0.28 0.56 0.28 0.59 0.34 0.59 0.34 0.44 0.41 0.44 0.41 0.09 0.48 0.09 0.48 0.21 0.55 0.21 0.55 0.03 0.61 0.03 0.61 0.03 0.68 0.03 0.68 0.00 0.75 0.00 0.75 0.03 0.82 0.03 0.82 0.00 /
\end{sparkline}
 &   2.0 & 1000.0 &   2.0 \\ 
Sockeye,Pink                        & Mesozooplankton                     &   1.01e-11 & 
\begin{sparkline}{10}
\spark 0.14 0.00 0.14 0.26 0.19 0.26 0.19 0.41 0.23 0.41 0.23 0.74 0.27 0.74 0.27 0.62 0.32 0.62 0.32 0.32 0.36 0.32 0.36 0.41 0.40 0.41 0.40 0.09 0.44 0.09 0.44 0.06 0.49 0.06 0.49 0.00 0.53 0.00 0.53 0.00 0.57 0.00 0.57 0.00 0.62 0.00 0.62 0.00 0.66 0.00 0.66 0.03 0.70 0.03 0.70 0.00 /
\end{sparkline}
 &   2.0 & 1000.0 &   2.0 \\ 
Sockeye,Pink                        & Gelatinous zooplankton              &   4.17e-13 & 
\begin{sparkline}{10}
\spark 0.09 0.00 0.09 0.24 0.15 0.24 0.15 0.44 0.20 0.44 0.20 0.79 0.26 0.79 0.26 0.50 0.31 0.50 0.31 0.32 0.36 0.32 0.36 0.29 0.42 0.29 0.42 0.15 0.47 0.15 0.47 0.09 0.52 0.09 0.52 0.00 0.58 0.00 0.58 0.06 0.63 0.06 0.63 0.00 0.69 0.00 0.69 0.00 0.74 0.00 0.74 0.00 0.79 0.00 0.79 0.06 0.85 0.06 0.85 0.00 /
\end{sparkline}
 &   2.0 & 1000.0 &   2.0 \\ 
Sockeye,Pink                        & Copepods                            &   4.87e-13 & 
\begin{sparkline}{10}
\spark 0.11 0.00 0.11 0.24 0.16 0.24 0.16 0.44 0.21 0.44 0.21 0.79 0.26 0.79 0.26 0.50 0.31 0.50 0.31 0.38 0.36 0.38 0.36 0.29 0.41 0.29 0.41 0.12 0.47 0.12 0.47 0.06 0.52 0.06 0.52 0.03 0.57 0.03 0.57 0.09 0.62 0.09 0.62 0.00 /
\end{sparkline}
 &   2.0 & 1000.0 &   2.0 \\ 
Fin,sei whales                      & Neon flying squid                   &   4.63e-14 & 
\begin{sparkline}{10}
\spark 0.10 0.00 0.10 0.53 0.16 0.53 0.16 0.59 0.22 0.59 0.22 0.50 0.29 0.50 0.29 0.74 0.35 0.74 0.35 0.09 0.41 0.09 0.41 0.21 0.48 0.21 0.48 0.09 0.54 0.09 0.54 0.06 0.60 0.06 0.60 0.09 0.67 0.09 0.67 0.06 0.73 0.06 0.73 0.00 /
\end{sparkline}
 &   2.0 & 1000.0 &   2.0 \\ 
Fin,sei whales                      & Boreal clubhook squid               &   1.20e-15 & 
\begin{sparkline}{10}
\spark 0.07 0.00 0.07 0.44 0.15 0.44 0.15 0.65 0.22 0.65 0.22 0.68 0.30 0.68 0.30 0.50 0.38 0.50 0.38 0.32 0.45 0.32 0.45 0.15 0.53 0.15 0.53 0.15 0.60 0.15 0.60 0.06 0.68 0.06 0.68 0.00 /
\end{sparkline}
 &   2.0 & 1000.0 &   2.0 \\ 
Fin,sei whales                      & Chinook,coho,steelhead              &   6.25e-16 & 
\begin{sparkline}{10}
\spark 0.12 0.00 0.12 0.32 0.17 0.32 0.17 0.47 0.23 0.47 0.23 0.76 0.28 0.76 0.28 0.53 0.33 0.53 0.33 0.38 0.38 0.38 0.38 0.15 0.43 0.15 0.43 0.15 0.48 0.15 0.48 0.18 0.54 0.18 0.54 0.00 /
\end{sparkline}
 &   2.0 & 1000.0 &   2.0 \\ 
Fin,sei whales                      & Pomfret                             &   5.86e-15 & 
\begin{sparkline}{10}
\spark 0.11 0.00 0.11 0.35 0.17 0.35 0.17 0.71 0.23 0.71 0.23 0.65 0.29 0.65 0.29 0.50 0.36 0.50 0.36 0.38 0.42 0.38 0.42 0.29 0.48 0.29 0.48 0.00 0.54 0.00 0.54 0.03 0.61 0.03 0.61 0.00 0.67 0.00 0.67 0.03 0.73 0.03 0.73 0.00 /
\end{sparkline}
 &   2.0 & 1000.0 &   2.0 \\ 
Fin,sei whales                      & Large gonatid squid                 &   2.98e-15 & 
\begin{sparkline}{10}
\spark 0.09 0.00 0.09 0.35 0.15 0.35 0.15 0.59 0.20 0.59 0.20 0.56 0.26 0.56 0.26 0.47 0.32 0.47 0.32 0.50 0.37 0.50 0.37 0.15 0.43 0.15 0.43 0.06 0.49 0.06 0.49 0.09 0.54 0.09 0.54 0.03 0.60 0.03 0.60 0.03 0.66 0.03 0.66 0.00 0.72 0.00 0.72 0.06 0.77 0.06 0.77 0.03 0.83 0.03 0.83 0.00 0.89 0.00 0.89 0.00 0.94 0.00 0.94 0.00 1.00 0.00 1.00 0.00 /
\end{sparkline}
 &   2.0 & 1000.0 &   2.0 \\ 
Fin,sei whales                      & Sockeye,Pink                        &   3.04e-15 & 
\begin{sparkline}{10}
\spark 0.11 0.00 0.11 0.15 0.16 0.15 0.16 0.41 0.20 0.41 0.20 0.68 0.25 0.68 0.25 0.56 0.30 0.56 0.30 0.50 0.35 0.50 0.35 0.29 0.40 0.29 0.40 0.06 0.44 0.06 0.44 0.15 0.49 0.15 0.49 0.15 0.54 0.15 0.54 0.00 /
\end{sparkline}
 &   2.0 & 1000.0 &   2.0 \\ 
Fin,sei whales                      & Micronektonic squid                 &   5.00e-14 & 
\begin{sparkline}{10}
\spark 0.08 0.00 0.08 0.44 0.15 0.44 0.15 0.59 0.23 0.59 0.23 0.76 0.30 0.76 0.30 0.38 0.37 0.38 0.37 0.44 0.45 0.44 0.45 0.18 0.52 0.18 0.52 0.09 0.59 0.09 0.59 0.03 0.66 0.03 0.66 0.03 0.74 0.03 0.74 0.00 /
\end{sparkline}
 &   2.0 & 1000.0 &   2.0 \\ 
Fin,sei whales                      & Mesopelagic fish                    &   1.22e-13 & 
\begin{sparkline}{10}
\spark 0.07 0.00 0.07 0.18 0.13 0.18 0.13 0.65 0.18 0.65 0.18 0.59 0.24 0.59 0.24 0.44 0.30 0.44 0.30 0.44 0.36 0.44 0.36 0.18 0.42 0.18 0.42 0.21 0.48 0.21 0.48 0.03 0.53 0.03 0.53 0.12 0.59 0.12 0.59 0.03 0.65 0.03 0.65 0.03 0.71 0.03 0.71 0.00 0.77 0.00 0.77 0.00 0.83 0.00 0.83 0.00 0.88 0.00 0.88 0.03 0.94 0.03 0.94 0.00 1.00 0.00 1.00 0.00 /
\end{sparkline}
 &   2.0 & 1000.0 &   2.0 \\ 
Fin,sei whales                      & Pelagic forage fish                 &   1.46e-13 & 
\begin{sparkline}{10}
\spark 0.08 0.00 0.08 0.41 0.15 0.41 0.15 0.74 0.22 0.74 0.22 0.62 0.30 0.62 0.30 0.44 0.37 0.44 0.37 0.35 0.44 0.35 0.44 0.21 0.51 0.21 0.51 0.09 0.59 0.09 0.59 0.06 0.66 0.06 0.66 0.03 0.73 0.03 0.73 0.00 /
\end{sparkline}
 &   2.0 & 1000.0 &   2.0 \\ 
Fin,sei whales                      & Saury                               &   1.35e-14 & 
\begin{sparkline}{10}
\spark 0.06 0.00 0.06 0.24 0.13 0.24 0.13 0.56 0.20 0.56 0.20 0.74 0.26 0.74 0.26 0.53 0.33 0.53 0.33 0.29 0.40 0.29 0.40 0.29 0.47 0.29 0.47 0.12 0.53 0.12 0.53 0.03 0.60 0.03 0.60 0.15 0.67 0.15 0.67 0.00 /
\end{sparkline}
 &   2.0 & 1000.0 &   2.0 \\ 
Fin,sei whales                      & Chum salmon                         &   1.46e-15 & 
\begin{sparkline}{10}
\spark 0.10 0.00 0.10 0.15 0.15 0.15 0.15 0.29 0.19 0.29 0.19 0.53 0.23 0.53 0.23 0.50 0.28 0.50 0.28 0.53 0.32 0.53 0.32 0.47 0.36 0.47 0.36 0.09 0.41 0.09 0.41 0.18 0.45 0.18 0.45 0.09 0.50 0.09 0.50 0.06 0.54 0.06 0.54 0.06 0.58 0.06 0.58 0.00 /
\end{sparkline}
 &   2.0 & 1000.0 &   2.0 \\ 
Fin,sei whales                      & Chaetognaths                        &   1.02e-13 & 
\begin{sparkline}{10}
\spark 0.07 0.00 0.07 0.18 0.13 0.18 0.13 0.47 0.19 0.47 0.19 0.71 0.25 0.71 0.25 0.59 0.31 0.59 0.31 0.35 0.37 0.35 0.37 0.24 0.43 0.24 0.43 0.18 0.49 0.18 0.49 0.09 0.55 0.09 0.55 0.03 0.61 0.03 0.61 0.12 0.67 0.12 0.67 0.00 /
\end{sparkline}
 &   2.0 & 1000.0 &   2.0 \\ 
Fin,sei whales                      & Predatory zooplankton               &   8.39e-14 & 
\begin{sparkline}{10}
\spark 0.07 0.00 0.07 0.21 0.13 0.21 0.13 0.56 0.20 0.56 0.20 0.74 0.27 0.74 0.27 0.56 0.33 0.56 0.33 0.32 0.40 0.32 0.40 0.35 0.46 0.35 0.46 0.06 0.53 0.06 0.53 0.03 0.60 0.03 0.60 0.12 0.66 0.12 0.66 0.00 /
\end{sparkline}
 &   2.0 & 1000.0 &   2.0 \\ 
Fin,sei whales                      & Sergestid shrimp                    &   7.50e-14 & 
\begin{sparkline}{10}
\spark 0.07 0.00 0.07 0.24 0.14 0.24 0.14 0.65 0.21 0.65 0.21 0.56 0.27 0.56 0.27 0.56 0.34 0.56 0.34 0.50 0.40 0.50 0.40 0.21 0.47 0.21 0.47 0.12 0.54 0.12 0.54 0.09 0.60 0.09 0.60 0.03 0.67 0.03 0.67 0.00 /
\end{sparkline}
 &   2.0 & 1000.0 &   2.0 \\ 
Fin,sei whales                      & Mesozooplankton                     &   7.16e-13 & 
\begin{sparkline}{10}
\spark 0.09 0.00 0.09 0.09 0.14 0.09 0.14 0.44 0.19 0.44 0.19 0.65 0.25 0.65 0.25 0.71 0.30 0.71 0.30 0.50 0.36 0.50 0.36 0.21 0.41 0.21 0.41 0.21 0.47 0.21 0.47 0.06 0.52 0.06 0.52 0.06 0.57 0.06 0.57 0.00 0.63 0.00 0.63 0.03 0.68 0.03 0.68 0.00 /
\end{sparkline}
 &   2.0 & 1000.0 &   2.0 \\ 
Fin,sei whales                      & Copepods                            &   5.52e-13 & 
\begin{sparkline}{10}
\spark 0.11 0.00 0.11 0.35 0.16 0.35 0.16 0.38 0.21 0.38 0.21 0.79 0.27 0.79 0.27 0.47 0.32 0.47 0.32 0.35 0.37 0.35 0.37 0.21 0.42 0.21 0.42 0.12 0.47 0.12 0.47 0.12 0.52 0.12 0.52 0.15 0.57 0.15 0.57 0.00 /
\end{sparkline}
 &   2.0 & 1000.0 &   2.0 \\ 
Micronektonic squid                 & Micronektonic squid                 &   1.12e-11 & 
\begin{sparkline}{10}
\spark 0.03 0.00 0.03 0.09 0.07 0.09 0.07 0.35 0.10 0.35 0.10 0.38 0.14 0.38 0.14 0.26 0.17 0.26 0.17 0.41 0.21 0.41 0.21 0.35 0.24 0.35 0.24 0.18 0.28 0.18 0.28 0.18 0.32 0.18 0.32 0.06 0.35 0.06 0.35 0.12 0.39 0.12 0.39 0.03 0.42 0.03 0.42 0.12 0.46 0.12 0.46 0.06 0.50 0.06 0.50 0.03 0.53 0.03 0.53 0.06 0.57 0.06 0.57 0.06 0.60 0.06 0.60 0.00 0.64 0.00 0.64 0.03 0.68 0.03 0.68 0.00 0.71 0.00 0.71 0.00 0.75 0.00 0.75 0.06 0.78 0.06 0.78 0.00 0.82 0.00 0.82 0.00 0.86 0.00 0.86 0.00 0.89 0.00 0.89 0.00 0.93 0.00 0.93 0.00 0.96 0.00 0.96 0.03 1.00 0.03 1.00 0.00 /
\end{sparkline}
 &   2.0 & 1000.0 &   2.0 \\ 
Micronektonic squid                 & Chaetognaths                        &   1.32e-11 & 
\begin{sparkline}{10}
\spark 0.05 0.00 0.05 0.21 0.09 0.21 0.09 0.26 0.13 0.26 0.13 0.47 0.18 0.47 0.18 0.50 0.22 0.50 0.22 0.32 0.26 0.32 0.26 0.24 0.31 0.24 0.31 0.26 0.35 0.26 0.35 0.18 0.39 0.18 0.39 0.06 0.44 0.06 0.44 0.21 0.48 0.21 0.48 0.09 0.52 0.09 0.52 0.00 0.57 0.00 0.57 0.00 0.61 0.00 0.61 0.00 0.65 0.00 0.65 0.03 0.70 0.03 0.70 0.00 0.74 0.00 0.74 0.03 0.78 0.03 0.78 0.00 0.83 0.00 0.83 0.03 0.87 0.03 0.87 0.00 0.91 0.00 0.91 0.00 0.96 0.00 0.96 0.00 1.00 0.00 1.00 0.00 /
\end{sparkline}
 &   2.0 & 1000.0 &   2.0 \\ 
Micronektonic squid                 & Predatory zooplankton               &   1.08e-11 & 
\begin{sparkline}{10}
\spark 0.03 0.00 0.03 0.09 0.07 0.09 0.07 0.29 0.11 0.29 0.11 0.41 0.15 0.41 0.15 0.50 0.20 0.50 0.20 0.38 0.24 0.38 0.24 0.24 0.28 0.24 0.28 0.15 0.32 0.15 0.32 0.24 0.37 0.24 0.37 0.29 0.41 0.29 0.41 0.09 0.45 0.09 0.45 0.06 0.49 0.06 0.49 0.09 0.54 0.09 0.54 0.00 0.58 0.00 0.58 0.00 0.62 0.00 0.62 0.00 0.66 0.00 0.66 0.00 0.70 0.00 0.70 0.00 0.75 0.00 0.75 0.00 0.79 0.00 0.79 0.00 0.83 0.00 0.83 0.00 0.87 0.00 0.87 0.03 0.92 0.03 0.92 0.00 0.96 0.00 0.96 0.00 1.00 0.00 1.00 0.00 /
\end{sparkline}
 &   2.0 & 1000.0 &   2.0 \\ 
Micronektonic squid                 & Sergestid shrimp                    &   1.01e-11 & 
\begin{sparkline}{10}
\spark 0.06 0.00 0.06 0.59 0.13 0.59 0.13 0.56 0.19 0.56 0.19 0.41 0.26 0.41 0.26 0.50 0.33 0.50 0.33 0.32 0.39 0.32 0.39 0.15 0.46 0.15 0.46 0.03 0.53 0.03 0.53 0.12 0.60 0.12 0.60 0.09 0.66 0.09 0.66 0.12 0.73 0.12 0.73 0.00 0.80 0.00 0.80 0.00 0.87 0.00 0.87 0.00 0.93 0.00 0.93 0.00 1.00 0.00 1.00 0.00 /
\end{sparkline}
 &   2.0 & 1000.0 &   2.0 \\ 
Micronektonic squid                 & Mesozooplankton                     &   9.18e-11 & 
\begin{sparkline}{10}
\spark 0.06 0.00 0.06 0.26 0.11 0.26 0.11 0.35 0.15 0.35 0.15 0.53 0.19 0.53 0.19 0.29 0.23 0.29 0.23 0.32 0.28 0.32 0.28 0.32 0.32 0.32 0.32 0.24 0.36 0.24 0.36 0.15 0.40 0.15 0.40 0.21 0.45 0.21 0.45 0.03 0.49 0.03 0.49 0.03 0.53 0.03 0.53 0.03 0.57 0.03 0.57 0.03 0.62 0.03 0.62 0.00 0.66 0.00 0.66 0.03 0.70 0.03 0.70 0.03 0.74 0.03 0.74 0.00 0.79 0.00 0.79 0.00 0.83 0.00 0.83 0.00 0.87 0.00 0.87 0.00 0.91 0.00 0.91 0.00 0.96 0.00 0.96 0.00 1.00 0.00 1.00 0.00 /
\end{sparkline}
 &   2.0 & 1000.0 &   2.0 \\ 
Micronektonic squid                 & Copepods                            &   7.17e-11 & 
\begin{sparkline}{10}
\spark 0.06 0.00 0.06 0.44 0.12 0.44 0.12 0.65 0.18 0.65 0.18 0.38 0.25 0.38 0.25 0.47 0.31 0.47 0.31 0.41 0.37 0.41 0.37 0.24 0.44 0.24 0.44 0.15 0.50 0.15 0.50 0.06 0.56 0.06 0.56 0.00 0.62 0.00 0.62 0.03 0.69 0.03 0.69 0.00 0.75 0.00 0.75 0.00 0.81 0.00 0.81 0.00 0.87 0.00 0.87 0.00 0.94 0.00 0.94 0.03 1.00 0.03 1.00 0.00 /
\end{sparkline}
 &   2.0 & 1000.0 &   2.0 \\ 
Mesopelagic fish                    & Chaetognaths                        &   2.58e-11 & 
\begin{sparkline}{10}
\spark 0.05 0.00 0.05 0.41 0.14 0.41 0.14 0.91 0.23 0.91 0.23 0.62 0.31 0.62 0.31 0.35 0.40 0.35 0.40 0.29 0.49 0.29 0.49 0.21 0.58 0.21 0.58 0.09 0.67 0.09 0.67 0.03 0.76 0.03 0.76 0.03 0.85 0.03 0.85 0.00 /
\end{sparkline}
 &   2.0 & 1000.0 &   2.0 \\ 
Mesopelagic fish                    & Predatory zooplankton               &   5.22e-12 & 
\begin{sparkline}{10}
\spark 0.03 0.00 0.03 0.50 0.12 0.50 0.12 0.68 0.21 0.68 0.21 0.71 0.30 0.71 0.30 0.41 0.38 0.41 0.38 0.26 0.47 0.26 0.47 0.09 0.56 0.09 0.56 0.15 0.65 0.15 0.65 0.06 0.74 0.06 0.74 0.03 0.82 0.03 0.82 0.06 0.91 0.06 0.91 0.00 /
\end{sparkline}
 &   2.0 & 1000.0 &   2.0 \\ 
Mesopelagic fish                    & Sergestid shrimp                    &   5.40e-12 & 
\begin{sparkline}{10}
\spark 0.04 0.00 0.04 0.50 0.12 0.50 0.12 0.68 0.20 0.68 0.20 0.50 0.28 0.50 0.28 0.53 0.36 0.53 0.36 0.32 0.44 0.32 0.44 0.09 0.52 0.09 0.52 0.09 0.60 0.09 0.60 0.06 0.68 0.06 0.68 0.03 0.76 0.03 0.76 0.09 0.84 0.09 0.84 0.03 0.92 0.03 0.92 0.00 1.00 0.00 1.00 0.00 /
\end{sparkline}
 &   2.0 & 1000.0 &   2.0 \\ 
Mesopelagic fish                    & Mesozooplankton                     &   7.85e-11 & 
\begin{sparkline}{10}
\spark 0.06 0.00 0.06 0.50 0.14 0.50 0.14 0.62 0.21 0.62 0.21 0.74 0.29 0.74 0.29 0.41 0.37 0.41 0.37 0.21 0.45 0.21 0.45 0.26 0.52 0.26 0.52 0.03 0.60 0.03 0.60 0.03 0.68 0.03 0.68 0.03 0.76 0.03 0.76 0.09 0.83 0.09 0.83 0.03 0.91 0.03 0.91 0.00 /
\end{sparkline}
 &   2.0 & 1000.0 &   2.0 \\ 
Mesopelagic fish                    & Copepods                            &   6.19e-11 & 
\begin{sparkline}{10}
\spark 0.05 0.00 0.05 0.41 0.13 0.41 0.13 0.56 0.20 0.56 0.20 0.62 0.28 0.62 0.28 0.56 0.35 0.56 0.35 0.35 0.42 0.35 0.42 0.21 0.50 0.21 0.50 0.09 0.57 0.09 0.57 0.06 0.64 0.06 0.64 0.03 0.72 0.03 0.72 0.06 0.79 0.06 0.79 0.00 /
\end{sparkline}
 &   2.0 & 1000.0 &   2.0 \\ 
Pelagic forage fish                 & Chaetognaths                        &   4.06e-12 & 
\begin{sparkline}{10}
\spark 0.03 0.00 0.03 0.38 0.11 0.38 0.11 0.68 0.20 0.68 0.20 0.88 0.29 0.88 0.29 0.29 0.38 0.29 0.38 0.26 0.47 0.26 0.47 0.15 0.55 0.15 0.55 0.09 0.64 0.09 0.64 0.06 0.73 0.06 0.73 0.06 0.82 0.06 0.82 0.09 0.91 0.09 0.91 0.00 /
\end{sparkline}
 &   2.0 & 1000.0 &   2.0 \\ 
Pelagic forage fish                 & Predatory zooplankton               &   2.99e-12 & 
\begin{sparkline}{10}
\spark 0.06 0.00 0.06 0.68 0.13 0.68 0.13 0.56 0.21 0.56 0.21 0.62 0.29 0.62 0.29 0.47 0.37 0.47 0.37 0.18 0.45 0.18 0.45 0.15 0.53 0.15 0.53 0.06 0.61 0.06 0.61 0.06 0.69 0.06 0.69 0.06 0.77 0.06 0.77 0.03 0.85 0.03 0.85 0.09 0.93 0.09 0.93 0.00 /
\end{sparkline}
 &   2.0 & 1000.0 &   2.0 \\ 
Pelagic forage fish                 & Sergestid shrimp                    &   2.83e-12 & 
\begin{sparkline}{10}
\spark 0.06 0.00 0.06 0.62 0.14 0.62 0.14 0.71 0.23 0.71 0.23 0.71 0.32 0.71 0.32 0.38 0.41 0.38 0.41 0.12 0.49 0.12 0.49 0.24 0.58 0.24 0.58 0.09 0.67 0.09 0.67 0.00 0.76 0.00 0.76 0.09 0.84 0.09 0.84 0.00 /
\end{sparkline}
 &   2.0 & 1000.0 &   2.0 \\ 
Pelagic forage fish                 & Mesozooplankton                     &   2.80e-11 & 
\begin{sparkline}{10}
\spark 0.07 0.00 0.07 0.53 0.14 0.53 0.14 0.59 0.22 0.59 0.22 0.71 0.30 0.71 0.30 0.44 0.37 0.44 0.37 0.21 0.45 0.21 0.45 0.18 0.53 0.18 0.53 0.15 0.60 0.15 0.60 0.03 0.68 0.03 0.68 0.12 0.75 0.12 0.75 0.00 /
\end{sparkline}
 &   2.0 & 1000.0 &   2.0 \\ 
Pelagic forage fish                 & Copepods                            &   1.99e-11 & 
\begin{sparkline}{10}
\spark 0.07 0.00 0.07 0.41 0.14 0.41 0.14 0.68 0.21 0.68 0.21 0.56 0.28 0.56 0.28 0.68 0.35 0.68 0.35 0.12 0.42 0.12 0.42 0.26 0.49 0.26 0.49 0.06 0.57 0.06 0.57 0.00 0.64 0.00 0.64 0.09 0.71 0.09 0.71 0.03 0.78 0.03 0.78 0.06 0.85 0.06 0.85 0.00 /
\end{sparkline}
 &   2.0 & 1000.0 &   2.0 \\ 
Saury                               & Chaetognaths                        &   2.89e-12 & 
\begin{sparkline}{10}
\spark 0.05 0.00 0.05 0.29 0.10 0.29 0.10 0.41 0.15 0.41 0.15 0.59 0.20 0.59 0.20 0.26 0.25 0.26 0.25 0.35 0.30 0.35 0.30 0.44 0.35 0.44 0.35 0.12 0.40 0.12 0.40 0.03 0.45 0.03 0.45 0.03 0.50 0.03 0.50 0.06 0.55 0.06 0.55 0.03 0.60 0.03 0.60 0.09 0.65 0.09 0.65 0.03 0.70 0.03 0.70 0.09 0.75 0.09 0.75 0.03 0.80 0.03 0.80 0.03 0.85 0.03 0.85 0.03 0.90 0.03 0.90 0.00 0.95 0.00 0.95 0.00 1.00 0.00 1.00 0.00 /
\end{sparkline}
 &   2.0 & 1000.0 &   2.0 \\ 
Saury                               & Predatory zooplankton               &   2.18e-12 & 
\begin{sparkline}{10}
\spark 0.06 0.00 0.06 0.50 0.14 0.50 0.14 0.76 0.21 0.76 0.21 0.79 0.29 0.79 0.29 0.21 0.37 0.21 0.37 0.24 0.45 0.24 0.45 0.12 0.53 0.12 0.53 0.06 0.61 0.06 0.61 0.06 0.69 0.06 0.69 0.06 0.76 0.06 0.76 0.06 0.84 0.06 0.84 0.03 0.92 0.03 0.92 0.03 1.00 0.03 1.00 0.00 /
\end{sparkline}
 &   2.0 & 1000.0 &   2.0 \\ 
Saury                               & Sergestid shrimp                    &   2.34e-12 & 
\begin{sparkline}{10}
\spark 0.03 0.00 0.03 0.26 0.08 0.26 0.08 0.53 0.13 0.53 0.13 0.56 0.18 0.56 0.18 0.29 0.23 0.29 0.23 0.38 0.27 0.38 0.27 0.26 0.32 0.26 0.32 0.09 0.37 0.09 0.37 0.09 0.42 0.09 0.42 0.09 0.47 0.09 0.47 0.00 0.52 0.00 0.52 0.03 0.56 0.03 0.56 0.00 0.61 0.00 0.61 0.06 0.66 0.06 0.66 0.03 0.71 0.03 0.71 0.03 0.76 0.03 0.76 0.03 0.81 0.03 0.81 0.00 0.85 0.00 0.85 0.03 0.90 0.03 0.90 0.03 0.95 0.03 0.95 0.00 1.00 0.00 1.00 0.00 /
\end{sparkline}
 &   2.0 & 1000.0 &   2.0 \\ 
Saury                               & Mesozooplankton                     &   2.00e-11 & 
\begin{sparkline}{10}
\spark 0.05 0.00 0.05 0.44 0.11 0.44 0.11 0.53 0.17 0.53 0.17 0.53 0.23 0.53 0.23 0.41 0.29 0.41 0.29 0.32 0.35 0.32 0.35 0.12 0.41 0.12 0.41 0.09 0.47 0.09 0.47 0.12 0.53 0.12 0.53 0.18 0.59 0.18 0.59 0.09 0.65 0.09 0.65 0.00 0.70 0.00 0.70 0.03 0.76 0.03 0.76 0.00 0.82 0.00 0.82 0.03 0.88 0.03 0.88 0.03 0.94 0.03 0.94 0.00 1.00 0.00 1.00 0.00 /
\end{sparkline}
 &   2.0 & 1000.0 &   2.0 \\ 
Saury                               & Copepods                            &   2.76e-11 & 
\begin{sparkline}{10}
\spark 0.06 0.00 0.06 0.47 0.12 0.47 0.12 0.62 0.19 0.62 0.19 0.62 0.25 0.62 0.25 0.47 0.31 0.47 0.31 0.12 0.37 0.12 0.37 0.15 0.44 0.15 0.44 0.12 0.50 0.12 0.50 0.09 0.56 0.09 0.56 0.06 0.62 0.06 0.62 0.06 0.69 0.06 0.69 0.03 0.75 0.03 0.75 0.06 0.81 0.06 0.81 0.06 0.87 0.06 0.87 0.00 0.94 0.00 0.94 0.00 1.00 0.00 1.00 0.00 /
\end{sparkline}
 &   2.0 & 1000.0 &   2.0 \\ 
Chum salmon                         & Micronektonic squid                 &   3.78e-13 & 
\begin{sparkline}{10}
\spark 0.05 0.00 0.05 0.12 0.12 0.12 0.12 0.41 0.18 0.41 0.18 0.82 0.25 0.82 0.25 0.65 0.31 0.65 0.31 0.35 0.38 0.35 0.38 0.21 0.44 0.21 0.44 0.21 0.51 0.21 0.51 0.09 0.57 0.09 0.57 0.03 0.64 0.03 0.64 0.03 0.70 0.03 0.70 0.03 0.77 0.03 0.77 0.00 /
\end{sparkline}
 &   2.0 & 1000.0 &   2.0 \\ 
Chum salmon                         & Mesopelagic fish                    &   8.61e-14 & 
\begin{sparkline}{10}
\spark 0.09 0.00 0.09 0.56 0.16 0.56 0.16 0.74 0.23 0.74 0.23 0.65 0.30 0.65 0.30 0.41 0.37 0.41 0.37 0.21 0.44 0.21 0.44 0.15 0.52 0.15 0.52 0.06 0.59 0.06 0.59 0.12 0.66 0.12 0.66 0.00 0.73 0.00 0.73 0.06 0.80 0.06 0.80 0.00 /
\end{sparkline}
 &   2.0 & 1000.0 &   2.0 \\ 
Chum salmon                         & Pelagic forage fish                 &   7.60e-14 & 
\begin{sparkline}{10}
\spark 0.07 0.00 0.07 0.41 0.14 0.41 0.14 0.53 0.21 0.53 0.21 0.65 0.28 0.65 0.28 0.47 0.35 0.47 0.35 0.50 0.42 0.50 0.42 0.18 0.49 0.18 0.49 0.09 0.56 0.09 0.56 0.09 0.63 0.09 0.63 0.00 0.70 0.00 0.70 0.00 0.77 0.00 0.77 0.03 0.84 0.03 0.84 0.00 /
\end{sparkline}
 &   2.0 & 1000.0 &   2.0 \\ 
Chum salmon                         & Chaetognaths                        &   3.89e-15 & 
\begin{sparkline}{10}
\spark 0.09 0.00 0.09 0.32 0.15 0.32 0.15 0.76 0.22 0.76 0.22 0.62 0.28 0.62 0.28 0.50 0.34 0.50 0.34 0.32 0.41 0.32 0.41 0.06 0.47 0.06 0.47 0.15 0.54 0.15 0.54 0.09 0.60 0.09 0.60 0.03 0.67 0.03 0.67 0.09 0.73 0.09 0.73 0.00 /
\end{sparkline}
 &   2.0 & 1000.0 &   2.0 \\ 
Chum salmon                         & Predatory zooplankton               &   1.45e-13 & 
\begin{sparkline}{10}
\spark 0.09 0.00 0.09 0.53 0.17 0.53 0.17 0.82 0.24 0.82 0.24 0.44 0.31 0.44 0.31 0.53 0.39 0.53 0.39 0.35 0.46 0.35 0.46 0.12 0.54 0.12 0.54 0.06 0.61 0.06 0.61 0.09 0.68 0.09 0.68 0.00 /
\end{sparkline}
 &   2.0 & 1000.0 &   2.0 \\ 
Chum salmon                         & Mesozooplankton                     &   2.10e-12 & 
\begin{sparkline}{10}
\spark 0.11 0.00 0.11 0.24 0.16 0.24 0.16 0.71 0.22 0.71 0.22 0.65 0.27 0.65 0.27 0.59 0.33 0.59 0.33 0.18 0.38 0.18 0.38 0.26 0.43 0.26 0.43 0.15 0.49 0.15 0.49 0.09 0.54 0.09 0.54 0.03 0.59 0.03 0.59 0.00 0.65 0.00 0.65 0.03 0.70 0.03 0.70 0.03 0.75 0.03 0.75 0.00 /
\end{sparkline}
 &   2.0 & 1000.0 &   2.0 \\ 
Chum salmon                         & Gelatinous zooplankton              &   3.79e-12 & 
\begin{sparkline}{10}
\spark 0.10 0.00 0.10 0.41 0.17 0.41 0.17 0.79 0.24 0.79 0.24 0.50 0.31 0.50 0.31 0.59 0.38 0.59 0.38 0.44 0.45 0.44 0.45 0.18 0.52 0.18 0.52 0.03 0.60 0.03 0.60 0.00 /
\end{sparkline}
 &   2.0 & 1000.0 &   2.0 \\ 
Chum salmon                         & Copepods                            &   2.87e-12 & 
\begin{sparkline}{10}
\spark 0.11 0.00 0.11 0.18 0.16 0.18 0.16 0.56 0.21 0.56 0.21 0.62 0.26 0.62 0.26 0.68 0.32 0.68 0.32 0.38 0.37 0.38 0.37 0.29 0.42 0.29 0.42 0.09 0.47 0.09 0.47 0.09 0.53 0.09 0.53 0.06 0.58 0.06 0.58 0.00 /
\end{sparkline}
 &   2.0 & 1000.0 &   2.0 \\ 
Large jellyfish                     & Chaetognaths                        &   1.33e-11 & 
\begin{sparkline}{10}
\spark 0.07 0.00 0.07 0.74 0.15 0.74 0.15 0.68 0.23 0.68 0.23 0.56 0.31 0.56 0.31 0.38 0.40 0.38 0.40 0.12 0.48 0.12 0.48 0.12 0.56 0.12 0.56 0.15 0.64 0.15 0.64 0.09 0.73 0.09 0.73 0.06 0.81 0.06 0.81 0.06 0.89 0.06 0.89 0.00 /
\end{sparkline}
 &   2.0 & 1000.0 &   2.0 \\ 
Large jellyfish                     & Predatory zooplankton               &   1.15e-11 & 
\begin{sparkline}{10}
\spark 0.05 0.00 0.05 1.00 0.17 1.00 0.17 0.79 0.28 0.79 0.28 0.41 0.40 0.41 0.40 0.26 0.52 0.26 0.52 0.29 0.64 0.29 0.64 0.06 0.76 0.06 0.76 0.06 0.88 0.06 0.88 0.03 1.00 0.03 1.00 0.00 /
\end{sparkline}
 &   2.0 & 1000.0 &   2.0 \\ 
Large jellyfish                     & Sergestid shrimp                    &   1.10e-11 & 
\begin{sparkline}{10}
\spark 0.05 0.00 0.05 0.88 0.15 0.88 0.15 0.65 0.25 0.65 0.25 0.56 0.35 0.56 0.35 0.32 0.45 0.32 0.45 0.26 0.56 0.26 0.56 0.09 0.66 0.09 0.66 0.09 0.76 0.09 0.76 0.06 0.86 0.06 0.86 0.03 0.96 0.03 0.96 0.00 /
\end{sparkline}
 &   2.0 & 1000.0 &   2.0 \\ 
Large jellyfish                     & Mesozooplankton                     &   9.25e-11 & 
\begin{sparkline}{10}
\spark 0.06 0.00 0.06 0.50 0.14 0.50 0.14 0.68 0.21 0.68 0.21 0.56 0.29 0.56 0.29 0.50 0.37 0.50 0.37 0.18 0.44 0.18 0.44 0.21 0.52 0.21 0.52 0.18 0.60 0.18 0.60 0.06 0.68 0.06 0.68 0.06 0.75 0.06 0.75 0.03 0.83 0.03 0.83 0.00 /
\end{sparkline}
 &   2.0 & 1000.0 &   2.0 \\ 
Large jellyfish                     & Gelatinous zooplankton              &   3.80e-11 & 
\begin{sparkline}{10}
\spark 0.06 0.00 0.06 0.38 0.12 0.38 0.12 0.68 0.18 0.68 0.18 0.26 0.24 0.26 0.24 0.62 0.30 0.62 0.30 0.35 0.36 0.35 0.36 0.18 0.41 0.18 0.41 0.09 0.47 0.09 0.47 0.12 0.53 0.12 0.53 0.03 0.59 0.03 0.59 0.09 0.65 0.09 0.65 0.03 0.71 0.03 0.71 0.03 0.77 0.03 0.77 0.03 0.82 0.03 0.82 0.00 0.88 0.00 0.88 0.00 0.94 0.00 0.94 0.03 1.00 0.03 1.00 0.00 /
\end{sparkline}
 &   2.0 & 1000.0 &   2.0 \\ 
Large jellyfish                     & Copepods                            &   2.85e-10 & 
\begin{sparkline}{10}
\spark 0.07 0.00 0.07 0.71 0.16 0.71 0.16 0.74 0.24 0.74 0.24 0.56 0.32 0.56 0.32 0.35 0.41 0.35 0.41 0.21 0.49 0.21 0.49 0.24 0.58 0.24 0.58 0.09 0.66 0.09 0.66 0.00 0.75 0.00 0.75 0.03 0.83 0.03 0.83 0.00 0.92 0.00 0.92 0.00 1.00 0.00 1.00 0.00 /
\end{sparkline}
 &   2.0 & 1000.0 &   2.0 \\ 
Chaetognaths                        & Mesozooplankton                     &   2.08e-10 & 
\begin{sparkline}{10}
\spark 0.08 0.00 0.08 0.35 0.16 0.35 0.16 0.76 0.24 0.76 0.24 0.94 0.32 0.94 0.32 0.21 0.40 0.21 0.40 0.35 0.48 0.35 0.48 0.18 0.55 0.18 0.55 0.12 0.63 0.12 0.63 0.00 0.71 0.00 0.71 0.03 0.79 0.03 0.79 0.00 /
\end{sparkline}
 & 1000.0 &  14.7 &   2.0 \\ 
Chaetognaths                        & Copepods                            &   8.71e-10 & 
\begin{sparkline}{10}
\spark 0.08 0.00 0.08 0.26 0.14 0.26 0.14 0.56 0.21 0.56 0.21 0.79 0.27 0.79 0.27 0.47 0.33 0.47 0.33 0.38 0.39 0.38 0.39 0.21 0.46 0.21 0.46 0.09 0.52 0.09 0.52 0.06 0.58 0.06 0.58 0.06 0.64 0.06 0.64 0.03 0.71 0.03 0.71 0.00 0.77 0.00 0.77 0.00 0.83 0.00 0.83 0.03 0.89 0.03 0.89 0.00 /
\end{sparkline}
 & 1000.0 &   3.3 &   2.0 \\ 
Predatory zooplankton               & Mesozooplankton                     &   1.93e-10 & 
\begin{sparkline}{10}
\spark 0.06 0.00 0.06 0.53 0.14 0.53 0.14 0.76 0.23 0.76 0.23 0.65 0.31 0.65 0.31 0.38 0.40 0.38 0.40 0.26 0.48 0.26 0.48 0.18 0.57 0.18 0.57 0.03 0.65 0.03 0.65 0.09 0.74 0.09 0.74 0.03 0.82 0.03 0.82 0.00 0.91 0.00 0.91 0.03 0.99 0.03 0.99 0.00 /
\end{sparkline}
 & 1000.0 &  19.6 &   2.0 \\ 
Predatory zooplankton               & Copepods                            &   8.24e-10 & 
\begin{sparkline}{10}
\spark 0.05 0.00 0.05 0.26 0.12 0.26 0.12 0.65 0.19 0.65 0.19 0.79 0.27 0.79 0.27 0.47 0.34 0.47 0.34 0.29 0.41 0.29 0.41 0.12 0.48 0.12 0.48 0.09 0.55 0.09 0.55 0.09 0.63 0.09 0.63 0.06 0.70 0.06 0.70 0.03 0.77 0.03 0.77 0.03 0.84 0.03 0.84 0.03 0.91 0.03 0.91 0.03 0.98 0.03 0.98 0.00 /
\end{sparkline}
 & 1000.0 &   7.8 &   2.0 \\ 
Sergestid shrimp                    & Mesozooplankton                     &   2.32e-10 & 
\begin{sparkline}{10}
\spark 0.04 0.00 0.04 0.32 0.12 0.32 0.12 0.59 0.19 0.59 0.19 0.68 0.27 0.68 0.27 0.59 0.34 0.59 0.34 0.29 0.42 0.29 0.42 0.12 0.50 0.12 0.50 0.15 0.57 0.15 0.57 0.09 0.65 0.09 0.65 0.06 0.72 0.06 0.72 0.06 0.80 0.06 0.80 0.00 /
\end{sparkline}
 & 1000.0 &  13.3 &   2.0 \\ 
Sergestid shrimp                    & Copepods                            &   9.38e-10 & 
\begin{sparkline}{10}
\spark 0.04 0.00 0.04 0.29 0.11 0.29 0.11 0.47 0.18 0.47 0.18 0.71 0.26 0.71 0.26 0.59 0.33 0.59 0.33 0.38 0.40 0.38 0.40 0.12 0.48 0.12 0.48 0.18 0.55 0.18 0.55 0.09 0.62 0.09 0.62 0.03 0.69 0.03 0.69 0.09 0.77 0.09 0.77 0.00 /
\end{sparkline}
 & 1000.0 &   3.7 &   2.0 \\ 
Mesozooplankton                     & Copepods                            &   1.86e-09 & 
\begin{sparkline}{10}
\spark 0.10 0.00 0.10 0.24 0.16 0.24 0.16 0.53 0.21 0.53 0.21 0.65 0.27 0.65 0.27 0.53 0.33 0.53 0.33 0.53 0.38 0.53 0.38 0.24 0.44 0.24 0.44 0.15 0.50 0.15 0.50 0.03 0.55 0.03 0.55 0.06 0.61 0.06 0.61 0.00 /
\end{sparkline}
 & 1000.0 &  22.2 &   2.0 \\ 
Mesozooplankton                     & Microzooplankton                    &   1.88e-09 & 
\begin{sparkline}{10}
\spark 0.09 0.00 0.09 0.18 0.14 0.18 0.14 0.44 0.20 0.44 0.20 0.53 0.25 0.53 0.25 0.71 0.30 0.71 0.30 0.41 0.35 0.41 0.35 0.24 0.40 0.24 0.40 0.29 0.46 0.29 0.46 0.09 0.51 0.09 0.51 0.00 0.56 0.00 0.56 0.03 0.61 0.03 0.61 0.00 0.66 0.00 0.66 0.03 0.72 0.03 0.72 0.00 /
\end{sparkline}
 & 1000.0 &  19.8 &   2.0 \\ 
Mesozooplankton                     & Large phytoplankton                 &   9.30e-10 & 
\begin{sparkline}{10}
\spark 0.10 0.00 0.10 0.32 0.15 0.32 0.15 0.35 0.20 0.35 0.20 0.53 0.26 0.53 0.26 0.62 0.31 0.62 0.31 0.47 0.37 0.47 0.37 0.35 0.42 0.35 0.42 0.24 0.47 0.24 0.47 0.03 0.53 0.03 0.53 0.03 0.58 0.03 0.58 0.00 /
\end{sparkline}
 & 1000.0 &  32.2 &   2.0 \\ 
Gelatinous zooplankton              & Copepods                            &   2.11e-10 & 
\begin{sparkline}{10}
\spark 0.10 0.00 0.10 0.74 0.17 0.74 0.17 0.74 0.24 0.74 0.24 0.41 0.31 0.41 0.31 0.44 0.38 0.44 0.38 0.21 0.45 0.21 0.45 0.18 0.52 0.18 0.52 0.12 0.59 0.12 0.59 0.03 0.66 0.03 0.66 0.03 0.72 0.03 0.72 0.03 0.79 0.03 0.79 0.00 0.86 0.00 0.86 0.00 0.93 0.00 0.93 0.00 1.00 0.00 1.00 0.00 /
\end{sparkline}
 & 1000.0 &  12.4 &   2.0 \\ 
Gelatinous zooplankton              & Microzooplankton                    &   2.18e-10 & 
\begin{sparkline}{10}
\spark 0.08 0.00 0.08 0.29 0.13 0.29 0.13 0.62 0.18 0.62 0.18 0.53 0.23 0.53 0.23 0.32 0.27 0.32 0.27 0.29 0.32 0.29 0.32 0.38 0.37 0.38 0.37 0.09 0.42 0.09 0.42 0.06 0.47 0.06 0.47 0.15 0.52 0.15 0.52 0.00 0.56 0.00 0.56 0.00 0.61 0.00 0.61 0.09 0.66 0.09 0.66 0.00 0.71 0.00 0.71 0.03 0.76 0.03 0.76 0.00 0.81 0.00 0.81 0.00 0.85 0.00 0.85 0.00 0.90 0.00 0.90 0.03 0.95 0.03 0.95 0.00 1.00 0.00 1.00 0.00 /
\end{sparkline}
 & 1000.0 &  10.8 &   2.0 \\ 
Gelatinous zooplankton              & Large phytoplankton                 &   4.20e-10 & 
\begin{sparkline}{10}
\spark 0.09 0.00 0.09 0.47 0.14 0.47 0.14 0.56 0.20 0.56 0.20 0.62 0.25 0.62 0.25 0.29 0.30 0.29 0.30 0.29 0.36 0.29 0.36 0.21 0.41 0.21 0.41 0.18 0.46 0.18 0.46 0.06 0.52 0.06 0.52 0.12 0.57 0.12 0.57 0.03 0.62 0.03 0.62 0.03 0.68 0.03 0.68 0.03 0.73 0.03 0.73 0.00 0.79 0.00 0.79 0.00 0.84 0.00 0.84 0.03 0.89 0.03 0.89 0.00 0.95 0.00 0.95 0.00 1.00 0.00 1.00 0.00 /
\end{sparkline}
 & 1000.0 &   8.6 &   2.0 \\ 
Copepods                            & Microzooplankton                    &   6.13e-09 & 
\begin{sparkline}{10}
\spark 0.17 0.00 0.17 0.15 0.21 0.15 0.21 0.71 0.25 0.71 0.25 0.68 0.28 0.68 0.28 0.56 0.32 0.56 0.32 0.62 0.36 0.62 0.36 0.24 0.40 0.24 0.40 0.00 /
\end{sparkline}
 & 1000.0 &  10.7 &   2.0 \\ 
Copepods                            & Small phytoplankton                 &   6.57e-09 & 
\begin{sparkline}{10}
\spark 0.15 0.00 0.15 0.03 0.18 0.03 0.18 0.18 0.21 0.18 0.21 0.56 0.24 0.56 0.24 0.53 0.27 0.53 0.27 0.71 0.30 0.71 0.30 0.38 0.33 0.38 0.33 0.15 0.36 0.15 0.36 0.15 0.39 0.15 0.39 0.12 0.42 0.12 0.42 0.06 0.45 0.06 0.45 0.06 0.49 0.06 0.49 0.00 0.52 0.00 0.52 0.03 0.55 0.03 0.55 0.00 /
\end{sparkline}
 & 1000.0 &   1.8 &   2.0 \\ 
Copepods                            & Large phytoplankton                 &   8.46e-09 & 
\begin{sparkline}{10}
\spark 0.14 0.00 0.14 0.09 0.18 0.09 0.18 0.41 0.22 0.41 0.22 0.76 0.26 0.76 0.26 0.62 0.30 0.62 0.30 0.35 0.34 0.35 0.34 0.38 0.38 0.38 0.38 0.21 0.42 0.21 0.42 0.09 0.45 0.09 0.45 0.03 0.49 0.03 0.49 0.00 /
\end{sparkline}
 & 1000.0 &   7.4 &   2.0 \\ 
Microzooplankton                    & Small phytoplankton                 &   3.39e-08 & 
\begin{sparkline}{10}
\spark 0.18 0.00 0.18 0.18 0.21 0.18 0.21 0.29 0.24 0.29 0.24 0.53 0.26 0.53 0.26 0.62 0.29 0.62 0.29 0.71 0.31 0.71 0.31 0.21 0.34 0.21 0.34 0.24 0.37 0.24 0.37 0.09 0.39 0.09 0.39 0.06 0.42 0.06 0.42 0.03 0.45 0.03 0.45 0.00 /
\end{sparkline}
 & 1000.0 &   2.9 &   2.0 \\
}







\bibliographystyle{meps}
\phantomsection\addcontentsline{toc}{section}{\refname}\bibliography{/Users/kakearney/Documents/Research/Literature/libraryEdited}

\end{document}
